%%%%%%%%%%%%%%%%%%%%%%%%%% lecture-16
%\begin{frame}[shrink]
%\frametitle{lecture-16 主要内容}
%\framesubtitle{建立自己的数据类型}
%%\tableofcontents[hideallsubsections]
%\tableofcontents
%\medskip
%\textbf{\textcolor{blue}{训练编程逻辑思维方式:}}
%\begin{itemize}
%	\item 领会结构化(模块化)编程思想。 
%	\item 用结构体管理相互关联的数据, 使其成为结构化的整体数据, 便于统一处理。 
%	\item 大问题分解为小问题, 设计函数解决小问题, 各个子函数彼此之间相互独立(便于调试, 不易出错),又可通过参数和返回值传递数据。
%	\item 主程序调用各个子函数, 解决``大"问题。
%\end{itemize}
%\end{frame}

\begin{frame}{进一步训练程序设计逻辑思维能力}
\begin{itemize}
	\setlength{\itemsep}{.5cm}
	\item 结构体是一种``高内聚, 低耦合''的数据结构。 
	\item 用结构体管理相互关联的数据, 使其成为结构化的整体数据, 便于统一处理。 
	\item 领会结构化(模块化)编程思想。
	\item 大问题分解为小问题, 相互关联的数据组织在一起,设计函数解决小问题, 各个子函数彼此之间相互独立(便于调试, 不易出错),又可通过参数和返回值传递数据。
	\item 主程序调用各个子函数, 解决``大"问题。
\end{itemize}
\end{frame}

\note{高内聚: 结构体使得各个不同类型的数据可以``聚合''在一起. 低耦合: 至与外界整体联系,比如学号,姓名,成绩。便于组织程序逻辑.}


\section{定义结构体数据类型:  struct\, 结构体类型名\{...\};}

\begin{frame}[shrink,fragile]{定义结构体数据类型:  struct 结构体类型名\{...\};}
\begin{columns}
\column{0.3\textwidth}<1->
定义结构体数据类型
\begin{lstlisting}
struct 结构体类型名
{
   数据类型 成员名1;
   数据类型 成员名2;
   ... // 定义其它成员
};
\end{lstlisting}
\column{0.7\textwidth}<2->
例: 定义结构体数据类型, 描述如下信息
\begin{tabular}{|c|c|c|c|c|c|}
\hline 
num & name & sex & age & score & addr \\ 
\hline 
10010 & Li Fang & M & 18 & 87.5 & Xian \\ 
\hline 
\end{tabular} 
\begin{lstlisting}
struct Student // 定义结构体数据类型struct Student
{
   int num;        // 学号为整型 
   char name[20];  // 姓名为字符串 
   char sex;       // 性别为字符型 
   int age;        // 年龄为整型
   float score;    // 成绩为实型 
   char addr[30];  // 地址为字符串 
};                 // 注意最后有一个分号 
\end{lstlisting}
\end{columns}
\end{frame}

\begin{frame}[shrink,fragile]{\small 嵌套定义结构体: 结构体的成员, 也可以是另外一个结构体变量。}
\vspace{-0.4cm}
\begin{lstlisting}
struct Date // 定义结构体类型 struct Date 
{
   int month; // 月
   int day;   // 日
   int year;  // 年
}; 
struct Student // 定义结构体类型 struct Student
{ 
   int num;  char name[20];  char sex;  int age;
   struct Date birthday; //成员birthday是struct Date类型变量
   char addr[30]; 
};
\end{lstlisting}
\begin{tabular}{|c|c|c|c|c|c|c|c|}
	\hline 
	num & name & sex & age & \multicolumn{3}{c|}{birthday} & addr \\ 
	\hline 
	10010 & Li Fang & M & 18 & month & day & year & Xian \\ 
	\hline 
\end{tabular} 
\medskip
\end{frame}

\section{定义结构体类型变量: struct\, 结构体类型名\, 变量名;}

\begin{frame}[shrink,fragile]{定义结构体类型变量: struct\, 结构体类型名\, 变量名;}
定义结构体类型后, 可以当作内置类型(如, int)一样进行变量定义和使用。
\vspace{-0.2cm}
\begin{columns}[T]
\column{0.5\textwidth}<1->
\begin{lstlisting}
#include<stdio.h>
....
// 定义结构体数据类型struct Student
struct Student 
{
   int num;
   char name[20]; 
   char sex; 
   int age; 
   float score; 
   char addr[30]; 
};         
\end{lstlisting}
\column{0.5\textwidth}<2->
\begin{lstlisting}[frame=leftline]
void fun1()
{
   // 定义结构体类型变量
   struct Student stu1, stu2; 
   .... 
}

// 结构体类型变量可作为函数的形式参数
void fun2(struct Student stu)
{
   ....
}
\end{lstlisting}
\end{columns}
\end{frame}

\begin{frame}[shrink,fragile]{结构体类型变量的初始化}
\begin{columns}[T]
\column{0.47\textwidth}
\begin{lstlisting}
#include<stdio.h>
....
// 定义结构体数据类型struct Student
struct Student 
{
   int num;
   char name[20]; 
   char sex; 
   int age; 
   float score; 
   char addr[30]; 
};         
\end{lstlisting}
\column{0.53\textwidth}
\begin{lstlisting}[frame=leftline]
void fun()
{
   // 定义结构体变量时, 初始化各成员变量
   struct Student stu1={10101,"LiLin",'M',18,90,"Xidian"},stu2; 
   .... 
}
\end{lstlisting}
\end{columns}
\end{frame}

\section{引用结构体变量的数据成员: 结构体变量名.成员名}

\begin{frame}[shrink,fragile]{引用结构体变量的数据成员: 结构体变量名.成员名}
定义结构体类型和结构体变量后, 通过\textcolor{blue}{``."成员运算符},引用(读写)其数据成员。
\vspace{-0.3cm}
\begin{columns}[T]
\column{0.5\textwidth}
\begin{lstlisting}
#include<stdio.h>
....
// 定义结构体数据类型struct Student
struct Student 
{
  int num;
  char name[20]; 
  char sex; 
  int age; 
  float score; 
  char addr[30]; 
};         
\end{lstlisting}
\column{0.5\textwidth}
\begin{lstlisting}[frame=leftline]
void fun()
{
  // 定义结构体变量时, 初始化各成员变量
  struct Student stu1={10101,"LiLin",'M',18,90,"Xidian"},stu2; 
  printf("%d,%s,%c,%d,%.2f,%s",stu1.num,stu1.name,stu1.sex,stu1.age,stu1.score,stu1.addr);
  stu2.num=10102;
  // 注: 字符串用数组表示, 不能直接赋值。
  strcpy(stu2.name,"Wanghong");   
}
\end{lstlisting}
\end{columns}
\end{frame}

\begin{frame}[shrink,fragile]{\small 对结构体变量的成员可以像普通变量一样进行各种运算}
\vspace{-0.5cm}
\begin{columns}[T]
\column{0.4\textwidth}
\begin{lstlisting}
struct Date  
{
  int month; 
  int day;   
  int year;  
}; 
struct Student 
{ 
  int num;
  char name[20];
  char sex;
  int age;
  struct Date birthday;
  char addr[30]; 
};
\end{lstlisting}
\column{0.6\textwidth}
\begin{lstlisting}[frame=leftline]
struct Student stu1,stu2;
stu1.num=10010;
stu1.sex='F';
stu1.birthday.month=6;// 逐级引用
stu1.age=18;
stu2.age=stu1.age+1;
stu2.age++;
stu2=stu1; // stu1中的各数据成员赋值给stu2
scanf("%d",&stu2.num);
gets(stu2.name); // 接受键盘输入的字符串
stu2.sex=stu1.sex; // 同性别
\end{lstlisting}
\textbf{\textcolor{blue}{``."成员运算符在所有运算符中优先级最高}}, stu2.num可以当作一个整体看待。
\end{columns}
\medskip
\end{frame}

\section{结构体指针变量和结构体数组}

\begin{frame}[shrink,fragile]{结构体指针变量. 数据成员引用: \lstinline|指针变量名->成员名|}
\vspace{-0.5cm}
\begin{columns}[T]
\column{0.4\textwidth}
\begin{lstlisting}
// 定义结构体
struct Student 
{ 
  int num;
  char name[20];
  char sex;
  int age;
  char addr[30]; 
};
\end{lstlisting}
\begin{block}{问题}<2->
	对\lstinline|stu2|的间接访问对吗?
\end{block}
\column{0.5\textwidth}
\begin{lstlisting}[frame=leftline]
// stu2是结构体指针变量
struct Student stu1,*stu2;
stu1.num=10010;
stu1.sex='F';
// 引用结构体指针变量指向的数据成员
(*stu2).num=10011; // 一定要有括号
stu2->num=10011; // 或
scanf("%d",&(*stu2).num);
scanf("%d",&(stu2->num)); // 或
gets((*stu2).name); 
gets(stu2->name); // 或
(*stu2).sex=stu1.sex; 
stu2->sex=stu1.sex; // 或
\end{lstlisting}
\end{columns}
\medskip
\end{frame}

\note{对\lstinline|stu2|的间接访问对吗?
	
	\begin{lstlisting}
	struct Student stu1, *stu2=&stu1;
	\end{lstlisting}
	要有指向!
}

\begin{frame}[shrink,fragile]{结构体数组}
\begin{columns}[T]
\column{0.4\textwidth}
\begin{lstlisting}
// 定义结构体
struct Student 
{ 
  int num;
  char name[20];
  char sex;
  int age;
  char addr[30]; 
};
\end{lstlisting}
\column{0.5\textwidth}
\begin{lstlisting}[frame=leftline]
// 结构体数组
struct Student stu[100];
scanf("%d",&stu[0].num);
gets(stu[0].name);
scanf("%c",&stu[0].sex);
stu[0].sex=getchar(); // 或
strcpy(stu[0].addr,"Xian");
\end{lstlisting}
\end{columns}
\medskip
\end{frame}

\section{结构体编程举例}

\begin{frame}[shrink,fragile]{例1: 输入学生信息并输出统计数据}
\vspace{-0.5cm}
\begin{columns}[T]
\column{0.4\textwidth}
\begin{lstlisting}
struct Student 
{ 
   int num;
   char name[20];
   // 5门课成绩
   float score[5];
};
\end{lstlisting}
\begin{block}{问题}<2->
	设计name成员, 为何不用指针形式:
	
	\lstinline|char *name;| 
\end{block}
\column{0.5\textwidth}
\begin{lstlisting}[frame=leftline]
int main()
{ 
  struct Student stu;
  int i; float sum=0;
  scanf("%d%s",&stu.num,stu.name);
  for(i=0;i<5;i++) 
     scanf("%f",&stu.score[i]); 
  // 计算与输出
  printf("%d %s",stu.num,stu.name);
  for(i=0;i<5;i++)
  { 
     printf(" %.2f ",stu.score[i]);
     sum=sum+stu.score[i];
  }
  printf("%.2f\n",sum/5.0);//平均成绩 
  return 0;
}
\end{lstlisting}
\end{columns}
\medskip
\end{frame}

%%%%%%%%%%% \note{lstinline中\%}
\note{指针变量name, 无法通过\lstinline|scanf("\%s",stu.name), gets(stu.name)|读入变量的值, 因为name无指向. 仅能name="Li"; 字符串常量给其赋值。}

\begin{frame}[shrink,fragile]{例2: 输入学生信息并输出统计数据(函数实现)}
\vspace{-0.5cm}
\begin{columns}[T]
\column{0.54\textwidth}
\begin{lstlisting}
struct Student 
{ 
  int num;
  char name[20];
  // 5门课成绩
  float score[5];
};
// 地址传递
void input(struct Student *stu)
{
  int i;
  scanf("%d%s",&(stu->num),stu->name);
  for(i=0;i<5;i++) 
    scanf("%f",&(stu->score[i])); 
    // 等效
    //scanf("%f",&((*stu).score[i]));
}
\end{lstlisting}
\column{0.50\textwidth}
\begin{lstlisting}[frame=leftline]
void print(struct Student stu)
{
  int i; float sum=0;
  printf("%d %s",stu.num,stu.name);
  for(i=0;i<5;i++)
  { 
     printf(" %.2f ",stu.score[i]);
     sum=sum+stu.score[i];
  }
  printf("%.2f\n",sum/5.0);
}
int main()
{ 
  struct Student stu;
  input(&stu); print(stu); 
  return 0;
}
\end{lstlisting}
\end{columns}
\medskip
\end{frame}

\begin{frame}[shrink,fragile]{例3: 输入$n$个学生信息(结构体数组)}
\begin{columns}[T]
\column{0.3\textwidth}
\begin{lstlisting}
// 定义结构体
struct Student 
{ 
  int num;
  char name[20];
  // 5门课成绩
  float score[5];
};
\end{lstlisting}
\column{0.6\textwidth}
\begin{lstlisting}[frame=leftline]
// 输入n个学生信息
//void input(struct Student stu[],int n)
void input(struct Student *stu, int n)
{
  int i,j;
  for(i=0;i<n;i++) 
  {
     scanf("%d%s",&stu[i].num,stu[i].name);
     for(j=0;j<5;j++) 
       scanf("%f",&stu[i].score[j]);
  } 
}
\end{lstlisting}
\end{columns}
\end{frame}

\begin{frame}[shrink,fragile]{例3: 输出$n$个学生信息以及统计数据(结构体数组)}
\vspace{-0.5cm}
\begin{columns}[T]
\column{0.6\textwidth}
\begin{lstlisting}
//void print(struct Student stu[],int n)
void print(struct Student *stu,int n)
{
  int i,j; float sum;
  for(i=0;i<n;i++)
  {
    printf("%d %s",stu[i].num,stu[i].name);
    sum=0;//注意在此处置0
    for(j=0;j<5;j++)
    { 
      printf(" %.0f ",stu[i].score[j]);
      sum=sum+stu[i].score[j];
    }
    printf("%.2f\n",sum/5.0);
  }
}
\end{lstlisting}
\column{0.4\textwidth}
\begin{lstlisting}[frame=leftline]
int main()
{ 
  struct Student stu[100];
  input(stu,2); // 输入2个学生
  print(stu,2); // 输出2个学生
  return 0;
}
\end{lstlisting}
例如, 输入两个学生的数据\\
1 zhang3 10 20 30 40 50\\
2 wang5 100 90 80 70 60\\
1 zhang3 10  20  30  40  50 30.00\\
2 wang5 100  90  80  70  60 80.00
\end{columns}
\medskip
\end{frame}

\begin{frame}[shrink,fragile]{例4: 输入学生信息并输出统计数据(结构体数组), 输入}
\vspace{-0.5cm}
\begin{columns}[T]
\column{0.35\textwidth}
\begin{lstlisting}
// 含平均成绩的结构体
struct Student 
{ 
  int num;
  char name[20];
  // 5门课成绩
  float score[5];
  // 平均成绩
  float aver;
};
\end{lstlisting}
\column{0.73\textwidth}
\begin{lstlisting}[frame=leftline]
// 输入并计算平均成绩, 返回平均成绩最高者(假定唯一)
struct Student input(struct Student *stu, int n)
{
  int i,j,maxIndex=0; float sum;
  for(i=0;i<n;i++) 
  {
    sum=0;
    scanf("%d%s",&stu[i].num,stu[i].name);
    for(j=0;j<5;j++)
    { 
      scanf("%f",&stu[i].score[j]);
      sum=sum+stu[i].score[j];
    }
    stu[i].aver=sum/5.0; //填充平均成绩
    //非唯一最大者,这里添加条件。复杂条件可以设计独立函数
    if(stu[i].aver > stu[maxIndex].aver) maxIndex=i;
  } 
  return stu[maxIndex];
}
\end{lstlisting}
\end{columns}
\medskip
\end{frame}

\begin{frame}[shrink,fragile]{例4: 输入学生信息并输出统计数据(结构体数组), 输出}
\begin{columns}[T]
\column{0.55\textwidth}
\begin{lstlisting}
void print(struct Student *stu,int n)
{
  int i,j;
  for(i=0;i<n;i++)
  {
    printf("%d %s ",stu[i].num,
                    stu[i].name);
    for(j=0;j<5;j++) 
      printf("%.2f ",stu[i].score[j]);
    printf("%.2f\n",stu[i].aver);
  }
}
\end{lstlisting}
\column{0.5\textwidth}
\begin{lstlisting}[frame=leftline]
int main()
{ 
  struct Student stu[100],maxStu;
  maxStu=input(stu,3); // 输入3个学生 
  print(stu,3); // 输出3个学生
  // 可复用print函数
  print(&maxStu,1); // 平均成绩最大者
  return 0;
}
\end{lstlisting}
\end{columns}
\medskip
\end{frame}

\begin{frame}[shrink,fragile]{\small 例5: 输入学生信息并输出统计数据(结构体数组),交换函数}
\begin{columns}[T]
\column{0.8\textwidth}
\begin{lstlisting}
// 交换两个结构体指针的内容, 地址传递
// 两个结构体的各数据成员互相交换 
void swap(struct Student *stu1, struct Student *stu2)
{
   struct Student tmp;
   tmp = *stu1; *stu1 = *stu2; *stu2 = tmp;
}

\end{lstlisting}
%\column{0.5\textwidth}
\end{columns}
\medskip
\end{frame}

\begin{frame}[shrink,fragile]{例5: 输入学生信息并输出统计数据(结构体数组), 排序}
\vspace{-0.5cm}
\begin{tikzpicture}
\node[text width=0.7\textwidth] (t) {
\begin{lstlisting}
void sort(struct Student stu[],int n)            
{
  int i,j,k;
  for(i = 0; i < n-1; i++) 
  {
    k = i; // 未经排序较大者
    for(j = i + 1; j < n; j++)
    {
       if(stu[j].aver > stu[k].aver) k=j;
       else if(stu[j].aver == stu[k].aver)
       { 
         if(stu[j].score[0] > stu[k].score[0] || 
            stu[j].score[1] > stu[k].score[1])
            k = j; 
       } 
    }
    if(k != i) swap(&stu[i],&stu[k]); // 交换
  } 
}
\end{lstlisting}
};
\node[anchor=west,text width=0.5\textwidth,draw] at(t.east) {
	定义排序函数(选择法,降序): 按平均值排序,如果平均值相同,按照前2门课比较单科成绩. 
};
\end{tikzpicture}
\end{frame}



