%%%%%%%%%%%%%%%%%%%%% chapter.tex %%%%%%%%%%%%%%%%%%%%%%%%%%%%%%%%%
%
% sample chapter
%
% Use this file as a template for your own input.
%
%%%%%%%%%%%%%%%%%%%%%%%% Springer-Verlag %%%%%%%%%%%%%%%%%%%%%%%%%%
%\motto{Use the template \emph{chapter.tex} to style the various elements of your chapter content.}
\chapter{第5次机试练习: 流程控制, 字符串, 数组}

\section{歌德巴赫猜想}
德巴赫猜想:任意一个大偶数都能分解为两个素数的和,对于输入的一个正偶数,写一个程序来验证歌德巴赫猜想。

由于每个正偶数可能分解成多组素数和,仅输出分解值分别是最小和最大素数的一组,按从小到大顺序输出。

输入说明	

输入一个正偶数n,1<n<1000。

输出说明
	
输出分解出的两个最小和最大素数。

输入样例
	
10

输出样例
	
3 7

\begin{lstlisting}
#include <stdio.h>
#include <math.h>

// 判断参数n是否素数, 如果是返回1, 否则返回0 
int isPrime(int n)
{
	int i;
	if(n < 2) return 0; // 最小素数是2, 1不是素数也不是合数, 题意(1<n<1000)不包含1,因此此语句不是必须的 
	//for(i = 2; i <= sqrt((double)n); i++) // vs2013编译器要求数学函数严格按原型解释
	for(i = 2; i <= sqrt(n); i++) // 或条件表达式:i*i<=n 
	{
		if (n%i == 0) return 0; // n不是素数 
	}
	return 1; // n是素数 
}

// 二重循环, 从最小素数开始迭代计算,获取符合题意的两个素数 
int main()
{
	int j,k,num; // num大偶数 
	int flag; // 标志变量:用于标识是否找到符合要求的素数对
	scanf("%d",&num);
	// 对于大偶数num,分解为两个素数 
	for(j = 2; j < num; j++)  //  找出第一个素数(最小的) for #1 
	{
		if(!isPrime(j)) continue; // 如果j不是素数, 继续下一轮迭代 
		
		flag = 0; // 初始化,未找出素数对 
		for(k = j + 1; k < num; k++) // //  找出第二个素数 for #2
		{
			if(isPrime(k) && j+k == num) // j是最小素数, k必然是最大素数
			{
				printf("%d %d\n",j,k);
				flag = 1; // 找出素数对, 如果没有此设置, 将会输出多组, 例如num=2020时会有多组素数 
				break; // break for #2
			}
		}
		if(flag) break; // break for #1 
	}
	return 0;
}

// 不用标志变量版本 
int main2()
{
	int j,k,num; // num大偶数 
	scanf("%d",&num);
	// 对于大偶数num,分解为两个素数 
	for(j = 2; j < num; j++)  //  找出第一个素数(最小的)  for #1 
	{
		if(!isPrime(j)) continue; // 如果j不是素数, 继续下一轮迭代 
		
		for(k = j + 1; k < num; k++) //  找出第二个素数 for #2
		{
			if(isPrime(k) && j+k == num) // j是最小素数, k必然是最大素数 
			{
				printf("%d %d\n",j,k);
				return 0; // 找出素数对,结束主函数。 如果不结束, 将会输出多组 
			}
		}
	}
	return 0;
}

// 优化,根据题意找出一组:最小素数+最大素数=偶数 
int main3()
{
	int j,k,num; // num大偶数 
	scanf("%d",&num);
	// 对于大偶数num,分解为两个素数 
	for(j = 2; j < num; j++)  //  找出第一个素数(最小的)  for #1 
	{
		if(!isPrime(j)) continue; // 如果j不是素数, 继续下一轮迭代 
		
		for(k = num-1; k>=2; k--) //  找出第二个素数(最大的) for #2
		{
			// j是最小素数, 判断k是否是最大素数并且二者之和=num 
			if(isPrime(k) && j+k == num) 
			{
				printf("%d %d\n",j,k);
				return 0; // 找出素数对,结束主函数。如果不结束, 将会输出多组 
			}
		}
	}
	return 0;
}

// 一重循环, 从最小素数开始迭代计算,获取符合题意的两个素数 
int main4() 
{ 
	int num,i;
	scanf("%d",&num);
	
	for(i = 2;i < num;i++)
	{
		if(isPrime(i)) // i是素数 
		{
			if(isPrime(num-i)) // 如果num-i也是素数, 即满足题意num=i+j 
			{
				printf("%d %d\n",i,num-i); 
				break;
			} 
		}
	}
	return 0; 
} 
\end{lstlisting}

\begin{note}[要点]
	再次体会标志变量的用法及内层循环前的初始化。仔细审题,本题要求一组输出: 最小素数+最大素数=偶数.
	
	如果要求找出最接近的一组素数, 因为$n=\frac{n}{2}+\frac{n}{2}$, 如果$\frac{n}{2}$是素数,即为所求。 否则,所求的两个素数在$\frac{n}{2}$附近, 分别小于和大于$\frac{n}{2}$。
	
	方法1: 修改main4()中的\lstinline|if(isPrime(num-i))为if(isPrime(num-i) && num-i<=i)|, 则输出\lstinline|num-i,i;|即可。 
	
	方法2: 由$i=\frac{n}{2}\to i=2$迭代, 第一个找到的素数$i$和$num-i$即为所求。
\end{note}

\section{回文数}
若一个非负整数其各位数字按照正反顺序读完全相同,则称之为回文数,例如12321。
判断输入的整数是否是回文数。若是,则输出该整数各位数字之和,否则输出no。

输入说明	

输入为一个整数n,0<=n<100000000。

输出说明	

若该整数为回文数,则输出整数各位数字之和,否则输出no。

输入样例
	
样例1输入

131

样例2输入

24

输出样例

样例1输出
	
5

样例2输出

no

\begin{lstlisting}
#include <stdio.h>
#include <string.h>

// 思路1:  求该整数的反序组成的整数,如整数1234, 其反序整数即为4321, 如果二者相等即为回文数 
// 判断num是否是回文数,是:返回1;不是,返回0
// 指针参数*sum,返回这个数的各位之和  
int isPalindromic1(int num,int *sum)
{
	int reverse = 0, tmp = num;
	*sum = 0; // 初始化指针内容 
	while(tmp)
	{
		reverse = reverse*10 + tmp%10;
		*sum += tmp%10;
		tmp /= 10;
	}
	if(reverse == num ) return 1;
	else return 0;	
}

// 思路2:  构造数组reverse, 反序存储该整数各位数,按照数组下标,前后数组元素相等则为回文数 
// 判断num是否是回文数,是:返回1;不是,返回0
// 指针参数*sum,返回这个数的各位之和  
int isPalindromic2(int num,int *sum)
{
	int reverse[9], len=0, i=0; // 依题意数组最大长度为9,最多存储9位数。实际长度用len变量表示 
	
	*sum = 0; // 初始换指针内容 
	//构造数组reverse, 反序存储该整数各位数
	while(num)
	{
		reverse[i]=num%10;  
		*sum += num%10;   // 累加各位数字 
		num /= 10;
		len++; // 计算数组实际长度 
		i++; 
	}
	//按照数组下标,前后数组元素相等则为回文数
	for(i=0; i<len/2;i++)
	{
		if(reverse[i] != reverse[len-i-1]) return 0;  // 不是回文数 
	}
	return 1;	// 至此,必然是回文数 
}

// 测试方法1和2的主程序
int main12()
{
	int i,num,sum;
	
	scanf("%d",&num); 
	//if(isPalindromic1(num,&sum)) printf("%d\n",sum);
	if(isPalindromic2(num,&sum)) printf("%d\n",sum);
	else printf("no\n");
	return 0;
} 

// 思路3:  按照字符串处理输入的整数,前后数组元素相等则为回文数 
int main3()
{
	char s[10]; // 留出'\0', 最多存储9位数。实际长度用len变量表示
	int sum=0, len=0,i=0; 
	
	// 以字符串形式接收输入的整数, 末尾自动追加'\0'  
	gets(s);
	
	// 计算len, 或者len=strlen(s), 同时计算各位数字之和 
	for(len=0;s[i]!='\0';) 
	{
		len++;
		sum=sum+s[i]-'0'; // 计算各位数字之和
		i++;  
	}
	
	//按照数组下标,前后数组元素相等则为回文数
	for(i=0; i<len/2;i++)
	{
		if(s[i] != s[len-i-1]) // 不是回文数  
		{
			printf("no\n");
			return 0;  // 主函数结束 
		} 
	}
	// 至此,必然是回文数,包含一位数不进入上面的for循环的情况 
	printf("%d\n",sum);
	return 0;
}

// 思路4:  按照字符串处理输入的整数,前后数组元素相等则为回文数。
// 使用指针操作 
int main()
{
	char s[10]; // 留出'\0', 最多存储9位数。实际长度用len变量表示
	int sum=0; 
	char *p1=s,*p2=s; // 用于正序和反序遍历s数组,初始指向第一个元素 
	
	// 以字符串形式接收输入的整数,末尾自动追加'\0' 
	gets(s);
	
	// 用p2遍历字符串,同时计算各位数字之和 
	for(;*p2!='\0';p2++) 
	{
		sum=sum+(*p2)-'0'; // 计算各位数字之和
	}
	// 至此,p2指向最后一个元素'\0', 我们使它指向最后一个有效元素: 
	p2--; 
	
	//按照数组下标,前后数组元素相等则为回文数
	for(;p1<p2;p1++,p2--)
	{
		if(*p1 != *p2) // 不是回文数 
		{
			printf("no\n");
			return 0;  // 主函数结束 
		} 
	}
	// 至此,必然是回文数 
	printf("%d\n",sum);
	return 0;
}
\end{lstlisting}

\begin{note}[要点]
	\begin{enumerate}
		\item 掌握函数的地址传递方法。
		\item 使用两个指针变量p1, p2, 其中p1指向待查找子串的首字母, 另一个指向末尾, \lstinline|p1++, p2--|; 是判断字符串是否是回文的有效技巧。
	\end{enumerate}
\end{note}

\section{寻找最大整数}
从键盘输入四个整数,找出其中的最大值并将其输出。

输入说明
	
输入4个整数,用空格分隔

输出说明
	
输出值最大的一个整数

输入样例
	
25 99 -46 0

输出样例
	
99

\begin{lstlisting}
#include <stdio.h>
// 不用存储整数序列, 采用一条循环语句,合并输入和计算,减少出错概率。 
int main() 
{
	int i, num, max;
	// 输入, 并计算 
	for(i = 0; i < 4; i++) 
	{ 
		if (i==0) scanf("%d",&max); // 假定第一个数就是最大的数 
		else
		{
			scanf("%d",&num);
			if(num > max) max=num; 
		} 
	} 
	printf("%d\n",max);
	return 0;
} 

int main1() // 另解, 存储整数序列
{
	int i, num[4], max;
	// 输入 
	for(i = 0; i < 4; i++) 
		scanf("%d",&num[i]);
	// 假定的最大值必须是实际存在的,不要想当然是0,9999,等等。
	max = num[0]; 
	for(i = 0; i < 4; i++)  
		if(max < num[i]) max = num[i];
	
	printf("%d\n",max);
	return 0;
} 
\end{lstlisting}

\begin{note}[要点]
	题目虽然简单,你能体会哪种解法更好? 特别注意假定变量的值必须是实际存在的数。
\end{note}

\section{ISBN号码}
每一本正式出版的图书都有一个ISBN号码与之对应,ISBN码包括9位数字、1位识别码和3位分隔符,其规定格式如"x-xxx-xxxxx-x",
其中符号“-”是分隔符(键盘上的减号),最后一位是识别码,例如0-670-82162-4就是一个标准的ISBN码。

ISBN码的首位数字表示书籍的出版语言,例如0代表英语;

第一个分隔符“-”之后的三位数字代表出版社,例如670代表维京出版社;

第二个分隔之后的五位数字代表该书在出版社的编号;

最后一位为识别码。识别码的计算方法如下:

首位数字乘以1加上次位数字乘以2……以此类推,用所得的结果mod 11,所得的余数即为识别码,如果余数为10,则识别码为大写字母X。
例如ISBN号码0-670-82162-4中的识别码4是这样得到的:
对067082162这9个数字,从左至右,分别乘以1,2,\dots,9,再求和,即$0\times 1+6\times 2+\dots +2\times 9=158$,然后取158 mod 11的结果4作为识别码。

编写程序判断输入的ISBN号码中识别码是否正确,如果正确,则仅输出``Right"; 如果错误,则输出正确的ISBN号码。

输入说明	

输入只有一行,是一个字符序列,表示一本书的ISBN号码(保证输入符合ISBN号码的格式要求)。

输出说明	

输出一行,假如输入的ISBN号码的识别码正确,那么输出``Right",否则,按照规定的格式,输出正确的ISBN号码(包括分隔符``-")。

输入样例	

样例输入1

0-670-82162-4

样例输入2

0-670-82162-0

输出样例	

样例输出1

Right

样例输出2

0-670-82162-4

\begin{lstlisting}
#include <stdio.h>
int main()
{
	char ISBN[14] = "0-670-82162-4";  // ISBN[13]='\0'
	int i,j, code1 = 0, code2;
	
	// 末尾自动添加'\0'.
	scanf("%s",ISBN);  // 或 gets(ISBN); 
	code2 = ISBN[12] == 'X' ? 'X' : ISBN[12]-'0';
	for(i = 0,j = 1; i < 11; i++)
	{
		if (ISBN[i]=='-') continue;
		// 整数与单个数字字符的关系:9 = '9' -'0' 
		code1 += (ISBN[i]-'0')*j;  
		j++;
	}   
	code1 %= 11;
	if (code1 == 10) code1 = 'X';  
	if(code1 == code2) printf("Right\n");
	else 
	{
		if(code1 == 'X') ISBN[12] = 'X';
		else ISBN[12] = code1 + '0';
		printf("%s\n",ISBN); // 或 puts(ISBN);
	}
	return 0;
} 
\end{lstlisting}

\begin{note}[要点]
	\begin{enumerate}
		\item 定义字符数组表示字符串时,且记给\lstinline|'\0'|留一个字符的位置, 表示字符串的结尾。
		\item \lstinline|ASCII编码(整数)=字符-'0';|
		\item \lstinline|字符=ASCII编码(整数)+'0';|
		\item 整数可以表示字符的ASCII编码(整数), 整数和字符类型可以``混用", 详见课件。
		\begin{lstlisting}
		int a; char c='A';
		a = c+1; // c当作整数运算
		printf("%d %c %d %c\n",a,a,c,c); //66 B 65 A 
		\end{lstlisting}
	\end{enumerate}
\end{note}

\section{密码强度}
每个人都有很多密码,你知道你的密码强度吗?假定密码由大写字母、小写字母、数字和非字母数字的符号这四类字符构成,密码强度计算规则如下:

1.	基础分:空密码(密码长度为零)0分, 非空密码1分

2.	加分项1:密码长度超过8位, +1分

3.	加分项2:密码包含两类不同字符+1分, 包含三类不同字符+2分, 包含四类不同字符+3分

按照此规则计算的密码强度为$0\sim 5$。请你设计一个程序计算给出的密码的强度。

输入说明
	
输入为一个密码字符串,字符串长度不超过50个字符。

输出说明
	
输出一个整数表示该密码的强度。

输入样例
	
输入样例1

abcd

输入样例2

ab123

输出样例
	
样例1输出:

1

样例2输出

2

\begin{lstlisting}	
#include <stdio.h>
#include <string.h>
int main()
{
	char p[51]; // 记得给'\0'留位置
	int i,strength = 0;
	// class4[0]=1大写字母, class4[1]=1小写字母, class4[2]=1数字, class4[3]=1非字母数字 
	int class4[4] = {0,0,0,0};  
	
	//scanf("%s",p);  // 不能完整接收含空格的字符串和空密码 
	gets(p); // last char: '\0',直接回车,就是空密码 
	
	// 	1.	基础分:空密码(密码长度为零)0分,非空密码1分 
	if(strlen(p) == 0) strength += 0;
	else  strength += 1;
	
	// 2.	加分项1:密码长度超过8位,+1分 
	if(strlen(p) > 8) strength += 1;
	
	// 3.	加分项2:密码包含两类不同字符+1分,包含三类不同字符+2分,包含四类不同字符+3分 
	for(i = 0; p[i] != '\0'; i++)
	{
		if(p[i] >= 'A' && p[i] <= 'Z') class4[0] = 1; 
		else if(p[i] >= 'a' && p[i] <= 'z') class4[1] = 1; 
		else if(p[i] >= '0' && p[i] <= '9') class4[2] = 1;
		else class4[3] = 1; 
	}
	int c = 0;
	for(i = 0; i < 4; i++) c += class4[i];
	
	if(c >= 4) strength += 3;
	else if(c >= 3) strength += 2;
	else if(c >= 2) strength += 1;
	
	printf("%d\n",strength);
	
	return 0;
}
\end{lstlisting}

\begin{note}[要点]
	字符串处理的典型问题: \lstinline|'\0'|, 字符串相关函数\lstinline|char s1[81],s2[81]; strlen(s1), strcmp(s1,s2), strcpy(s1,s2); scanf("%s",s1), gets(s1)|的区别等, 应该充分掌握。
\end{note}

