%%%%%%%%%%%%%%%%%%%%% chapter.tex %%%%%%%%%%%%%%%%%%%%%%%%%%%%%%%%%
%
% sample chapter
%
% Use this file as a template for your own input.
%
%%%%%%%%%%%%%%%%%%%%%%%% Springer-Verlag %%%%%%%%%%%%%%%%%%%%%%%%%%
%\motto{Use the template \emph{chapter.tex} to style the various elements of your chapter content.}
\chapter{第4次机试练习: 继续练习基本输入输出语句,分支与循环,简单数组应用}

\section{最小差值}
给定n个数,请找出其中相差(差的绝对值)最小的两个数,输出它们的差值的绝对值。

输入格式

输入第一行包含一个整数n。

第二行包含n个正整数,相邻整数之间使用一个空格分隔。

输出格式

输出一个整数,表示答案。

样例输入

5

1 5 4 8 20

样例输出

1

样例说明

相差最小的两个数是5和4,它们之间的差值是1。

样例输入

5

9 3 6 1 3

样例输出

0

样例说明

有两个相同的数3,它们之间的差值是0.

数据规模和约定

对于所有评测用例,$2\le n\le1000$,每个给定的整数都是不超过10000的正整数。

\begin{lstlisting}
#include <stdio.h>
#include <math.h>
#define N 10000 // 估计数组num的最大长度 
int main()
{
	int i,j,n,num[N],smallest,temp; 
	scanf("%d",&n);
	// 输入数组各元素
	for(i=0;i<n;i++) // 实际数组的最大长度n, 下标由0到(n-1)
	{
		scanf("%d",&num[i]);
	}
	// 初始的最小值就是前两个数的差值, 注意初始化值必须是实际存在的值,而不能想当然给值。
	smallest=(int)fabs(num[0]-num[1]); // 整数绝对值函数int abs(int x)在低版本编译器中有问题, 此处用双精度绝对值函数代替, 其结果转换为整数。  
	
	// 前后两项比较
	for(i=0;i<=n-2;i++) // 循环变量i用于访问数组元素, 注意数组边界问题
	{
		for(j=i+1;j<n;j++)
		{
			temp=(int)fabs(num[i]-num[j]);
			if(smallest>temp) smallest=temp;
		}
	}
	printf("%d\n",smallest);
	return 0;
} 
\end{lstlisting}

\begin{note}[整数求绝对值函数]
	\lstinline|int abs(int x);| 在有些低版本编译器中,\lstinline|math.h|头文件无此函数原型说明,可用\lstinline|double fabs(double x);|代替。见本例。
\end{note}

\section{车牌限行}
受雾霾天气影响,某市决定当雾霾指数超过设定值时对车辆进行限行,假设车牌号全为数字,且长度不超过6位,限行规则如下: 
\begin{enumerate}
	\item 限行时间段只包括周一至周五,周六周日不限行;
	\item 如果雾霾指数低于200,不限行;
	\item 如果雾霾指数大于等于200且低于400,每天限行两个尾号的汽车,周一限行1和6,周二限行2和7,周三限行3和8,周四限行4和9,周五限行5和0;
	\item 如果雾霾指数大于等于400,每天限行五个尾号的汽车,周一、周三和周五限行1,3,5,7,9,周二和周四限行0,2,4,6,8。 	
\end{enumerate}
现在给出星期几、雾霾指数和车牌号,判断该车牌号是否限行。 

输入说明

输入分为三个整数,第一个整数表示星期几(1~7,1表示周一,2表示周二,依次类推,7表示周日),

第二个整数表示雾霾指数(0~600),第三个整数表示车牌号,整数之间用空格分隔。

输出说明

输出为两个部分,第一部分为车牌最后一位数字,第二部分为限行情况,限行输出yes,不限行输出no。

输入样例

输入样例1 

4 230 80801 

输入样例2 

3 300 67008

输出样例

输出样例1 

1 no 

输出样例2 

8 yes 
\begin{lstlisting}
int main1()
{
	int  week, hazeIndex, No; // 星期几, 雾霾指数, 车牌号码
	int LastNo; // 车牌号最后一位数字
	int control=0; // 标志变量,0:不限行; 1: 限行 
	
	scanf("%d%d%d",&week,&hazeIndex,&No);
	LastNo=No%10; 
	switch(week)
	{
		case 1: 
			if(hazeIndex>=200 && hazeIndex<400 && (LastNo==1 || LastNo==6)) 
				control=1;
			if(hazeIndex>=400 && (LastNo%2 != 0)) 
				control=1;  
			break; 
		case 2: 
			if(hazeIndex>=200 && hazeIndex<400 && (LastNo==2 || LastNo==7)) 
				control=1;
			if(hazeIndex>=400 && (LastNo%2 == 0)) 
				control=1; 
			break;
		case 3: 
			if(hazeIndex>=200 && hazeIndex<400 && (LastNo==3 || LastNo==8)) 
				control=1;
			if(hazeIndex>=400 && (LastNo%2 != 0)) 
				control=1; 
			break; 
		case 4: 
			if(hazeIndex>=200 && hazeIndex<400 && (LastNo==4 || LastNo==9))
				control=1;
			if(hazeIndex>=400 && (LastNo%2 == 0)) 
				control=1; 
			break;
		case 5: 
			if(hazeIndex>=200 && hazeIndex<400 && (LastNo==5 || LastNo==0))
				control=1;
			if(hazeIndex>=400 && (LastNo%2 != 0)) 
				control=1; 
			break;
		case 6:
		case 7: break;
	} 
	if(control==1) printf("%d yes",LastNo);
	else printf("%d no",LastNo);
	return 0;
} 

int main1() // 另解(数组标志变量)
{
	int  week, hazeIndex, No; // 星期几, 雾霾指数, 车牌号码
	int LastNo; // 车牌号最后一位数字
	int control[2][5][10]={
		// hazeIndex>=200 && hazeIndex<400
		{ 
			{0,1,0,0,0,1,0,0,0,0}, // 周一 
			{0,0,1,0,0,0,0,1,0,0}, // 周二 
			{0,0,0,1,0,0,0,0,1,0}, // 周三
			{0,0,0,0,1,0,0,0,0,1}, // 周四 
			{1,0,0,0,1,0,0,0,0,0}, // 周五 
		}, 
		// hazeIndex>=400
		{
			{0,1,0,1,0,1,0,1,0,1}, // 周一 
			{1,0,1,0,1,0,1,0,1,0}, // 周二 
			{0,1,0,1,0,1,0,1,0,1}, // 周三 
			{1,0,1,0,1,0,1,0,1,0}, // 周四 
			{0,1,0,1,0,1,0,1,0,1}, // 周五  
		}}; 

	scanf("%d%d%d",&week,&hazeIndex,&No);
	LastNo=No%10; 
	if(hazeIndex>=200 && hazeIndex<400)
	{
		if(control[0][week-1][LastNo]) printf("%d yes",LastNo);
		else printf("%d no",LastNo);
	} 
	else if(hazeIndex>=400)
	{
		if(control[1][week-1][LastNo]) printf("%d yes",LastNo);
		else printf("%d no",LastNo);
	}
	else
	{
		printf("%d no",LastNo);
	}
	return 0;
} 
\end{lstlisting}

\begin{note}[要点]
	\begin{enumerate}
		\item 首先假定标志变量的值(如,\lstinline|int control=0;|), 再根据题目要求, 计算它的真实值, 是基本技巧。
		\item 用数组作为标志变量(如,\lstinline|control[2][5][10];|是另一技巧。
	\end{enumerate}	
\end{note}

\section{PM2.5}
给出一组PM2.5数据,按以下分级标准统计各级天气的天数,并计算出PM2.5平均值。
PM2.5分级标准为:\\
一级优(0<=PM2.5<=50)\\
二级良(1<=PM2.5<=100)\\
三级轻度污染(101<=PM2.5<=150)\\
四级中度污染(151<=PM2.5<=200)\\
五级重度污染(201<=PM2.5<=300)\\
六级严重污染(PM2.5>300)\\

输入说明
	
输入分为两行,

第一行是一个整数n表示天数(1<n<=100);

第二行为n个非负整数Pi(0<=Pi<=1000),表示每天的PM2.5值,整数之间用空格分隔。

输出说明
	
输出两行数据,

第一行为PM2.5平均值,结果保留2位小数;

第二行依次输出一级优,二级良,三级轻度污染,四级中度污染,五级重度污染,六级严重污染的天数。

输入样例
	
10

50 100 120 80 200 350 400 220 180 165

输出样例	

186.50

1 2 1 3 1 2

\begin{lstlisting}
#include <stdio.h>

int main()
{
	// 用数组变量day存储数据,避免设置6个变量存储。
	int i =0,n,pm25,day[6] = {0,0,0,0,0,0},sum = 0;
	scanf("%d",&n);
	while(i < n) 
	{
		scanf("%d",&pm25);
		sum += pm25;
		if(pm25 >= 0 && pm25 <= 50 ) day[0]++;
		else if(pm25 >= 51 && pm25 <= 100 ) day[1]++;
		else if(pm25 >= 101 && pm25 <= 150 ) day[2]++;
		else if(pm25 >= 151 && pm25 <= 200 ) day[3]++;
		else if(pm25 >= 201 && pm25 <= 300 ) day[4]++;
		else day[5]++;
		i++;
	} 
	printf("%.2f\n",(float)sum/n);
	for(i = 0; i < 6; i++)  // 视作一条语句, 省略{ }
		if(i == 5) printf("%d\n",day[i]);
		else  printf("%d ",day[i]);
	
	return 0;
} 
\end{lstlisting}

\begin{note}[要点]
	\lstinline|if() { } else if() { } else { }|的用法, 循环语句的\{ \}.
	
	体会数组变量day的使用, 避免变量过多,难于管理的麻烦。
\end{note}

\section{气温波动}	
最近一段时间气温波动较大。已知连续若干天的气温,请给出这几天气温的最大波动值是多少,即在这几天中某天气温与前一天气温之差的绝对值最大是多少。

输入说明	

输入数据分为两行。

第一行包含了一个整数n,表示给出了连续n天的气温值,$2\le n\le 30$。

第二行包含n个整数,依次表示每天的气温,气温为-20到40之间的整数。

输出说明
	
输出一个整数,表示气温在这n天中的最大波动值。

输入样例	

6

2 5 5 7 -3 5

输出样例
	
10

\begin{lstlisting}
#include <stdio.h>
#include <math.h>
// 不使用数组存储气温值
int main1()
{
	// last: 前一天的气温, temperature: 当天气温, undulation: 波动值 
	int i, n, last ,temperature, undulation = 0;
	scanf("%d",&n);
	
	// 当天气温temperature与前一天气温比较
	for(i = 0; i < n; i++)
	{
		scanf("%d",&temperature);
		// 注意i==0时, last无值
		if (i != 0 && fabs(temperature - last) > undulation) 
		undulation = fabs(temperature - last);
		last = temperature;
	} 
	printf("%d\n",undulation); 
	return 0;
} 

// 使用数组存储最多30个气温值 
int main()
{
	int i, n,temperature[30], undulation = 0; // temperature数组: 气温值,  undulation: 波动值 
	scanf("%d",&n);
	for(i = 0; i < n; i++)
	{
		scanf("%d",&temperature[i]);
	} 
	
	undulation= fabs(temperature[0]-temperature[1]); // 初始波动值
	
	for(i = 2; i < n-1; i++) // 注意数组边界,保证数组不越界 
	{
		if (fabs(temperature[i] - temperature[i+1]) > undulation) 
		undulation = fabs(temperature[i] - temperature[i+1]);
	} 
	
	printf("%d\n",undulation); 
	return 0;
} 
\end{lstlisting}

\begin{note}
	借助变量last表示前一天的气温,即可不用数组存储所有数据,是基本技巧。
\end{note}
	

\section{折点计数}	
给定n个整数表示一个商店连续n天的销售量。如果某天之前销售量在增长,而后一天销售量减少,则称这一天为折点,反过来如果之前销售量减少而后一天销售量增长,也称这一天为折点,其他的天都不是折点。如图所示,第3天和第6天是折点。

\includegraphics*[scale=0.7]{points}

给定n个整数a1, a2, \dots, an表示连续n天中每天的销售量。请计算出这些天总共有多少个折点。

输入说明
	
输入的第一行包含一个整数n。

第二行包含n个整数,用空格分隔,分别表示a1, a2, \dots, an。

$3\le n\le 100$,每天的销售量是不超过1000的非负整数。为了减少歧义,输入数据保证:在这n天中相邻两天的销售量总是不同的,即$ai-1\ne ai$。

输出说明	

输出一个整数,表示折点数量。

输入样例	

7

5 4 1 2 3 6 4

输出样例
	
2

\begin{lstlisting}
#include <stdio.h>
int main()
{
    // 估计数组a的实际长度为100, 实际长度是n(待输入的值)
	int i = 0, points = 0, n = 7, a[100] = {5,4,1,2,3,6,4}; // 把样例数据作为初始化,方便了调试
	int up; // 标志变量 
	
	// 有了初始化数据,这些输入语句在调试时就可注释掉,达到快速调试程序逻辑的目的。
	scanf("%d",&n); // 输入数组a的实际长度
	// 输入各元数值
	for(i = 0; i < n; i++) scanf("%d",&a[i]);
	
	// 标志变量up的初始值必须是真实存在的值,不要想当然。
	up = a[1] > a[0] ? 1 : -1; // 如果a[1] > a[0]成立, up=1,否则up=-1
	for(i = 2; i < n; i++)
	{
		if((a[i] > a[i-1] && up < 0) || (a[i] < a[i-1] && up > 0))
			points++;
		up = a[i] > a[i-1] ? 1 : -1;
	}
	
	printf("%d\n",points);
	return 0;
} 
\end{lstlisting}

\begin{note}[要点]
	善用标志变量, 标志变量的初始值必须是真实存在的值,不要想当然。
	
	样例数据作为初始化数据,调试时注释掉输入语句,便于进行快速调试程序。
\end{note}
