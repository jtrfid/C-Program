%%%%%%%%%%%%%%%%%%%%% chapter.tex %%%%%%%%%%%%%%%%%%%%%%%%%%%%%%%%%
%
% sample chapter
%
% Use this file as a template for your own input.
%
%%%%%%%%%%%%%%%%%%%%%%%% Springer-Verlag %%%%%%%%%%%%%%%%%%%%%%%%%%
%\motto{Use the template \emph{chapter.tex} to style the various elements of your chapter content.}
\chapter{第1次机试练习: 熟悉DEV-C++开发平台, 基本输入输出语句练习}

\section{计算球体重量}
已知铁的比重是7.86(克/立方厘米),金的比重是19.3(克/立方厘米)。写一个程序,分别计算出给定直径的铁球与金球的质量,假定PI=3.1415926

输入说明:\\
输入两个整数,分别表示铁球与金球的直径(单位为毫米)

输出说明:\\
输出两个浮点数,分别表示铁球与金球的质量(单位为克),小数点后保留3位小数,两个浮点数之间用空格分隔

输入样例:\\
100 100

输出样例:\\
4115.486  10105.456

提示: \\
用\lstinline|scanf|输入,用\lstinline|printf|输出,保留3位小数的格式控制字符为\lstinline|%.3f|

\begin{lstlisting}
#include<stdio.h>
#include<math.h>   // 数学库函数 
#define PI 3.1415926       
int main()                   
{  
	int a,b;
	scanf("%d%d",&a,&b);
	float v1= 4.0/3.0*pow(a/2.0/10,3)*PI;
	float v2= 4.0/3.0*pow(b/2.0/10,3)*PI;
	printf("%.3f %.3f\n",7.86*v1,19.3*v2); 
	return 0;           
}                   
\end{lstlisting}

\begin{note}[要点]
	\begin{enumerate}
		\item 整数除以整数, 结果为整数。\\
		4.0/3.0结果是浮点数,4/3结果是整数
		\item 化简公式会引起精度问题, 不要随意化简公式。
		\item pow函数原型: \lstinline|double pow(double x,double y)|\\
		当形参数是整数时, 由于精度问题,不要使用此函数计算$x^y$. 推荐使用循环语句, 易计算$x^y$。 如果必要, 可自定义函数: \lstinline|int mypow(int x,int y)|。 见课件。
	\end{enumerate}
\end{note}

\section{温度转化}
已知华氏温度到摄氏温度的转换公式为:摄氏温度= (华氏温度- 32)×5/9,写程序将给定的华氏温度转换为摄氏温度输出。

输入说明:\\
只有一个整数,表示输入的华氏温度

输出说明:\\
输出一个表示摄氏温度的实数,小数点后保留2位有效数字,多余部分四舍五入

输入样例:\\
50

输出样例:\\
10.00

提示:\\
用scanf输入,用printf输出,保留2位小数的格式控制字符为%.2f

\begin{lstlisting}
#include <stdio.h>

int main()
{
	int f;
	float c;
	scanf("%d",&f);
	c = (f-32)*5.0/9;     // (1)
	//c = (f-32)*5/9;     // (2)
	printf("%.2f\n",c);
	return 0;
} 
\end{lstlisting}

\begin{note}[思考]
	为何语句(1),(2)计算结果不一致, 哪一条语句正确?
\end{note}


\section{整数简单运算}
编写程序,计算用户输入的两个整数的和、差、乘积(*)和商(/)。

输入格式:输入两个整数,整数之间用空格分隔。

输出格式:输出四个整数结果,分别表示和、差、积和商,每输出一个结果换行。

输入样例:\\
3 4

输出样例:\\
7  \\
-1 \\
12 \\
0

\begin{lstlisting}
#include<stdio.h>
int main()                   
{  
	int a,b;
	scanf("%d%d",&a,&b);
	printf("%d\n%d\n%d\n%d\n",a+b,a-b,a*b,a/b); 
	return 0;           
}             
\end{lstlisting}

\begin{note}[思考]
	b=0时如何处理?
\end{note}

\section{A+B+C}
通过键盘输入三个整数a,b,c,求3个整数之和。

输入说明:

三整形数据通过键盘输入,输入的数据介于-100000和100000之间,整数之间以空格、跳格或换行分隔。 

输出说明:

输出3个数的和。

输入样例:

-6  0  39

输出样例:

33

\begin{lstlisting}
#include<stdio.h>  
int main()                   
{  
	int a,b,c;
	scanf("%d%d%d",&a,&b,&c);
	printf("%d\n",a+b+c); 
	return 0;           
}           
\end{lstlisting}

\section{字符输入输出}
通过键盘输入5个大写字母,输出其对应的小写字母,并在末尾加上“!”。

输入说明:

5个大写字母通过键盘输入,字母之间以竖线“|”分隔。

输出说明:

输出5个大写字母对应的小写字母,之间无分隔,并在末尾加上‘!’。

输入样例:

H|E|L|L|O

输出样例:

hello!

\begin{lstlisting}
#include<stdio.h>     
int main()                   
{  
	char c1,c2,c3,c4,c5;
	scanf("%c|%c|%c|%c|%c",&c1,&c2,&c3,&c4,&c5);
	c1+=32; c2+=32; c3+=32; c4+=32; c5+=32;
	printf("%c%c%c%c%c!",c1,c2,c3,c4,c5); 
	return 0;           
}           
\end{lstlisting}

\begin{note}[要点]
	\lstinline|scanf("原样输入",...);|
\end{note}

\begin{note}(大小写字符转化关系)
	小写字符ASCII码=大写字符ASCII码+32
\end{note}

\section{数字字符}
通过键盘输入1个整数\lstinline|a(0<=a<=4)|,1个数字字符\lstinline|b('0'<=b<='5')|求a+b。

输入说明:

整形数据、数字字符通过键盘输入,输入的整形数据介于0和4之间,输入的数字字符介于`0'和`5'之间,二个输入数之间用``,"分隔。

输出说明:

分别以整数形式及字符形式输出a+b,输出的二个数之间用``,"分隔。

输入样例:

3,5

输出样例:

56,8

\newpage

\begin{lstlisting}
#include<stdio.h>    
int main()                   
{  
	int a;
	char b;
	scanf("%d,%c",&a,&b);
	printf("%d,%c",a+b,a+b); 
	return 0;           
}             
\end{lstlisting}

\begin{note}(scanf函数)
	\lstinline|scanf("原样输入",...); |
\end{note}

\begin{note}(整型数值与字符混合运算)
	字符对应的ASCII编码参与整数运算, 其结果也是整数。注意\lstinline|'0'|与\lstinline|0|不同, 本例中输入0,0, 则\lstinline|a=0|,\lstinline|b='0'|, 变量a的值是整数0, 变量b的值是字符\lstinline|'0'|对应的ASCII编码, 即整数48。
	
	如果本题改为计算整数a+(字符b对应的数字), 则, \lstinline|printf("%d",a+b-'0');|
\end{note}

\section{实数运算}
通过键盘输入长方体的长、宽、高,求长方体的体积V(单精度)。

输入说明:

十进制形式输入长、宽、高,输入数据间用空格分隔。

输出说明:

单精度形式输出长方体体积V,保留小数点后3位,左对齐。

输入样例:

15  8.12  6.66

输出样例:

811.188


\begin{lstlisting}
#include<stdio.h>   
int main()                   
{  
	float a,b,c;
	scanf("%f%f%f",&a,&b,&c);
	printf("%.3f",a*b*c); 
	return 0;           
}                  
\end{lstlisting}

\begin{note}(精度问题)
	32位编译器: \lstinline|a*b*c| 与 \lstinline|a*c*b| 结果一致。
	但是在64位编译器中, 二者不一致。
	
	因此, 浮点数运算会存在精度问题, 不要随意改变运算顺序。
\end{note}
