%%%%%%%%%%%%%%%%%%%% book.tex %%%%%%%%%%%%%%%%%%%%%%%%%%%%%
%
% sample root file for the chapters of your "monograph"
%
% Use this file as a template for your own input.
%
%%%%%%%%%%%%%%%% Springer-Verlag %%%%%%%%%%%%%%%%%%%%%%%%%%


% RECOMMENDED %%%%%%%%%%%%%%%%%%%%%%%%%%%%%%%%%%%%%%%%%%%%%%%%%%%
\documentclass[graybox,envcountchap,sectrefs]{svmono}

% choose options for [] as required from the list
% in the Reference Guide

\usepackage{mathptmx}
\usepackage{helvet}
\usepackage{courier}
\usepackage{amsmath}

\usepackage{amssymb}
\usepackage{mathrsfs}
\usepackage{amsfonts}
%
\usepackage{type1cm}

\usepackage{makeidx}         % allows index generation
\usepackage{graphicx}        % standard LaTeX graphics tool
                             % when including figure files
\usepackage{multicol}        % used for the two-column index
\usepackage[bottom]{footmisc}% places footnotes at page bottom

\usepackage{wrapfig}
\usepackage{picins}
%xxx\usepackage{custom}

\usepackage{tikz}
\usetikzlibrary{positioning,backgrounds}
\usetikzlibrary{fadings}
\usetikzlibrary{patterns}
\usetikzlibrary{calc}
\usetikzlibrary{shadings}
\usetikzlibrary{shapes.symbols,shapes.misc,plotmarks,decorations.text,shapes.geometric} % shapes.symbols
\usetikzlibrary{arrows,automata,matrix,arrows,graphs,shadows,calc,decorations.pathmorphing,backgrounds,positioning,fit,petri}

\usetikzlibrary[intersections,mindmap,quotes,chains] %

\usetikzlibrary{overlay-beamer-styles}

\usetikzlibrary{fit}
%\usegdlibray{tree,layered}

\usepgfmodule{decorations}  %decorations
\usepgflibrary{decorations.pathmorphing}

\usetikzlibrary{cd} % commutative diagram

\usetikzlibrary{matrix,arrows}

\usepackage[justification=centering]{caption}
\usepackage{array}
\usepackage{multirow} 

\usepackage{verbatim} % comment

%中文包
\usepackage[UTF8]{ctex}

% subfigure
\usepackage{subfigure}

% figure目录
\graphicspath{{fig/},} 


%%%% 插入程序代码
\definecolor{mygreen}{rgb}{0,0.6,0}
\usepackage{listings}
\usepackage{xcolor}
\lstset{
	language=C++,
	%numbers=left, 
	%numberstyle= \tiny, 
	keywordstyle= \color{ blue!70},
	%commentstyle= \color{red!50!green!50!blue!50}, 
	commentstyle=\color{mygreen},    % comment style
	frame=shadowbox, % 阴影效果
	rulesepcolor= \color{ red!20!green!20!blue!20} ,
	%escapeinside=``, % 英文分号中可写入中文
	escapeinside={\%*}{*)},          % if you want to add LaTeX within your code
	xleftmargin=1em,xrightmargin=1em, aboveskip=1em,
	framexleftmargin=1em,
	breaklines=true,  % automatic line breaking only at whitespace
	extendedchars=false,  % 代码跨页问题
	keepspaces=true,
	showspaces=false,
	tabsize=4,
	showtabs=false,                  % 不显示tab键,不作为
	breaklines=true,  % automatic line breaking only at whitespace
	extendedchars=false,  % 代码跨页问题
	%keepspaces=true,
	mathescape=true  % 注释可以用latex math, eg. $a_{ij}$
} 

%% e.g. 如果mathescape=true, 如下输出$ \lstinline[mathescape=false]|sdfghjkla!#$0123456789@EIOQRTUWXYa|

%%%%%%%%%%%%%%%%%%%%%%%%%%%%%%%%%%%%%%%%%%
\usepackage[top=2cm, bottom=2cm, left=2cm, right=2cm]{geometry}  
\usepackage{algorithm}  
\usepackage{algorithmicx}  
\usepackage{algpseudocode} 
%\floatname{algorithm}{算法}  
%\renewcommand{\algorithmicrequire}{\textbf{输入:}}  
%\renewcommand{\algorithmicensure}{\textbf{输出:}}  
\renewcommand{\algorithmicrequire}{\textbf{Input:}}  
\renewcommand{\algorithmicensure}{\textbf{Output:}}  

%自定义列表编号
\usepackage{enumerate} %\begin{enumerate}[(i)]

%书签
\usepackage[breaklinks,colorlinks,linkcolor=black,citecolor=black,urlcolor=black]{hyperref}

\newtheorem{convention}{Convention} 

%\renewcommand{\thefootnote}{}  % footnote不编号

% see the list of further useful packages
% in the Reference Guide

\usepackage{tabulary} %tabulary tries to balance the column widths so that each column has at least its natural width, without exceeding the maximum length. 

\usepackage{array}

\graphicspath{{../figures/},} 

%%%%%%x%%%%%%%%%%%%%%%%%%%%%%%%%%%%%%%%%%%%%%%%%%%%%%%%%%%%%%%%%%%%%%%

\begin{document}

\author{段江涛}
\title{ 计算机导论与程序设计[CS006001038,X07]}
\subtitle{机试练习参考程序代码 }

%\includegraphics[height=1cm]{xd.jpg}


\date{\zhtoday}

\maketitle  % 生成书名

%\frontmatter

% \include{dedic}    %dedication 赠言
%%%%%%%%%%%%%%%%%%%%%%%%%%%%%%%%%%%foreword.tex%%%%%%%%%%%%%%%%%%%%%%%%%%%%%%%%%
% sample foreword
%
% Use this file as a template for your own input.
%
%%%%%%%%%%%%%%%%%%%%%%%% Springer %%%%%%%%%%%%%%%%%%%%%%%%%%

\foreword

%% Please have the foreword written here
%Use the template \textit{foreword.tex} together with the Springer document class SVMono (monograph-type books) or SVMult (edited books) to style your foreword\index{foreword} in the Springer layout.

%The foreword covers introductory remarks preceding the text of a book that are written by a \textit{person other than the author or editor} of the book. If applicable, the foreword precedes the preface which is written by the author or editor of the book.

本机试练习参考程序代码供学习$\ll$计算机导论与程序设计$\gg$课程的学生使用。希望同学们认真阅读总结代码中体现的课程知识点,并比对自己的程序, 建议从以下几个方面总结程序设计要领:
\begin{enumerate}
	\item 通过大量编程实践,训练自己的逻辑思维能力。
	\item 学习结构化、模块化程序设计技术。
	\item 重点题型: 字符串处理, 数组(矩阵)应用, 结构体数据组织以及循环迭代。
	\item  数组应用注意越界问题, 字符串注意\lstinline|'\0'|问题。
	\item  循环迭代基本思想是根据变量的上一轮值计算下一轮的值, 注意进入内层循环前的初始化问题。
	\item 各种类型的排序程序必须熟练掌握, 包括整数、字符、字符串、结构体等数组类型的排序,冒泡或选择法排序均可。
	\item 课件及本代码集是复习主要参考资料, 融汇贯通整个课程内容。
	\item 准备自己的函数库或程序片段, 做到信手拈来。
	\item 学习程序调试技巧, 每写好一段程序框架就编译运行一次,排除低级语法错误。千万不要整个程序写完才开始编译调试。
	\item 采用模块化分段调试技术, 每完成一部分功能, 调试该部分功能, 保证其正确性。不吝惜使用输出语句, 观察程序的执行结果是否与所设想的一致。测试输入变量的正确值, 是调试程序的首要点, 尤其是数字, 字符, 字符串混合形式的输入。
\end{enumerate}

\vspace{\baselineskip}
\begin{flushright}\noindent
Xi'an, China, December, 2019\hfill {\it 段江涛}\\
\end{flushright}


 %foreword 前言

\tableofcontents   %生成目录

%%表示文章的正文部分,在生成目录后将从第一页开始

%%%%%%%%%%%%%%%%%%%%% chapter.tex %%%%%%%%%%%%%%%%%%%%%%%%%%%%%%%%%
%
% sample chapter
%
% Use this file as a template for your own input.
%
%%%%%%%%%%%%%%%%%%%%%%%% Springer-Verlag %%%%%%%%%%%%%%%%%%%%%%%%%%
%\motto{Use the template \emph{chapter.tex} to style the various elements of your chapter content.}
\chapter{第1次机试练习: 熟悉DEV-C++开发平台, 基本输入输出语句练习}

\section{计算球体重量}
已知铁的比重是7.86(克/立方厘米),金的比重是19.3(克/立方厘米)。写一个程序,分别计算出给定直径的铁球与金球的质量,假定PI=3.1415926

输入说明:\\
输入两个整数,分别表示铁球与金球的直径(单位为毫米)

输出说明:\\
输出两个浮点数,分别表示铁球与金球的质量(单位为克),小数点后保留3位小数,两个浮点数之间用空格分隔

输入样例:\\
100 100

输出样例:\\
4115.486  10105.456

提示: \\
用\lstinline|scanf|输入,用\lstinline|printf|输出,保留3位小数的格式控制字符为\lstinline|%.3f|

\begin{lstlisting}
#include<stdio.h>
#include<math.h>   // 数学库函数 
#define PI 3.1415926       
int main()                   
{  
	int a,b;
	scanf("%d%d",&a,&b);
	float v1= 4.0/3.0*pow(a/2.0/10,3)*PI;
	float v2= 4.0/3.0*pow(b/2.0/10,3)*PI;
	printf("%.3f %.3f\n",7.86*v1,19.3*v2); 
	return 0;           
}                   
\end{lstlisting}

\begin{note}[要点]
	\begin{enumerate}
		\item 整数除以整数, 结果为整数。\\
		4.0/3.0结果是浮点数,4/3结果是整数
		\item 化简公式会引起精度问题, 不要随意化简公式。
		\item pow函数原型: \lstinline|double pow(double x,double y)|\\
		当形参数是整数时, 由于精度问题,不要使用此函数计算$x^y$. 推荐使用循环语句, 易计算$x^y$。 如果必要, 可自定义函数: \lstinline|int mypow(int x,int y)|。 见课件。
	\end{enumerate}
\end{note}

\section{温度转化}
已知华氏温度到摄氏温度的转换公式为:摄氏温度= (华氏温度- 32)×5/9,写程序将给定的华氏温度转换为摄氏温度输出。

输入说明:\\
只有一个整数,表示输入的华氏温度

输出说明:\\
输出一个表示摄氏温度的实数,小数点后保留2位有效数字,多余部分四舍五入

输入样例:\\
50

输出样例:\\
10.00

提示:\\
用scanf输入,用printf输出,保留2位小数的格式控制字符为%.2f

\begin{lstlisting}
#include <stdio.h>

int main()
{
	int f;
	float c;
	scanf("%d",&f);
	c = (f-32)*5.0/9;     // (1)
	//c = (f-32)*5/9;     // (2)
	printf("%.2f\n",c);
	return 0;
} 
\end{lstlisting}

\begin{note}[思考]
	为何语句(1),(2)计算结果不一致, 哪一条语句正确?
\end{note}


\section{整数简单运算}
编写程序,计算用户输入的两个整数的和、差、乘积(*)和商(/)。

输入格式:输入两个整数,整数之间用空格分隔。

输出格式:输出四个整数结果,分别表示和、差、积和商,每输出一个结果换行。

输入样例:\\
3 4

输出样例:\\
7  \\
-1 \\
12 \\
0

\begin{lstlisting}
#include<stdio.h>
int main()                   
{  
	int a,b;
	scanf("%d%d",&a,&b);
	printf("%d\n%d\n%d\n%d\n",a+b,a-b,a*b,a/b); 
	return 0;           
}             
\end{lstlisting}

\begin{note}[思考]
	b=0时如何处理?
\end{note}

\section{A+B+C}
通过键盘输入三个整数a,b,c,求3个整数之和。

输入说明:

三整形数据通过键盘输入,输入的数据介于-100000和100000之间,整数之间以空格、跳格或换行分隔。 

输出说明:

输出3个数的和。

输入样例:

-6  0  39

输出样例:

33

\begin{lstlisting}
#include<stdio.h>  
int main()                   
{  
	int a,b,c;
	scanf("%d%d%d",&a,&b,&c);
	printf("%d\n",a+b+c); 
	return 0;           
}           
\end{lstlisting}

\section{字符输入输出}
通过键盘输入5个大写字母,输出其对应的小写字母,并在末尾加上“!”。

输入说明:

5个大写字母通过键盘输入,字母之间以竖线“|”分隔。

输出说明:

输出5个大写字母对应的小写字母,之间无分隔,并在末尾加上‘!’。

输入样例:

H|E|L|L|O

输出样例:

hello!

\begin{lstlisting}
#include<stdio.h>     
int main()                   
{  
	char c1,c2,c3,c4,c5;
	scanf("%c|%c|%c|%c|%c",&c1,&c2,&c3,&c4,&c5);
	c1+=32; c2+=32; c3+=32; c4+=32; c5+=32;
	printf("%c%c%c%c%c!",c1,c2,c3,c4,c5); 
	return 0;           
}           
\end{lstlisting}

\begin{note}[要点]
	\lstinline|scanf("原样输入",...);|
\end{note}

\begin{note}(大小写字符转化关系)
	小写字符ASCII码=大写字符ASCII码+32
\end{note}

\section{数字字符}
通过键盘输入1个整数\lstinline|a(0<=a<=4)|,1个数字字符\lstinline|b('0'<=b<='5')|求a+b。

输入说明:

整形数据、数字字符通过键盘输入,输入的整形数据介于0和4之间,输入的数字字符介于`0'和`5'之间,二个输入数之间用``,"分隔。

输出说明:

分别以整数形式及字符形式输出a+b,输出的二个数之间用``,"分隔。

输入样例:

3,5

输出样例:

56,8

\newpage

\begin{lstlisting}
#include<stdio.h>    
int main()                   
{  
	int a;
	char b;
	scanf("%d,%c",&a,&b);
	printf("%d,%c",a+b,a+b); 
	return 0;           
}             
\end{lstlisting}

\begin{note}(scanf函数)
	\lstinline|scanf("原样输入",...); |
\end{note}

\begin{note}(整型数值与字符混合运算)
	字符对应的ASCII编码参与整数运算, 其结果也是整数。注意\lstinline|'0'|与\lstinline|0|不同, 本例中输入0,0, 则\lstinline|a=0|,\lstinline|b='0'|, 变量a的值是整数0, 变量b的值是字符\lstinline|'0'|对应的ASCII编码, 即整数48。
\end{note}

\section{实数运算}
通过键盘输入长方体的长、宽、高,求长方体的体积V(单精度)。

输入说明:

十进制形式输入长、宽、高,输入数据间用空格分隔。

输出说明:

单精度形式输出长方体体积V,保留小数点后3位,左对齐。

输入样例:

15  8.12  6.66

输出样例:

811.188


\begin{lstlisting}
#include<stdio.h>   
int main()                   
{  
	float a,b,c;
	scanf("%f%f%f",&a,&b,&c);
	printf("%.3f",a*b*c); 
	return 0;           
}                  
\end{lstlisting}

\begin{note}(精度问题)
	32位编译器: \lstinline|a*b*c| 与 \lstinline|a*c*b| 结果一致。
	但是在64位编译器中, 二者不一致。
	
	因此, 浮点数运算会存在精度问题, 不要随意改变运算顺序。
\end{note}
   % 第1次机试练习: 熟悉DEV-C++开发平台, 基本输入输出语句练习

%%%%%%%%%%%%%%%%%%%%% chapter.tex %%%%%%%%%%%%%%%%%%%%%%%%%%%%%%%%%
%
% sample chapter
%
% Use this file as a template for your own input.
%
%%%%%%%%%%%%%%%%%%%%%%%% Springer-Verlag %%%%%%%%%%%%%%%%%%%%%%%%%%
%\motto{Use the template \emph{chapter.tex} to style the various elements of your chapter content.}
\chapter{第2次机试练习: 选择与循环语句练习}

\section*{--- 特别提示 ---}

\begin{note}[不该再次发生的常见错误,输入输出格式转换符不对应, 导致的严重错误]
	\begin{lstlisting}
	int a; float b; double c; char d;
	scanf("%d",a); // 遗忘变量前的取地址符$\&$
	scanf("%d\n",&a); // 多余'\n', 导致不能正常输入
	scanf("%d%f%lf%c",&a,&b,&c,%d); // 正确对应关系
	scanf("%d%c%f",&a,&d,&b); // 正确对应关系
	printf("%d,%f,%lf,%c",a,b,c,d); // 正确对应关系
	\end{lstlisting}
\end{note}

\begin{note}[不该再次发生的常见错误, 由`;'引发的悲剧]
	\begin{lstlisting}
	if();
	{
		...
	}                 
	
	while();
	{
		...
	} 
	
	for(;;);
	{
		...
	}                  
	\end{lstlisting}
\end{note}

\newpage
\begin{note}[用C语言关系表达式准确表达数学含义]
\begin{lstlisting}
int a;
if(110<=a<=210) // 错误
{  }
if(110<=a && a<=210) // 正确 
{  }
if(a>=110 && a<=210) // 正确 
{  }
\end{lstlisting}	
\end{note}

\begin{note}[学习体会编程技巧]
	\begin{itemize}
		\item 使用\lstinline|printf()|语句, 追踪程序执行细节, 查找出错原因。
		\item 对于条件结构,循环结构,首先书写整体结构,再添加细节,避免低级错误。
		\item 提倡一题多解, 举一反三, 体会编程技巧。	
	\end{itemize}
\end{note}


\section{四则运算}
输入两个整数和一个四则运算符,根据运算符计算并输出其运算结果(和、差、积、商、余之一)。注意做整除及求余运算时,除数不能为零。

输入说明:

使用scanf()函数输入两个整数和一个运算符,格式见输入样例。

输出说明:

输出使用printf()函数,格式见输出样例。

输入样例:

\lstinline|5%2|

输出样例:

\lstinline|5%2=1|

\begin{lstlisting}
#include<stdio.h>    
int main()                   
{  
	int a,b;
	char op;
	scanf("%d%c%d",&a,&op,&b);
	switch(op)
	{
		case '+': printf("%d%c%d=%d\n",a,op,b,a+b); break;
		case '-': printf("%d%c%d=%d\n",a,op,b,a-b); break;
		case '*': printf("%d%c%d=%d\n",a,op,b,a*b); break;
		// 注意分母为0时, 不会正确运算/,%
		case '/': if (b!=0) printf("%d%c%d=%d\n",a,op,b,a/b); break;
		case '%': if (b!=0) printf("%d%c%d=%d\n",a,op,b,a%b); break;
	}
	return 0;           
}                 
\end{lstlisting}

\begin{note}[printf双引号中的\%输出, \%\%表示输出\%]
	\begin{lstlisting}
		int a,b;
		char op;
		printf("%d%%%d=%d\n",a,b,a%b);
		// 或当 op='%'时
		printf("%d%c%d=%d\n",a,op,b,a%b);
	\end{lstlisting}
\end{note}
	

\section{数位输出}
输入一个5位整数,求出其各数位数值,并按照从高位到低位的顺序输出,如:输入12345,输出为1 2 3 4 5。

输入说明:

输入一个五位正整数。

输出说明:

按数位从高到低依次输出,各数位之间以一个空格相分隔。

输入样例:

96237

输出样例:

9 6 2 3 7
\newpage
\begin{lstlisting}
#include<stdio.h> 
/******************************************************
 5位整数已知. 首先用10000除以整数a(分子), 得到分子最高位。 
 改变分子分母, 循环迭代, 依次获得分子的最高位。
*******************************************************/        
int main1()              
{  
	int a,b=10000,i=5; // i记录整数a的初始位数
	scanf("%d",&a);
	while(i>=1)
	{
		if (i==1) printf("%d\n",a/b); // 输出当前a的最高位
		else printf("%d ",a/b);
		a = a-a/b*b;  // 去除当前a的最高位,准备下轮迭代的分子a
		b/=10; // b=b/10, 准备下轮迭代的分母b
		i--;
	} 
	return 0;           
}  
             
/******************************************************
 假设不知整数a的位数。
 除10取余,迭代循环,可方便获取整数a的个位,十位,百位,千位, ... 
 利用数组存储个位,十位,百位,千位, ... 最后反序输出即是所求。 
*******************************************************/        
int main2()              
{  
	int a, tmp[100]; // tmp数组存储100(估计的最大值)个整数,用tmp[0],tmp[1],tmp[2],...读写各个整数。
	int i=0, j;         // i: 记录整数a的位数 
	scanf("%d",&a);
	if(a==0) // 考虑整数0的特殊情况,直接输出即可。
	{
		printf("%d\n",a); 
	}
	else // 因为循环语句判断a是否为0, 因此要有上述判断才能考虑到所有可能情况的发生 
	{
		while(a!=0) // 迭代逆序求出整数a的各位数字 
		{
			tmp[i]=a%10; // 存储本轮循环a的末位数
			//printf("调式查看tmp[i]=%d\n",tmp[i]); // 提交时,别忘了注释或删除调试语句 
			a=a/10;      // 改变分子, 准备下轮循环
			i++;         // 位数递增 
		}
		//printf("调式查看i=%d\n",i); 
		//  逆序输出tmp, 此时的i是整数a的位数, 注意tmp的下标从i-1开始到下标0结束。 
		for(j=i-1;j>=0;j--)
		{
			printf("%d ",tmp[j]);
		} 
	}
	return 0;           
}    

/******************************************************
 假设不知整数a的位数,用递归函数求解。
*******************************************************/    
void output(int a)              
{  
	if(a>=10) // 大于等于两位数时
	{
		// '栈'是一种'先进后出'的数据结构
		// 递归调用, 函数参数会自动存储在系统维护的'栈'中。 
		output(a/10); 
	} 
	// 从内部存储'栈'中,依次弹出各位数,输出之。
	printf("%d ",a%10);   
} 

int main()
{
	int a;
	scanf("%d",&a);
	output(a); // 函数调用, 完成正序输出。  
	return 0; 
}
\end{lstlisting}

\begin{figure}[H]
\centering
\caption{递归函数\lstinline|void output(int a)|中系统内部维护的`栈'结构示意图}	
\begin{tikzpicture}
\node[] (tab) {
\begin{tabular}{|c|c|c|}
\hline
参数a&output(a)&递归调用\lstinline|output(a/10); printf("%d ",a%10);|\\
\hline
1&output(1)& 递归结束, \lstinline|printf("%d ",1);| 然后printf逐个出栈.\\
\hline
12&output(12)&\lstinline|output(1); printf("%d ",2);| 栈顶\\
\hline
123&output(123)&\lstinline|output(12); printf("%d ",3);|\\
\hline
1234&output(1234)&\lstinline|output(123); printf("%d ",4);|\\
\hline
\hline
12345&output(12345)&\lstinline|output(1234); printf("%d ",5);|\\
\hline
\end{tabular}
};
\path[->,very thick]
%($(tab.north west)+(-0.5cm,0)$) edge[] node[left] {入栈} +(0,-4cm)
($(tab.north west)+(-0.5cm,0)$) edge[red] node[left] {入栈} ($(tab.south west)+(-0.5cm,0)$)
($(tab.south east)+(0.5cm,0)$) edge[blue] node[right] {出栈} ($(tab.north east)+(0.5cm,0)$)
; 
\end{tikzpicture}
\end{figure}

\begin{note}[知识点]
	\begin{enumerate}
		\item 体会除10取余, 迭代循环的整数分解技巧;
		\item 第一种解法的b=1000初值是可计算的, 这样就可扩充此解法为任意位的整数a。
		\begin{lstlisting}[frame=none]
		// 因为a要在main1()函数的while循环中使用。
		// 因此, 定义临时变量, 存储a的值, 用于计算b的初值。
		int tmp; 
		b=1; tmp=a;
		while(tmp!=0)
		{
			b=b*10;
			tmp=tmp/10;
		}
		\end{lstlisting}
		\item 预习数组使用技巧;
		\item 预习函数定义及调用;
		\item 预习递归函数的定义, 体会系统维护的内部存储'栈'的数据存储特点。
	\end{enumerate}
\end{note}

\section{阶梯电价计费}
电价分三个档次,[0,110]度电,每度电0.5元;(110,210]度电,超出110部分每度电0.55元,超过210度电,超出210部分每度电0.70元,给出一个家庭一月用电量,请计算出应缴的电费(四舍五入,保留小数点后两位小数)。

输入说明:

输入数据为一个正实数,表示一月用电量

输出说明:

输出应缴电费,四舍五入保留2位小数。

输入样例:

输入样例1

100

输入样例2

200

输入样例3

329.75

输出样例:

输出样例1

50.00

输出样例2

104.50

输出样例3

193.83

\begin{lstlisting}
#include <stdio.h>
int main()
{
	float sum,u1=0.5,u2=0.55,u3=0.70; // 用电量,每度电单价
	float fee = 0; // 应缴电费
	
	scanf("%f",&sum);
	
	if (sum > 210) 
	{
		fee = (sum-210)*u3;
		sum = 210;
	}
	if (sum > 110)
	{
		fee += (sum-110)*u2; // fee=fee+(sum-110)*u2;
		sum = 110;
	}
	fee += sum*u1;
	
	printf("%.2f\n",fee);
	return 0;
} 

int main2()  // 另解
{
	float sum,u1=0.5,u2=0.55,u3=0.70; // 用电量,每度电单价
	float fee = 0; // 应缴电费 
	scanf("%f",&sum);
	
	if (sum >= 210)
		fee = 110*u1 + (210-110)*u2 + (sum-210)*u3;
	else if (sum >= 110)
		fee = 110*u1 + (sum-110)*u2;
	else 
		fee = sum*u1; 

	printf("%.2f\n",fee);
	return 0;
} 

int main3() // 另解
{
	float sum,u1=0.5,u2=0.55,u3=0.70; // 用电量,每度电单价
	float fee = 0; // 应缴电费 
	
	scanf("%f",&sum);
	
	if (sum <= 110) fee = sum*u1;
	else if (sum <= 210)
	{
		fee = 110*u1;
		sum -= 110; // sum=sum-110;
		fee += sum*u2;
	}
	else // sum > 210
	{
		fee = 110*u1; 
		fee += (210-110)*u2; // fee = fee+(210-110)*u2
		sum -= 210;  // sum=sum-210;
		fee += sum*u3; // fee = fee+ sum*u3;
	}
	
	printf("%.2f\n",fee);
	return 0;
} 
\end{lstlisting}

\begin{note}[四舍五入问题]
	不同的编译系统,处理结果可能不一致,\lstinline|printf("%.2f\n",fee);|默认输出即可。
\end{note}

\begin{note}
	练习if语句的不同组合形式, 杜绝出现\lstinline|if(110<=sum<=210)|的错误形式。
\end{note}

\section{计算某月天数}
每年的1,3,5,7,8,10,12月有31天,4,6,9,11月有30天,闰年2月29天,其他年份2月28天,给定年份和月份求该月的天数

输入说明:

输入由两个正整数a和b构成,a表示年份,b表示月份,a和b之间用空格分隔

输出说明:

根据年份和月份计算该月天数并输出

输入样例	

输入样例1

2000 3

输入样例2

2001 2

输出样例
	
输出样例1

31

输出样例2

28

\begin{lstlisting}
#include <stdio.h>
int main()
{
	int a,b,t = 0;
	scanf("%d%d",&a,&b);
	if((a%4 == 0 && a%100 !=0) || (a%100 == 0 && a%400 == 0))
	{
		if(b == 2) t = 29;
	}
	else if (b == 2) t = 28;
	
	if(b == 1 || b == 3 || b == 5 || b == 7 || b == 8 || b == 10 || b == 12) t = 31;
	else if(b == 4 || b == 6 || b == 9 || b == 11) t = 30; 
	
	printf("%d\n",t);
	return 0;	
}
\end{lstlisting}

\begin{note}(逻辑运算符)
	\&\&, ||, !, 练习符合逻辑的各种组合形式。
\end{note}

\begin{note}(闰年判断)
	ppt中已详细说明,还有同学写错。
\end{note}

\section{计算整数各位数字之和}
假设n是一个由最多9位数字(d9,\dots, d1)组成的正整数。编写一个程序计算n的每一位数字之和。

输入说明:
	
输入数据为一个正整数n

输出说明:
	
对整数n输出它的各位数字之和后换行

输入样例:	

3704

输出样例:	

14
\newpage
\begin{lstlisting}
#include <stdio.h>
// 体会除10取余, 迭代循环的整数分解技巧;
int main1()
{
	int n,sum = 0; // 注意初始化sum
	scanf("%d",&n);
	while(n) // 等效于n!=0 
	{
		sum += n%10; // 累加本轮循环的末位数
		n /= 10;     // 准备下轮循环的分子
	}
	printf("%d",sum);
	return 0;	
}

// 另解:定义递归函数, 返回整数n的各位数之和 
int sum(int n)
{
	if(n!=0)
	{
	    // 递归调用, 累加本轮循环的末位数
		return (sum(n/10)+n%10); 
	}
	else // n==0时, 结束递归调用 
	{
		return 0; // 函数结束, 返回整数0
	}	
}

int main()
{
	int n;
	scanf("%d",&n);
	printf("%d\n",sum(n)); // 函数调用。 
	return 0; 
}
\end{lstlisting}

\begin{figure}[H]
	\centering
	\caption{递归函数\lstinline|int sum(int n)|中系统内部维护的`栈'结构示意图}	
	\begin{tikzpicture}
	\node[] (tab) {
		\begin{tabular}{|c|c|}
		\hline
		参数n&递归调用\lstinline|sum(n)=sum(n/10)+n%10;|\\
		\hline
		0&sum(0)=0; 结束递归, 开始出栈\\
		\hline
		1&\lstinline|sum(1)=sum(1/10)+1%10=sum(0)+1;|\\
		\hline
		12&\lstinline|sum(12)=sum(12/10)+12%10=sum(1)+2;|\\
		\hline
		123&\lstinline|sum(123)=sum(123/10)+123%10=sum(12)+3;|\\
		\hline
		1234&\lstinline|sum(1234)=sum(1234/10)+1234%10=sum(123)+4;|\\
		\hline
		\hline
		12345&\lstinline|sum(12345)=sum(12345/10)+12345%10=sum(1234)+5;|\\
		\hline
		\end{tabular}
	};
	\path[->,very thick]
	%($(tab.north west)+(-0.5cm,0)$) edge[] node[left] {入栈} +(0,-4cm)
	($(tab.north west)+(-0.5cm,0)$) edge[red] node[left] {入栈} ($(tab.south west)+(-0.5cm,0)$)
	($(tab.south east)+(0.5cm,0)$) edge[blue] node[right] {出栈} ($(tab.north east)+(0.5cm,0)$)
	; 
	\end{tikzpicture}
\end{figure}

\begin{note}[知识点]
	\begin{enumerate}
		\item 体会除10取余, 迭代循环的整数分解技巧;
		\item 预习递归函数定义及调用。
	\end{enumerate}
\end{note}

\section{完数}\label{perfect number}
请写一个程序,给出指定整数范围[a,b]内的所有完数,0 < a < b < 10000。
一个数如果恰好等于除它本身外的所有因子之和,这个数就称为``完数"。
例如6是完数,因为6=1+2+3

输入说明	

输入为两个整数a和b,a和b之间用空格分隔

输出说明	

输出[a,b]内的所有完数,每个数字占一行

输入样例	

1 10

输出样例	

6

\begin{lstlisting}
#include <stdio.h>
/**********************************************
 采用两层循环方案
 (1) 外层循环使整数i递增,完成区间[n1,n2]区间的完数计算 
 (2) 内层循环, 累加整数i的各因子
 (3) 判断整数i是否是完数, 如果是,输出之 
**********************************************/ 
int main1()
{
	int i,j,n1,n2,sum = 0;
	scanf("%d%d",&n1,&n2);
	for(i = n1; i <= n2; i++) // 外层循环使整数i递增,完成区间[n1,n2]区间的完数计算 
	{
		if(i == 1) continue; // 避免输出1,1不是完数 
		// i不等于1, 计算各因子 
		sum = 1;// 不要忘记, 内层循环前sum的初始化。1总是一个整数的合法因子 
		for(j = 2; j < i; j++) // 累加整数i的所有因子 
		{
			if(i%j == 0) sum += j; // 如果j是i的因子,累加之。 
		}
		if(sum == i) printf("%d\n",i); // 如果i是完数,输出之。 
	} 
	
	return 0;	
}

/**********************************************
 采用一重循环 + 调用函数方案
 (1) 一重循环使整数i递增,函数compute调用,完成区间[n1,n2]区间的完数计算 
 (2) 定义函数compute, 判断整数参数是否是完数, 如果是,返回它, 否则返回-1 
**********************************************/ 

// 定义函数compute, 判断整数参数a是否是完数, 如果是,返回a, 否则返回-1 
int compute(int a)
{
	int i,s=1; // s用于存储a的各因子累加值, 1总是一个整数的合法因子 
	if(a == 1) 
	{
		return -1; // 1不是完数
	}
	// a不为1, 计算各因子 
	for(i = 2; i < a; i++) // 累加整数a的所有因子 
	{
		if(a%i == 0) s += i; // 如果i是a的因子,累加之。 
	}
	if(s == a) 
	{
		return a; // 如果a是完数,返回之。
	} 
	// 如果程序执行到此处必然不是完数 
	return -1;
}

// 另一种方式定义函数compute, 判断整数参数a是否是完数, 如果是,返回a, 否则返回-1 
// 一条return函数返回语句 
int compute1(int a)
{
	int i,s=1; // s用于存储a的各因子累加值, 1总是一个整数的合法因子 
	int ret=-1; // 用于返回值,默认为-1 
	
	for(i = 2; i < a; i++) // 累加整数a的所有因子 
	{
		if(a%i == 0) s += i; // 如果i是a的因子,累加之。 
	}
	if(s == a && a!=1) // 如果a是完数,返回值是本身。1不是完数
	{
		ret = a; 
	} 
	else // a不是完数 
	{
		ret = -1;
	}
	return ret;
}

int main()
{
	int i,n1,n2;
	scanf("%d%d",&n1,&n2);
	for(i = n1; i <= n2; i++) // 调用函数compute,完成区间[n1,n2]区间的完数计算 
	{
		
		if(compute(i)!=-1) printf("%d\n",i); // 如果i是完数,输出之。 
		// 测试函数compute1的调用 
		// if(compute1(i)!=-1) printf("%d\n",i); // 如果i是完数,输出之。 
	} 
	
	return 0;	
}
\end{lstlisting}

\begin{note}[特别注意]
	且记:进入内层循环前, 相关变量的初始化问题。	
\end{note}

\begin{note}[函数定义和调用]
	\begin{itemize}
		\item 函数定义: \lstinline|返回类型 函数名(参数列表) { 函数体 }|
		\item \lstinline|int fun1(float a,float b) { return a/b; // 返回整数部分 }| 
		\item \lstinline|void fun2(float a,float b) { printf(a/b); // 输出整数部分 }| 
		\item 函数调用
		\begin{lstlisting}[frame=none]
		float m,n;
		int ret;
		ret = fun1(m,n); // 调用函数fun1, 其返回值赋值给变量ret;
		fun2(m,n); // 调用函数fun2, 无返回值可用;
		\end{lstlisting}
	\end{itemize}	
\end{note}

   % 第2次机试练习: 选择与循环语句练习

%%%%%%%%%%%%%%%%%%%%% chapter.tex %%%%%%%%%%%%%%%%%%%%%%%%%%%%%%%%%
%
% sample chapter
%
% Use this file as a template for your own input.
%
%%%%%%%%%%%%%%%%%%%%%%%% Springer-Verlag %%%%%%%%%%%%%%%%%%%%%%%%%%
%\motto{Use the template \emph{chapter.tex} to style the various elements of your chapter content.}
\chapter{第3次机试练习: 继续分支与循环练习}

\section{最大公约数}	
最大公约数(GCD)指某几个整数共有因子中最大的一个,最大公约数具有如下性质,

gcd(a,0)=a

gcd(a,1)=1

因此当两个数中有一个为0时,gcd是不为0的那个整数,当两个整数互质时最大公约数为1。

输入两个整数a和b,求最大公约数

输入说明:

输入为两个正整数a和b(0<=a,b<10000),a和b之间用空格分隔,

输出说明:

输出其最大公约数

输入样例:

样例1输入

2 4

样例2输入:

12 6

样例3输入:

3 5

输出样例:

样例1输出

2

样例2输出

6

样例3输出

1

\newpage
\begin{lstlisting}
#include <stdio.h>
// 递归函数
int gcd(int a,int b)
{
	if(b==0) return a;   // 公约数就是a
	return gcd(b,a%b);   // 递归调用
}

int main() // 调用递归函数
{
	int a,b,t;
	scanf("%d%d",&a,&b);
	if(a<b) { t=a; a=b; b=t; } // 交换a,b 
	printf("%d\n",gcd(a,b));   // 函数调用 
	return 0;	
}

int main1() // 暴力循环求解, 效率低。
{
	int a,b,t=-1,i;//t给初值是好习惯,否则下面程序逻辑有可能使t得到随机值。 
	scanf("%d%d",&a,&b); // 机试系统不要想当然给提示语句, 除非题目要求  
	if(a<b) { t=a; a=b; b=t; } // 交换a,b,使a是较大者 
	if(b==0) 
	{
		t=a; // 考虑分母为0的情况,比如:5,0的最大公约数为5 
	} 
	else
	{
		for(i=b;i>0;i--)
		{
			if(a%i==0 && b%i==0)
			{
				t=i; break; // 求得最大公约数,a,b互质, 必然t=1 
			}
		}
	}
	printf("%d\n",t);
	return 0;	
}

int main2() // 利用欧几里得定理循环求解, 效率高。
{
	int a,b,r,t;
	scanf("%d%d",&a,&b); // 机试系统不要想当然给提示语句, 除非题目要求
	if(a<b) { t=a; a=b; b=t; } // 交换a,b,使a是较大者
	while(1)
	{
		if(b==0) { t=a; break; } // 分母为0时, a就是最大公约数
		r = a%b; 
		if(r==0) {t=b; break;} // b就是最大公约数
		a=b; b=r; // 准备下一轮迭代   
	}
	printf("%d\n",t);// 输出最大公约数
	return 0;
}

int main3() // 利用欧几里得定理循环求解, 效率高。
{
	int a,b,r,t;
	scanf("%d%d",&a,&b); // 机试系统不要想当然给提示语句, 除非题目要求
	if(a<b) { t=a; a=b; b=t; } // 交换a,b,使a是较大者
	if (b==0) // 考虑分母为0的情况,比如:5,0的最大公约数为5 
	{
		printf("%d\n",a);
	}
	else
	{
		// 排除了分母为0时不能求余数的情况 
		while((r=a%b)!=0) // a/b的余数赋值给r,r不等于0时执行循环体 
		{	
			a=b; 
			b=r; 
		}
		printf("%d\n",b);
	}
	return 0; // 主函数结束
}	

int main4() // 体会函数结束语句return的使用
{
	int a,b,r,t;
	scanf("%d%d",&a,&b); // 机试系统不要想当然给提示语句, 除非题目要求
	if(a<b) { t=a; a=b; b=t; } // 交换a,b,使a是较大者
	if (b==0) // 考虑分母为0的情况,比如:5,0的最大公约数为5 
	{
		printf("%d\n",a);
		return 0;  // 主函数结束 
	}
	// 排除了分母为0时不能求余数的情况 
	while((r=a%b)!=0) // a/b的余数赋值给r,r不等于0时执行循环体 
	{
		a=b; b=r; // 准备下一轮迭代   
	}
	printf("%d\n",b);
	return 0;	// 主函数结束 
}
\end{lstlisting}

\begin{figure}[H]
	\centering
	\caption{递归函数\lstinline|int gcd(int a,int b)|中系统内部维护的`栈'结构示意图}	
	\begin{tikzpicture}
	\node[] (tab) {
		\begin{tabular}{|c|c|}
		\hline
		参数a,b & 递归调用\lstinline|gcd(a,b)=gcd(b,a%b);|\\
		\hline
		a=6,b=0 & gcd(6,0)=6; 6就是公约数, 结束递归, 开始出栈\\
		\hline
		a=12,b=6 & \lstinline|gcd(12,6)=gcd(6,12%6)=gcd(6,0);|\\
		\hline
		a=18,b=12 & \lstinline|gcd(18,12)=gcd(12,18%12)=gcd(12,6);|\\
		\hline
		\end{tabular}
	};
	\node[above] at(tab.north) {例1: 计算gcd(18,12)};
	\path[->,very thick]
	($(tab.north west)+(-0.5cm,0)$) edge[red] node[left] {入栈} ($(tab.south west)+(-0.5cm,0)$)
	($(tab.south east)+(0.5cm,0)$) edge[blue] node[right] {出栈} ($(tab.north east)+(0.5cm,0)$)
	($(tab.south west)+(-0.8cm,-0.2cm)$) edge[-,double] ($(tab.south east)+(0.8cm,-0.2cm)$)
	; 
	\end{tikzpicture}
	
	\begin{tikzpicture}
	\node[] (tab) {
		\begin{tabular}{|c|c|}
		\hline
		参数a,b & 递归调用\lstinline|gcd(a,b)=gcd(b,a%b);|\\
		\hline
		a=1,b=0 & gcd(1,0)=1; 1就是公约数, 结束递归, 开始出栈\\
		\hline
		a=2,b=1 & \lstinline|gcd(2,1)=gcd(1,2%1)=gcd(1,0);|\\
		\hline
		a=3,b=2 & \lstinline|gcd(3,2)=gcd(2,3%2)=gcd(2,1);|\\
		\hline
		a=2,b=3 & \lstinline|gcd(2,3)=gcd(3,2%3)=gcd(3,2);|\\
		\hline
		a=5,b=2 & \lstinline|gcd(5,2)=gcd(2,5%2)=gcd(2,3);|\\
		\hline
		a=32,b=5 & \lstinline|gcd(32,5)=gcd(5,32%5)=gcd(5,2);|\\
		\hline
		a=37,b=32 & \lstinline|gcd(37,32)=gcd(32,37%32)=gcd(32,5);|\\
		\hline
		\end{tabular}
	};
	\node[above] at(tab.north) {例2: 计算gcd(37,32)};
	\path[->,very thick]
	($(tab.north west)+(-0.5cm,0)$) edge[red] node[left] {入栈} ($(tab.south west)+(-0.5cm,0)$)
	($(tab.south east)+(0.5cm,0)$) edge[blue] node[right] {出栈} ($(tab.north east)+(0.5cm,0)$)
	; 
	\end{tikzpicture}
\end{figure}

\begin{note}[欧几里得定理]
	\begin{lstlisting}[frame=none]
	a(大),b(小)的最大公约数: 因为: a=mb+r, m=a/b; r=a%b, $\Rightarrow$a,b的公约数能整除b和r.
	r=a%b,r为0, 则b就是最大公约数。否则迭代循环, a=b,b=r, 直到余数为零, 则分母就是最大公约数。
	\end{lstlisting}
\end{note}

\begin{note}
	预习函数及递归函数的使用。
\end{note}

\section{角谷定理}
角谷定理定义如下:
对于一个大于1的整数n,如果n是偶数,则n = n / 2。如果n是奇数,则n = 3 * n +1,反复操作后,n一定为1。

例如输入22的变化过程: 22 ->11 -> 34 -> 17 -> 52 -> 26 -> 13 -> 40 -> 20 -> 10 -> 5 -> 16 -> 8 -> 4 -> 2 -> 1,数据变化次数为15。

输入一个大于1的整数,求经过多少次变化可得到自然数1。

输入说明	

输入为一个整数n,1<n<100000。

输出说明	

输出变为1需要的次数

输入样例

样例1输入

22

样例2输入

33

输出样例

样例1输出

15

样例2输出

26

\begin{lstlisting}
#include <stdio.h>
int main()
{
	int n,i=0; // 变量i用于计数的辅助变量
	scanf("%d",&n);
	// 因为题目输入假设$n>1$, 因此不必考虑n=1时的情况
	while(n!=1)  // n不等于1时执行循环体中的语句
	{
		if(n%2==0) n=n/2;
		else n=3*n+1;
		i++;
	} 
	printf("%d\n",i);
	return 0;	
}

// 含程序调试语句,不吝惜写一些printf语句,观察程序的执行过程。 
int main()
{
	int n=22,i=0; // 变量i用于计数的辅助变量
	//scanf("%d",&n); // 调试时可以注释掉输入语句, 改变变量n的值,观察执行过程 
	
	printf("%d->",n);
	while(n!=1)  // n不等于1时执行循环体中的语句
	{
		if(n%2==0) 
		{ 
			n=n/2;
		} 
		else 
		{
			n=3*n+1;
		}
		printf("%d->",n);
		i++;
	} 
	printf("\n总共变化次数%d\n",i);
	return 0;	
}
\end{lstlisting}

\begin{note}
	试着用\lstinline|do{ }while(); for(;;)|改写此程序, 执行相同功能。
\end{note}

\section{整数分析}
给出一个整数n(0<=n<=100000000)。求出该整数的位数,以及组成该整数的所有数字中的最大数字和最小数字。

输入说明
	
输入一个整数n(0<=n<=100000000)

输出说明
	
在一行上依次输出整数n的位数,以及组成该整数的所有数字中的最大数字和最小数字,各个数字之间用空格分隔。

输入样例
	
217

输出样例
	
3 7 1

\begin{lstlisting}
#include <stdio.h>
// 循环除10取余是整数分解的基本技巧
int main()
{
	int i = 0, n, bit, max, min;
	scanf("%d",&n);
	while(n) // 等效于 while(n!=0)
	{
		bit = n%10; // 获取n的最低为
		// 切记: 初始化时,假设的max和min必须是实际存在的数。
		if(i == 0) // 初始化: 原始n的最低位设为最大和最小数字
		{
			max = min = bit;
		}  
		else
		{
			if(bit > max) max = bit;
			if(bit < min) min = bit;
		}
		n /= 10; // 去除最低位
		i++;
	} 
	// (i == 0 ? 1 : i)是条件表达式, 表达式的值是:
	// 如果i==0,则表达式的值为1否则表达式的值是i
	printf("%d %d %d\n",(i == 0 ? 1 : i),max,min); //考虑原始n==0的情况 
	return 0;
} 
\end{lstlisting}

\begin{note}[知识点]
  \begin{enumerate}
  	\item 整数数位分解是基本编程练习之一。
  	\item 切记: 初始化时,假设的max和min必须是实际存在的数。比如不能想当然假设max=1000, min=0.
  	\item 注意审题: ``输入一个整数$n,(0<=n<=100000000)$'', 因此, 0也是一个合法输入。
  \end{enumerate}
  
\end{note}

\section{冰箱温度预测}
编写一个程序,用于预测冰箱断电后经过时间t(以小时为单位)后的温度T。已知计算公式如下所示
\[ T=\frac{4t^2}{t+2}-20 \]

输入说明	

输入两个整数h和m表示冰箱断电后经过的时间,h表示小时,m表示分钟

输出说明
	
输出冰箱断电后经过时间t(以小时为单位)后的温度T,保留两位小数

输入样例
	
2 0

输出样例
	
-16.00

\begin{lstlisting}
#include <stdio.h>
int main()
{
	int h,m;
	float t,T;
	scanf("%d%d",&h,&m);
	t = h + m/60.0;      // 必须是60.0, why?
	T = 4*t*t/(t+2)-20;  // 优先级保证了计算的正确性, why?
	printf("%.2f\n",T);
	
	return 0;
}
\end{lstlisting}

\begin{note}[知识点]
	整数/整数, 表达式的值是整数部分, 自动舍去小数部分。
\end{note}

\section{除法计算器}	
小明的弟弟刚开始学习除法,为了检查弟弟的计算结果是否正确,小明决定设计一个简单计算器程序来验算。

输入说明
	
输入数据由四个整数m,n,q,r构成,m为被除数,n为除数,q和r为小明的弟弟计算出的商和余数。整数之间用空格分隔,所有整数取值范围在$(-100000\sim 100000)$, n不为0。

输出说明
	
如果验算结果正确,输出yes,否则输出正确的商和余数

输入样例:

样例1:

10 3 3 1

样例2:

10 3 3 2

输出样例	

样例1输出:

yes

样例2输出:

3 1

\begin{lstlisting}
#include <stdio.h>
int main()
{
	int m,n,q,r;
	scanf("%d%d%d%d",&m,&n,&q,&r);
	if(m==q*n+r && q==m/n && r==m%n) printf("yes\n");
	else printf("%d %d\n",m/n,m%n); 
	return 0;
} 
\end{lstlisting}

\begin{note}
	改变题设条件,修改此程序,进行各种表达式计算练习, 分析优先级。 如果n=0时, 如何处理。
\end{note}


\section{自然数分解}
任何一个自然数m的立方均可写成m个连续奇数之和。例如:
\begin{align*}
1^3 &=1\\
2^3 &=3+5\\
3^3 &=7+9+11\\
4^3 &=13+15+17+19
\end{align*}

编程实现:输入一自然数n, 求组成$n^3$的n个连续奇数。

输入说明

一个正整数n, 0<n<30 。

输出说明

输出n个连续奇数, 数据之间用空格隔开,并换行

输入样例

4

输出样例

13 15 17 19

\begin{lstlisting}
#include <stdio.h>
// 从估计的第一个奇数开始,循环迭代求解。
int main()
{
	int n,i,j,sum,first;
	scanf("%d",&n);
	
	// 第一个可能的奇数:
	if(n%2) first = n;    // n是奇数 
	else first = n + 1;   // n是偶数 
	
	while(1)
	{
		sum = 0; // 每趟内层循环前,必须置0 
		// 从first开始,n个连续奇数, i:表示连续奇数,j:计数。 
		for(i = first,j = 1; j <= n; i += 2,j++ ) 
		{
			sum += i; // 连续奇数累加 
			if(sum == n*n*n) 
			{
				// 输出 
				for(i = first,j = 1; j <= n; i += 2,j++) 
				{
					if (j == n)printf("%d\n",i);
					else printf("%d ",i);
				}
				return 0; // 函数结束 
			}
		}
		first += 2; 
	}
	return 0;
} 
\end{lstlisting}

\begin{note}[要点]
	再次强调进入内层循环前, 相关变量的初始化; 以及标志变量(如本例first)的使用技巧。
\end{note}

\section{选号程序}
小明决定申请一个新的QQ号码,系统随机生成了若干个号码供他选择。小明的选号原则是:
\begin{enumerate}
	\item 选择所有号码中各位数字之和最大的号码。
	\item 如果有多个号码各位数字之和相同则选择数值最大的号码。
\end{enumerate}

请你写一个程序帮助小明选择一个QQ号码。

输入说明

输入数据由两行构成,第一行为一个整数n表示有n个待选号码(0<n<100),第二行有n个正整数,表示各个待选的号码,每个号码长度不超过9位数。每个号码之间用空格分隔,且每个号码都不相同。

输出说明

输出根据小明的选号原则选出的号码。

输入样例

5

10000 11111 22222 333 1234

输出样例

22222

\begin{lstlisting}
#include <stdio.h>
// 在循环语句中, 读取备选qq号, 计算各位之和, 依据筛选条件选取qq号
int main() 
{ 
	// 关键变量含义说明: 
	// select_qq,select_sum表示备选qq及其各位之和
	// qq,sum表示当前读取的qq及其各位和 
	int i,n,select_qq,select_sum,qq,sum,tmp;
	scanf("%d",&n);
	for(i=0;i<n;i++) // 注意条件表达式, 表明i的最大值是n-1, 因为i是0开始的, 因此共执行n次循环
	{
		scanf("%d",&qq); // 读取当前备选qq号
		tmp=qq; // 保存到临时变量中,因为下面的循环语句要更改。 
		sum=0;  // 当前读取qq号的各位之和。 注意: 一定要初始化,否则上一个备选号的sum值会带入本轮循环中。 
		while(tmp) // 计算各位之和 
		{
			sum+=tmp%10;
			tmp/=10;
		}
		// 第1轮迭代(i==0), 当前读取的qq就是所选, 其它根据题设条件选号
		// 因为三个表达式为||运算, 从左到右依次计算各表达式的值, 如果为真,则不会计算后边表达式。
		// 因此, 当i==0时不会计算其它两个表达式的值, if条件为真。 
		if(i==0 || sum>select_sum || (sum==select_sum && qq>select_qq))
		{ 
			select_qq=qq;
			select_sum=sum; // i==0时,select_sum初值为第一个号码各位之和.
		} 
	}
	printf("%d",select_qq);
	return 0;
} 

// 解法2: 用二维数组存储所有qq号及其各位和 
#define N 100 // 估计最大数组长度 
int main1() 
{
	// 二维数组No, 第一列表示qq号, 第二列表示该qq号的各位数字之和。 
	int i,n,No[N][2],tmp,sum,max=0,largest=0,select;
	scanf("%d",&n);
	// 筛选条件2
	for(i=0;i<n;i++)
	{
		scanf("%d",&No[i][0]);
		tmp=No[i][0];
		sum=0;  // 一定初始化 
		while(tmp)
		{
			sum+=tmp%10;
			tmp/=10;
		}
		No[i][1]=sum;
		if(sum>=max) max=sum; 
	}
	// 筛选条件1
	for(i=0;i<n;i++)
	{
		if(No[i][1]==max) // 备选号码
		{
			if(No[i][0]>=largest)
			{
				select=No[i][0];
				largest=No[i][0];
			}
		} 
	}
	printf("%d",select);
	return 0;
} 
\end{lstlisting}

\begin{note}[要点]
	\begin{enumerate}
		\item ||和 \&\&运算从左到右执行,取得结果,则不执行后面的表达式。\\
		取得结果的含义是: \\
		if (条件1||条件2||条件3)运算中, 只要有一个条件表达式为真(非0),即整个条件()结果即为真。 \\
		if (条件1 \&\&条件2 \&\& 条件3)运算中, 只要有一个条件表达式为假(0),即整个条件()结果即为假。
		\item 比较两种解法的优缺点。
		\item 本例是循环迭代的范例, 应反复演练, 领会迭代程序的编程技巧。
		\item 试着定义函数, 改写此程序。
		\item 本题不必使用排序算法,使程序复杂化。
	\end{enumerate}	
\end{note}


   % 第3次机试练习: 继续分支与循环练习

%%%%%%%%%%%%%%%%%%%%% chapter.tex %%%%%%%%%%%%%%%%%%%%%%%%%%%%%%%%%
%
% sample chapter
%
% Use this file as a template for your own input.
%
%%%%%%%%%%%%%%%%%%%%%%%% Springer-Verlag %%%%%%%%%%%%%%%%%%%%%%%%%%
%\motto{Use the template \emph{chapter.tex} to style the various elements of your chapter content.}
\chapter{第4次机试练习: 继续练习基本输入输出语句,分支与循环,简单数组应用}

\section{最小差值}
给定n个数,请找出其中相差(差的绝对值)最小的两个数,输出它们的差值的绝对值。

输入格式

输入第一行包含一个整数n。

第二行包含n个正整数,相邻整数之间使用一个空格分隔。

输出格式

输出一个整数,表示答案。

样例输入

5

1 5 4 8 20

样例输出

1

样例说明

相差最小的两个数是5和4,它们之间的差值是1。

样例输入

5

9 3 6 1 3

样例输出

0

样例说明

有两个相同的数3,它们之间的差值是0.

数据规模和约定

对于所有评测用例,$2\le n\le1000$,每个给定的整数都是不超过10000的正整数。

\begin{lstlisting}
#include <stdio.h>
#include <math.h>
#define N 10000 // 估计数组num的最大长度 
int main()
{
	int i,j,n,num[N],smallest,temp; 
	scanf("%d",&n);
	// 输入数组各元素
	for(i=0;i<n;i++) // 实际数组的最大长度n, 下标由0到(n-1)
	{
		scanf("%d",&num[i]);
	}
	// 初始的最小值就是前两个数的差值, 注意初始化值必须是实际存在的值,而不能想当然给值。
	smallest=(int)fabs(num[0]-num[1]); // 整数绝对值函数int abs(int x)在低版本编译器中有问题, 此处用双精度绝对值函数代替, 其结果转换为整数。  
	
	// 前后两项比较
	for(i=0;i<=n-2;i++) // 循环变量i用于访问数组元素, 注意数组边界问题
	{
		for(j=i+1;j<n;j++)
		{
			temp=(int)fabs(num[i]-num[j]);
			if(smallest>temp) smallest=temp;
		}
	}
	printf("%d\n",smallest);
	return 0;
} 
\end{lstlisting}

\begin{note}[整数求绝对值函数]
	\lstinline|int abs(int x);| 在有些低版本编译器中,\lstinline|math.h|头文件无此函数原型说明,可用\lstinline|double fabs(double x);|代替。见本例。
\end{note}

\section{车牌限行}
受雾霾天气影响,某市决定当雾霾指数超过设定值时对车辆进行限行,假设车牌号全为数字,且长度不超过6位,限行规则如下: 
\begin{enumerate}
	\item 限行时间段只包括周一至周五,周六周日不限行;
	\item 如果雾霾指数低于200,不限行;
	\item 如果雾霾指数大于等于200且低于400,每天限行两个尾号的汽车,周一限行1和6,周二限行2和7,周三限行3和8,周四限行4和9,周五限行5和0;
	\item 如果雾霾指数大于等于400,每天限行五个尾号的汽车,周一、周三和周五限行1,3,5,7,9,周二和周四限行0,2,4,6,8。 	
\end{enumerate}
现在给出星期几、雾霾指数和车牌号,判断该车牌号是否限行。 

输入说明

输入分为三个整数,第一个整数表示星期几(1~7,1表示周一,2表示周二,依次类推,7表示周日),

第二个整数表示雾霾指数(0~600),第三个整数表示车牌号,整数之间用空格分隔。

输出说明

输出为两个部分,第一部分为车牌最后一位数字,第二部分为限行情况,限行输出yes,不限行输出no。

输入样例

输入样例1 

4 230 80801 

输入样例2 

3 300 67008

输出样例

输出样例1 

1 no 

输出样例2 

8 yes 
\begin{lstlisting}
int main1()
{
	int  week, hazeIndex, No; // 星期几, 雾霾指数, 车牌号码
	int LastNo; // 车牌号最后一位数字
	int control=0; // 0:不限行; 1: 限行 
	
	scanf("%d%d%d",&week,&hazeIndex,&No);
	LastNo=No%10; 
	switch(week)
	{
		case 1: 
			if(hazeIndex>=200 && hazeIndex<400 && (LastNo==1 || LastNo==6)) 
				control=1;
			if(hazeIndex>=400 && (LastNo%2 != 0)) 
				control=1;  
			break; 
		case 2: 
			if(hazeIndex>=200 && hazeIndex<400 && (LastNo==2 || LastNo==7)) 
				control=1;
			if(hazeIndex>=400 && (LastNo%2 == 0)) 
				control=1; 
			break;
		case 3: 
			if(hazeIndex>=200 && hazeIndex<400 && (LastNo==3 || LastNo==8)) 
				control=1;
			if(hazeIndex>=400 && (LastNo%2 != 0)) 
				control=1; 
			break; 
		case 4: 
			if(hazeIndex>=200 && hazeIndex<400 && (LastNo==4 || LastNo==9))
				control=1;
			if(hazeIndex>=400 && (LastNo%2 == 0)) 
				control=1; 
			break;
		case 5: 
			if(hazeIndex>=200 && hazeIndex<400 && (LastNo==5 || LastNo==0))
				control=1;
			if(hazeIndex>=400 && (LastNo%2 != 0)) 
				control=1; 
			break;
		case 6:
		case 7: break;
	} 
	if(control==1) printf("%d yes",LastNo);
	else printf("%d no",LastNo);
	return 0;
} 

int main1() // 另解(数组标志变量)
{
	int  week, hazeIndex, No; // 星期几, 雾霾指数, 车牌号码
	int LastNo; // 车牌号最后一位数字
	int control[2][5][10]={
		// hazeIndex>=200 && hazeIndex<400
		{ 
			{0,1,0,0,0,1,0,0,0,0}, // 周一 
			{0,0,1,0,0,0,0,1,0,0}, // 周二 
			{0,0,0,1,0,0,0,0,1,0}, // 周三
			{0,0,0,0,1,0,0,0,0,1}, // 周四 
			{1,0,0,0,1,0,0,0,0,0}, // 周五 
		}, 
		// hazeIndex>=400
		{
			{0,1,0,1,0,1,0,1,0,1}, // 周一 
			{1,0,1,0,1,0,1,0,1,0}, // 周二 
			{0,1,0,1,0,1,0,1,0,1}, // 周三 
			{1,0,1,0,1,0,1,0,1,0}, // 周四 
			{0,1,0,1,0,1,0,1,0,1}, // 周五  
		}}; 

	scanf("%d%d%d",&week,&hazeIndex,&No);
	LastNo=No%10; 
	if(hazeIndex>=200 && hazeIndex<400)
	{
		if(control[0][week-1][LastNo]) printf("%d yes",LastNo);
		else printf("%d no",LastNo);
	} 
	else if(hazeIndex>=400)
	{
		if(control[1][week-1][LastNo]) printf("%d yes",LastNo);
		else printf("%d no",LastNo);
	}
	else
	{
		printf("%d no",LastNo);
	}
	return 0;
} 
\end{lstlisting}

\begin{note}[要点]
	\begin{enumerate}
		\item 首先假定变量的值(如,\lstinline|int control=0;|), 再根据题目要求, 计算它的真实值, 是基本技巧。
		\item 用数组作为标志变量(如,\lstinline|control[2][5][10];|是另一技巧。
	\end{enumerate}	
\end{note}

\section{PM2.5}
给出一组PM2.5数据,按以下分级标准统计各级天气的天数,并计算出PM2.5平均值。
PM2.5分级标准为:\\
一级优(0<=PM2.5<=50)\\
二级良(1<=PM2.5<=100)\\
三级轻度污染(101<=PM2.5<=150)\\
四级中度污染(151<=PM2.5<=200)\\
五级重度污染(201<=PM2.5<=300)\\
六级严重污染(PM2.5>300)\\

输入说明
	
输入分为两行,

第一行是一个整数n表示天数(1<n<=100);

第二行为n个非负整数Pi(0<=Pi<=1000),表示每天的PM2.5值,整数之间用空格分隔。

输出说明
	
输出两行数据,

第一行为PM2.5平均值,结果保留2位小数;

第二行依次输出一级优,二级良,三级轻度污染,四级中度污染,五级重度污染,六级严重污染的天数。

输入样例
	
10

50 100 120 80 200 350 400 220 180 165

输出样例	

186.50

1 2 1 3 1 2

\begin{lstlisting}
#include <stdio.h>

int main()
{
	// 用数组变量day存储数据,避免设置6个变量存储。
	int i =0,n,pm25,day[6] = {0,0,0,0,0,0},sum = 0;
	scanf("%d",&n);
	while(i < n) 
	{
		scanf("%d",&pm25);
		sum += pm25;
		if(pm25 >= 0 && pm25 <= 50 ) day[0]++;
		else if(pm25 >= 51 && pm25 <= 100 ) day[1]++;
		else if(pm25 >= 101 && pm25 <= 150 ) day[2]++;
		else if(pm25 >= 151 && pm25 <= 200 ) day[3]++;
		else if(pm25 >= 201 && pm25 <= 300 ) day[4]++;
		else day[5]++;
		i++;
	} 
	printf("%.2f\n",(float)sum/n);
	for(i = 0; i < 6; i++)  // 视作一条语句, 省略{ }
		if(i == 5) printf("%d\n",day[i]);
		else  printf("%d ",day[i]);
	
	return 0;
} 
\end{lstlisting}

\begin{note}[要点]
	\lstinline|if() { } else if() { } else { }|的用法, 循环语句的\{ \}.
	
	体会数组变量day的使用, 避免变量过多,难于管理的麻烦。
\end{note}

\section{气温波动}	
最近一段时间气温波动较大。已知连续若干天的气温,请给出这几天气温的最大波动值是多少,即在这几天中某天气温与前一天气温之差的绝对值最大是多少。

输入说明	

输入数据分为两行。

第一行包含了一个整数n,表示给出了连续n天的气温值,2 ≤ n ≤ 30。

第二行包含n个整数,依次表示每天的气温,气温为-20到40之间的整数。

输出说明
	
输出一个整数,表示气温在这n天中的最大波动值。

输入样例	

6

2 5 5 7 -3 5

输出样例
	
10

\begin{lstlisting}
#include <stdio.h>
#include <math.h>
int main()
{
	// last表示前一天的气温, temperature表示当天气温
	int i, n, last ,temperature, undulation = 0;
	scanf("%d",&n);
	
	// 当天气温temperature与前一天气温比较
	for(i = 0; i < n; i++)
	{
		scanf("%d",&temperature);
		if (i != 0 && fabs(temperature - last) > undulation) 
		undulation = fabs(temperature - last);
		last = temperature;
	} 
	printf("%d\n",undulation); 
	return 0;
} 
\end{lstlisting}

\begin{note}
	借助变量last表示前一天的气温,即可不用数组存储所有数据,是基本技巧。
\end{note}
	

\section{折点计数}	
给定n个整数表示一个商店连续n天的销售量。如果某天之前销售量在增长,而后一天销售量减少,则称这一天为折点,反过来如果之前销售量减少而后一天销售量增长,也称这一天为折点,其他的天都不是折点。如图所示,第3天和第6天是折点。

\includegraphics*[scale=0.7]{points}

给定n个整数a1, a2, \dots, an表示连续n天中每天的销售量。请计算出这些天总共有多少个折点。

输入说明
	
输入的第一行包含一个整数n。

第二行包含n个整数,用空格分隔,分别表示a1, a2, \dots, an。

3 ≤ n ≤ 100,每天的销售量是不超过1000的非负整数。为了减少歧义,输入数据保证:在这n天中相邻两天的销售量总是不同的,即$ai-1\ne ai$。

输出说明	

输出一个整数,表示折点数量。

输入样例	

7

5 4 1 2 3 6 4

输出样例
	
2

\begin{lstlisting}
#include <stdio.h>
int main()
{
    // 估计数组a的实际长度为100, 实际长度是n(待输入的值)
	int i = 0, points = 0, n = 7, a[100] = {5,4,1,2,3,6,4}; // 把样例数据作为初始化,方便了调试
	int up; // 标志变量 
	
	// 有了初始化数据,这些输入语句在调试时就可注释掉,达到快速调试程序逻辑的目的。
	scanf("%d",&n); // 输入数组a的实际长度
	// 输入各元数值
	for(i = 0; i < n; i++) scanf("%d",&a[i]);
	
	// 标志变量up的初始值必须是真实存在的值,不要想当然。
	up = a[1] > a[0] ? 1 : -1; // 如果a[1] > a[0]成立, up=1,否则up=-1
	for(i = 2; i < n; i++)
	{
		if((a[i] > a[i-1] && up < 0) || (a[i] < a[i-1] && up > 0))
			points++;
		up = a[i] > a[i-1] ? 1 : -1;
	}
	
	printf("%d\n",points);
	return 0;
} 
\end{lstlisting}

\begin{note}[要点]
	善用标志变量, 标志变量的初始值必须是真实存在的值,不要想当然。
	
	样例数据作为初始化数据,调试时注释掉输入语句,便于进行快速调试程序。
\end{note}

%%%%%%%%%% 留作下次练习
\begin{comment}

\section{数字分解排序}
输入一个9位以内的正整数n,按数值从高到低的顺序输出n的各位数字。

输入说明	

一个正整数n(0<n<1000000000)

输出说明	

按数值从高到低的顺序输出n的各位数字,数字之间用空格隔开

输入样例	

564391

输出样例
	
9 6 5 4 3 1

\begin{lstlisting}
#include <stdio.h>

void order(int a[],int n);  // 排序函数说明

int main()
{
	int i = 0,j,k,n,num[9];
	scanf("%d",&n);
	k=0; // 数字个数 
	while(n)
	{
		num[i++] = n%10;
		n /= 10;
		k++;
    } 

	// 输出验证,调试技巧之一。 
	//for(j = 0; j < k; j++) printf("%d ",num[j]);
	//   printf("\n");

	order(num,k);  // 排序函数
	
	// 输出
	for(j = 0; j < k; j++) printf("%d ",num[j]);
	
	printf("\n");
	return 0;
} 

// 冒泡排序函数 
void order(int a[],int n)
{ 
	int i,j,t,flag; 
	for(j = 1; j <= n-1; j++)
	{ 
		flag=0; // 且记! 必须在进入内层循环前初始化。
		for(i = 0; i < n - j; i++)
		if (a[i] < a[i+1]) 
		{ 
			t = a[i]; a[i] = a[i+1]; a[i+1] = t; 
			flag=1;
		}
		if(!flag) break;
	}
}
\end{lstlisting}

\begin{note}[要点]
	排序函数必须掌握,注意检查数组是否越界问题。
\end{note}

\end{comment}
   % 第4次机试练习: 继续分支与循环练习

%%%%%%%%%%%%%%%%%%%%% chapter.tex %%%%%%%%%%%%%%%%%%%%%%%%%%%%%%%%%
%
% sample chapter
%
% Use this file as a template for your own input.
%
%%%%%%%%%%%%%%%%%%%%%%%% Springer-Verlag %%%%%%%%%%%%%%%%%%%%%%%%%%
%\motto{Use the template \emph{chapter.tex} to style the various elements of your chapter content.}
\chapter{第5次机试练习: 继续练习基本输入输出语句,分支与循环,简单数组应用}

\section{最小差值}
给定n个数,请找出其中相差(差的绝对值)最小的两个数,输出它们的差值的绝对值。

输入格式

输入第一行包含一个整数n。

第二行包含n个正整数,相邻整数之间使用一个空格分隔。

输出格式

输出一个整数,表示答案。

样例输入

5

1 5 4 8 20

样例输出

1

样例说明

相差最小的两个数是5和4,它们之间的差值是1。

样例输入

5

9 3 6 1 3

样例输出

0

样例说明

有两个相同的数3,它们之间的差值是0.

数据规模和约定

对于所有评测用例,$2\le n\le1000$,每个给定的整数都是不超过10000的正整数。

\begin{lstlisting}
#include <stdio.h>
#include <math.h>
#define N 10000 // 估计数组num的最大长度 
int main()
{
	int i,j,n,num[N],smallest,temp; 
	scanf("%d",&n);
	// 输入数组各元素
	for(i=0;i<n;i++) // 实际数组的最大长度n, 下标由0到(n-1)
	{
		scanf("%d",&num[i]);
	}
	// 初始的最小值就是前两个数的差值, 注意初始化值必须是实际存在的值,而不能想当然给值。
	smallest=(int)fabs(num[0]-num[1]); // 整数绝对值函数int abs(int x)在低版本编译器中有问题, 此处用双精度绝对值函数代替, 其结果转换为整数。  
	
	// 前后两项比较
	for(i=0;i<=n-2;i++) // 循环变量i用于访问数组元素, 注意数组边界问题
	{
		for(j=i+1;j<n;j++)
		{
			temp=(int)fabs(num[i]-num[j]);
			if(smallest>temp) smallest=temp;
		}
	}
	printf("%d\n",smallest);
	return 0;
} 
\end{lstlisting}

\begin{note}[整数求绝对值函数]
	\lstinline|int abs(int x);| 在有些低版本编译器中,\lstinline|math.h|头文件无此函数原型说明,可用\lstinline|double fabs(double x);|代替。见本例。
\end{note}

\section{PM2.5}
给出一组PM2.5数据,按以下分级标准统计各级天气的天数,并计算出PM2.5平均值。
PM2.5分级标准为:\\
一级优(0<=PM2.5<=50)\\
二级良(1<=PM2.5<=100)\\
三级轻度污染(101<=PM2.5<=150)\\
四级中度污染(151<=PM2.5<=200)\\
五级重度污染(201<=PM2.5<=300)\\
六级严重污染(PM2.5>300)\\

输入说明
	
输入分为两行,

第一行是一个整数n表示天数(1<n<=100);

第二行为n个非负整数Pi(0<=Pi<=1000),表示每天的PM2.5值,整数之间用空格分隔。

输出说明
	
输出两行数据,

第一行为PM2.5平均值,结果保留2位小数;

第二行依次输出一级优,二级良,三级轻度污染,四级中度污染,五级重度污染,六级严重污染的天数。

输入样例
	
10

50 100 120 80 200 350 400 220 180 165

输出样例	

186.50

1 2 1 3 1 2

\begin{lstlisting}
#include <stdio.h>

int main()
{
	// 用数组变量day存储数据,避免设置6个变量存储。
	int i =0,n,pm25,day[6] = {0,0,0,0,0,0},sum = 0;
	scanf("%d",&n);
	while(i < n) 
	{
		scanf("%d",&pm25);
		sum += pm25;
		if(pm25 >= 0 && pm25 <= 50 ) day[0]++;
		else if(pm25 >= 51 && pm25 <= 100 ) day[1]++;
		else if(pm25 >= 101 && pm25 <= 150 ) day[2]++;
		else if(pm25 >= 151 && pm25 <= 200 ) day[3]++;
		else if(pm25 >= 201 && pm25 <= 300 ) day[4]++;
		else day[5]++;
		i++;
	} 
	printf("%.2f\n",(float)sum/n);
	for(i = 0; i < 6; i++)  // 视作一条语句, 省略{ }
		if(i == 5) printf("%d\n",day[i]);
		else  printf("%d ",day[i]);
	
	return 0;
} 
\end{lstlisting}

\begin{note}[要点]
	\lstinline|if() { } else if() { } else { }|的用法, 循环语句的\{ \}.
	
	体会数组变量day的使用, 避免变量过多,难于管理的麻烦。
\end{note}

\section{折点计数}	
给定n个整数表示一个商店连续n天的销售量。如果某天之前销售量在增长,而后一天销售量减少,则称这一天为折点,反过来如果之前销售量减少而后一天销售量增长,也称这一天为折点,其他的天都不是折点。如图所示,第3天和第6天是折点。

\includegraphics*[scale=0.7]{points}

给定n个整数a1, a2, \dots, an表示连续n天中每天的销售量。请计算出这些天总共有多少个折点。

输入说明
	
输入的第一行包含一个整数n。

第二行包含n个整数,用空格分隔,分别表示a1, a2, \dots, an。

$3\le n\le 100$,每天的销售量是不超过1000的非负整数。为了减少歧义,输入数据保证:在这n天中相邻两天的销售量总是不同的,即$ai-1\ne ai$。

输出说明	

输出一个整数,表示折点数量。

输入样例	

7

5 4 1 2 3 6 4

输出样例
	
2

\begin{lstlisting}
#include <stdio.h>
// 解法一, 使用标志变量up, 判断是否后一数据大于前一数据.
int main()
{
    // 估计数组a的实际长度为100, 实际长度是n(待输入的值)
	int i = 0, points = 0, n = 7, a[100] = {5,4,1,2,3,6,4}; // 把样例数据作为初始化,方便调试
	int up; // 标志变量 
	
	// 有了初始化数据,这些输入语句在调试时就可注释掉,达到快速调试程序逻辑的目的。
	scanf("%d",&n); // 输入数组a的实际长度
	// 输入各元数值
	for(i = 0; i < n; i++) scanf("%d",&a[i]);
	
	// 标志变量up的初始值必须是真实存在的值,不要想当然。
	up = a[1] > a[0] ? 1 : -1; // 如果a[1] > a[0]成立, up=1,否则up=-1
	for(i = 2; i < n; i++)
	{
		if((a[i] > a[i-1] && up < 0) || (a[i] < a[i-1] && up > 0))
			points++;
		up = a[i] > a[i-1] ? 1 : -1;
	}
	
	printf("%d\n",points);
	return 0;
} 

// 解法二, 通过判断a[i]前后相邻两数据, 确定是否折点. 
int main()
{
	// 估计数组a的实际长度为100, 实际长度是n(待输入的值)
	int i = 0, points = 0, n = 7, a[100] = {5,4,1,2,3,6,4}; // 把样例数据作为初始化,方便调试
	scanf("%d",&n);
	for(i=0;i<n;i++) scanf("%d",&a[i]);
	
	for(i=1;i<n-1;i++) // 注意数组边界,不要越界。 
	{
		if((a[i]<a[i-1] && a[i]<a[i+1]) || (a[i]>a[i-1] && a[i]>a[i+1]))
			points++;
	}
	printf("%d\n",points);
	return 0;
}
\end{lstlisting}

\begin{note}[要点]
	善用标志变量, 标志变量的初始值必须是真实存在的值,不要想当然。
	
	样例数据作为初始化数据,调试时注释掉输入语句,便于进行快速调试程序。
\end{note}

\section{寻找最大整数}
从键盘输入四个整数,找出其中的最大值并将其输出。

输入说明

输入4个整数,用空格分隔

输出说明

输出值最大的一个整数

输入样例

25 99 -46 0

输出样例

99

\begin{lstlisting}
#include <stdio.h>
// 不用存储整数序列, 采用一条循环语句,合并输入和计算,减少出错概率。 
int main() 
{
	int i, num, max;
	// 输入, 并计算 
	for(i = 0; i < 4; i++) 
	{ 
		if (i==0) scanf("%d",&max); // 假定第一个数就是最大的数 
		else
		{
			scanf("%d",&num);
			if(num > max) max=num; 
		} 
	} 
	printf("%d\n",max);
	return 0;
} 

int main1() // 另解, 存储整数序列
{
	int i, num[4], max;
	// 输入 
	for(i = 0; i < 4; i++) 
		scanf("%d",&num[i]);
	// 假定的最大值必须是实际存在的,不要想当然是0,9999,等等。
	max = num[0]; 
	for(i = 0; i < 4; i++)  
		if(max < num[i]) max = num[i];
	
	printf("%d\n",max);
	return 0;
} 
\end{lstlisting}

\begin{note}[要点]
	题目虽然简单,你能体会哪种解法更好? 特别注意假定变量的值必须是实际存在的数。
\end{note}

\section{ISBN号码}
每一本正式出版的图书都有一个ISBN号码与之对应,ISBN码包括9位数字、1位识别码和3位分隔符,其规定格式如"x-xxx-xxxxx-x",
其中符号“-”是分隔符(键盘上的减号),最后一位是识别码,例如0-670-82162-4就是一个标准的ISBN码。

ISBN码的首位数字表示书籍的出版语言,例如0代表英语;

第一个分隔符“-”之后的三位数字代表出版社,例如670代表维京出版社;

第二个分隔之后的五位数字代表该书在出版社的编号;

最后一位为识别码。识别码的计算方法如下:

首位数字乘以1加上次位数字乘以2……以此类推,用所得的结果mod 11,所得的余数即为识别码,如果余数为10,则识别码为大写字母X。
例如ISBN号码0-670-82162-4中的识别码4是这样得到的:
对067082162这9个数字,从左至右,分别乘以1,2,\dots,9,再求和,即$0\times 1+6\times 2+\dots +2\times 9=158$,然后取158 mod 11的结果4作为识别码。

编写程序判断输入的ISBN号码中识别码是否正确,如果正确,则仅输出``Right"; 如果错误,则输出正确的ISBN号码。

输入说明	

输入只有一行,是一个字符序列,表示一本书的ISBN号码(保证输入符合ISBN号码的格式要求)。

输出说明	

输出一行,假如输入的ISBN号码的识别码正确,那么输出``Right",否则,按照规定的格式,输出正确的ISBN号码(包括分隔符``-")。

输入样例	

样例输入1

0-670-82162-4

样例输入2

0-670-82162-0

输出样例	

样例输出1

Right

样例输出2

0-670-82162-4

\begin{lstlisting}
#include <stdio.h>
int main()
{
	// 按题意, 存储ISBN需要13个字符, 再加上字符串结束字符'\0'.
	// 样例数据作为初始化,方便调试
	char ISBN[14] = "0-670-82162-4";  // ISBN[13]='\0'
	//char ISBN[14] = "0-670-82162-0";  
	int i,j,sum=0,r; // r=sum % 11 
	
	// 调试时,注释输入语句 
	// 末尾自动添加'\0'.
	scanf("%s",ISBN);  // 或 gets(ISBN);
	
	// 对识别码之前的数字求和, 注意边界.
	for(i = 0,j = 1; i < 11; i++)
	{
		if (ISBN[i]=='-') continue;
		sum += (ISBN[i]-'0')*j;   // 整数与单个数字字符的关系:9 = '9' -'0' 
		j++;
	}   
	
	r=sum%11;
	
	// 特别处理识别码 ISBN[12]
	if (r==10)
	{
		if(ISBN[12]=='X') printf("Right\n");
		else 
		{
			ISBN[12]='X';
			printf("%s\n",ISBN); // 或 puts(ISBN);
		}
	}
	else
	{
		// 数字转字符 r+'0'
		if(ISBN[12]== r +'0') printf("Right\n");
		else 
		{
			ISBN[12]= r +'0';
			printf("%s\n",ISBN); // 或 puts(ISBN);
		}
	} 
	
	return 0;
} 
\end{lstlisting}

\begin{note}[要点]
	\begin{enumerate}
		\item 仔细审题, 考虑输入的识别码, 可能是'X'的情况.
		\item 定义字符数组表示字符串时,且记给\lstinline|'\0'|留一个字符的位置, 表示字符串的结尾。
		\item \lstinline|ASCII编码(整数)=字符-'0';|
		\item \lstinline|字符=ASCII编码(整数)+'0';|
		\item 整数可以表示字符的ASCII编码(整数), 整数和字符类型可以``混用", 详见课件。
		\begin{lstlisting}
		int a; char c='A';
		a = c+1; // c当作整数运算
		printf("%d %c %d %c\n",a,a,c,c); //66 B 65 A 
		\end{lstlisting}
	\end{enumerate}
\end{note}

   % 第5次机试练习: 继续练习流程控制, 简单数组应用

\end{document}

%%%%%%%%%%%%%%%%%%%%% chapter.tex %%%%%%%%%%%%%%%%%%%%%%%%%%%%%%%%%
%
% sample chapter
%
% Use this file as a template for your own input.
%
%%%%%%%%%%%%%%%%%%%%%%%% Springer-Verlag %%%%%%%%%%%%%%%%%%%%%%%%%%
%\motto{Use the template \emph{chapter.tex} to style the various elements of your chapter content.}
\chapter{第6次机试练习: 函数, 流程控制, 字符串, 数组}

\section{歌德巴赫猜想}
德巴赫猜想:任意一个大偶数都能分解为两个素数的和,对于输入的一个正偶数,写一个程序来验证歌德巴赫猜想。

由于每个正偶数可能分解成多组素数和,仅输出分解值分别是最小和最大素数的一组,按从小到大顺序输出。

输入说明	

输入一个正偶数n,1<n<1000。

输出说明
	
输出分解出的两个最小和最大素数。

输入样例
	
10

输出样例
	
3 7

\begin{lstlisting}
#include <stdio.h>
#include <math.h>

// 判断参数n是否素数, 如果是返回1, 否则返回0 
int isPrime(int n)
{
	int i;
	if(n < 2) return 0; // 最小素数是2, 1不是素数也不是合数, 题意(1<n<1000)不包含1,因此此语句不是必须的 
	//for(i = 2; i <= sqrt((double)n); i++) // vs2013编译器要求数学函数严格按原型解释
	for(i = 2; i <= sqrt(n); i++) // 或条件表达式:i*i<=n 
	{
		if (n%i == 0) return 0; // n不是素数 
	}
	return 1; // n是素数 
}

// 解法一: 递归函数
// 第1个参数: i是递归可能的素数(第一个可能的素数是big_even-1), 第2个参数: big_even是大偶数 
void  recursiveComputing(int i, int big_even)
{
	if(isPrime(i) && isPrime(big_even-i)) // 两个素数和是big_even
	{
		// 注意, 较小的在前 
		printf("%d %d\n",big_even - i, i); // 输出后, return; 结束递归 
	} 
	else // 递归 
	{
		recursiveComputing(i-1,big_even); // 递归调用 
	} 
	return; // 递归结束
} 

int mian()
{
	int num; // num大偶数
	scanf("%d",&num);
	recursiveComputing(num-1,num);
	return 0; 
}

// 解法二: 二重循环, 从最小素数开始迭代计算,获取符合题意的两个素数 
int main1()
{
	int j,k,num; // num大偶数 
	int flag; // 标志变量:用于标识是否找到符合要求的素数对
	scanf("%d",&num);
	// 对于大偶数num,分解为两个素数 
	for(j = 2; j < num; j++)  //  找出第一个素数(最小的) for #1 
	{
		if(!isPrime(j)) continue; // 如果j不是素数, 继续下一轮迭代 
		
		flag = 0; // 初始化,未找出素数对 
		for(k = j + 1; k < num; k++) // //  找出第二个素数 for #2
		{
			if(isPrime(k) && j+k == num) // j是最小素数, k必然是最大素数
			{
				printf("%d %d\n",j,k);
				flag = 1; // 找出素数对, 如果没有此设置, 将会输出多组, 例如num=2020时会有多组素数 
				break; // break for #2
			}
		}
		if(flag) break; // break for #1 
	}
	return 0;
}

// 解法三: 不用标志变量版本 
int main2()
{
	int j,k,num; // num大偶数 
	scanf("%d",&num);
	// 对于大偶数num,分解为两个素数 
	for(j = 2; j < num; j++)  //  找出第一个素数(最小的)  for #1 
	{
		if(!isPrime(j)) continue; // 如果j不是素数, 继续下一轮迭代 
		
		for(k = j + 1; k < num; k++) //  找出第二个素数 for #2
		{
			if(isPrime(k) && j+k == num) // j是最小素数, k必然是最大素数 
			{
				printf("%d %d\n",j,k);
				return 0; // 找出素数对,结束主函数。 如果不结束, 将会输出多组 
			}
		}
	}
	return 0;
}

// 解法四: 优化,根据题意找出一组:最小素数+最大素数=偶数 
int main3()
{
	int j,k,num; // num大偶数 
	scanf("%d",&num);
	// 对于大偶数num,分解为两个素数 
	for(j = 2; j < num; j++)  //  找出第一个素数(最小的)  for #1 
	{
		if(!isPrime(j)) continue; // 如果j不是素数, 继续下一轮迭代 
		
		for(k = num-1; k>=2; k--) //  找出第二个素数(最大的) for #2
		{
			// j是最小素数, 判断k是否是最大素数并且二者之和=num 
			if(isPrime(k) && j+k == num) 
			{
				printf("%d %d\n",j,k);
				return 0; // 找出素数对,结束主函数。如果不结束, 将会输出多组 
			}
		}
	}
	return 0;
}

// 解法五: 一重循环, 从最小素数开始迭代计算,获取符合题意的两个素数 
int main4() 
{ 
	int num,i;
	scanf("%d",&num);
	
	for(i = 2;i < num;i++)
	{
		if(isPrime(i)) // i是素数 
		{
			if(isPrime(num-i)) // 如果num-i也是素数, 即满足题意num=i+j 
			{
				printf("%d %d\n",i,num-i); 
				break;
			} 
		}
	}
	return 0; 
} 
\end{lstlisting}

\begin{note}[要点]
	再次体会标志变量的用法及内层循环前的初始化。仔细审题,本题要求一组输出: 最小素数+最大素数=偶数.
	
	如果要求找出最接近的一组素数, 因为$n=\frac{n}{2}+\frac{n}{2}$, 如果$\frac{n}{2}$是素数,即为所求。 否则,所求的两个素数在$\frac{n}{2}$附近, 分别小于和大于$\frac{n}{2}$。
	
	例如: 修改main4()中的\lstinline|if(isPrime(num-i))为if(isPrime(num-i) && num-i<=i)|, 则输出\lstinline|num-i,i;|即可。 
	
	或者: 由$i=\frac{n}{2}\to i=2$迭代, 第一个找到的素数$i$和$num-i$即为所求。
\end{note}

\section{矩阵}
请写一个程序,对于一个m行m列$(2<m<20)$的方阵,求其每一行、每一列及主、辅对角线元素之和,然后按照从大到小的顺序依次输出这些值。

注:主对角线是方阵从左上角到右下角的一条斜线,辅对角线是方阵从右上角到左下角的一条斜线。

输入说明

输入数据的第一行为一个正整数m;

接下来为m行、每行m个整数表示方阵的元素。

输出说明	

从大到小排列的一行整数,每个整数后跟一个空格,最后换行。

输入样例

\begin{tabular}{llll}
	4&&&\\
	15  &~8   &~-2   &~6\\
	31  &~24  &~18  &~71\\
	-3  &~-9  &~27  &~13\\
	17  &~21  &~38  &~69
\end{tabular}

输出样例

159 145 144 135 81 60 44 32 28 27

\begin{lstlisting}
#include <stdio.h>
// 估计方阵行列数 
#define M 20

// input, m是实际方阵行列数 
void input(int matrix[][M], int m)
{
	int i,j; 
	for(i = 0; i < m; i++)
	for(j = 0; j < m; j++) 
	scanf("%d",&matrix[i][j]);
}

// 计算主对角线之和, m是实际方阵行列数 
int main_diagonal(int matrix[][M], int m)
{	
	int i,j,sum = 0; 
	for(i = 0; i < m; i++)
	{
		for(j = 0; j < m; j++) 
		{
			if(i == j) sum += matrix[i][j];   // 主对角线 
		}
	}
return sum;
}

// 计算副对角线之和, m是实际方阵行列数 
int counter_diagonal(int matrix[][M], int m)
{	
	int i,j,sum = 0;
	for(i = 0; i < m; i++)
	{
		for(j = 0; j < m; j++) 
		{ 
			if(j == m-i-1) sum += matrix[i][j]; // 副对角线之和 
		}
	}
	return sum;
}

// 计算第i行之和, m是实际方阵行列数 
int sumI(int matrix[][M], int m, int i)
{ 
	int j,sum = 0; 
	for(j = 0; j < m; j++) // 遍历列 
	{
		sum += matrix[i][j];  // 第i行之和
	}
	return sum; 
} 

// 计算第j行之和, m是实际方阵行列数 
int sumJ(int matrix[][M], int m, int j)
{ 
	int i,sum = 0; 
	for(i = 0; i < m; i++) // 遍历行 
	{
		sum += matrix[i][j];  // 第j行之和
	}
	return sum;
} 

// 交换两个元素值 
void swap(int *p1, int *p2)
{
	int temp;
	temp = *p1; *p1 = *p2; *p2 = temp;
}

// 选择法排序(降序)
void sorts(int a[], int n)
{
	int i,j,k;
	for(i = 0; i < n-1; i++)
	{
		k = i;
		for (j = i+1; j < n; j++)
			if(a[j] > a[k])	k = j;
		if (k != i) swap(&a[i],&a[k]);
	} 
}

int main()
{
	int matrix[M][M],a[2*M+2]; // 以估计行列数,定义数组 
	int i,m; 
	scanf("%d",&m); // 实际方阵行列数 
	
	input(matrix,m); // input 
	
	// 调用各函数,装配数组a
	int n = 0; // 记录数组a的实际长度		
	for(i = 0; i < m; i++)
	{
		a[n++] = sumI(matrix,m,i); // 第i行之和
		a[n++] = sumJ(matrix,m,i); // 第i列之和  
	}
	
	a[n++] = main_diagonal(matrix,m);    // 主对角线之和 
	a[n++] = counter_diagonal(matrix,m); // 副对角线之和  
	
	// 排序数组a 
	sorts(a,n);
	
	// 输出 
	for(i = 0; i < n; i++)
		printf("%d ",a[i]);
	
	printf("\n");
	
	return 0;
}
\end{lstlisting}

\begin{note}[要点]
	\begin{enumerate}
		\item 思路:定义功能单一的函数,实现简单功能,主程序调用各个函数。
		\item 一维数组a[2*M+2]存储相关函数计算结果,排序数组a即是所求。 
		\item 避免过多循环嵌套,不易出错,简化程序设计。 
		\item 但是缺点是在各函数中分别循环遍历方阵,效率低。
		\item 优化方案是不采用独立函数计算,在主函数中一次遍历方阵,计算各值。 
		\item 二维数组表示矩阵, 是常见题型, 必须熟练掌握元素的下标规律及其遍历技巧。
	\end{enumerate}
\end{note}

\section{回文数}
若一个非负整数其各位数字按照正反顺序读完全相同,则称之为回文数,例如12321。
判断输入的整数是否是回文数。若是,则输出该整数各位数字之和,否则输出no。

输入说明	

输入为一个整数n,0<=n<100000000。

输出说明	

若该整数为回文数,则输出整数各位数字之和,否则输出no。

输入样例
	
样例1输入

131

样例2输入

24

输出样例

样例1输出
	
5

样例2输出

no

\begin{lstlisting}
#include <stdio.h>
#include <string.h>

// 思路1:  求该整数的反序组成的整数,如整数1234, 其反序整数即为4321, 如果二者相等即为回文数 
// 判断num是否是回文数,是:返回1;不是,返回0
// 指针参数*sum,返回这个数的各位之和  
int isPalindromic1(int num,int *sum)
{
	int reverse = 0, tmp = num;
	*sum = 0; // 初始化指针内容 
	while(tmp)
	{
		reverse = reverse*10 + tmp%10;
		*sum += tmp%10;
		tmp /= 10;
	}
	if(reverse == num ) return 1;
	else return 0;	
}

// 思路2:  构造数组reverse, 反序存储该整数各位数,按照数组下标,前后数组元素相等则为回文数 
// 判断num是否是回文数,是:返回1;不是,返回0
// 指针参数*sum,返回这个数的各位之和  
int isPalindromic2(int num,int *sum)
{
	int reverse[9], len=0, i=0; // 依题意数组最大长度为9,最多存储9位数。实际长度用len变量表示 
	
	*sum = 0; // 初始换指针内容 
	//构造数组reverse, 反序存储该整数各位数
	while(num)
	{
		reverse[i]=num%10;  
		*sum += num%10;   // 累加各位数字 
		num /= 10;
		len++; // 计算数组实际长度 
		i++; 
	}
	//按照数组下标,前后数组元素相等则为回文数
	for(i=0; i<len/2;i++)
	{
		if(reverse[i] != reverse[len-i-1]) return 0;  // 不是回文数 
	}
	return 1;	// 至此,必然是回文数 
}

// 测试方法1和2的主程序
int main12()
{
	int i,num,sum;
	
	scanf("%d",&num); 
	//if(isPalindromic1(num,&sum)) printf("%d\n",sum);
	if(isPalindromic2(num,&sum)) printf("%d\n",sum);
	else printf("no\n");
	return 0;
} 

// 思路3:  按照字符串处理输入的整数,前后数组元素相等则为回文数 
int main3()
{
	char s[10]; // 留出'\0', 最多存储9位数。实际长度用len变量表示
	int sum=0, len=0,i=0; 
	
	// 以字符串形式接收输入的整数, 末尾自动追加'\0'  
	gets(s);
	
	// 计算len, 或者len=strlen(s), 同时计算各位数字之和 
	for(len=0;s[i]!='\0';) 
	{
		len++;
		sum=sum+s[i]-'0'; // 计算各位数字之和
		i++;  
	}
	
	//按照数组下标,前后数组元素相等则为回文数
	for(i=0; i<len/2;i++)
	{
		if(s[i] != s[len-i-1]) // 不是回文数  
		{
			printf("no\n");
			return 0;  // 主函数结束 
		} 
	}
	// 至此,必然是回文数,包含一位数不进入上面的for循环的情况 
	printf("%d\n",sum);
	return 0;
}

// 思路4:  按照字符串处理输入的整数,前后数组元素相等则为回文数。
// 使用指针操作 
int main()
{
	char s[10]; // 留出'\0', 最多存储9位数。实际长度用len变量表示
	int sum=0; 
	char *p1=s,*p2=s; // 用于正序和反序遍历s数组,初始指向第一个元素 
	
	// 以字符串形式接收输入的整数,末尾自动追加'\0' 
	gets(s);
	
	// 用p2遍历字符串,同时计算各位数字之和 
	for(;*p2!='\0';p2++) 
	{
		sum=sum+(*p2)-'0'; // 计算各位数字之和
	}
	// 至此,p2指向最后一个元素'\0', 我们使它指向最后一个有效元素: 
	p2--; 
	
	//按照数组下标,前后数组元素相等则为回文数
	for(;p1<p2;p1++,p2--)
	{
		if(*p1 != *p2) // 不是回文数 
		{
			printf("no\n");
			return 0;  // 主函数结束 
		} 
	}
	// 至此,必然是回文数 
	printf("%d\n",sum);
	return 0;
}
\end{lstlisting}

\begin{note}[要点]
	\begin{enumerate}
		\item 掌握函数的地址传递方法。
		\item 使用两个指针变量p1, p2, 其中p1指向待查找子串的首字母, 另一个指向末尾, \lstinline|p1++, p2--|; 是判断字符串是否是回文的有效技巧。
	\end{enumerate}
\end{note}

\section{马鞍点}	
若一个矩阵中的某元素在其所在行最小而在其所在列最大,则该元素为矩阵的一个马鞍点。请写一个程序,找出给定矩阵的马鞍点。

输入说明

输入数据第一行只有两个整数m和n(0<m<100,0<n<100),分别表示矩阵的行数和列数;

接下来的m行、每行n个整数表示矩阵元素(矩阵中的元素互不相同), 整数之间以空格间隔。

输出说明

在一行上输出马鞍点的行号、列号(行号和列号从0开始计数)及元素的值(用一个空格分隔), 之后换行;

若不存在马鞍点,则输出一个字符串``no"后换行。

输入样例

\begin{tabular}{p{1cm}p{1cm}p{1cm}p{1cm}l}
	4  &3&\\
	11    &13    &121\\
	407   &72    &88\\
	23    &58    &1\\
	134   &30    &62
\end{tabular} 

输出样例

1 1 72

\begin{lstlisting}
#include <stdio.h>
// 估计的二维数组最大行列数 
#define M 100
#define N 100

// 判断a[row,col]是否是马鞍点, 是: 返回1; 否则返回0
// m,n是二维数组实际行列数 
int compute(int a[][N], int m, int n, int row, int col)
{
	int i,element = a[row][col]; 
	// element在其所在行最小而在其所在列最大 
	for(i = 0; i < n; i++)
		if(a[row][i] < element) return 0; // 不是马鞍点,直接返回0 
	for(i = 0; i < m; i++)
		if(a[i][col] > element) return 0; // 不是马鞍点,直接返回0
	return 1;	// 如果执行至此,肯定是马鞍点,直接返回0 
}

int main()
{
	int matrix[M][N]; // 按照估计的最大行列数定义二维数组 
	int i,j,m,n,flag = 0; 
	
	scanf("%d%d",&m,&n); // 实际行列数 
	
	// input
	for(i = 0; i < m; i++)
		for(j = 0; j < n; j++) 
			scanf("%d",&matrix[i][j]);
	
	// 遍历二维数组, 判断马鞍点		
	for(i = 0; i < m; i++)
	{
		for(j = 0; j < n; j++) 
		{
			if(compute(matrix,m,n,i,j))
			{
				printf("%d %d %d\n",i,j,matrix[i][j]);
				flag = 1;
			}
		}
	}	
	if (!flag) printf("no\n");
	
	return 0;
}
\end{lstlisting}

\begin{note}[要点]
	\begin{enumerate}
		\item 思路:定义函数计算单个元素a[i,j]是否是马鞍点,主程序遍历二维数组,调用此函数。
		\item 避免过多循环嵌套,不易出错,简化程序设计。
		\item 掌握二维数组作为函数参数的定义, 调用方式。
	\end{enumerate}
\end{note}

\section{密码强度}
每个人都有很多密码,你知道你的密码强度吗?假定密码由大写字母、小写字母、数字和非字母数字的符号这四类字符构成,密码强度计算规则如下:

1.	基础分:空密码(密码长度为零)0分, 非空密码1分

2.	加分项1:密码长度超过8位, +1分

3.	加分项2:密码包含两类不同字符+1分, 包含三类不同字符+2分, 包含四类不同字符+3分

按照此规则计算的密码强度为$0\sim 5$。请你设计一个程序计算给出的密码的强度。

输入说明

输入为一个密码字符串,字符串长度不超过50个字符。

输出说明

输出一个整数表示该密码的强度。

输入样例

输入样例1

abcd

输入样例2

ab123

输出样例

样例1输出:

1

样例2输出

2

\begin{lstlisting}	
#include <stdio.h>
#include <string.h>
int main()
{
	char p[51]; // 记得给'\0'留位置
	int i,strength = 0;
	// class4[0]=1大写字母, class4[1]=1小写字母, class4[2]=1数字, class4[3]=1非字母数字 
	int class4[4] = {0,0,0,0};  
	
	//scanf("%s",p);  // 不能完整接收含空格的字符串和空密码 
	gets(p); // last char: '\0',直接回车,就是空密码 
	
	// 	1.	基础分:空密码(密码长度为零)0分,非空密码1分 
	if(strlen(p) == 0) strength += 0;
	else  strength += 1;
	
	// 2.	加分项1:密码长度超过8位,+1分 
	if(strlen(p) > 8) strength += 1;
	
	// 3.	加分项2:密码包含两类不同字符+1分,包含三类不同字符+2分,包含四类不同字符+3分 
	for(i = 0; p[i] != '\0'; i++)
	{
		if(p[i] >= 'A' && p[i] <= 'Z') class4[0] = 1; 
		else if(p[i] >= 'a' && p[i] <= 'z') class4[1] = 1; 
		else if(p[i] >= '0' && p[i] <= '9') class4[2] = 1;
		else class4[3] = 1; 
	}
	int c = 0;
	for(i = 0; i < 4; i++) c += class4[i];
	
	if(c >= 4) strength += 3;
	else if(c >= 3) strength += 2;
	else if(c >= 2) strength += 1;
	
	printf("%d\n",strength);
	
	return 0;
}
\end{lstlisting}

\begin{note}[要点]
	字符串处理的典型问题: \lstinline|'\0'|, 字符串相关函数\lstinline|char s1[81],s2[81]; strlen(s1), strcmp(s1,s2), strcpy(s1,s2); scanf("%s",s1), gets(s1)|的区别等, 应该充分掌握。
\end{note}



\section{数字分解排序}
输入一个9位以内的正整数n,按数值从高到低的顺序输出n的各位数字。

输入说明	

一个正整数n(0<n<1000000000)

输出说明	

按数值从高到低的顺序输出n的各位数字,数字之间用空格隔开

输入样例	

564391

输出样例

9 6 5 4 3 1

\begin{lstlisting}
#include <stdio.h>

void order(int a[],int n);  // 排序函数说明

int main()
{
	int i = 0,j,k,n,num[9];
	scanf("%d",&n);
	k=0; // 数字个数 
	while(n) // 构造数组num存储n的各位数字(从最低位到最高位存储)
	{
		num[i++] = n%10;
		n /= 10;
		k++;
	} 

	// 输出验证,调试技巧之一。 
	//for(j = 0; j < k; j++) printf("%d ",num[j]);
	//   printf("\n");
	
	order(num,k);  // 排序函数
	
	// 输出
	for(j = 0; j < k; j++) printf("%d ",num[j]);
	
	printf("\n");
	return 0;
} 

// 冒泡排序函数 
void order(int a[],int n)
{ 
	int i,j,t,flag; 
	for(j = 1; j <= n-1; j++)
	{ 
		flag=0; // 且记! 必须在进入内层循环前初始化。
		for(i = 0; i < n - j; i++)
			if (a[i] < a[i+1]) 
			{ 
				t = a[i]; a[i] = a[i+1]; a[i+1] = t; 
				flag=1;
			}
		if(!flag) break;
	}
}
\end{lstlisting}

\begin{note}[要点]
	其它解法及其变种见课件。
	
	排序函数必须掌握,注意检查数组是否越界问题。
\end{note}

   % 第6次机试练习: 流程控制, 字符串, 数组

%%%%%%%%%%%%%%%%%%%%% chapter.tex %%%%%%%%%%%%%%%%%%%%%%%%%%%%%%%%%
%
% sample chapter
%
% Use this file as a template for your own input.
%
%%%%%%%%%%%%%%%%%%%%%%%% Springer-Verlag %%%%%%%%%%%%%%%%%%%%%%%%%%
%\motto{Use the template \emph{chapter.tex} to style the various elements of your chapter content.}
\chapter{第7次机试练习: 函数综合练习}

\section{歌德巴赫猜想}
德巴赫猜想:任意一个大偶数都能分解为两个素数的和,对于输入的一个正偶数,写一个程序来验证歌德巴赫猜想。

由于每个正偶数可能分解成多组素数和,仅输出分解值分别是最小和最大素数的一组,按从小到大顺序输出。

输入说明	

输入一个正偶数n,1<n<1000。

输出说明
	
输出分解出的两个最小和最大素数。

输入样例
	
10

输出样例
	
3 7

\begin{lstlisting}
#include <stdio.h>
#include <math.h>

// 判断参数n是否素数, 如果是返回1, 否则返回0 
int isPrime(int n)
{
	int i;
	if(n < 2) return 0; // 最小素数是2, 1不是素数也不是合数, 题意(1<n<1000)不包含1,因此此语句不是必须的 
	//for(i = 2; i <= sqrt((double)n); i++) // vs2013编译器要求数学函数严格按原型解释
	for(i = 2; i <= sqrt(n); i++) // 或条件表达式:i*i<=n 
	{
		if (n%i == 0) return 0; // n不是素数 
	}
	return 1; // n是素数 
}

// 解法一: 递归函数
// 第1个参数: i是递归可能的素数(第一个可能的素数是big_even-1), 第2个参数: big_even是大偶数 
void  recursiveComputing(int i, int big_even)
{
	if(isPrime(i) && isPrime(big_even-i)) // 两个素数和是big_even
	{
		// 注意, 较小的在前 
		printf("%d %d\n",big_even - i, i); // 输出后, return; 结束递归 
	} 
	else // 递归 
	{
		recursiveComputing(i-1,big_even); // 递归调用 
	} 
	return; // 递归结束
} 

int mian()
{
	int num; // num大偶数
	scanf("%d",&num);
	recursiveComputing(num-1,num);
	return 0; 
}

// 解法二: 二重循环, 从最小素数开始迭代计算,获取符合题意的两个素数 
int main1()
{
	int j,k,num; // num大偶数 
	int flag; // 标志变量:用于标识是否找到符合要求的素数对
	scanf("%d",&num);
	// 对于大偶数num,分解为两个素数 
	for(j = 2; j < num; j++)  //  找出第一个素数(最小的) for #1 
	{
		if(!isPrime(j)) continue; // 如果j不是素数, 继续下一轮迭代 
		
		flag = 0; // 初始化,未找出素数对 
		for(k = j + 1; k < num; k++) // //  找出第二个素数 for #2
		{
			if(isPrime(k) && j+k == num) // j是最小素数, k必然是最大素数
			{
				printf("%d %d\n",j,k);
				flag = 1; // 找出素数对, 如果没有此设置, 将会输出多组, 例如num=2020时会有多组素数 
				break; // break for #2
			}
		}
		if(flag) break; // break for #1 
	}
	return 0;
}

// 解法三: 不用标志变量版本 
int main2()
{
	int j,k,num; // num大偶数 
	scanf("%d",&num);
	// 对于大偶数num,分解为两个素数 
	for(j = 2; j < num; j++)  //  找出第一个素数(最小的)  for #1 
	{
		if(!isPrime(j)) continue; // 如果j不是素数, 继续下一轮迭代 
		
		for(k = j + 1; k < num; k++) //  找出第二个素数 for #2
		{
			if(isPrime(k) && j+k == num) // j是最小素数, k必然是最大素数 
			{
				printf("%d %d\n",j,k);
				return 0; // 找出素数对,结束主函数。 如果不结束, 将会输出多组 
			}
		}
	}
	return 0;
}

// 解法四: 优化,根据题意找出一组:最小素数+最大素数=偶数 
int main3()
{
	int j,k,num; // num大偶数 
	scanf("%d",&num);
	// 对于大偶数num,分解为两个素数 
	for(j = 2; j < num; j++)  //  找出第一个素数(最小的)  for #1 
	{
		if(!isPrime(j)) continue; // 如果j不是素数, 继续下一轮迭代 
		
		for(k = num-1; k>=2; k--) //  找出第二个素数(最大的) for #2
		{
			// j是最小素数, 判断k是否是最大素数并且二者之和=num 
			if(isPrime(k) && j+k == num) 
			{
				printf("%d %d\n",j,k);
				return 0; // 找出素数对,结束主函数。如果不结束, 将会输出多组 
			}
		}
	}
	return 0;
}

// 解法五: 一重循环, 从最小素数开始迭代计算,获取符合题意的两个素数 
int main4() 
{ 
	int num,i;
	scanf("%d",&num);
	
	for(i = 2;i < num;i++)
	{
		if(isPrime(i)) // i是素数 
		{
			if(isPrime(num-i)) // 如果num-i也是素数, 即满足题意num=i+j 
			{
				printf("%d %d\n",i,num-i); 
				break;
			} 
		}
	}
	return 0; 
} 
\end{lstlisting}

\begin{note}[要点]
	再次体会标志变量的用法及内层循环前的初始化。仔细审题,本题要求一组输出: 最小素数+最大素数=偶数.
	
	如果要求找出最接近的一组素数, 因为$n=\frac{n}{2}+\frac{n}{2}$, 如果$\frac{n}{2}$是素数,即为所求。 否则,所求的两个素数在$\frac{n}{2}$附近, 分别小于和大于$\frac{n}{2}$。
	
	例如: 修改main4()中的\lstinline|if(isPrime(num-i))为if(isPrime(num-i) && num-i<=i)|, 则输出\lstinline|num-i,i;|即可。 
	
	或者: 由$i=\frac{n}{2}\to i=2$迭代, 第一个找到的素数$i$和$num-i$即为所求。
\end{note}

\section{矩阵}
请写一个程序,对于一个m行m列$(2<m<20)$的方阵,求其每一行、每一列及主、辅对角线元素之和,然后按照从大到小的顺序依次输出这些值。

注:主对角线是方阵从左上角到右下角的一条斜线,辅对角线是方阵从右上角到左下角的一条斜线。

输入说明

输入数据的第一行为一个正整数m;

接下来为m行、每行m个整数表示方阵的元素。

输出说明	

从大到小排列的一行整数,每个整数后跟一个空格,最后换行。

输入样例

\begin{tabular}{llll}
	4&&&\\
	15  &~8   &~-2   &~6\\
	31  &~24  &~18  &~71\\
	-3  &~-9  &~27  &~13\\
	17  &~21  &~38  &~69
\end{tabular}

输出样例

159 145 144 135 81 60 44 32 28 27

\begin{lstlisting}
#include <stdio.h>
// 估计方阵行列数 
#define M 20

// input, m是实际方阵行列数 
void input(int matrix[][M], int m)
{
	int i,j; 
	for(i = 0; i < m; i++)
		for(j = 0; j < m; j++) 
			scanf("%d",&matrix[i][j]);
}

// 计算主对角线之和, m是实际方阵行列数 
int main_diagonal(int matrix[][M], int m)
{	
	int i,j,sum = 0; 
	for(i = 0; i < m; i++)
	{
		for(j = 0; j < m; j++) 
		{
			if(i == j) sum += matrix[i][j];   // 主对角线 
		}
	}
	return sum;
}

// 计算副对角线之和, m是实际方阵行列数 
int counter_diagonal(int matrix[][M], int m)
{	
	int i,j,sum = 0;
	for(i = 0; i < m; i++)
	{
		for(j = 0; j < m; j++) 
		{ 
			if(j == m-i-1) sum += matrix[i][j]; // 副对角线之和 
		}
	}
	return sum;
}

// 计算第i行之和, m是实际方阵行列数 
int sumI(int matrix[][M], int m, int i)
{ 
	int j,sum = 0; 
	for(j = 0; j < m; j++) // 遍历列 
	{
		sum += matrix[i][j];  // 第i行之和
	}
	return sum; 
} 

// 计算第j行之和, m是实际方阵行列数 
int sumJ(int matrix[][M], int m, int j)
{ 
	int i,sum = 0; 
	for(i = 0; i < m; i++) // 遍历行 
	{
		sum += matrix[i][j];  // 第j行之和
	}
	return sum;
} 

// 交换两个元素值 
void swap(int *p1, int *p2)
{
	int temp;
	temp = *p1; *p1 = *p2; *p2 = temp;
}

// 选择法排序(降序)
void sorts(int a[], int n)
{
	int i,j,k;
	for(i = 0; i < n-1; i++)
	{
		k = i; // 未经排序的第1个元素设a[k]
		for (j = i+1; j < n; j++)
			if(a[j] > a[k])	k = j;
		if (k != i) swap(&a[i],&a[k]);
	} 
}

int main()
{
	int matrix[M][M],a[2*M+2]; // 以估计行列数,定义数组 
	int i,m; 
	scanf("%d",&m); // 实际方阵行列数 
	
	input(matrix,m); // input 
	
	// 调用各函数,装配数组a
	int n = 0; // 记录数组a的实际长度		
	for(i = 0; i < m; i++)
	{
		a[n++] = sumI(matrix,m,i); // 第i行之和
		a[n++] = sumJ(matrix,m,i); // 第i列之和  
	}
	
	a[n++] = main_diagonal(matrix,m);    // 主对角线之和 
	a[n++] = counter_diagonal(matrix,m); // 副对角线之和  
	
	// 排序数组a 
	sorts(a,n);
	
	// 输出 
	for(i = 0; i < n; i++)
		printf("%d ",a[i]);
	
	printf("\n");
	
	return 0;
}
\end{lstlisting}

\begin{note}[要点]
	\begin{enumerate}
		\item 思路:定义功能单一的函数,实现简单功能,主程序调用各个函数。
		\item 一维数组a[2*M+2]存储相关函数计算结果,排序数组a即是所求。 
		\item 避免过多循环嵌套,不易出错,简化程序设计。 
		\item 但是缺点是在各函数中分别循环遍历方阵,效率低。
		\item 优化方案是不采用独立函数计算,在主函数中一次遍历方阵,计算各值。 
		\item 二维数组表示矩阵, 是常见题型, 必须熟练掌握元素的下标规律及其遍历技巧。
	\end{enumerate}
\end{note}

\section{回文数}
若一个非负整数其各位数字按照正反顺序读完全相同,则称之为回文数,例如12321。
判断输入的整数是否是回文数。若是,则输出该整数各位数字之和,否则输出no。

输入说明	

输入为一个整数n,0<=n<100000000。

输出说明	

若该整数为回文数,则输出整数各位数字之和,否则输出no。

输入样例
	
样例1输入

131

样例2输入

24

输出样例

样例1输出
	
5

样例2输出

no

\begin{lstlisting}
#include <stdio.h>
#include <string.h>

// 思路1:  求该整数的反序组成的整数,如整数1234, 其反序整数即为4321, 如果二者相等即为回文数 
// 判断num是否是回文数,是:返回1;不是,返回0
// 指针参数*sum,返回这个数的各位之和  
int isPalindromic1(int num,int *sum)
{
	int reverse = 0, tmp = num;
	*sum = 0; // 初始化指针内容 
	while(tmp)
	{
		reverse = reverse*10 + tmp%10;
		*sum += tmp%10;
		tmp /= 10;
	}
	if(reverse == num ) return 1;
	else return 0;	
}

// 思路2:  构造数组reverse, 反序存储该整数各位数,按照数组下标,前后数组元素相等则为回文数 
// 判断num是否是回文数,是:返回1;不是,返回0
// 指针参数*sum,返回这个数的各位之和  
int isPalindromic2(int num,int *sum)
{
	int reverse[9], len=0, i=0; // 依题意数组最大长度为9,最多存储9位数。实际长度用len变量表示 
	
	*sum = 0; // 初始换指针内容 
	//构造数组reverse, 反序存储该整数各位数
	while(num)
	{
		reverse[i]=num%10;  
		*sum += num%10;   // 累加各位数字 
		num /= 10;
		len++; // 计算数组实际长度 
		i++; 
	}
	//按照数组下标,前后数组元素相等则为回文数
	for(i=0; i<len/2;i++)
	{
		if(reverse[i] != reverse[len-i-1]) return 0;  // 不是回文数 
	}
	return 1;	// 至此,必然是回文数 
}

// 测试方法1和2的主程序
int main12()
{
	int i,num,sum;
	
	scanf("%d",&num); 
	//if(isPalindromic1(num,&sum)) printf("%d\n",sum);
	if(isPalindromic2(num,&sum)) printf("%d\n",sum);
	else printf("no\n");
	return 0;
} 

// 思路3:  按照字符串处理输入的整数,前后数组元素相等则为回文数 
int main3()
{
	char s[10]; // 留出'\0', 最多存储9位数。实际长度用len变量表示
	int sum=0, len=0,i=0; 
	
	// 以字符串形式接收输入的整数, 末尾自动追加'\0'  
	gets(s);
	
	// 计算len, 或者len=strlen(s), 同时计算各位数字之和 
	for(len=0;s[i]!='\0';) 
	{
		len++;
		sum=sum+s[i]-'0'; // 计算各位数字之和
		i++;  
	}
	
	//按照数组下标,前后数组元素相等则为回文数
	for(i=0; i<len/2;i++)
	{
		if(s[i] != s[len-i-1]) // 不是回文数  
		{
			printf("no\n");
			return 0;  // 主函数结束 
		} 
	}
	// 至此,必然是回文数,包含一位数不进入上面的for循环的情况 
	printf("%d\n",sum);
	return 0;
}

// 思路4:  按照字符串处理输入的整数,前后数组元素相等则为回文数。
// 使用指针操作 
int main()
{
	char s[10]; // 留出'\0', 最多存储9位数。实际长度用len变量表示
	int sum=0; 
	char *p1=s,*p2=s; // 用于正序和反序遍历s数组,初始指向第一个元素 
	
	// 以字符串形式接收输入的整数,末尾自动追加'\0' 
	gets(s);
	
	// 用p2遍历字符串,同时计算各位数字之和 
	for(;*p2!='\0';p2++) 
	{
		sum=sum+(*p2)-'0'; // 计算各位数字之和
	}
	// 至此,p2指向最后一个元素'\0', 我们使它指向最后一个有效元素: 
	p2--; 
	
	//按照数组下标,前后数组元素相等则为回文数
	for(;p1<p2;p1++,p2--)
	{
		if(*p1 != *p2) // 不是回文数 
		{
			printf("no\n");
			return 0;  // 主函数结束 
		} 
	}
	// 至此,必然是回文数 
	printf("%d\n",sum);
	return 0;
}
\end{lstlisting}

\begin{note}[要点]
	\begin{enumerate}
		\item 掌握函数的地址传递方法。
		\item 使用两个指针变量p1, p2, 其中p1指向待查找子串的首字母, 另一个指向末尾, \lstinline|p1++, p2--|; 是判断字符串是否是回文的有效技巧。
	\end{enumerate}
\end{note}

\section{马鞍点}	
若一个矩阵中的某元素在其所在行最小而在其所在列最大,则该元素为矩阵的一个马鞍点。请写一个程序,找出给定矩阵的马鞍点。

输入说明

输入数据第一行只有两个整数m和n(0<m<100,0<n<100),分别表示矩阵的行数和列数;

接下来的m行、每行n个整数表示矩阵元素(矩阵中的元素互不相同), 整数之间以空格间隔。

输出说明

在一行上输出马鞍点的行号、列号(行号和列号从0开始计数)及元素的值(用一个空格分隔), 之后换行;

若不存在马鞍点,则输出一个字符串``no"后换行。

输入样例

\begin{tabular}{p{1cm}p{1cm}p{1cm}p{1cm}l}
	4  &3&\\
	11    &13    &121\\
	407   &72    &88\\
	23    &58    &1\\
	134   &30    &62
\end{tabular} 

输出样例

1 1 72

\begin{lstlisting}
#include <stdio.h>
// 估计的二维数组最大行列数 
#define M 100
#define N 100

// 判断a[row,col]是否是马鞍点, 是: 返回1; 否则返回0
// m,n是二维数组实际行列数 
int compute(int a[][N], int m, int n, int row, int col)
{
	int i,element = a[row][col]; 
	// element在其所在行最小而在其所在列最大 
	for(i = 0; i < n; i++)
		if(a[row][i] < element) return 0; // 不是马鞍点,直接返回0 
	for(i = 0; i < m; i++)
		if(a[i][col] > element) return 0; // 不是马鞍点,直接返回0
	return 1;	// 如果执行至此,肯定是马鞍点,直接返回0 
}

int main()
{
	int matrix[M][N]; // 按照估计的最大行列数定义二维数组 
	int i,j,m,n,flag = 0; 
	
	scanf("%d%d",&m,&n); // 实际行列数 
	
	// input
	for(i = 0; i < m; i++)
		for(j = 0; j < n; j++) 
			scanf("%d",&matrix[i][j]);
	
	// 遍历二维数组, 判断马鞍点		
	for(i = 0; i < m; i++)
	{
		for(j = 0; j < n; j++) 
		{
			if(compute(matrix,m,n,i,j))
			{
				printf("%d %d %d\n",i,j,matrix[i][j]);
				flag = 1;
			}
		}
	}	
	if (!flag) printf("no\n");
	
	return 0;
}
\end{lstlisting}

\begin{note}[要点]
	\begin{enumerate}
		\item 思路:定义函数计算单个元素a[i,j]是否是马鞍点,主程序遍历二维数组,调用此函数。
		\item 避免过多循环嵌套,不易出错,简化程序设计。
		\item 掌握二维数组作为函数参数的定义, 调用方式。
	\end{enumerate}
\end{note}


\section{数字分解排序}
输入一个9位以内的正整数n,按数值从高到低的顺序输出n的各位数字。

输入说明	

一个正整数n(0<n<1000000000)

输出说明	

按数值从高到低的顺序输出n的各位数字,数字之间用空格隔开

输入样例	

564391

输出样例

9 6 5 4 3 1

\begin{lstlisting}
#include <stdio.h>

void order(int a[],int n);  // 排序函数说明

int main()
{
	int i = 0,j,k,n,num[9];
	scanf("%d",&n);
	k=0; // 数字个数 
	while(n) // 构造数组num存储n的各位数字(从最低位到最高位存储)
	{
		num[i++] = n%10;
		n /= 10;
		k++;
	} 

	// 输出验证,调试技巧之一。 
	//for(j = 0; j < k; j++) printf("%d ",num[j]);
	//   printf("\n");
	
	order(num,k);  // 排序函数
	
	// 输出
	for(j = 0; j < k; j++) printf("%d ",num[j]);
	
	printf("\n");
	return 0;
} 

// 冒泡排序函数 
void order(int a[],int n)
{ 
	int i,j,t,flag; 
	for(j = 1; j <= n-1; j++)
	{ 
		flag=0; // 且记! 必须在进入内层循环前初始化。
		for(i = 0; i < n - j; i++)
			if (a[i] < a[i+1]) 
			{ 
				t = a[i]; a[i] = a[i+1]; a[i+1] = t; 
				flag=1;
			}
		if(!flag) break;
	}
}
\end{lstlisting}

\begin{note}[要点]
	其它解法及其变种见课件。
	
	排序函数必须掌握,注意检查数组是否越界问题。
\end{note}

\section{消除类游戏}
消除类游戏是深受大众欢迎的一种游戏,游戏在一个包含有n行m列的游戏棋盘上进行,棋盘的每一行每一列的方格上放着一个有颜色的棋子,当一行或一列上有连续三个或更多的相同颜色的棋子时,这些棋子都被消除。当有多处可以被消除时,这些地方的棋子将同时被消除。

现在给你一个n行m列的棋盘,棋盘中的每一个方格上有一个棋子,请给出经过一次消除后的棋盘。

请注意:一个棋子可能在某一行和某一列同时被消除。

输入格式

输入的第一行包含两个整数n, m,用空格分隔,分别表示棋盘的行数和列数。

接下来n行,每行m个整数,用空格分隔,分别表示每一个方格中的棋子的颜色。颜色使用1至9编号。

输出格式

输出n行,每行m个整数,相邻的整数之间使用一个空格分隔,表示经过一次消除后的棋盘。如果一个方格中的棋子被消除,则对应的方格输出0,否则输出棋子的颜色编号。

样例输入1

4 5

2 2 3 1 2

3 4 5 1 4

2 3 2 1 3

2 2 2 4 4

样例输出1

2 2 3 0 2

3 4 5 0 4

2 3 2 0 3

0 0 0 4 4

样例说明

棋盘中第4列的1和第4行的2可以被消除,其他的方格中的棋子均保留。

样例输入2

4 5

2 2 3 1 2

3 1 1 1 1

2 3 2 1 3

2 2 3 3 3

样例输出2

2 2 3 0 2

3 0 0 0 0

2 3 2 0 3

2 2 0 0 0

样例说明

棋盘中所有的1以及最后一行的3可以被同时消除,其他的方格中的棋子均保留。
评测用例规模与约定

所有的评测用例满足:$1 \le n, m \le 30$。

\section*{基本思路}

\begin{enumerate}
	\item 二维数组checker, 表示$n$行$m$列的棋盘。
	\item 二维数组F, 为$n$行$m$列的标志矩阵。 初始化为全0。经计算, 如果F[i][j]=1, 表示checker[i][j]将被删除(置0)。
	\item 设计按行扫描函数, 当一行上有连续三个或更多的相同颜色的棋子时,这些棋子都被消除。checker每行中符合删除棋子条件,其对应的F矩阵中的元素置1.  
	
	如, 扫描第$i(=0,\dots,n-1)$行, $j=1,\dots,m-2$.
	
	\lstinline|if (checker[i][j-1]==checker[i][j]==checker[i][j+1]) F[i][j-1]=F[i][j]=F[i][j+1]=1;|
	
	\item 设计按列扫描函数, 当一列上有连续三个或更多的相同颜色的棋子时,这些棋子都被消除。checker每行中符合删除棋子条件,其对应的F矩阵中的元素置1.  
	
	如, 扫描第$j(=0,\dots,m-1)$列, $i=1,\dots,n-2$.
	
	\lstinline|if (checker[i-1][j]==checker[i][j]==checker[i+1][j]) F[i-1][j]=F[i][j]=F[i+1][j]=1;|
\end{enumerate}

\begin{lstlisting}
#include <stdio.h>
#include <math.h>
#define M 30 // 估计数组最大长度

/* 按行扫描: 当一行上有连续三个或更多的相同颜色的棋子时,这些棋子都被消除。
checker每行中符合删除棋子条件,其对应的F矩阵中的元素置1. 
checker棋盘, F: 标志矩阵, n: 行数, m: 列数
***********************************************/ 
void scanRow(int checker[][M], int F[][M], int n, int m)
{
	int i,j; 
	for(i=0;i<n;i++) // 行扫描 
	{
		for(j=1;j<m-1;j++) // 列,注意下标  
		{
			// 3个以上同行棋子连续同色,置F删除标志 
			if(checker[i][j-1] == checker[i][j] && checker[i][j] == checker[i][j+1]) 
			{
				F[i][j-1]=F[i][j]=F[i][j+1]=1;
			}
		}
	}	
}

/* 按列扫描: 当一列上有连续三个或更多的相同颜色的棋子时,这些棋子都被消除。
checker每列中符合删除棋子条件,其对应的F矩阵中的元素置1. 
checker棋盘, F: 标志矩阵, n: 行数, m: 列数
***********************************************/ 
void scanCol(int checker[][M], int F[][M], int n, int m)
{
	int i,j; 
	for(j=0;j<m;j++) // 列扫描 
	{
		for(i=1;i<n-1;i++) // 行,注意下标 
		{
			// 3个以上同列棋子连续同色,置F删除标志 
			if(checker[i-1][j] == checker[i][j] && checker[i][j] == checker[i+1][j]) 
			{
				F[i-1][j]=F[i][j]=F[i+1][j]=1;
			}
		}
	}		
}

// 读棋盘, n: 行数, m: 列数
void Read(int checker[][M],int n, int m)
{
	int i,j;
	for(i=0;i<n;i++)
	{
		for(j=0;j<m;j++)
		{
			scanf("%d",&checker[i][j]);
		}
	}
}

// 输出棋盘, n: 行数, m: 列数
void output(int checker[][M],int n, int m)
{
	int i,j;
	for(i=0;i<n;i++) // 行 
	{
		for(j=0;j<m;j++) // 列 
		{
			printf("%d ",checker[i][j]); 
		}
		printf("\n");
	} 
}

// 清除所有可删除的元素, F: 删除标志矩阵, n: 行数, m: 列数 
void del(int checker[][M], int F[][M],int n, int m)
{
	int i,j;
	for(i=0;i<n;i++)
	{
		for(j=0;j<m;j++)
		{
			if(F[i][j]==1) checker[i][j]=0;
		}
	}
}

int main()
{
	// 棋盘,初始化是为了便于测试 
	int checker[M][M]={{2,2,3,1,2},
		{3,4,5,1,4},
		{2,3,2,1,3},
		{2,2,2,4,4}};
	int n=4,m=5,i,j; // n行,m列 
	/************
	int checker[M][M]={{2,2,3,1,2},
		{3,1,1,1,1},
		{2,3,2,1,3},
		{2,2,3,3,3}};
	int n=4,m=5,i,j; // n行,m列 
	*************/
	// 标志矩阵 
	int F[M][M]; // 对应元素为1,则表示删除 
	
	scanf("%d%d",&n,&m);
	// 读棋盘 
	Read(checker,n,m);
	
	// 初始化F 
	for(i=0;i<n;i++)
	for(j=0;j<m;j++)
	F[i][j]=0;
	
	// 扫描行 
	scanRow(checker, F, n, m);
	
	// 扫描列 
	scanCol(checker, F, n, m);
	
	// 测试F   
	//output(F,n,m);
	
	// 清除 
	del(checker, F, n, m);
	
	// 输出棋盘 
	output(checker,n,m);
	
	return 0;
} 
\end{lstlisting}

\section*{附: 按行(列)扫描函数的另一种设计方法}

\begin{lstlisting}
/* 按扫描: 当一行上有连续三个或更多的相同颜色的棋子时,这些棋子都被消除。
checker每行中符合删除棋子条件,其对应的F矩阵中的元素置1. 
checker棋盘, F: 标志矩阵, n: 行数, m: 列数
***********************************************/ 
void scanRow(int checker[][M], int F[][M], int n, int m)
{
	int i,j,k,last,count; // last是最后检查的棋子颜色, count是连续同色棋子个数 
	for(i=0;i<n;i++) // 行扫描 
	{
		last=checker[i][0]; // 每行中第一个元素初始为last
		count=0; 
		for(j=1;j<m;j++) // 列 
		{
			if(last == checker[i][j]) count++;
			else // 不同色
			{
				if(count>=2) // 设置同色棋子的删除标志 
				{
					for(k=j-1;count>=0; k--,count--) F[i][k]=1;//置删除标志 
				}
				last=checker[i][j]; count=0;  
			} 
		}
		// 直到最后一列,仍同色 
		if(count>=2) // 设置同色棋子的删除标志 
		{
			for(k=j-1;count>=0; k--,count--) F[i][k]=1;//置删除标志 
		} 
	}	
}

/* 按列扫描: 当一列上有连续三个或更多的相同颜色的棋子时,这些棋子都被消除。
checker每列中符合删除棋子条件,其对应的F矩阵中的元素置1. 
checker棋盘, F: 标志矩阵, n: 行数, m: 列数
***********************************************/ 
void scanCol(int checker[][M], int F[][M], int n, int m)
{
	int i,j,k,last,count; // last是最后检查的棋子颜色, count是连续同色棋子个数 
	for(j=0;j<m;j++) // 列扫描 
	{
		last=checker[0][j]; // 每列中第一个元素初始为last
		count=0; 
		for(i=1;i<n;i++) // 行 
		{
			if(last == checker[i][j]) count++;
			else // 不同色
			{
				if(count>=2) // 设置同色棋子的删除标志 
				{
					for(k=i-1;count>=0; k--,count--) F[k][j]=1;//置删除标志 
				}
				last=checker[i][j]; count=0;  
			} 
		}
		// 直到最后一行,仍同色 
		if(count>=2) // 设置同色棋子的删除标志 
		{
			for(k=i-1;count>=0; k--,count--) F[k][j]=1;//置删除标志 
		} 
	}	
}
\end{lstlisting}

\begin{note}[要点]
	标志矩阵,体会模块化编程思想,初始化变量的程序调试技巧。
\end{note}

\section{表达式求值}
表达式由两个非负整数x,y和一个运算符op构成,求表达式的值。
这两个整数和运算符的顺序是随机的,可能是``x op y", ``op x y"或者``x y op",例如, ``25 + 3"表示25加3, ``5 30 *"表示5乘以30, "/ 600 15"表示600除以15。

输入说明

输入为一个表达式,表达式由两个非负整数x,y和一个运算符op构成,x,y和op之间以空格分隔,但顺序不确定。

x和y均不大于10000000,op可以是+, -,*, /, \%中的任意一种, 分表表示加法, 减法, 乘法, 除法和求余。

除法按整数除法求值, 输入数据保证除法和求余运算的y值不为0。

输出说明	

输出表达式的值。

输入样例

样例1输入

5 20 *

样例2输入

4 + 8

样例3输入

/ 8 4

输出样例

样例1输出

100

样例2输出

12

样例3输出

2

\begin{lstlisting}
#include <stdio.h>
// 估计字符串最大长度,存储有效字符(N-1)个,预留最后一个字符'\0' 
#define N 20 

// 根据参数,计算表达式的值 
int compute(char op,int x,int y)
{
	int result = -1;
	switch(op)
	{
		case '+': result = x+y; break;
		case '-': result = x-y; break;
		case '*': result = x*y; break;
		case '/': if(y != 0) result = x/y; break;
		case '%': if(y != 0) result = x%y; break;
	}
	return result;
}

// 数字字符串s转为int, 要求s以'\0'结尾 
int strToInt(char *s)// int toInt(char s[]) 
{
	int result=0;
	while(*s) // 等效while(*s != '\0')或while(*s!=0)
	{
		result=result*10+ (*s-'0');
		s++; //移至下一字符 
	}
	return result;
} 

/******************************************************
提取子串函数 
忽略s中空格前缀,复制s中的字符串到subs中,遇空格或'\0'结束
返回subs不含空格。 返回复制后s指针指向(地址) 
要求s和subs以'\0'结尾。
*******************************************************/ 
char* getSubs(char *s, char *subs) 
{
	int start=0; 
	while(*s)
	{
		if(*s==' ') 
		{
			if(start==0) s++; // 忽略s的前缀空格 
			else break; // 是有效字符串后的一个空格 
		}
		else
		{
			start=1; // 开始复制 
			*subs=*s;
			s++;
			subs++;
		}
	}
	*subs='\0'; // 不要忘记结尾符 
	return s;
}

// 解析s, 以空格为分隔符, 分解s为3个字符串 
void parse(char *s,char result[][N])
{
	char *p;
	p=getSubs(s,result[0]);
	p=getSubs(p,result[1]);
	p=getSubs(p,result[2]);
}	

// 如果s是操作符,返回1, 参数op返回该操作符
// 否则, 返回0 
int isOp(char *s, char *op)
{
	if(*s >= '0' && *s <= '9') // 数字
		return 0;
	else // 操作符
	{
		*op=*s;
		return 1;
	}
} 

int main()
{
	char s[N],op;
	char s3[3][N];   
	int x,y;
	gets(s); 
	
	parse(s,s3); // s被分解为3个字符串 
	if(isOp(s3[0],&op)) // op x y
	{
		x=strToInt(s3[1]);
		y=strToInt(s3[2]);
	}
	else if(isOp(s3[1],&op)) // x op y
	{
		x=strToInt(s3[0]);
		y=strToInt(s3[2]);
	}
	else if(isOp(s3[2],&op)) // x y op
	{
		x=strToInt(s3[0]);
		y=strToInt(s3[1]);
	}

	printf("%d\n",compute(op,x,y)); 
	return 0;
}

// 简单实现, 直接读取三个子串, 就不用分解了
int main1() 
{
	char s[N],op;
	char s3[3][N];   
	int x,y;
	scanf("%s%s%s",s3[0],s3[1],s3[2]); // 利用"%s"读字符串遇空格结束特点,直接读取3个字符串。 
	
	// parse(s,s3); // s被分解为3个字符串 
	if(isOp(s3[0],&op)) // op x y
	{
		x=strToInt(s3[1]);
		y=strToInt(s3[2]);
	}
	else if(isOp(s3[1],&op)) // x op y
	{
		x=strToInt(s3[0]);
		y=strToInt(s3[2]);
	}
	else if(isOp(s3[2],&op)) // x y op
	{
		x=strToInt(s3[0]);
		y=strToInt(s3[1]);
	}
	
	printf("%d\n",compute(op,x,y)); 
	return 0;
}
\end{lstlisting}

\begin{note}[要点]
	体会字符串处理, 指针应用, 模块化程序设计思想。
\end{note}

   % 第7次机试练习: 函数, 矩阵, 数组, 字符串, 排序
%%%%%%%%%%%%%%%%%%%%% chapter.tex %%%%%%%%%%%%%%%%%%%%%%%%%%%%%%%%%
%
% sample chapter
%
% Use this file as a template for your own input.
%
%%%%%%%%%%%%%%%%%%%%%%%% Springer-Verlag %%%%%%%%%%%%%%%%%%%%%%%%%%
%\motto{Use the template \emph{chapter.tex} to style the various elements of your chapter content.}
\chapter{第8次机试练习: 函数, 数组, 字符串}

\section{累加和校验}
数据传输中一种常见的校验方式是累加和校验。其实现方式是在一次通讯数据包的最后加入一个字节的校验数据。

这个字节内容为前面数据包中所有数据按字节累加所得结果的最后一个字节。例如: 

要传输的信息为: TEST (ASCII码为0x54,0x45,0x53,0x54)

四个字节的累加和为:0x54+0x45+0x53+0x54=0x140 

校验和为累加和的最后一个字节,即0x40,也就是十进制的64 

现在请设计一个程序计算给出的待传输信息的累加校验和 

输入说明

输入为一个字符串,字符串长度不超过100个字符

输出说明

输出一个十进制整数,表示输入字符串的累加校验和。

输入样例

TEST

输出样例

64

\begin{lstlisting}
#include <stdio.h>
int main()
{
	char s[101];
	int i,check,sum = 0; 
	
	gets(s); // scanf("%s",s);  // 不能完整接收含空格的字符串 
	for(i = 0; s[i] != '\0'; i++) sum += s[i];
	// 计算最后一个字节值。
	check=sum%256; // 一个字节表示256个数(0 -- 255),recall:10进制的个位数计算, sum%10. 
	// 或 check=sum%(0xFF+1); // 0x表示16进制, 一个字节表示0 -- FF 
	// 或 check = (sum/16%16)*16+sum%16; // 最后一个字节就是最后两位16进制数 
	printf("%d\n",check);
	return 0;
}
\end{lstlisting}

\begin{note}[要点]
	一个字节用两位十六进制数表示,16进制分解与10进制是类似的。整数与字符类型可混合运算。
\end{note}

\section{购票系统}
请实现一个铁路购票系统的简单座位分配算法,来处理一节车厢的座位分配。 
假设一节车厢有20排、每一排5个座位。为方便起见,我们用1到100来给所有的座位编号,第一排是1到5号,第二排是6到10号,依次类推,第20排是96到100号。 
购票时,一个人可能购一张或多张票,最多不超过5张。如果这几张票可以安排在同一排编号相邻的座位,则应该安排在编号最小的相邻座位。否则应该安排在编号最小的几个空座位中(不考虑是否相邻)。 

假设初始时车票全部未被购买,现在给了一些购票指令,请你处理这些指令,输出购票结果。 

例如:若一次购买2,5,4,2张票得到的购票结果为: 

1) 购2张票,得到座位1、2。 

2) 购5张票,得到座位6至10。 

3) 购4张票,得到座位11至14。 

4) 购2张票,得到座位3、4。 

输入说明

输入由两行构成。 

第一行包含一个整数n,表示购票指令的数量,$1\le n\le 100$。 

第二行包含n个整数,每个整数p在1到5之间,表示要购入的票数,相邻的两个整数之间使用一个空格分隔,所有购票数量之和不超过100。 

输出说明

输出n行,每行对应一条购票指令的处理结果。即对于购票指令p,按从小到大排序输出p张车票的编号。 

输入样例

4 

2 5 4 2 

输出样例

1 2 

6 7 8 9 10 

11 12 13 14 

3 4 

\begin{lstlisting}
#include <stdio.h>
// 估计二维数组最大行数和列数 
#define M 20
#define N 5 

//初始化, 二维数组seat代表座位,元素为0值,表示座位未售出, 元素为1代表对应座位已输出。m,n是seat实际的行数和列数 
void init(int seat[][N],int m,int n)
{
	int i,j;
	for(i = 0; i < m; i++) // 排(行)
		for(j = 0; j < n; j++) // 列
			seat[i][j] = 0;	 
}

//根据购票数,查询本购票指令,首座位对应的行索引(从0开始) 
//二维数组seat代表座位,元素为0值,表示座位未售出, 元素为1代表对应座位已输出。m,n是seat实际的行数和列数,tickets购票数  
int search(int seat[][N],int m,int n,int tickets)
{
	int i,j;
	int zeroNum; // 表示某行未售出的座位数
	int row = 0; 
	for(i = 0; i < m; i++) // 排(行)
	{
		zeroNum = 0;
		for(j = 0; j < n; j++) // 列
		{
			if (seat[i][j] == 0) zeroNum++;
		}
		// 如果这几张票可以安排在同一排编号相邻的座位,则应该安排在编号最小的相邻座位
		if(zeroNum >= tickets) // 在一排中
		{
			row = i;
			break;  // 保证编号最小
		}
	}
	return row;  // 首座位对应的行索引(从0开始) 
}

//售票 
//二维数组seat代表座位,元素为0值,表示座位未售出, 元素为1代表对应座位已输出。m,n是seat实际的行数和列数,tickets购票数  
void salse(int seat[][N],int m,int n,int tickets)
{
	int i,j,count = 0, flag = 0;
	int row = search(seat,m,n,tickets); // 获取本购票指令,首座位对应的行索引(从0开始)  
	for(i = row; i < m; i++)  // 排(行)
	{
		for(j = 0; j < n; j++) // 列
		{
			if (seat[i][j] == 0) 
			{ 
				seat[i][j] = 1;
				printf("%d ",i*n+j+1); // 打印座位编号 
				count++;
				if(count >= tickets)
				{
					flag = 1; break;
				}
			} 
		}
		if(flag == 1) break;  
	}
	printf("\n"); // 换行 
}

int main()
{ 
	int seat[M][N];  // 二维数组seat代表座位,元素为0值,表示座位未售出, 元素为1代表对应座位已输出。
	int i, m = 20, n = 5; // m,n是seat实际的行数和列数
	int pNum,tickets;     // 售票指令数和票数 
	
	// 初始化 
	init(seat,m,n);
	
	scanf("%d",&pNum);
	// 输入同时,处理售票。因此不需要使用数组存储pNum和tickets 
	for(i = 0; i < pNum; i++)
	{
		scanf("%d",&tickets);
		salse(seat,m,n,tickets);
	}
	
	return 0;
}
\end{lstlisting}

\begin{note}[要点]
	\begin{enumerate}
		\item 思路:使用二维数组模拟售票座位,元素为0值,表示座位未售出, 元素为1代表对应座位已输出。
		\item 定义独立函数,根据售票指令查询待售座位;定义独立函数,模拟售票操作。 
		\item 注意:虽然要求排序输出,但是选票过程就是按照从小到大的顺序选座位,因此并不需要编写排序函数。 
		\item 体会模块化程序设计的优势。
	\end{enumerate}
\end{note}

\section{字符统计}
给出一个字符C和一行文字S,统计在这行文字S中字符C出现的次数。

程序还需要支持大小写敏感选项:

当选项打开时,表示同一个字母的大写和小写看作不同的字符;

当选项关闭时,表示同一个字母的大写和小写看作相同的字符。

输入说明	

输入数据由两行构成。

第一行包含一个字符C和一个数字n。字符C为大小或小写英文字母。数字n表示大小写敏感选项,当数字n为0时表示大小写不敏感,当数字n为1时表示大小写敏感。字符C和数字n之间用空格分隔。

第二行为一个字符串S,字符串由大小写英文字母组成,不含空格和其他字符。字符串S长度不超过100。

输出说明
	
输出字符C在字符串S中出现的次数。

输入样例

样例1输入

L 1

HELLOWorld

样例2输入

L 0

HELLOWorld

输出样例

样例1输出	

2

样例2输出

3

\section*{测试接收输入数据的正确性, 是本题的重要要点。}

\begin{lstlisting}
#include <stdio.h>
// 估计字符串长度,实际容纳(N-1)个有效字符,预留最后一个字符'\0'
#define N 101 

// 使用gets函数接收两行字符串最安全,可正确消费每行的回车键.
int main()
{
	// 第一行字符串, 包括中间的空格共3个字符, 再加上字符串结尾字符'\0' 
	char info[4]; 
	char s[N]; // 第二行字符串 
	int i, count=0;
	
	// 第一行,info[0]是待统计的字符, info[1]空格, info[2]='1'或'0'表示大小写敏感信息
	gets(info);  
	
	// 大小写不敏感, 统一转为大写比较 
	if(info[2]=='0' && info[0]>='a' && info[0]<='z') 
	{
		info[0] = info[0]-32; // 小写转大写 
	} 
	
	gets(s); // 读取第二行字符串 
	for(i=0; s[i]!='\0'; i++) 
	{
		if(info[2]=='0') // 大小写不敏感, 统一转为大写比较 
		{
			if(s[i]>='a' && s[i]<='z') s[i] = s[i] - 32;  // 小写转大写 
		}
		if(s[i] == info[0]) count++; // 计数 
	}
	printf("%d\n",count);
	return 0;
} 

// 测试接收输入数据的正确性, 是本题的重要要点。}
// scanf("%c%d",--)形式不会消费行尾的回车键 
int main1()
{
	char S[N],C;
	int n, i, count=0;
	
	/************ 测试, 这样的输入得不到正确的值 
	scanf("%c%d",&C,&n); // 回车键保留在输入缓冲区, 导致下面的gets(S)仅得到一个'\0' 
	gets(S); // 遇回车结束.
	printf("===%c,%d,%s\n",C,n,S); // 测试输入, 错误 
	*************/ 
	
	/************ 测试, 这样的输入得不到正确的值
	scanf("%c%d%s",&C,&n,S); // S中如果有空格则接收不完整 
	printf("===%c,%d,%s\n",C,n,S); // 测试输入, 错误 
	*************/ 
	
	// 测试以下输入正确 
	C=getchar();    // 读取C 
	scanf("%d",&n); // 读取n 
	getchar();      // 消费回车键 
	gets(S);        // 读取S 
	//printf("===%c,%d,%s\n",C,n,S); // 测试输入, 正确 
	
	
	if(n == 0 && C >= 'a' && C <= 'z') // 大小写不敏感, 统一转为大写比较 
	{
		C = C-32; // 小写转大写 
	} 
	
	for(i=0; S[i] != '\0'; i++) 
	{
		if(n==0) // 大小写不敏感, 统一转为大写比较 
		{
			if(S[i]>='a' && S[i]<='z') S[i] = S[i] - 32;  // 小写转大写 
		}
		if(S[i] == C) count++; // 计数 
	}
	printf("%d\n",count);
	return 0;
} 
\end{lstlisting}

\begin{note}[要点]
	\begin{itemize}
		\item 测试输入变量的正确值, 是调试程序的首要点, 尤其是数字, 字符, 字符串混合形式的输入。
		\item 使用gets函数接收两行字符串最安全。
	\end{itemize}
	
\end{note}

\section{目录操作}
在操作系统中,文件系统一般采用层次化的组织形式,由目录(或者文件夹)和文件构成,形成一棵树的形状。

有一个特殊的目录被称为根目录,是整个文件系统形成的这棵树的根节点,在类Linux系统中用一个单独的 ``/"符号表示。

因此一个目录的绝对路径可以表示为``/d2/d3"这样的形式。

当前目录表示用户目前正在工作的目录。为了切换到文件系统中的某个目录,可以使用``cd"命令。

假设当前目录为``/d2/d3", 下图给出了cd命令的几种形式,以及执行命令之后的当前目录。

\begin{tabular}{|l|l|l|l|}
	\hline 
	\textbf{当前目录} & \textbf{目录切换命令} & \textbf{命令含义} & \textbf{执行命令后的当前目录} \\ 
	\hline 
	/d2/d3 & cd / & 切换到根目录 & / \\ 
	\hline 
	/d2/d3 & cd .. & 切换到当前目录的上级目录 & /d2 \\ 
	\hline 
	/d2/d3 & cd d4/d5 & 切换到当前目录下的某个子目录 & /d2/d3/d4/d5 \\ 
	\hline 
	/d2/d3 & cd /d1/d5 & 切换到某个绝对路径所指的目录 & /d1/d5 \\ 
	\hline 
\end{tabular} 

现在给出初始时的当前目录和一系列目录操作指令,请给出操作完成后的当前目录。

输入说明	

第一行包含一个字符串,表示当前目录。

后续若干行,每行包含一个字符串,表示需要进行的目录切换命令。

最后一行为pwd命令,表示输出当前目录

注意:

1.	所有目录的名字只包含小写字母和数字,cd命令和pwd命令也都是小写。最长目录长度不超过200个字符。

2.	当前目录已经是根目录时,cd .. 和cd /不会产生任何作用

输出说明

输出一个字符串,表示经过一系列目录操作后的当前目录

输入样例	

/d2/d3/d7

cd ..

cd /

cd /d1/d6

cd d4/d5

pwd

输出样例

/d1/d6/d4/d5

\begin{lstlisting}
#include <stdio.h>
#include <string.h>
// 估计字符串最大长度,存储有效字符(N-1)个,预留最后一个字符'\0' 
#define N 201 

/******************************************************
提取子串函数 
忽略s中空格前缀,复制s中的字符串到subs中,遇空格或'\0'结束
返回subs不含空格。 返回复制后s指针指向(地址) 
要求s和subs以'\0'结尾。
*******************************************************/ 
char* getSubs(char *s, char *subs) 
{
	int start=0; 
	while(*s)
	{
		if(*s==' ') 
		{
			if(start==0) s++; // 忽略s的前缀空格 
			else break;       // 是有效字符串后的一个空格 
		}
		else
		{
			start=1; // 开始复制 
			*subs=*s;
			s++;
			subs++;
		}
	}
	*subs='\0'; // 不要忘记结尾符 
	return s;
}

// cd .. 命令, 参数pwd返回当前目录的上一级目录
// pwd="/a/b/c" 或 "/a" 
void dotdot(char *pwd)
{
	int i,len;
	if(strcmp(pwd,"/")==0) return; // 如果已经是根目录, 无动作 
	len=strlen(pwd); 
	for(i=len-1;i>=0;i--)
	{
		if(pwd[i] == '/')
		{
			if(i==0) strcpy(pwd,"/"); // "/a" --> pwd="/"  
			else pwd[i]='\0';         // "/a/b/c"  --> pwd="/a/b" 
			break;
		}
	} 
}

// 解析command, 参数pwd返回当前目录 
void parse(char *command, char *pwd)
{
	char *p, str[N]; // str是cd后的字符串(不含空格) 
	p=getSubs(command, str); // 获得"cd" 
	getSubs(p,str); // cd后的字符串 
	// cd / 或 cd /a/b/c 或 cd a/b/c 
	if(strcmp(str,"/")==0) strcpy(pwd,"/");    // cd /
	else if(strcmp(str,"..")==0)  dotdot(pwd); // cd ..
	else if(str[0]=='/') strcpy(pwd,str);      // cd /a/b/c
	else // cd a/b/c, pwd为pwd/str 
	{
		// 如果pwd最后一个字符是非'/' 
		if(pwd[strlen(pwd)-1]!='/') strcat(pwd,"/"); 
		strcat(pwd,str);
	}
} 


int main()
{
	char pwd[N]; // 当前目录
	char command[N]; // cd ..  或 cd / 或 cd /a/b/c
	
	gets(pwd);   // 第一行, 当前目录 
	while(1) // 逐行获得命令, 并处理
	{ 
		gets(command); 
		if(strcmp(command,"pwd")==0) break;
		parse(command,pwd);
	} 
	puts(pwd);
	return 0;
}
\end{lstlisting}

\begin{note}[要点]
	\begin{itemize}
		\item 字符串处理, 指针函数参数的使用, 模块化程序设计。
		\item 可以使用\lstinline|char s1[N], s2[N]; scanf("%s%s",s1,s2)|遇空结束特点, 获取cd及其参数两部分字符串。
	\end{itemize}
\end{note}

\section{处理字符串}
从键盘输入一个字符串,将该字符串按下述要求处理后输出: 

将ASCII码大于原首字符的各字符按原来相互间的顺序关系集中在原首字符的左边,
将ASCII码小于等于原首字符的各字符按升序集中在原首字符的右边。

输入说明	

输入一行字符串,字符串c不长度超过100.

输出说明	

输出处理后的一行字符串

输入样例

\lstinline[mathescape=false]|aQWERsdfg7654!@#$hjklTUIO3210X98aY|

输出样例

\lstinline[mathescape=false]|sdfghjkla!#$0123456789@EIOQRTUWXYa|


\begin{lstlisting}
#include <stdio.h>
#include <string.h> // str前缀的字符串处理函数需要此头文件
// 估计字符串长度,实际容纳(N-1)个有效字符,预留最后一个字符'\0'  
#define N 101

// 排序函数(升序) 
void sort(char a[],int n)
{ 
	int i,j,t;
	// 冒泡排序
	for(j = 1; j <= n-1; j++) 
		for(i = 0; i < n - j; i++)
			if (a[i] > a[i+1]) 
				{ t = a[i]; a[i] = a[i+1]; a[i+1] = t; }
}


int main()
{
	char s[N],left[N],right[N];
	char *p = s;
	int i = 0,j = 0;
	gets(s); // 不能使用scanf("%s",lines),因其遇空格结束
	// 使用指针处理较方便 
	while(*p)
	{
		if(*p > s[0]) left[i++] = *p;
		else if(p != s) right[j++] = *p; // 原首字符不加入到right 
		p++; 
	}
	left[i++] = s[0]; // 原首字符加入到left中 
	left[i] = '\0';   // left,right添加结束字符 
	right[j] = '\0';
	sort(right,strlen(right)); // 排序right 
	
	// 可以不必进行字符串连接操作,printf("%s%s\n",left,right); 亦可。
	strcat(left,right); // 连接两个字符串 
	puts(left);
	return 0;
} 
\end{lstlisting}

\begin{note}[要点]
	\begin{itemize}
		\item 独立排序函数, 便于模块化程序设计。
		\item 借助指针进行字符串处理是常见技巧, 应深刻领会。
		\item 且记字符串结尾字符\lstinline|'\0'|.
	\end{itemize}
\end{note}

   % 第8次机试练习: 数组,排序,字符串练习
%%%%%%%%%%%%%%%%%%%%% chapter.tex %%%%%%%%%%%%%%%%%%%%%%%%%%%%%%%%%
%
% sample chapter
%
% Use this file as a template for your own input.
%
%%%%%%%%%%%%%%%%%%%%%%%% Springer-Verlag %%%%%%%%%%%%%%%%%%%%%%%%%%
%\motto{Use the template \emph{chapter.tex} to style the various elements of your chapter content.}
\chapter{第9次机试练习: 结构体}

\section{复试筛选}	
考研初试成绩公布后需要对m个学生的成绩进行排序,筛选出可以进入复试的前n名学生。
排序规则为首先按照总分排序,总分相同则按英语单科成绩排序,总分和英语成绩也相同时考号小者排在前面。

现给出这m个学生的考研初试成绩,请筛选出可以进入复试的n名学生并按照排名从高到低的顺序依次输出。

输入说明	

输入为m+1行,第一行为两个整数m和n,分别表示总人数和可以进入复试人数,m和n之间用空格分隔,0<n<m<200。

接下来为m行数据,每行包括三项信息,分别表示一个学生的考号(长度不超过20的字符串)、总成绩(小于500的整数)和英语单科成绩(小于100的整数), 这三项之间用空格分隔。

输出说明

按排名从高到低的顺序输出进入复试的这n名学生的信息。

输入样例	

5 3

XD20160001 330 65

XD20160002 330 70

XD20160003 340 60

XD20160004 310 80

XD20160005 360 75

输出样例	

XD20160005 360 75

XD20160003 340 60

XD20160002 330 70

\begin{lstlisting}
// 思路:定义结构体类型和结构体数组,对结构体数组进行排序。 
#include <stdio.h>
#include <string.h>

// 估计结构体数组最大长度 
#define N 200

struct Student 
{
	char no[20];   // 考号 
	int  total;    // 总成绩 
	int  english;  // 英语成绩  
};

// 输入m个考生信息
void input(struct Student *stus, int m)
{
	int i;
	for(i=0;i < m;i++) 
		scanf("%s%d%d",	stus[i].no,&stus[i].total,&stus[i].english); 
}

// 输入n个考生信息
void print(struct Student *stus, int n)
{
	int i;
	for(i=0;i<n;i++) 
		printf("%s %d %d\n",stus[i].no,stus[i].total,stus[i].english); 
}

// 交换两个结构体对象 
void swap(struct Student *p1, struct Student *p2)
{
	struct Student temp;
	temp = *p1; *p1 = *p2; *p2 = temp;
}

/************************************************* 
 选择法排序(降序)
 排序规则为首先按照总分排序,总分相同则按英语单科成绩排序
**************************************************/ 
void sorts(struct Student a[], int n)
{
	int i,j,k;
	for(i = 0; i < n - 1; i++)
	{
		k = i;
		for (j = i+1; j < n; j++)
		{
			// 条件判断语句:善用&&、||运算,简化if else结构 
			// 注意(条件1||条件2||$\cdots$)的截断语义
			if (a[j].total>a[k].total || (a[j].total==a[k].total &&  a[j].english>a[k].english)
			|| (a[j].total==a[k].total &&  a[j].english==a[k].english && strcmp(a[j].no,a[k].no)<0))  
			k = j;
		}
		if (k != i) swap(&a[i],&a[k]);
	} 
}

int main()
{
	struct Student stus[N]; 
	int m,n,i;
	scanf("%d%d",&m,&n);
	input(stus,m); 
	sorts(stus,m);
	print(stus,n);
	return 0;
}
\end{lstlisting}

\begin{note}[要点]
	\begin{itemize}
		\item 条件判断语句:善用 \lstinline$&&, ||$ 运算,简化\lstinline|if else|结构
		\item \lstinline!if (条件1||条件2||条件3||$\cdots$)!, 从左到右计算条件表达式, 如果任一条件为\lstinline|true|, 不必进行后续条件的判断。称为``或''条件的截断语义。
		\item 本例是结构体数组排序, 模块化程序设计的典型案例,  详见课件。 
	\end{itemize}
\end{note}

\section{画图}
在一个定义了直角坐标系的纸上,画一个(x1,y1)到(x2,y2)的矩形,指将横坐标范围从x1到x2,纵坐标范围从y1到y2之间的区域涂上颜色。    
下图给出了一个画了两个矩形的例子。第一个矩形是(1,1) 到(4, 4),用绿色和紫色表示。第二个矩形是(2, 3)到(6, 5),用蓝色和紫色表示。

\begin{center}
	\includegraphics*[scale=0.4]{paint}
\end{center}

图中,一共有15个单位的面积被涂上颜色,其中紫色部分被涂了两次,但在计算面积时只计算一次。在实际的涂色过程中,所有的矩形 都涂成统一的颜色,图中显示不同颜色仅为说明方便。给出所有要画的矩形,请问总共有多少个单位的面积被涂上颜色。

输入说明 

输入的第一行包含一个整数n,表示要画的矩形的个数,1<=n<=100   

接下来n行,每行4个非负整数,分别表示要画的矩形的左下角的横坐标与纵坐标,以及右上角的横坐标与纵坐标。0<=横坐标、纵坐标<=100。

输出说明	

输出一个整数,表示有多少个单位的面积被涂上颜色。

输入样例

2 

1 1 4 4 

2 3 6 5 

输出样例

15

\begin{lstlisting}
#define N 100
//定义矩形结构体 
struct Rec
{
	int leftBottomX;
	int leftBottomY;
	int rightTopX;
	int rightTopY;
};

// 返回最大矩形
struct Rec Largest(struct Rec res[], int n)
{
	struct Rec largest=res[0];
	int i;
	for(i=1;i<n;i++)
	{
		if(res[i].leftBottomX < largest.leftBottomX) largest.leftBottomX=res[i].leftBottomX;
		if(res[i].leftBottomY < largest.leftBottomY) largest.leftBottomY=res[i].leftBottomY;
		if(res[i].rightTopX > largest.rightTopX) largest.rightTopX=res[i].rightTopX;
		if(res[i].rightTopY > largest.rightTopY) largest.rightTopY=res[i].rightTopY;
	}
	return largest; 
} 

// rec区域, grid数组元素置1 
void grid1(struct Rec rec, int grid[][N])
{
	int i,j;
	for(i=rec.leftBottomY;i<rec.rightTopY;i++)
	{
		for(j=rec.leftBottomX;j<rec.rightTopX;j++)
			grid[i][j]=1;
	} 
}

// largest区域, grid数组元素初始化为0 
void grid0(struct Rec largest, int grid[][N])
{
	int i,j;
	for(i=largest.leftBottomY;i<largest.rightTopY;i++)
	{
		for(j=largest.leftBottomX;j<largest.rightTopX;j++)
			grid[i][j]=0;
	} 
}

// largest区域, 返回grid数组元素为1的单元数量,即被覆盖的单位面积数 
int gridNum(struct Rec largest, int grid[][N])
{
	int i,j,num=0;
	for(i=largest.leftBottomY;i<largest.rightTopY;i++)
	{
		for(j=largest.leftBottomX;j<largest.rightTopX;j++)
			if(grid[i][j]) num++;
	} 
	return num;
}

void input(struct Rec *recs, int n)
{
	int i;
	for(i=0;i<n;i++) 
		scanf("%d%d%d%d",&recs[i].leftBottomX,&recs[i].leftBottomY,
	&recs[i].rightTopX,&recs[i].rightTopY);
} 

void print(struct Rec *recs, int n)
{
	int i;
	for(i=0;i<n;i++) 
		printf("%d,%d,%d,%d\n",recs[i].leftBottomX,recs[i].leftBottomY,
	recs[i].rightTopX,recs[i].rightTopY);
} 

int main()
{
	int grid[N][N],n,i,num;
	struct Rec recs[N],largest;
	scanf("%d",&n);
	input(recs,n);
	// print(recs,n); // 检查输入 
	largest=Largest(recs,n); // 计算最大矩形 
	// print(&largest,1); // 查看最大矩形是否正确 
	grid0(largest,grid); // 初始化 
	for(i=0;i<n;i++)
	{
		grid1(recs[i],grid); // 置1 
	}
	printf("%d\n",gridNum(largest,grid)); // 统计输出 
	return 0;
}
\end{lstlisting}

\begin{note}[要点]
详见课件。注意使用标志矩阵表示最大矩阵的使用技巧。 进一步练习模块化程序设计方法。
\end{note}


   % 第9次机试练习: 函数, 数组, 字符串, 结构体

%%%%%%%%%%%%%%%%%%%%%%%%%%%%%%%%%%%%%%%%%%%%%%%%%%%%%%%%%%%%%%%%%%%%%%
\end{document}