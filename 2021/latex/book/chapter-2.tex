%%%%%%%%%%%%%%%%%%%%% chapter.tex %%%%%%%%%%%%%%%%%%%%%%%%%%%%%%%%%
%
% sample chapter
%
% Use this file as a template for your own input.
%
%%%%%%%%%%%%%%%%%%%%%%%% Springer-Verlag %%%%%%%%%%%%%%%%%%%%%%%%%%
%\motto{Use the template \emph{chapter.tex} to style the various elements of your chapter content.}
\chapter{第2次机试练习: 选择与循环语句练习}

\section*{--- 特别提示 ---}

\begin{note}[不该再次发生的常见错误,输入输出格式转换符不对应, 导致的严重错误]
	\begin{lstlisting}
	int a; float b; double c; char d;
	scanf("%d",a); // 遗忘变量前的取地址符$\&$
	scanf("%d\n",&a); // 多余'\n', 导致不能正常输入
	scanf("%d%f%lf%c",&a,&b,&c,%d); // 正确对应关系
	scanf("%d%c%f",&a,&d,&b); // 正确对应关系
	printf("%d,%f,%lf,%c",a,b,c,d); // 正确对应关系
	\end{lstlisting}
\end{note}

\begin{note}[不该再次发生的常见错误, 由`;'引发的悲剧]
	\begin{lstlisting}
	if();
	{
		...
	}                 
	
	while();
	{
		...
	} 
	
	for(;;);
	{
		...
	}                  
	\end{lstlisting}
\end{note}

\newpage
\begin{note}[用C语言关系表达式准确表达数学含义]
\begin{lstlisting}
int a;
if(110<=a<=210) // 错误
{  }
if(110<=a && a<=210) // 正确 
{  }
if(a>=110 && a<=210) // 正确 
{  }
\end{lstlisting}	
\end{note}

\begin{note}[学习体会编程技巧]
	\begin{itemize}
		\item 使用\lstinline|printf()|语句, 追踪程序执行细节, 查找出错原因。
		\item 对于条件结构,循环结构,首先书写整体结构,再添加细节,避免低级错误。
		\item 提倡一题多解, 举一反三, 体会编程技巧。	
	\end{itemize}
\end{note}


\section{四则运算}
输入两个整数和一个四则运算符,根据运算符计算并输出其运算结果(和、差、积、商、余之一)。注意做整除及求余运算时,除数不能为零。

输入说明:

使用scanf()函数输入两个整数和一个运算符,格式见输入样例。

输出说明:

输出使用printf()函数,格式见输出样例。

输入样例:

\lstinline|5%2|

输出样例:

\lstinline|5%2=1|

\begin{lstlisting}
#include<stdio.h>    
int main()                   
{  
	int a,b;
	char op;
	scanf("%d%c%d",&a,&op,&b);
	switch(op)
	{
		case '+': printf("%d%c%d=%d\n",a,op,b,a+b); break;
		case '-': printf("%d%c%d=%d\n",a,op,b,a-b); break;
		case '*': printf("%d%c%d=%d\n",a,op,b,a*b); break;
		// 注意分母为0时, 不会正确运算/,%
		case '/': if (b!=0) printf("%d%c%d=%d\n",a,op,b,a/b); break;
		case '%': if (b!=0) printf("%d%c%d=%d\n",a,op,b,a%b); break;
	}
	return 0;           
}                 
\end{lstlisting}

\begin{note}[printf双引号中的\%输出, \%\%表示输出\%]
	\begin{lstlisting}
		int a,b;
		char op;
		printf("%d%%%d=%d\n",a,b,a%b);
		// 或当 op='%'时
		printf("%d%c%d=%d\n",a,op,b,a%b);
	\end{lstlisting}
\end{note}
	

\section{数位输出}
输入一个5位整数,求出其各数位数值,并按照从高位到低位的顺序输出,如:输入12345,输出为1 2 3 4 5。

输入说明:

输入一个五位正整数。

输出说明:

按数位从高到低依次输出,各数位之间以一个空格相分隔。

输入样例:

96237

输出样例:

9 6 2 3 7
\newpage
\begin{lstlisting}
#include<stdio.h> 
/******************************************************
 5位整数已知. 首先用10000除以整数a(分子), 得到分子最高位。 
 改变分子分母, 循环迭代, 依次获得分子的最高位。
*******************************************************/        
int main1()              
{  
	int a,b=10000,i=5; // i记录整数a的初始位数
	scanf("%d",&a);
	while(i>=1)
	{
		if (i==1) printf("%d\n",a/b); // 输出当前a的最高位
		else printf("%d ",a/b);
		a = a-a/b*b;  // 去除当前a的最高位,准备下轮迭代的分子a
		b/=10; // b=b/10, 准备下轮迭代的分母b
		i--;
	} 
	return 0;           
}  
             
/******************************************************
 假设不知整数a的位数。
 除10取余,迭代循环,可方便获取整数a的个位,十位,百位,千位, ... 
 利用数组存储个位,十位,百位,千位, ... 最后反序输出即是所求。 
*******************************************************/        
int main2()              
{  
	int a, tmp[100]; // tmp数组存储100(估计的最大值)个整数,用tmp[0],tmp[1],tmp[2],...读写各个整数。
	int i=0, j;         // i: 记录整数a的位数 
	scanf("%d",&a);
	if(a==0) // 考虑整数0的特殊情况,直接输出即可。
	{
		printf("%d\n",a); 
	}
	else // 因为循环语句判断a是否为0, 因此要有上述判断才能考虑到所有可能情况的发生 
	{
		while(a!=0) // 迭代逆序求出整数a的各位数字 
		{
			tmp[i]=a%10; // 存储本轮循环a的末位数
			//printf("调式查看tmp[i]=%d\n",tmp[i]); // 提交时,别忘了注释或删除调试语句 
			a=a/10;      // 改变分子, 准备下轮循环
			i++;         // 位数递增 
		}
		//printf("调式查看i=%d\n",i); 
		//  逆序输出tmp, 此时的i是整数a的位数, 注意tmp的下标从i-1开始到下标0结束。 
		for(j=i-1;j>=0;j--)
		{
			printf("%d ",tmp[j]);
		} 
	}
	return 0;           
}    

/******************************************************
 假设不知整数a的位数。
 利用递归函数求解, a==0的情况在函数外处理输出较方便。
 因此该函数仅考虑a!=0的情况。 
*******************************************************/    
void output(int a)              
{  
	if(a!=0) // 如果考虑a==0的情况, 不好判断是初始a=0还是迭代后a=0的情况。这里考虑后者。前者的处理留给调用它的程序。 
	{
		// '栈'是一种'先进后出'的数据结构 
		output(a/10); // 递归调用, 函数参数会自动存储在系统维护的'栈'中。 
		printf("%d ",a%10);  // 从内部存储'栈'中,依次弹出各位数,输出之。 
	}
	else // a==0 ,可省略else语句,隐含结束递归调用
	{
		return; // 函数结束, 注意本函数无返回值,因此return后无表达式。 
	} 
} 

int main()
{
	int a;
	scanf("%d",&a);
	if(a==0) // 考虑整数0的特殊情况,直接输出即可。
	{
		printf("%d\n",a); 
	}
	else
	{
		output(a); // 函数调用, 完成逆序输出。 
	} 
	return 0; 
}
\end{lstlisting}

\begin{figure}[H]
\centering
\caption{递归函数\lstinline|void output(int a)|中系统内部维护的`栈'结构示意图}	
\begin{tikzpicture}
\node[] (tab) {
\begin{tabular}{|c|c|c|}
\hline
参数a&output(a)&递归调用\lstinline|output(a/10); printf("%d ",a%10);|\\
\hline
0&output(0)&return; 结束递归, 开始出栈\\
\hline
1&output(1)&\lstinline|output(0); printf("%d ",1);|\\
\hline
12&output(12)&\lstinline|output(1); printf("%d ",2);|\\
\hline
123&output(123)&\lstinline|output(12); printf("%d ",3);|\\
\hline
1234&output(1234)&\lstinline|output(123); printf("%d ",4);|\\
\hline
\hline
12345&output(12345)&\lstinline|output(1234); printf("%d ",5);|\\
\hline
\end{tabular}
};
\path[->,very thick]
%($(tab.north west)+(-0.5cm,0)$) edge[] node[left] {入栈} +(0,-4cm)
($(tab.north west)+(-0.5cm,0)$) edge[red] node[left] {入栈} ($(tab.south west)+(-0.5cm,0)$)
($(tab.south east)+(0.5cm,0)$) edge[blue] node[right] {出栈} ($(tab.north east)+(0.5cm,0)$)
; 
\end{tikzpicture}
\end{figure}

\begin{note}[知识点]
	\begin{enumerate}
		\item 体会除10取余, 迭代循环的整数分解技巧;
		\item 第一种解法的b=1000初值是可计算的, 这样就可扩充此解法为任意位的整数a。
		\begin{lstlisting}[frame=none]
		// 因为a要在main1()函数的while循环中使用。
		// 因此, 定义临时变量, 存储a的值, 用于计算b的初值。
		int tmp; 
		b=1; tmp=a;
		while(tmp!=0)
		{
			b=b*10;
			tmp=tmp/10;
		}
		\end{lstlisting}
		\item 预习数组使用技巧;
		\item 预习函数定义及调用;
		\item 预习递归函数的定义, 体会系统维护的内部存储'栈'的数据存储特点。
	\end{enumerate}
\end{note}

\section{阶梯电价计费}
电价分三个档次,[0,110]度电,每度电0.5元;(110,210]度电,超出110部分每度电0.55元,超过210度电,超出210部分每度电0.70元,给出一个家庭一月用电量,请计算出应缴的电费(四舍五入,保留小数点后两位小数)。

输入说明:

输入数据为一个正实数,表示一月用电量

输出说明:

输出应缴电费,四舍五入保留2位小数。

输入样例:

输入样例1

100

输入样例2

200

输入样例3

329.75

输出样例:

输出样例1

50.00

输出样例2

104.50

输出样例3

193.83

\begin{lstlisting}
#include <stdio.h>
int main()
{
	float sum,u1=0.5,u2=0.55,u3=0.70; // 用电量,每度电单价
	float fee = 0; // 应缴电费
	
	scanf("%f",&sum);
	
	if (sum > 210) 
	{
		fee = (sum-210)*u3;
		sum = 210;
	}
	if (sum > 110)
	{
		fee += (sum-110)*u2; // fee=fee+(sum-110)*u2;
		sum = 110;
	}
	fee += sum*u1;
	
	printf("%.2f\n",fee);
	return 0;
} 

int main2()  // 另解
{
	float sum,u1=0.5,u2=0.55,u3=0.70; // 用电量,每度电单价
	float fee = 0; // 应缴电费 
	scanf("%f",&sum);
	
	if (sum >= 210)
		fee = 110*u1 + (210-110)*u2 + (sum-210)*u3;
	else if (sum >= 110)
		fee = 110*u1 + (sum-110)*u2;
	else 
		fee = sum*u1; 

	printf("%.2f\n",fee);
	return 0;
} 

int main3() // 另解
{
	float sum,u1=0.5,u2=0.55,u3=0.70; // 用电量,每度电单价
	float fee = 0; // 应缴电费 
	
	scanf("%f",&sum);
	
	if (sum <= 110) fee = sum*u1;
	else if (sum <= 210)
	{
		fee = 110*u1;
		sum -= 110; // sum=sum-110;
		fee += sum*u2;
	}
	else // sum > 210
	{
		fee = 110*u1; 
		fee += (210-110)*u2; // fee = fee+(210-110)*u2
		sum -= 210;  // sum=sum-210;
		fee += sum*u3; // fee = fee+ sum*u3;
	}
	
	printf("%.2f\n",fee);
	return 0;
} 
\end{lstlisting}

\begin{note}[四舍五入问题]
	不同的编译系统,处理结果可能不一致,\lstinline|printf("%.2f\n",fee);|默认输出即可。
\end{note}

\begin{note}
	练习if语句的不同组合形式, 杜绝出现\lstinline|if(110<=sum<=210)|的错误形式。
\end{note}

\section{计算某月天数}
每年的1,3,5,7,8,10,12月有31天,4,6,9,11月有30天,闰年2月29天,其他年份2月28天,给定年份和月份求该月的天数

输入说明:

输入由两个正整数a和b构成,a表示年份,b表示月份,a和b之间用空格分隔

输出说明:

根据年份和月份计算该月天数并输出

输入样例	

输入样例1

2000 3

输入样例2

2001 2

输出样例
	
输出样例1

31

输出样例2

28

\begin{lstlisting}
#include <stdio.h>
int main()
{
	int a,b,t = 0;
	scanf("%d%d",&a,&b);
	if((a%4 == 0 && a%100 !=0) || (a%100 == 0 && a%400 == 0))
	{
		if(b == 2) t = 29;
	}
	else if (b == 2) t = 28;
	
	if(b == 1 || b == 3 || b == 5 || b == 7 || b == 8 || b == 10 || b == 12) t = 31;
	else if(b == 4 || b == 6 || b == 9 || b == 11) t = 30; 
	
	printf("%d\n",t);
	return 0;	
}
\end{lstlisting}

\begin{note}(逻辑运算符)
	\&\&, ||, !, 练习符合逻辑的各种组合形式。
\end{note}

\begin{note}(闰年判断)
	ppt中已详细说明,还有同学写错。
\end{note}

\section{计算整数各位数字之和}
假设n是一个由最多9位数字(d9,\dots, d1)组成的正整数。编写一个程序计算n的每一位数字之和。

输入说明:
	
输入数据为一个正整数n

输出说明:
	
对整数n输出它的各位数字之和后换行

输入样例:	

3704

输出样例:	

14
\newpage
\begin{lstlisting}
#include <stdio.h>
// 体会除10取余, 迭代循环的整数分解技巧;
int main1()
{
	int n,sum = 0; // 注意初始化sum
	scanf("%d",&n);
	while(n) // 等效于n!=0 
	{
		sum += n%10; // 累加本轮循环的末位数
		n /= 10;     // 准备下轮循环的分子
	}
	printf("%d",sum);
	return 0;	
}

// 另解:定义递归函数, 返回整数n的各位数之和 
int sum(int n)
{
	if(n!=0)
	{
	    // 递归调用, 累加本轮循环的末位数
		return (sum(n/10)+n%10); 
	}
	else // n==0时, 结束递归调用 
	{
		return 0; // 函数结束, 返回整数0
	}	
}

int main()
{
	int n;
	scanf("%d",&n);
	printf("%d\n",sum(n)); // 函数调用。 
	return 0; 
}
\end{lstlisting}

\begin{figure}[H]
	\centering
	\caption{递归函数\lstinline|int sum(int n)|中系统内部维护的`栈'结构示意图}	
	\begin{tikzpicture}
	\node[] (tab) {
		\begin{tabular}{|c|c|}
		\hline
		参数n&递归调用\lstinline|sum(n)=sum(n/10)+n%10;|\\
		\hline
		0&sum(0)=0; 结束递归, 开始出栈\\
		\hline
		1&\lstinline|sum(1)=sum(1/10)+1%10=sum(0)+1;|\\
		\hline
		12&\lstinline|sum(12)=sum(12/10)+12%10=sum(1)+2;|\\
		\hline
		123&\lstinline|sum(123)=sum(123/10)+123%10=sum(12)+3;|\\
		\hline
		1234&\lstinline|sum(1234)=sum(1234/10)+1234%10=sum(123)+4;|\\
		\hline
		\hline
		12345&\lstinline|sum(12345)=sum(12345/10)+12345%10=sum(1234)+5;|\\
		\hline
		\end{tabular}
	};
	\path[->,very thick]
	%($(tab.north west)+(-0.5cm,0)$) edge[] node[left] {入栈} +(0,-4cm)
	($(tab.north west)+(-0.5cm,0)$) edge[red] node[left] {入栈} ($(tab.south west)+(-0.5cm,0)$)
	($(tab.south east)+(0.5cm,0)$) edge[blue] node[right] {出栈} ($(tab.north east)+(0.5cm,0)$)
	; 
	\end{tikzpicture}
\end{figure}

\begin{note}[知识点]
	\begin{enumerate}
		\item 体会除10取余, 迭代循环的整数分解技巧;
		\item 预习递归函数定义及调用。
	\end{enumerate}
\end{note}

\section{完数}
请写一个程序,给出指定整数范围[a,b]内的所有完数,0 < a < b < 10000。
一个数如果恰好等于除它本身外的所有因子之和,这个数就称为``完数"。
例如6是完数,因为6=1+2+3

输入说明	

输入为两个整数a和b,a和b之间用空格分隔

输出说明	

输出[a,b]内的所有完数,每个数字占一行

输入样例	

1 10

输出样例	

6

\begin{lstlisting}
#include <stdio.h>
/**********************************************
 采用两层循环方案
 (1) 外层循环使整数i递增,完成区间[n1,n2]区间的完数计算 
 (2) 内层循环, 累加整数i的各因子
 (3) 判断整数i是否是完数, 如果是,输出之 
**********************************************/ 
int main1()
{
	int i,j,n1,n2,sum = 0;
	scanf("%d%d",&n1,&n2);
	for(i = n1; i <= n2; i++) // 外层循环使整数i递增,完成区间[n1,n2]区间的完数计算 
	{
		if(i == 1) continue; // 避免输出1,1不是完数 
		// i不等于1, 计算各因子 
		sum = 1;// 不要忘记, 内层循环前sum的初始化。1总是一个整数的合法因子 
		for(j = 2; j < i; j++) // 累加整数i的所有因子 
		{
			if(i%j == 0) sum += j; // 如果j是i的因子,累加之。 
		}
		if(sum == i) printf("%d\n",i); // 如果i是完数,输出之。 
	} 
	
	return 0;	
}

/**********************************************
 采用一重循环 + 调用函数方案
 (1) 一重循环使整数i递增,函数compute调用,完成区间[n1,n2]区间的完数计算 
 (2) 定义函数compute, 判断整数参数是否是完数, 如果是,返回它, 否则返回-1 
**********************************************/ 

// 定义函数compute, 判断整数参数a是否是完数, 如果是,返回a, 否则返回-1 
int compute(int a)
{
	int i,s=1; // s用于存储a的各因子累加值, 1总是一个整数的合法因子 
	if(a == 1) 
	{
		return -1; // 1不是完数
	}
	// a不为1, 计算各因子 
	for(i = 2; i < a; i++) // 累加整数a的所有因子 
	{
		if(a%i == 0) s += i; // 如果i是a的因子,累加之。 
	}
	if(s == a) 
	{
		return a; // 如果a是完数,返回之。
	} 
	// 如果程序执行到此处必然不是完数 
	return -1;
}

// 另一种方式定义函数compute, 判断整数参数a是否是完数, 如果是,返回a, 否则返回-1 
// 一条return函数返回语句 
int compute1(int a)
{
	int i,s=1; // s用于存储a的各因子累加值, 1总是一个整数的合法因子 
	int ret=-1; // 用于返回值,默认为-1 
	
	for(i = 2; i < a; i++) // 累加整数a的所有因子 
	{
		if(a%i == 0) s += i; // 如果i是a的因子,累加之。 
	}
	if(s == a && a!=1) // 如果a是完数,返回值是本身。1不是完数
	{
		ret = a; 
	} 
	else // a不是完数 
	{
		ret = -1;
	}
	return ret;
}

int main()
{
	int i,n1,n2;
	scanf("%d%d",&n1,&n2);
	for(i = n1; i <= n2; i++) // 调用函数compute,完成区间[n1,n2]区间的完数计算 
	{
		
		if(compute(i)!=-1) printf("%d\n",i); // 如果i是完数,输出之。 
		// 测试函数compute1的调用 
		// if(compute1(i)!=-1) printf("%d\n",i); // 如果i是完数,输出之。 
	} 
	
	return 0;	
}
\end{lstlisting}

\begin{note}[特别注意]
	且记:进入内层循环前, 相关变量的初始化问题。	
\end{note}

\begin{note}[函数定义和调用]
	\begin{itemize}
		\item 函数定义: \lstinline|返回类型 函数名(参数列表) { 函数体 }|
		\item \lstinline|int fun1(float a,float b) { return a/b; // 返回整数部分 }| 
		\item \lstinline|void fun2(float a,float b) { printf(a/b); // 输出整数部分 }| 
		\item 函数调用
		\begin{lstlisting}[frame=none]
		float m,n;
		int ret;
		ret = fun1(m,n); // 调用函数fun1, 其返回值赋值给变量ret;
		fun2(m,n); // 调用函数fun2, 无返回值可用;
		\end{lstlisting}
	\end{itemize}	
\end{note}

