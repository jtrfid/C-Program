%%%%%%%%%%%%%%%%%%%%% chapter.tex %%%%%%%%%%%%%%%%%%%%%%%%%%%%%%%%%
%
% sample chapter
%
% Use this file as a template for your own input.
%
%%%%%%%%%%%%%%%%%%%%%%%% Springer-Verlag %%%%%%%%%%%%%%%%%%%%%%%%%%
%\motto{Use the template \emph{chapter.tex} to style the various elements of your chapter content.}
\chapter{第6次机试练习: 函数, 矩阵, 数组, 字符串, 排序}


\section{矩阵}
请写一个程序,对于一个m行m列$(2<m<20)$的方阵,求其每一行、每一列及主、辅对角线元素之和,然后按照从大到小的顺序依次输出这些值。

注:主对角线是方阵从左上角到右下角的一条斜线,辅对角线是方阵从右上角到左下角的一条斜线。

输入说明

输入数据的第一行为一个正整数m;

接下来为m行、每行m个整数表示方阵的元素。

输出说明	

从大到小排列的一行整数,每个整数后跟一个空格,最后换行。

输入样例

\begin{tabular}{llll}
4&&&\\
15  &~8   &~-2   &~6\\
31  &~24  &~18  &~71\\
-3  &~-9  &~27  &~13\\
17  &~21  &~38  &~69
\end{tabular}

输出样例

159 145 144 135 81 60 44 32 28 27

\begin{lstlisting}
#include <stdio.h>
// 估计方阵行列数 
#define M 20

// input, m是实际方阵行列数 
void input(int matrix[][M], int m)
{
	int i,j; 
	for(i = 0; i < m; i++)
	for(j = 0; j < m; j++) 
	scanf("%d",&matrix[i][j]);
}

// 计算主对角线之和, m是实际方阵行列数 
int main_diagonal(int matrix[][M], int m)
{	
	int i,j,sum = 0; 
	for(i = 0; i < m; i++)
	{
		for(j = 0; j < m; j++) 
		{
			if(i == j) sum += matrix[i][j];   // 主对角线 
		}
	}
	return sum;
}

// 计算副对角线之和, m是实际方阵行列数 
int counter_diagonal(int matrix[][M], int m)
{	
	int i,j,sum = 0;
	for(i = 0; i < m; i++)
	{
		for(j = 0; j < m; j++) 
		{ 
			if(j == m-i-1) sum += matrix[i][j]; // 副对角线之和 
		}
	}
	return sum;
}

// 计算第i行之和, m是实际方阵行列数 
int sumI(int matrix[][M], int m, int i)
{ 
	int j,sum = 0; 
	for(j = 0; j < m; j++) // 遍历列 
	{
		sum += matrix[i][j];  // 第i行之和
	}
	return sum; 
} 

// 计算第j行之和, m是实际方阵行列数 
int sumJ(int matrix[][M], int m, int j)
{ 
	int i,sum = 0; 
	for(i = 0; i < m; i++) // 遍历行 
	{
		sum += matrix[i][j];  // 第j行之和
	}
	return sum;
} 

// 交换两个元素值 
void swap(int *p1, int *p2)
{
	int temp;
	temp = *p1; *p1 = *p2; *p2 = temp;
}

// 选择法排序(降序)
void sorts(int a[], int n)
{
	int i,j,k;
	for(i = 0; i < n-1; i++)
	{
		k = i;
		for (j = i+1; j < n; j++)
			if(a[j] > a[k])	k = j;
		if (k != i) swap(&a[i],&a[k]);
	} 
}

int main()
{
	int matrix[M][M],a[2*M+2]; // 以估计行列数,定义数组 
	int i,m; 
	scanf("%d",&m); // 实际方阵行列数 
	
	input(matrix,m); // input 
	
	// 调用各函数,装配数组a
	int n = 0; // 记录数组a的实际长度		
	for(i = 0; i < m; i++)
	{
		a[n++] = sumI(matrix,m,i); // 第i行之和
		a[n++] = sumJ(matrix,m,i); // 第i列之和  
	}

	a[n++] = main_diagonal(matrix,m);    // 主对角线之和 
	a[n++] = counter_diagonal(matrix,m); // 副对角线之和  
	
	// 排序数组a 
	sorts(a,n);
	
	// 输出 
	for(i = 0; i < n; i++)
	printf("%d ",a[i]);
	
	printf("\n");

	return 0;
}
\end{lstlisting}

\begin{note}[要点]
\begin{enumerate}
\item 思路:定义功能单一的函数,实现简单功能,主程序调用各个函数。
\item 一维数组a[2*M+2]存储相关函数计算结果,排序数组a即是所求。 
\item 避免过多循环嵌套,不易出错,简化程序设计。 
\item 但是缺点是在各函数中分别循环遍历方阵,效率低。
\item 优化方案是不采用独立函数计算,在主函数中一次遍历方阵,计算各值。 
\item 二维数组表示矩阵, 是常见题型, 必须熟练掌握元素的下标规律及其遍历技巧。
\end{enumerate}
\end{note}

\section{消除类游戏}
消除类游戏是深受大众欢迎的一种游戏,游戏在一个包含有n行m列的游戏棋盘上进行,棋盘的每一行每一列的方格上放着一个有颜色的棋子,当一行或一列上有连续三个或更多的相同颜色的棋子时,这些棋子都被消除。当有多处可以被消除时,这些地方的棋子将同时被消除。

现在给你一个n行m列的棋盘,棋盘中的每一个方格上有一个棋子,请给出经过一次消除后的棋盘。

请注意:一个棋子可能在某一行和某一列同时被消除。

输入格式

输入的第一行包含两个整数n, m,用空格分隔,分别表示棋盘的行数和列数。

接下来n行,每行m个整数,用空格分隔,分别表示每一个方格中的棋子的颜色。颜色使用1至9编号。

输出格式

输出n行,每行m个整数,相邻的整数之间使用一个空格分隔,表示经过一次消除后的棋盘。如果一个方格中的棋子被消除,则对应的方格输出0,否则输出棋子的颜色编号。

样例输入1

4 5

2 2 3 1 2

3 4 5 1 4

2 3 2 1 3

2 2 2 4 4

样例输出1

2 2 3 0 2

3 4 5 0 4

2 3 2 0 3

0 0 0 4 4

样例说明

棋盘中第4列的1和第4行的2可以被消除,其他的方格中的棋子均保留。

样例输入2

4 5

2 2 3 1 2

3 1 1 1 1

2 3 2 1 3

2 2 3 3 3

样例输出2

2 2 3 0 2

3 0 0 0 0

2 3 2 0 3

2 2 0 0 0

样例说明

棋盘中所有的1以及最后一行的3可以被同时消除,其他的方格中的棋子均保留。
评测用例规模与约定

所有的评测用例满足:$1 \le n, m \le 30$。

\begin{lstlisting}
#include <stdio.h>
#include <math.h>
#define M 30 // 估计数组最大长度

// 扫描curRow行中可删除的元素, F: 删除标志矩阵, m: 列数 
void delRow(int curRow, int checker[][M], int F[][M],int m)
{
	int i,j, current,count;
	for(i=0;i<m;i++) 
	{
		// if(F[curRow][i]==1) continue; // 如果先扫描行,这就不合适了 
		count=1; 
		current=checker[curRow][i];
		for(j=i+1;j<m;j++)
		{
			if(current==checker[curRow][j]) count++;
			else break;
		}
		if(count>=3)
		{
			for(j=i;j<m;j++)
			{
				if(checker[curRow][j]==current) F[curRow][j]=1;//置删除标志
				else break; 
			}
		}
	} 
}

// 扫描curCol列中可删除的元素, F: 删除标志矩阵, n: 行数 
void delCol(int curCol, int checker[][M], int F[][M],int n)
{
	int i,j, current,count;
	for(i=0;i<n;i++) 
	{
		count=1; 
		current=checker[i][curCol];
		for(j=i+1;j<n;j++)
		{
			if(current==checker[j][curCol]) count++;
			else break;
		}
		if(count>=3)
		{
			for(j=i;j<n;j++)
			{
				if(checker[j][curCol]==current) F[j][curCol]=1;//置删除标志
				else break; 
			}
		}
	} 
}

// 清除所有可删除的元素, F: 删除标志矩阵, n: 行数, m: 列数 
void del(int checker[][M], int F[][M],int n, int m)
{
	int i,j;
	for(i=0;i<n;i++)
	{
		for(j=0;j<m;j++)
		{
			if(F[i][j]==1) checker[i][j]=0;
		}
	}
}

// 读棋盘, n: 行数, m: 列数
void Read(int checker[][M],int n, int m)
{
	int i,j;
	for(i=0;i<n;i++)
	{
		for(j=0;j<m;j++)
		{
			scanf("%d",&checker[i][j]);
		}
	}
}

// 输出棋盘, n: 行数, m: 列数
void output(int checker[][M],int n, int m)
{
	int i,j;
	for(i=0;i<n;i++) // 行 
	{
		for(j=0;j<m;j++) // 列 
		{
			printf("%d ",checker[i][j]); 
		}
		printf("\n");
	} 
}

int main()
{
	// 棋盘,初始化是为了便于测试 
	int checker[M][M]={{2,2,3,1,2},
	{3,4,5,1,4},
	{2,3,2,1,3},
	{2,2,2,4,4}};
	int n=4,m=5,i,j; // n行,m列 
	/************
	int checker[M][M]={{2,2,3,1,2},
	{3,1,1,1,1},
	{2,3,2,1,3},
	{2,2,3,3,3}};
	int n=4,m=5,i,j; // n行,m列 
	*************/
	// 标志矩阵 
	int F[M][M]; // 对应元素为1,则表示删除 
	
	scanf("%d%d",&n,&m);
	// 读棋盘 
	Read(checker,n,m);
	
	// 初始化F 
	for(i=0;i<n;i++)
		for(j=0;j<m;j++)
			F[i][j]=0;
	
	// 扫描行 
	for(i=0;i<n;i++) // 行 
		delRow(i, checker, F, m);
	
	// 扫描列 
	for(j=0;j<m;j++) // 列 
		delCol(j, checker, F, n);
	
	// 清除 
	del(checker, F, n, m);
	
	// 输出棋盘 
	output(checker,n,m); 
	return 0;
} 
\end{lstlisting}

\begin{note}[要点]
标志矩阵,体会模块化编程思想,初始化变量的程序调试技巧。
\end{note}

\section{表达式求值}
表达式由两个非负整数x,y和一个运算符op构成,求表达式的值。
这两个整数和运算符的顺序是随机的,可能是``x op y", ``op x y"或者``x y op",例如, ``25 + 3"表示25加3, ``5 30 *"表示5乘以30, "/ 600 15"表示600除以15。

输入说明

输入为一个表达式,表达式由两个非负整数x,y和一个运算符op构成,x,y和op之间以空格分隔,但顺序不确定。

x和y均不大于10000000,op可以是+, -,*, /, \%中的任意一种, 分表表示加法, 减法, 乘法, 除法和求余。

除法按整数除法求值, 输入数据保证除法和求余运算的y值不为0。

输出说明	

输出表达式的值。

输入样例

样例1输入

5 20 *

样例2输入

4 + 8

样例3输入

/ 8 4

输出样例

样例1输出

100

样例2输出

12

样例3输出

2

\begin{lstlisting}
#include <stdio.h>
// 估计字符串最大长度,存储有效字符(N-1)个,预留最后一个字符'\0' 
#define N 20 

// 根据参数,计算表达式的值 
int compute(char op,int x,int y)
{
	int result = -1;
	switch(op)
	{
		case '+': result = x+y; break;
		case '-': result = x-y; break;
		case '*': result = x*y; break;
		case '/': if(y != 0) result = x/y; break;
		case '%': if(y != 0) result = x%y; break;
	}
	return result;
}

// 数字字符串s转为int, 要求s以'\0'结尾 
int strToInt(char *s)// int toInt(char s[]) 
{
	int result=0;
	while(*s) // 等效while(*s != '\0')或while(*s!=0)
	{
		result=result*10+ (*s-'0');
		s++; //移至下一字符 
	}
	return result;
} 

/******************************************************
提取子串函数 
忽略s中空格前缀,复制s中的字符串到subs中,遇空格或'\0'结束
返回subs不含空格。 返回复制后s指针指向(地址) 
要求s和subs以'\0'结尾。
*******************************************************/ 
char* getSubs(char *s, char *subs) 
{
	int start=0; 
	while(*s)
	{
		if(*s==' ') 
		{
			if(start==0) s++; // 忽略s的前缀空格 
			else break; // 是有效字符串后的一个空格 
		}
		else
		{
			start=1; // 开始复制 
			*subs=*s;
			s++;
			subs++;
		}
	}
	*subs='\0'; // 不要忘记结尾符 
	return s;
}

// 解析s, 以空格为分隔符, 分解s为3个字符串 
void parse(char *s,char result[][N])
{
	char *p;
	p=getSubs(s,result[0]);
	p=getSubs(p,result[1]);
	p=getSubs(p,result[2]);
}	

// 如果s是操作符,返回1, 参数op返回该操作符
// 否则, 返回0 
int isOp(char *s, char *op)
{
	if(*s >= '0' && *s <= '9') // 数字
		return 0;
	else // 操作符
	{
		*op=*s;
		return 1;
	}
} 

int main()
{
	char s[N],op;
	char s3[3][N];   
	int x,y;
	gets(s); 
	
	parse(s,s3); // s被分解为3个字符串 
	if(isOp(s3[0],&op)) // op x y
	{
		x=strToInt(s3[1]);
		y=strToInt(s3[2]);
	}
	else if(isOp(s3[1],&op)) // x op y
	{
		x=strToInt(s3[0]);
		y=strToInt(s3[2]);
	}
	else if(isOp(s3[2],&op)) // x y op
	{
		x=strToInt(s3[0]);
		y=strToInt(s3[1]);
	}

	printf("%d\n",compute(op,x,y)); 
	return 0;
}

int main1() // 另解,直接读取三个子串, 就不用分解了
{
	char s[N],op;
	char s3[3][N];   
	int x,y;
	scanf("%s%s%s",s3[0],s3[1],s3[2]); // 利用"%s"读字符串遇空格结束特点,直接读取3个字符串。 
	
	// parse(s,s3); // s被分解为3个字符串 
	if(isOp(s3[0],&op)) // op x y
	{
		x=strToInt(s3[1]);
		y=strToInt(s3[2]);
	}
	else if(isOp(s3[1],&op)) // x op y
	{
		x=strToInt(s3[0]);
		y=strToInt(s3[2]);
	}
	else if(isOp(s3[2],&op)) // x y op
	{
		x=strToInt(s3[0]);
		y=strToInt(s3[1]);
	}
	
	printf("%d\n",compute(op,x,y)); 
	return 0;
}
\end{lstlisting}

\begin{note}[要点]
本例是字符串处理, 指针应用, 模块化程序设计的范例, 上课时做了详细的讲解。详见课件。
\end{note}

\section{马鞍点}	
若一个矩阵中的某元素在其所在行最小而在其所在列最大,则该元素为矩阵的一个马鞍点。请写一个程序,找出给定矩阵的马鞍点。

输入说明

输入数据第一行只有两个整数m和n(0<m<100,0<n<100),分别表示矩阵的行数和列数;

接下来的m行、每行n个整数表示矩阵元素(矩阵中的元素互不相同), 整数之间以空格间隔。

输出说明

在一行上输出马鞍点的行号、列号(行号和列号从0开始计数)及元素的值(用一个空格分隔), 之后换行;

若不存在马鞍点,则输出一个字符串``no"后换行。

输入样例
	
\begin{tabular}{p{1cm}p{1cm}p{1cm}p{1cm}l}
4  &3&\\
11    &13    &121\\
407   &72    &88\\
23    &58    &1\\
134   &30    &62
\end{tabular} 

输出样例

1 1 72

\begin{lstlisting}
#include <stdio.h>
// 估计的二维数组最大行列数 
#define M 100
#define N 100

// 判断a[row,col]是否是马鞍点, 是: 返回1; 否则返回0
// m,n是二维数组实际行列数 
int compute(int a[][N], int m, int n, int row, int col)
{
	int i,element = a[row][col]; 
	// element在其所在行最小而在其所在列最大 
	for(i = 0; i < n; i++)
		if(a[row][i] < element) return 0; // 不是马鞍点,直接返回0 
	for(i = 0; i < m; i++)
		if(a[i][col] > element) return 0; // 不是马鞍点,直接返回0
	return 1;	// 如果执行至此,肯定是马鞍点,直接返回0 
}

int main()
{
	int matrix[M][N]; // 按照估计的最大行列数定义二维数组 
	int i,j,m,n,flag = 0; 
	
	scanf("%d%d",&m,&n); // 实际行列数 
	
	// input
	for(i = 0; i < m; i++)
		for(j = 0; j < n; j++) 
			scanf("%d",&matrix[i][j]);
	
	// 遍历二维数组, 判断马鞍点		
	for(i = 0; i < m; i++)
	{
		for(j = 0; j < n; j++) 
		{
			if(compute(matrix,m,n,i,j))
			{
				printf("%d %d %d\n",i,j,matrix[i][j]);
				flag = 1;
			}
		}
	}	
	if (!flag) printf("no\n");
	
	return 0;
}
\end{lstlisting}

\begin{note}[要点]
\begin{enumerate}
\item 思路:定义函数计算单个元素a[i,j]是否是马鞍点,主程序遍历二维数组,调用此函数。
\item 避免过多循环嵌套,不易出错,简化程序设计。
\item 掌握二维数组作为函数参数的定义, 调用方式。
\end{enumerate}
\end{note}

\section{数字分解排序}
输入一个9位以内的正整数n,按数值从高到低的顺序输出n的各位数字。

输入说明	

一个正整数n(0<n<1000000000)

输出说明	

按数值从高到低的顺序输出n的各位数字,数字之间用空格隔开

输入样例	

564391

输出样例

9 6 5 4 3 1

\begin{lstlisting}
#include <stdio.h>

void order(int a[],int n);  // 排序函数说明

int main()
{
	int i = 0,j,k,n,num[9];
	scanf("%d",&n);
	k=0; // 数字个数 
	while(n) // 构造数组num存储n的各位数字(从最低位到最高位存储)
	{
		num[i++] = n%10;
		n /= 10;
		k++;
	} 

	// 输出验证,调试技巧之一。 
	//for(j = 0; j < k; j++) printf("%d ",num[j]);
	//   printf("\n");
	
	order(num,k);  // 排序函数
	
	// 输出
	for(j = 0; j < k; j++) printf("%d ",num[j]);
	
	printf("\n");
	return 0;
} 

// 冒泡排序函数 
void order(int a[],int n)
{ 
	int i,j,t,flag; 
	for(j = 1; j <= n-1; j++)
	{ 
		flag=0; // 且记! 必须在进入内层循环前初始化。
		for(i = 0; i < n - j; i++)
			if (a[i] < a[i+1]) 
			{ 
				t = a[i]; a[i] = a[i+1]; a[i+1] = t; 
				flag=1;
			}
		if(!flag) break;
	}
}
\end{lstlisting}

\begin{note}[要点]
排序函数必须掌握,注意检查数组是否越界问题。
\end{note}



