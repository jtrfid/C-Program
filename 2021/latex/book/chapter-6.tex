%%%%%%%%%%%%%%%%%%%%% chapter.tex %%%%%%%%%%%%%%%%%%%%%%%%%%%%%%%%%
%
% sample chapter
%
% Use this file as a template for your own input.
%
%%%%%%%%%%%%%%%%%%%%%%%% Springer-Verlag %%%%%%%%%%%%%%%%%%%%%%%%%%
%\motto{Use the template \emph{chapter.tex} to style the various elements of your chapter content.}
\chapter{第6次机试练习: 函数, 流程控制, 字符串, 数组}

\section{歌德巴赫猜想}
德巴赫猜想:任意一个大偶数都能分解为两个素数的和,对于输入的一个正偶数,写一个程序来验证歌德巴赫猜想。

由于每个正偶数可能分解成多组素数和,仅输出分解值分别是最小和最大素数的一组,按从小到大顺序输出。

输入说明	

输入一个正偶数n,1<n<1000。

输出说明
	
输出分解出的两个最小和最大素数。

输入样例
	
10

输出样例
	
3 7

\begin{lstlisting}
#include <stdio.h>
#include <math.h>

// 判断参数n是否素数, 如果是返回1, 否则返回0 
int isPrime(int n)
{
	int i;
	if(n < 2) return 0; // 最小素数是2, 1不是素数也不是合数, 题意(1<n<1000)不包含1,因此此语句不是必须的 
	//for(i = 2; i <= sqrt((double)n); i++) // vs2013编译器要求数学函数严格按原型解释
	for(i = 2; i <= sqrt(n); i++) // 或条件表达式:i*i<=n 
	{
		if (n%i == 0) return 0; // n不是素数 
	}
	return 1; // n是素数 
}

// 解法一: 递归函数
// 第1个参数: i是递归可能的素数(第一个可能的素数是big_even-1), 第2个参数: big_even是大偶数 
void  recursiveComputing(int i, int big_even)
{
	if(isPrime(i) && isPrime(big_even-i)) // 两个素数和是big_even
	{
		// 注意, 较小的在前 
		printf("%d %d\n",big_even - i, i); // 输出后, return; 结束递归 
	} 
	else // 递归 
	{
		recursiveComputing(i-1,big_even); // 递归调用 
	} 
	return; // 递归结束
} 

int mian()
{
	int num; // num大偶数
	scanf("%d",&num);
	recursiveComputing(num-1,num);
	return 0; 
}

// 解法二: 二重循环, 从最小素数开始迭代计算,获取符合题意的两个素数 
int main1()
{
	int j,k,num; // num大偶数 
	int flag; // 标志变量:用于标识是否找到符合要求的素数对
	scanf("%d",&num);
	// 对于大偶数num,分解为两个素数 
	for(j = 2; j < num; j++)  //  找出第一个素数(最小的) for #1 
	{
		if(!isPrime(j)) continue; // 如果j不是素数, 继续下一轮迭代 
		
		flag = 0; // 初始化,未找出素数对 
		for(k = j + 1; k < num; k++) // //  找出第二个素数 for #2
		{
			if(isPrime(k) && j+k == num) // j是最小素数, k必然是最大素数
			{
				printf("%d %d\n",j,k);
				flag = 1; // 找出素数对, 如果没有此设置, 将会输出多组, 例如num=2020时会有多组素数 
				break; // break for #2
			}
		}
		if(flag) break; // break for #1 
	}
	return 0;
}

// 解法三: 不用标志变量版本 
int main2()
{
	int j,k,num; // num大偶数 
	scanf("%d",&num);
	// 对于大偶数num,分解为两个素数 
	for(j = 2; j < num; j++)  //  找出第一个素数(最小的)  for #1 
	{
		if(!isPrime(j)) continue; // 如果j不是素数, 继续下一轮迭代 
		
		for(k = j + 1; k < num; k++) //  找出第二个素数 for #2
		{
			if(isPrime(k) && j+k == num) // j是最小素数, k必然是最大素数 
			{
				printf("%d %d\n",j,k);
				return 0; // 找出素数对,结束主函数。 如果不结束, 将会输出多组 
			}
		}
	}
	return 0;
}

// 解法四: 优化,根据题意找出一组:最小素数+最大素数=偶数 
int main3()
{
	int j,k,num; // num大偶数 
	scanf("%d",&num);
	// 对于大偶数num,分解为两个素数 
	for(j = 2; j < num; j++)  //  找出第一个素数(最小的)  for #1 
	{
		if(!isPrime(j)) continue; // 如果j不是素数, 继续下一轮迭代 
		
		for(k = num-1; k>=2; k--) //  找出第二个素数(最大的) for #2
		{
			// j是最小素数, 判断k是否是最大素数并且二者之和=num 
			if(isPrime(k) && j+k == num) 
			{
				printf("%d %d\n",j,k);
				return 0; // 找出素数对,结束主函数。如果不结束, 将会输出多组 
			}
		}
	}
	return 0;
}

// 解法五: 一重循环, 从最小素数开始迭代计算,获取符合题意的两个素数 
int main4() 
{ 
	int num,i;
	scanf("%d",&num);
	
	for(i = 2;i < num;i++)
	{
		if(isPrime(i)) // i是素数 
		{
			if(isPrime(num-i)) // 如果num-i也是素数, 即满足题意num=i+j 
			{
				printf("%d %d\n",i,num-i); 
				break;
			} 
		}
	}
	return 0; 
} 
\end{lstlisting}

\begin{note}[要点]
	再次体会标志变量的用法及内层循环前的初始化。仔细审题,本题要求一组输出: 最小素数+最大素数=偶数.
	
	如果要求找出最接近的一组素数, 因为$n=\frac{n}{2}+\frac{n}{2}$, 如果$\frac{n}{2}$是素数,即为所求。 否则,所求的两个素数在$\frac{n}{2}$附近, 分别小于和大于$\frac{n}{2}$。
	
	例如: 修改main4()中的\lstinline|if(isPrime(num-i))为if(isPrime(num-i) && num-i<=i)|, 则输出\lstinline|num-i,i;|即可。 
	
	或者: 由$i=\frac{n}{2}\to i=2$迭代, 第一个找到的素数$i$和$num-i$即为所求。
\end{note}

\section{矩阵}
请写一个程序,对于一个m行m列$(2<m<20)$的方阵,求其每一行、每一列及主、辅对角线元素之和,然后按照从大到小的顺序依次输出这些值。

注:主对角线是方阵从左上角到右下角的一条斜线,辅对角线是方阵从右上角到左下角的一条斜线。

输入说明

输入数据的第一行为一个正整数m;

接下来为m行、每行m个整数表示方阵的元素。

输出说明	

从大到小排列的一行整数,每个整数后跟一个空格,最后换行。

输入样例

\begin{tabular}{llll}
	4&&&\\
	15  &~8   &~-2   &~6\\
	31  &~24  &~18  &~71\\
	-3  &~-9  &~27  &~13\\
	17  &~21  &~38  &~69
\end{tabular}

输出样例

159 145 144 135 81 60 44 32 28 27

\begin{lstlisting}
#include <stdio.h>
// 估计方阵行列数 
#define M 20

// input, m是实际方阵行列数 
void input(int matrix[][M], int m)
{
	int i,j; 
	for(i = 0; i < m; i++)
	for(j = 0; j < m; j++) 
	scanf("%d",&matrix[i][j]);
}

// 计算主对角线之和, m是实际方阵行列数 
int main_diagonal(int matrix[][M], int m)
{	
	int i,j,sum = 0; 
	for(i = 0; i < m; i++)
	{
		for(j = 0; j < m; j++) 
		{
			if(i == j) sum += matrix[i][j];   // 主对角线 
		}
	}
return sum;
}

// 计算副对角线之和, m是实际方阵行列数 
int counter_diagonal(int matrix[][M], int m)
{	
	int i,j,sum = 0;
	for(i = 0; i < m; i++)
	{
		for(j = 0; j < m; j++) 
		{ 
			if(j == m-i-1) sum += matrix[i][j]; // 副对角线之和 
		}
	}
	return sum;
}

// 计算第i行之和, m是实际方阵行列数 
int sumI(int matrix[][M], int m, int i)
{ 
	int j,sum = 0; 
	for(j = 0; j < m; j++) // 遍历列 
	{
		sum += matrix[i][j];  // 第i行之和
	}
	return sum; 
} 

// 计算第j行之和, m是实际方阵行列数 
int sumJ(int matrix[][M], int m, int j)
{ 
	int i,sum = 0; 
	for(i = 0; i < m; i++) // 遍历行 
	{
		sum += matrix[i][j];  // 第j行之和
	}
	return sum;
} 

// 交换两个元素值 
void swap(int *p1, int *p2)
{
	int temp;
	temp = *p1; *p1 = *p2; *p2 = temp;
}

// 选择法排序(降序)
void sorts(int a[], int n)
{
	int i,j,k;
	for(i = 0; i < n-1; i++)
	{
		k = i;
		for (j = i+1; j < n; j++)
			if(a[j] > a[k])	k = j;
		if (k != i) swap(&a[i],&a[k]);
	} 
}

int main()
{
	int matrix[M][M],a[2*M+2]; // 以估计行列数,定义数组 
	int i,m; 
	scanf("%d",&m); // 实际方阵行列数 
	
	input(matrix,m); // input 
	
	// 调用各函数,装配数组a
	int n = 0; // 记录数组a的实际长度		
	for(i = 0; i < m; i++)
	{
		a[n++] = sumI(matrix,m,i); // 第i行之和
		a[n++] = sumJ(matrix,m,i); // 第i列之和  
	}
	
	a[n++] = main_diagonal(matrix,m);    // 主对角线之和 
	a[n++] = counter_diagonal(matrix,m); // 副对角线之和  
	
	// 排序数组a 
	sorts(a,n);
	
	// 输出 
	for(i = 0; i < n; i++)
		printf("%d ",a[i]);
	
	printf("\n");
	
	return 0;
}
\end{lstlisting}

\begin{note}[要点]
	\begin{enumerate}
		\item 思路:定义功能单一的函数,实现简单功能,主程序调用各个函数。
		\item 一维数组a[2*M+2]存储相关函数计算结果,排序数组a即是所求。 
		\item 避免过多循环嵌套,不易出错,简化程序设计。 
		\item 但是缺点是在各函数中分别循环遍历方阵,效率低。
		\item 优化方案是不采用独立函数计算,在主函数中一次遍历方阵,计算各值。 
		\item 二维数组表示矩阵, 是常见题型, 必须熟练掌握元素的下标规律及其遍历技巧。
	\end{enumerate}
\end{note}

\section{回文数}
若一个非负整数其各位数字按照正反顺序读完全相同,则称之为回文数,例如12321。
判断输入的整数是否是回文数。若是,则输出该整数各位数字之和,否则输出no。

输入说明	

输入为一个整数n,0<=n<100000000。

输出说明	

若该整数为回文数,则输出整数各位数字之和,否则输出no。

输入样例
	
样例1输入

131

样例2输入

24

输出样例

样例1输出
	
5

样例2输出

no

\begin{lstlisting}
#include <stdio.h>
#include <string.h>

// 思路1:  求该整数的反序组成的整数,如整数1234, 其反序整数即为4321, 如果二者相等即为回文数 
// 判断num是否是回文数,是:返回1;不是,返回0
// 指针参数*sum,返回这个数的各位之和  
int isPalindromic1(int num,int *sum)
{
	int reverse = 0, tmp = num;
	*sum = 0; // 初始化指针内容 
	while(tmp)
	{
		reverse = reverse*10 + tmp%10;
		*sum += tmp%10;
		tmp /= 10;
	}
	if(reverse == num ) return 1;
	else return 0;	
}

// 思路2:  构造数组reverse, 反序存储该整数各位数,按照数组下标,前后数组元素相等则为回文数 
// 判断num是否是回文数,是:返回1;不是,返回0
// 指针参数*sum,返回这个数的各位之和  
int isPalindromic2(int num,int *sum)
{
	int reverse[9], len=0, i=0; // 依题意数组最大长度为9,最多存储9位数。实际长度用len变量表示 
	
	*sum = 0; // 初始换指针内容 
	//构造数组reverse, 反序存储该整数各位数
	while(num)
	{
		reverse[i]=num%10;  
		*sum += num%10;   // 累加各位数字 
		num /= 10;
		len++; // 计算数组实际长度 
		i++; 
	}
	//按照数组下标,前后数组元素相等则为回文数
	for(i=0; i<len/2;i++)
	{
		if(reverse[i] != reverse[len-i-1]) return 0;  // 不是回文数 
	}
	return 1;	// 至此,必然是回文数 
}

// 测试方法1和2的主程序
int main12()
{
	int i,num,sum;
	
	scanf("%d",&num); 
	//if(isPalindromic1(num,&sum)) printf("%d\n",sum);
	if(isPalindromic2(num,&sum)) printf("%d\n",sum);
	else printf("no\n");
	return 0;
} 

// 思路3:  按照字符串处理输入的整数,前后数组元素相等则为回文数 
int main3()
{
	char s[10]; // 留出'\0', 最多存储9位数。实际长度用len变量表示
	int sum=0, len=0,i=0; 
	
	// 以字符串形式接收输入的整数, 末尾自动追加'\0'  
	gets(s);
	
	// 计算len, 或者len=strlen(s), 同时计算各位数字之和 
	for(len=0;s[i]!='\0';) 
	{
		len++;
		sum=sum+s[i]-'0'; // 计算各位数字之和
		i++;  
	}
	
	//按照数组下标,前后数组元素相等则为回文数
	for(i=0; i<len/2;i++)
	{
		if(s[i] != s[len-i-1]) // 不是回文数  
		{
			printf("no\n");
			return 0;  // 主函数结束 
		} 
	}
	// 至此,必然是回文数,包含一位数不进入上面的for循环的情况 
	printf("%d\n",sum);
	return 0;
}

// 思路4:  按照字符串处理输入的整数,前后数组元素相等则为回文数。
// 使用指针操作 
int main()
{
	char s[10]; // 留出'\0', 最多存储9位数。实际长度用len变量表示
	int sum=0; 
	char *p1=s,*p2=s; // 用于正序和反序遍历s数组,初始指向第一个元素 
	
	// 以字符串形式接收输入的整数,末尾自动追加'\0' 
	gets(s);
	
	// 用p2遍历字符串,同时计算各位数字之和 
	for(;*p2!='\0';p2++) 
	{
		sum=sum+(*p2)-'0'; // 计算各位数字之和
	}
	// 至此,p2指向最后一个元素'\0', 我们使它指向最后一个有效元素: 
	p2--; 
	
	//按照数组下标,前后数组元素相等则为回文数
	for(;p1<p2;p1++,p2--)
	{
		if(*p1 != *p2) // 不是回文数 
		{
			printf("no\n");
			return 0;  // 主函数结束 
		} 
	}
	// 至此,必然是回文数 
	printf("%d\n",sum);
	return 0;
}
\end{lstlisting}

\begin{note}[要点]
	\begin{enumerate}
		\item 掌握函数的地址传递方法。
		\item 使用两个指针变量p1, p2, 其中p1指向待查找子串的首字母, 另一个指向末尾, \lstinline|p1++, p2--|; 是判断字符串是否是回文的有效技巧。
	\end{enumerate}
\end{note}

\section{马鞍点}	
若一个矩阵中的某元素在其所在行最小而在其所在列最大,则该元素为矩阵的一个马鞍点。请写一个程序,找出给定矩阵的马鞍点。

输入说明

输入数据第一行只有两个整数m和n(0<m<100,0<n<100),分别表示矩阵的行数和列数;

接下来的m行、每行n个整数表示矩阵元素(矩阵中的元素互不相同), 整数之间以空格间隔。

输出说明

在一行上输出马鞍点的行号、列号(行号和列号从0开始计数)及元素的值(用一个空格分隔), 之后换行;

若不存在马鞍点,则输出一个字符串``no"后换行。

输入样例

\begin{tabular}{p{1cm}p{1cm}p{1cm}p{1cm}l}
	4  &3&\\
	11    &13    &121\\
	407   &72    &88\\
	23    &58    &1\\
	134   &30    &62
\end{tabular} 

输出样例

1 1 72

\begin{lstlisting}
#include <stdio.h>
// 估计的二维数组最大行列数 
#define M 100
#define N 100

// 判断a[row,col]是否是马鞍点, 是: 返回1; 否则返回0
// m,n是二维数组实际行列数 
int compute(int a[][N], int m, int n, int row, int col)
{
	int i,element = a[row][col]; 
	// element在其所在行最小而在其所在列最大 
	for(i = 0; i < n; i++)
		if(a[row][i] < element) return 0; // 不是马鞍点,直接返回0 
	for(i = 0; i < m; i++)
		if(a[i][col] > element) return 0; // 不是马鞍点,直接返回0
	return 1;	// 如果执行至此,肯定是马鞍点,直接返回0 
}

int main()
{
	int matrix[M][N]; // 按照估计的最大行列数定义二维数组 
	int i,j,m,n,flag = 0; 
	
	scanf("%d%d",&m,&n); // 实际行列数 
	
	// input
	for(i = 0; i < m; i++)
		for(j = 0; j < n; j++) 
			scanf("%d",&matrix[i][j]);
	
	// 遍历二维数组, 判断马鞍点		
	for(i = 0; i < m; i++)
	{
		for(j = 0; j < n; j++) 
		{
			if(compute(matrix,m,n,i,j))
			{
				printf("%d %d %d\n",i,j,matrix[i][j]);
				flag = 1;
			}
		}
	}	
	if (!flag) printf("no\n");
	
	return 0;
}
\end{lstlisting}

\begin{note}[要点]
	\begin{enumerate}
		\item 思路:定义函数计算单个元素a[i,j]是否是马鞍点,主程序遍历二维数组,调用此函数。
		\item 避免过多循环嵌套,不易出错,简化程序设计。
		\item 掌握二维数组作为函数参数的定义, 调用方式。
	\end{enumerate}
\end{note}

\section{密码强度}
每个人都有很多密码,你知道你的密码强度吗?假定密码由大写字母、小写字母、数字和非字母数字的符号这四类字符构成,密码强度计算规则如下:

1.	基础分:空密码(密码长度为零)0分, 非空密码1分

2.	加分项1:密码长度超过8位, +1分

3.	加分项2:密码包含两类不同字符+1分, 包含三类不同字符+2分, 包含四类不同字符+3分

按照此规则计算的密码强度为$0\sim 5$。请你设计一个程序计算给出的密码的强度。

输入说明

输入为一个密码字符串,字符串长度不超过50个字符。

输出说明

输出一个整数表示该密码的强度。

输入样例

输入样例1

abcd

输入样例2

ab123

输出样例

样例1输出:

1

样例2输出

2

\begin{lstlisting}	
#include <stdio.h>
#include <string.h>
int main()
{
	char p[51]; // 记得给'\0'留位置
	int i,strength = 0;
	// class4[0]=1大写字母, class4[1]=1小写字母, class4[2]=1数字, class4[3]=1非字母数字 
	int class4[4] = {0,0,0,0};  
	
	//scanf("%s",p);  // 不能完整接收含空格的字符串和空密码 
	gets(p); // last char: '\0',直接回车,就是空密码 
	
	// 	1.	基础分:空密码(密码长度为零)0分,非空密码1分 
	if(strlen(p) == 0) strength += 0;
	else  strength += 1;
	
	// 2.	加分项1:密码长度超过8位,+1分 
	if(strlen(p) > 8) strength += 1;
	
	// 3.	加分项2:密码包含两类不同字符+1分,包含三类不同字符+2分,包含四类不同字符+3分 
	for(i = 0; p[i] != '\0'; i++)
	{
		if(p[i] >= 'A' && p[i] <= 'Z') class4[0] = 1; 
		else if(p[i] >= 'a' && p[i] <= 'z') class4[1] = 1; 
		else if(p[i] >= '0' && p[i] <= '9') class4[2] = 1;
		else class4[3] = 1; 
	}
	int c = 0;
	for(i = 0; i < 4; i++) c += class4[i];
	
	if(c >= 4) strength += 3;
	else if(c >= 3) strength += 2;
	else if(c >= 2) strength += 1;
	
	printf("%d\n",strength);
	
	return 0;
}
\end{lstlisting}

\begin{note}[要点]
	字符串处理的典型问题: \lstinline|'\0'|, 字符串相关函数\lstinline|char s1[81],s2[81]; strlen(s1), strcmp(s1,s2), strcpy(s1,s2); scanf("%s",s1), gets(s1)|的区别等, 应该充分掌握。
\end{note}



\section{数字分解排序}
输入一个9位以内的正整数n,按数值从高到低的顺序输出n的各位数字。

输入说明	

一个正整数n(0<n<1000000000)

输出说明	

按数值从高到低的顺序输出n的各位数字,数字之间用空格隔开

输入样例	

564391

输出样例

9 6 5 4 3 1

\begin{lstlisting}
#include <stdio.h>

void order(int a[],int n);  // 排序函数说明

int main()
{
	int i = 0,j,k,n,num[9];
	scanf("%d",&n);
	k=0; // 数字个数 
	while(n) // 构造数组num存储n的各位数字(从最低位到最高位存储)
	{
		num[i++] = n%10;
		n /= 10;
		k++;
	} 

	// 输出验证,调试技巧之一。 
	//for(j = 0; j < k; j++) printf("%d ",num[j]);
	//   printf("\n");
	
	order(num,k);  // 排序函数
	
	// 输出
	for(j = 0; j < k; j++) printf("%d ",num[j]);
	
	printf("\n");
	return 0;
} 

// 冒泡排序函数 
void order(int a[],int n)
{ 
	int i,j,t,flag; 
	for(j = 1; j <= n-1; j++)
	{ 
		flag=0; // 且记! 必须在进入内层循环前初始化。
		for(i = 0; i < n - j; i++)
			if (a[i] < a[i+1]) 
			{ 
				t = a[i]; a[i] = a[i+1]; a[i+1] = t; 
				flag=1;
			}
		if(!flag) break;
	}
}
\end{lstlisting}

\begin{note}[要点]
	其它解法及其变种见课件。
	
	排序函数必须掌握,注意检查数组是否越界问题。
\end{note}

