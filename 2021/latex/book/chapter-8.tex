%%%%%%%%%%%%%%%%%%%%% chapter.tex %%%%%%%%%%%%%%%%%%%%%%%%%%%%%%%%%
%
% sample chapter
%
% Use this file as a template for your own input.
%
%%%%%%%%%%%%%%%%%%%%%%%% Springer-Verlag %%%%%%%%%%%%%%%%%%%%%%%%%%
%\motto{Use the template \emph{chapter.tex} to style the various elements of your chapter content.}
\chapter{第7次机试练习: 数组,排序,字符串练习}

\section{排序}	
给定N个不同的整数,要求对这N个整数按如下规则排序并输出。

规则一:所有的偶数排在奇数前面。

规则二:在规则一的前提下按照从大到小的顺序排序。

输入说明	

数据由两行构成,第一行为整数n (n<=100), 表示待排序整数的数量。第二行是n个整数,每个整数的取值区间都为$[-32768\sim32767]$,整数之间以空格间隔。

输出说明
	
在一行输出排好序的整数,整数之间以空格间隔。

输入样例	

5 

1 2 3 4 5

输出样例
	
4 2 5 3 1

\begin{lstlisting}
# include <stdio.h>
#define N 100

// 选择法从大到小排序 
int sorts(int a[],int n)
{
	int i,j,k,tmp;
	for(i=0;i<n-1;i++)
	{
		k=i; // 未排序中较大者 
		for(j=i+1;j<n;j++)
		if(a[j]>a[k]) k=j;
		if(k!=i)
		{
			tmp=a[i];
			a[i]=a[k];
			a[k]=tmp;
		} 
	}
}

int main()
{
	int even[N],odd[N],n,i,num;
	int evenN=0, oddN=0;
	scanf("%d",&n);
	for(i=0;i<n;i++)
	{
		scanf("%d",&num);
		if(num%2==0) // even
		{
			even[evenN]=num;
			evenN++;
		}
		else
		{
			odd[oddN]=num;
			oddN++;
		}
	} 
	// 排序
	sorts(even,evenN);
	sorts(odd,oddN);
	// 输出 
	for(i=0;i<evenN;i++) printf("%d ",even[i]); 
	for(i=0;i<oddN;i++) printf("%d ",odd[i]); 
	printf("\n"); 
	return 0;
}
\end{lstlisting}

\begin{note}[要点]
	各种形式的排序问题(整数, 字符, 字符串, 结构体)是必须掌握的重点之一。选择法排序或冒泡法排序必须信手拈来。
\end{note}

\section{等差数列}	
请写一个程序,判断给定整数序列能否构成一个等差数列。

输入说明
	
输入数据由两行构成,第一行只有一个整数n(n<100), 表示序列长度(该序列中整数的个数);

第二行为n个整数,每个整数的取值区间都为$[-32768\sim 32767]$, 整数之间以空格间隔。

输出说明	

对输入数据进行判断,不能构成等差数列输出``no", 能构成等差数列输出表示数列公差(相邻两项的差)的绝对值的一个整数。

输入样例
	
样例1输入

6

23 15 4 18 35 11

样例2输入

5

2 6 8 4 10

输出样例
	
样例1输出

no

样例2输出

2

\begin{lstlisting}
# include <stdio.h>
#define N 100

// 选择法从大到小排序 
int sorts(int a[],int n)
{
	int i,j,k,tmp;
	for(i=0;i<n-1;i++)
	{
		k=i; // 未排序中较大者 
		for(j=i+1;j<n;j++)
		if(a[j]>a[k]) k=j;
		if(k!=i)
		{
			tmp=a[i];
			a[i]=a[k];
			a[k]=tmp;
		} 
	}
}

// 排序后的a是否构成等差数列,是返回1,否则返回0
// 若是等差数列, 指针参数tolerance返回公差的绝对值。 
int check(int a[],int n, int *tolerance)
{
	int i;
	*tolerance = a[0]-a[1]; // 大-小,不用求绝对值 
	for(i=1;i<n-1;i++) // 注意检查数组越界问题
	{
		if(a[i]-a[i+1] != *tolerance) return 0;
	} 
	return 1;
}

int main()
{
	int a[N],n,i,tolerance;
	scanf("%d",&n);
	for(i=0;i<n;i++)
	{
		scanf("%d",&a[i]);
	} 
	// 排序
	sorts(a,n);
	// check
	if(check(a,n, &tolerance)) printf("%d\n",tolerance);
	else printf("no\n");
	return 0;
}
\end{lstlisting}

\begin{note}[要点]
	数组越界问题, 排序, 地址传递, 模块化程序设计。
\end{note}

\section{0-1矩阵}	
查找一个只包含0和1的矩阵中每行最长的连续1序列。

输入说明
	
输入第一行为两个整数m和n (0<=m,n<=100)表示二维数组行数和列数,其后为m行数据,每行n个整数(0或1), 输入数据中不会出现同一行有两个最长1序列的情况。

输出说明
	
找出每一行最长的连续1序列,输出其起始位置(从0开始计算)和结束位置(从0开始计算),如果这一行没有1则输出两个-1,然后换行。

输入样例 
	
5 6

1 0 0 1 1 0

0 0 0 0 0 0

1 1 1 1 1 1

1 1 1 0 1 1

0 0 1 1 0 0

输出样例
	
3 4

-1 -1

0 5

0 2

2 3

\begin{lstlisting}
#include <stdio.h>
#define N 100

int main()
{
	// 标志变量start,end, maxLen: 记录本行最长连续1总体情况
	// 标志变量tmpStart, tmpLen: 标志是否开始连续1及连续长度 
	int a[N][N],m,n,i,j,start,end,tmpStart,maxLen,tmpLen;
	scanf("%d%d",&m,&n);
	for(i=0;i<m;i++)
		for(j=0;j<n;j++)
			scanf("%d",&a[i][j]);
	
	// 遍历各行		
	for(i=0;i<m;i++) // 行
	{
		start=-1; end=-1; maxLen=0; // 本行连续1总体情况
		tmpStart=-1; tmpLen=0; // 标志是否开始连续1及连续长度
		for(j=0;j<n;j++) // 列
		{
			if(a[i][j]==1) // 1开始 
			{
				if (tmpStart==-1) tmpStart=j; // 开始记录 
				tmpLen++;
			}
			else // 遇0结算 
			{
				if(tmpStart!=-1 && tmpLen>maxLen) 
				{
					start=tmpStart;
					end=j-1;
					maxLen=tmpLen;
					tmpStart=-1;
					tmpLen=0;
				}
			}
		}
		if(tmpStart!=-1 && tmpLen>maxLen) // 末尾是1的情况 
		{
			start=tmpStart;
			end=n-1;
		}
		printf("%d %d\n",start,end);
	}
	return 0;
}
\end{lstlisting}

\begin{note}[要点]
	合理使用标志变量, 训练自己的逻辑思维能力。
\end{note}

\section{寻找最长的行}	
寻找若干行文本中最长的一行

输入说明	

输入为多个字符串(每个字符串长度不超过100个字符),每个字符串占一行,输入的行为``***end***"时表示输入结束

输出说明
	
输出其中最长的一行长度后换行再输出最长行的内容,如果最长行不止一个,则输出其中的第一行。

输入样例
	
abce

abdf dlfd

***end***

输出样例
	
9

abdf dlfd 

\begin{lstlisting}
#include <stdio.h>
#include <string.h> // str前缀的字符串处理函数需要此头文件 

// 估计字符串长度,实际容纳(N-1)个有效字符,预留最后一个字符'\0' 
#define N 100

int main()
{
	char lines[N],maxLine[N];
	
	// 在输入的同时,即可处理,不必设置一个一维数组存储这些数。
	while(1)
	{
		gets(lines);  // 不能使用scanf("%s",lines),因其遇空格结束 
		if(strcmp(lines,"***end***") == 0) break;
		if(strlen(lines) > strlen(maxLine)) strcpy(maxLine,lines);
	}
	
	printf("%d\n%s\n",strlen(maxLine),maxLine); 
	
	return 0;
} 
\end{lstlisting}

\begin{note}[要点]
	简单处理即可, 不必用二维字符数组存储各个字符串,否则会导致超时。
\end{note}

\section{统计正整数的个数}
统计n个正整数中每个数出现的次数。

输入说明	

第一行是一个整数n(5<n<30), 表示要待统计整数的个数;

第二行是n个整数,每个整数均小于100000

输出说明
	
按照整数从小到大的顺序依次输出不同的整数及其出现次数, 整数和出现次数之间用冒号$(:)$分隔。

输入样例	

12
 
19 223 35 321 2 33 44 223 2 19 2 19

输出样例

2:3

19:3

33:1

35:1

44:1

223:2

321:1

\begin{lstlisting}
// 思路: 首先排序, 再统计每个数的个数.
#include <stdio.h>
int main()
{
	int n,num[30],i=0;
	scanf("%d",&n);
	while(i < n) 
	{ 
		scanf("%d",&num[i]);
		i++; 
	} 
	
	// 冒泡排序
	int p,q,t;
	for(p = 1; p <= n-1; p++)
	{
		for(q = 0; q < n-p; q++)
		{
			if(num[q] > num[q+1])
			{
				t = num[q];
				num[q] = num[q+1];
				num[q+1] = t;
			}
		}
	} 
	
	// 判断重复数
	int x,num_i = 0;
	x = num[0]; 
	for(i = 0; i < n; i++)
	{
		if(num[i] == x) num_i++;
		else 
		{
			printf("%d:%d\n",x,num_i);
			x = num[i]; // 新数 
			num_i = 1;  // 至少有一个数  
		}	
		
		if (i == n - 1)  // 最后一个数 
		printf("%d:%d\n",x,num_i);
	} 
	
	return 0;
} 
\end{lstlisting}

\begin{note}[要点]
	基本思路: 首先排序, 再统计每个数的个数.
\end{note}

\section{图像旋转}	
旋转是图像处理的基本操作,在这个问题中,你需要将一个图像顺时针旋转90度。
计算机中的图像可以用一个矩阵来表示,为了旋转一个图像,只需要将对应的矩阵旋转即可。例如,下面的矩阵(a)表示原始图像,矩阵(b)表示顺时针旋转90度后的图像。

\begin{minipage}[h]{0.4\linewidth}
	\centering
	\begin{tabular}{|>{\centering\arraybackslash}p{1cm}|>{\centering\arraybackslash}p{1cm}|>{\centering\arraybackslash}p{1cm}|}
		\hline 
		1 & 5 & 3 \\ 
		\hline 
		3 & 2 & 4 \\ 
		\hline 
	\end{tabular} \\ 
    \medskip
	(a) 原始图像
\end{minipage}
\begin{minipage}[h]{0.4\linewidth}
	\centering
	\begin{tabular}{|>{\centering\arraybackslash}p{1cm}|>{\centering\arraybackslash}p{1cm}|>{\centering\arraybackslash}p{1cm}|}
		\hline 
		3 & 1 \\ 
		\hline 
		2 & 5 \\ 
		\hline
		4 & 3 \\
		\hline 
	\end{tabular} \\
    \medskip
	(b) 顺时针旋转90度的图像
\end{minipage}

输入说明	

输入的第一行包含两个整数n和m,分别表示图像矩阵的行数和列数。$1\le n, m\le 100$。
接下来n行,每行包含m个非负整数,表示输入的图像,整数之间用空格分隔。

输出说明
	
输出m行,每行n个整数,表示顺时针旋转90度之后的矩阵,元素之间用空格分隔。

输入样例
	
2 3

1 5 3

3 2 4

输出样例
	
3 1

2 5

4 3	

\begin{lstlisting}
#include <stdio.h>
int main()
{
	int n, m, a[100][100],b[100][100];
	int i,j;
	
	scanf("%d%d",&n,&m);
	for(i = 0; i < n; i++)
	  for(j = 0; j < m; j++)
	    scanf("%d",&a[i][j]); 
	
	for(i = 0; i < n; i++)
	  for(j = 0; j < m; j++)
    	b[j][i] = a[n-1-i][j]; 
	
	for(i = 0; i < m; i++)
	{
		for(j = 0; j < n ; j++) printf("%d ",b[i][j]);
		printf("\n");
	}
	return 0;
}
\end{lstlisting}

\begin{note}[要点]
	常规二维数组操作。
\end{note}

\section{目录操作}
在操作系统中,文件系统一般采用层次化的组织形式,由目录(或者文件夹)和文件构成,形成一棵树的形状。

有一个特殊的目录被称为根目录,是整个文件系统形成的这棵树的根节点,在类Linux系统中用一个单独的 ``/"符号表示。

因此一个目录的绝对路径可以表示为``/d2/d3"这样的形式。

当前目录表示用户目前正在工作的目录。为了切换到文件系统中的某个目录,可以使用``cd"命令。

假设当前目录为``/d2/d3", 下图给出了cd命令的几种形式,以及执行命令之后的当前目录。

\begin{tabular}{|l|l|l|l|}
	\hline 
	\textbf{当前目录} & \textbf{目录切换命令} & \textbf{命令含义} & \textbf{执行命令后的当前目录} \\ 
	\hline 
	/d2/d3 & cd / & 切换到根目录 & / \\ 
	\hline 
	/d2/d3 & cd .. & 切换到当前目录的上级目录 & /d2 \\ 
	\hline 
	/d2/d3 & cd d4/d5 & 切换到当前目录下的某个子目录 & /d2/d3/d4/d5 \\ 
	\hline 
	/d2/d3 & cd /d1/d5 & 切换到某个绝对路径所指的目录 & /d1/d5 \\ 
	\hline 
\end{tabular} 

现在给出初始时的当前目录和一系列目录操作指令,请给出操作完成后的当前目录。

输入说明	

第一行包含一个字符串,表示当前目录。

后续若干行,每行包含一个字符串,表示需要进行的目录切换命令。

最后一行为pwd命令,表示输出当前目录

注意:

1.	所有目录的名字只包含小写字母和数字,cd命令和pwd命令也都是小写。最长目录长度不超过200个字符。

2.	当前目录已经是根目录时,cd .. 和cd /不会产生任何作用

输出说明
	
输出一个字符串,表示经过一系列目录操作后的当前目录

输入样例	

/d2/d3/d7

cd ..

cd /

cd /d1/d6

cd d4/d5

pwd

输出样例
	
/d1/d6/d4/d5

\begin{lstlisting}
#include <stdio.h>
#include <string.h>
// 估计字符串最大长度,存储有效字符(N-1)个,预留最后一个字符'\0' 
#define N 201 

/******************************************************
 提取子串函数 
 忽略s中空格前缀,复制s中的字符串到subs中,遇空格或'\0'结束
 返回subs不含空格。 返回复制后s指针指向(地址) 
 要求s和subs以'\0'结尾。
*******************************************************/ 
char* getSubs(char *s, char *subs) 
{
	int start=0; 
	while(*s)
	{
		if(*s==' ') 
		{
			if(start==0) s++; // 忽略s的前缀空格 
			else break;       // 是有效字符串后的一个空格 
		}
		else
		{
			start=1; // 开始复制 
			*subs=*s;
			s++;
			subs++;
		}
	}
	*subs='\0'; // 不要忘记结尾符 
	return s;
}

// cd .. 命令, 参数pwd返回当前目录的上一级目录
// pwd="/a/b/c" 或 "/a" 
void dotdot(char *pwd)
{
	int i,len;
	if(strcmp(pwd,"/")==0) return; // 如果已经是根目录, 无动作 
	len=strlen(pwd); 
	for(i=len-1;i>=0;i--)
	{
		if(pwd[i] == '/')
		{
			if(i==0) strcpy(pwd,"/"); // "/a" --> pwd="/"  
			else pwd[i]='\0';         // "/a/b/c"  --> pwd="/a/b" 
			break;
		}
	} 
}

// 解析command, 参数pwd返回当前目录 
void parse(char *command, char *pwd)
{
	char *p, str[N]; // str是cd后的字符串(不含空格) 
	p=getSubs(command, str); // 获得"cd" 
	getSubs(p,str); // cd后的字符串 
	// cd / 或 cd /a/b/c 或 cd a/b/c 
	if(strcmp(str,"/")==0) strcpy(pwd,"/");    // cd /
	else if(strcmp(str,"..")==0)  dotdot(pwd); // cd ..
	else if(str[0]=='/') strcpy(pwd,str);      // cd /a/b/c
	else // cd a/b/c, pwd为pwd/str 
	{
	    // 如果pwd最后一个字符是非'/' 
		if(pwd[strlen(pwd)-1]!='/') strcat(pwd,"/"); 
		strcat(pwd,str);
	}
} 


int main()
{
	char pwd[N]; // 当前目录
	char command[N]; // cd ..  或 cd / 或 cd /a/b/c
	
	gets(pwd);   // 第一行 
	while(1)
	{ 
		gets(command); 
		if(strcmp(command,"pwd")==0) break;
		parse(command,pwd);
	} 
	puts(pwd);
	return 0;
}
\end{lstlisting}

\begin{note}[要点]
	\begin{itemize}
		\item 字符串处理, 指针函数参数的使用, 模块化程序设计。
		\item 可以使用\lstinline|char s1[N], s2[N]; scanf("%s%s",s1,s2)|遇空结束特点, 获取cd及其参数两部分字符串。
	\end{itemize}
\end{note}

\section{处理字符串}
从键盘输入一个字符串,将该字符串按下述要求处理后输出: 

将ASCII码大于原首字符的各字符按原来相互间的顺序关系集中在原首字符的左边,
将ASCII码小于等于原首字符的各字符按升序集中在原首字符的右边。

输入说明	

输入一行字符串,字符串c不长度超过100.

输出说明	

输出处理后的一行字符串

输入样例

\lstinline[mathescape=false]|aQWERsdfg7654!@#$hjklTUIO3210X98aY|

输出样例
	
\lstinline[mathescape=false]|sdfghjkla!#$0123456789@EIOQRTUWXYa|


\begin{lstlisting}
#include <stdio.h>
#include <string.h> // str前缀的字符串处理函数需要此头文件
// 估计字符串长度,实际容纳(N-1)个有效字符,预留最后一个字符'\0'  
#define N 100

// 排序函数(升序) 
void sort(char a[],int n)
{ 
	int i,j,t;
	// 冒泡排序
	for(j = 1; j <= n-1; j++) 
	for(i = 0; i < n - j; i++)
	if (a[i] > a[i+1]) 
	{ t = a[i]; a[i] = a[i+1]; a[i+1] = t; }
}


int main()
{
	char s[N],left[N],right[N];
	char *p = s;
	int i = 0,j = 0;
	gets(s); // 不能使用scanf("%s",lines),因其遇空格结束
	// 使用指针处理较方便 
	while(*p)
	{
		if(*p > s[0]) left[i++] = *p;
		else if(p != s) right[j++] = *p; // 原首字符不加入到right 
		p++; 
	}
	left[i++] = s[0]; // 原首字符加入到left中 
	left[i] = '\0';   // left,right添加结束字符 
	right[j] = '\0';
	sort(right,strlen(right)); // 排序right 
	
	// 可以不必进行字符串连接操作,printf("%s%s\n",left,right); 亦可。
	strcat(left,right); // 连接两个字符串 
	puts(left);
	return 0;
} 
\end{lstlisting}

\begin{note}[要点]
	\begin{itemize}
		\item 借助指针进行字符串处理是常见技巧, 应深刻领会。
		\item 且记字符串结尾字符\lstinline|'\0'|.
	\end{itemize}
\end{note}
