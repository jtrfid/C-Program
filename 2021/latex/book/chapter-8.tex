%%%%%%%%%%%%%%%%%%%%% chapter.tex %%%%%%%%%%%%%%%%%%%%%%%%%%%%%%%%%
%
% sample chapter
%
% Use this file as a template for your own input.
%
%%%%%%%%%%%%%%%%%%%%%%%% Springer-Verlag %%%%%%%%%%%%%%%%%%%%%%%%%%
%\motto{Use the template \emph{chapter.tex} to style the various elements of your chapter content.}
\chapter{第8次机试练习: 函数, 数组, 字符串, 结构体}

\section{累加和校验}
数据传输中一种常见的校验方式是累加和校验。其实现方式是在一次通讯数据包的最后加入一个字节的校验数据。

这个字节内容为前面数据包中所有数据按字节累加所得结果的最后一个字节。例如: 

要传输的信息为: TEST (ASCII码为0x54,0x45,0x53,0x54)

四个字节的累加和为:0x54+0x45+0x53+0x54=0x140 

校验和为累加和的最后一个字节,即0x40,也就是十进制的64 

现在请设计一个程序计算给出的待传输信息的累加校验和 

输入说明

输入为一个字符串,字符串长度不超过100个字符

输出说明

输出一个十进制整数,表示输入字符串的累加校验和。

输入样例

TEST

输出样例

64

\begin{lstlisting}
#include <stdio.h>
int main()
{
	char s[101];
	int i,check,sum = 0; 
	
	gets(s); // scanf("%s",s);  // 不能完整接收含空格的字符串 
	for(i = 0; s[i] != '\0'; i++) sum += s[i];
	check = (sum/16%16)*16+sum%16; // 计算最后一个字节值。
	printf("%d\n",check);
	return 0;
}
\end{lstlisting}

\begin{note}[要点]
	一个字节用两位十六进制数表示,16进制分解与10进制是类似的。整数与字符类型可混合运算。
\end{note}

\section{购票系统}
请实现一个铁路购票系统的简单座位分配算法,来处理一节车厢的座位分配。 
假设一节车厢有20排、每一排5个座位。为方便起见,我们用1到100来给所有的座位编号,第一排是1到5号,第二排是6到10号,依次类推,第20排是96到100号。 
购票时,一个人可能购一张或多张票,最多不超过5张。如果这几张票可以安排在同一排编号相邻的座位,则应该安排在编号最小的相邻座位。否则应该安排在编号最小的几个空座位中(不考虑是否相邻)。 

假设初始时车票全部未被购买,现在给了一些购票指令,请你处理这些指令,输出购票结果。 

例如:若一次购买2,5,4,2张票得到的购票结果为: 

1) 购2张票,得到座位1、2。 

2) 购5张票,得到座位6至10。 

3) 购4张票,得到座位11至14。 

4) 购2张票,得到座位3、4。 

输入说明

输入由两行构成。 

第一行包含一个整数n,表示购票指令的数量,$1\le n\le 100$。 

第二行包含n个整数,每个整数p在1到5之间,表示要购入的票数,相邻的两个整数之间使用一个空格分隔,所有购票数量之和不超过100。 

输出说明

输出n行,每行对应一条购票指令的处理结果。即对于购票指令p,按从小到大排序输出p张车票的编号。 

输入样例

4 

2 5 4 2 

输出样例

1 2 

6 7 8 9 10 

11 12 13 14 

3 4 

\begin{lstlisting}
#include <stdio.h>
// 估计二维数组最大行数和列数 
#define M 20
#define N 5 

//初始化, 二维数组seat代表座位,元素为0值,表示座位未售出, 元素为1代表对应座位已输出。m,n是seat实际的行数和列数 
void init(int seat[][N],int m,int n)
{
	int i,j;
	for(i = 0; i < m; i++)
		for(j = 0; j < n; j++)
			seat[i][j] = 0;	 
}

//根据购票数,查询本购票指令,首座位对应的行索引(从0开始) 
//二维数组seat代表座位,元素为0值,表示座位未售出, 元素为1代表对应座位已输出。m,n是seat实际的行数和列数,tickets购票数  
int search(int seat[][N],int m,int n,int tickets)
{
	int i,j;
	int zeroNum; // 表示某行未售出的座位数
	int row = 0; 
	for(i = 0; i < m; i++)
	{
		zeroNum = 0;
		for(j = 0; j < n; j++)
		{
			if (seat[i][j] == 0) zeroNum++;
		}
		if(zeroNum >= tickets) 
		{
			row = i;
			break;
		}
	}
	return row;
}

//售票 
//二维数组seat代表座位,元素为0值,表示座位未售出, 元素为1代表对应座位已输出。m,n是seat实际的行数和列数,tickets购票数  
void salse(int seat[][N],int m,int n,int tickets)
{
	int i,j,count = 0, flag = 0;
	int row = search(seat,m,n,tickets); // 获取本购票指令,首座位对应的行索引(从0开始)  
	for(i = row; i < m; i++)
	{
		for(j = 0; j < n; j++)
		{
			if (seat[i][j] == 0) 
			{ 
				seat[i][j] = 1;
				printf("%d ",i*n+j+1); // 打印座位编号 
				count++;
				if(count >= tickets)
				{
					flag = 1; break;
				}
			} 
		}
		if(flag == 1) break;
	}
	printf("\n"); // 换行 
}

int main()
{ 
	int seat[M][N];  // 二维数组seat代表座位,元素为0值,表示座位未售出, 元素为1代表对应座位已输出。
	int i, m = 20, n = 5; // m,n是seat实际的行数和列数
	int pNum,tickets;     // 售票指令数和票数 
	
	// 初始化 
	init(seat,m,n);
	
	scanf("%d",&pNum);
	// 输入同时,处理售票。因此不需要使用数组存储pNum和tickets 
	for(i = 0; i < pNum; i++)
	{
		scanf("%d",&tickets);
		salse(seat,m,n,tickets);
	}
	
	return 0;
}
\end{lstlisting}

\begin{note}[要点]
	\begin{enumerate}
		\item 思路:使用二维数组模拟售票座位,元素为0值,表示座位未售出, 元素为1代表对应座位已输出。
		\item 定义独立函数,根据售票指令查询待售座位;定义独立函数,模拟售票操作。 
		\item 注意:随然要求排序输出,但是选票过程就是按照从小到大的顺序选座位,因此并不需要编写排序函数。 
		\item 体会模块化程序设计的优势。
	\end{enumerate}
\end{note}

\section{查找}	
给定一个包含n个整数的数列A0, A1, A2, \dots, An-1和一个整数k,依次输出 k在序列中出现的位置(从0开始计算)。

输入说明	

输入由两行构成,第一行为两个整数n和k,分别表示数列中整数个数和待查找整数k,n和k之间用空格分隔,0<n<100,0<k<10000。

第二行为n个整数,表示数列中的各个整数,整数之间用空格分隔,每个整数均不超过10000。

输出说明	

依次输出整数k在数列中出现的位置(从0开始计算),如果k未在数列中出现,则输出-1。

输入样例

样例1输入
	
5 20

10 20 30 20 5

样例2输入

5 20

10 30 25 34 44

输出样例
	
样例1输出

1 3

样例2输出

-1

\begin{lstlisting}
# include <stdio.h>
#define N 100
int main()
{
	// N是估计数组a的最大长度,实际长度是n; 标志变量flag表示是否k在a数组中
	int a[N],n,i,k,flag=0; 
	scanf("%d%d",&n,&k);
	for(i=0;i<n;i++)
	{
		scanf("%d",&a[i]);
		if(k==a[i]) 
		{
			printf("%d ",i);
			flag=1;
		}
	} 
	if(flag) printf("\n");
	else printf("-1\n");
	return 0;
}
\end{lstlisting}

\begin{note}[要点]
	简单处理,不要复杂化。
\end{note}

\section{中间数}
在一个整数序列$A_1, A_2,\dots, A_n$中,如果存在某个数,大于它的整数数量等于小于它的整数数量,则称其为中间数。

在一个序列中,可能存在多个下标不相同的中间数,这些中间数的值是相同的。 

给定一个整数序列,请找出这个整数序列的中间数的值。 

输入说明

输入的第一行包含了一个整数n,表示整数序列中数的个数,$1\le n\le 1000$。
 
第二行包含$n$个正整数,依次表示$A_1, A_2, \dots, A_n,~1\le A_i\le 1000$。 

输出说明

如果序列的中间数存在,则输出中间数的值,否则输出-1表示不存在中间数。

输入样例

输入样例1 

6 

2 6 5 6 3 5 

输入样例2 

4 

3 4 6 7 

输出样例

输出样例1 

5 

输出样例2 

-1

提示信息

样例1中比5小的数有2个,比5大的数也有2个。因此中间数是5 

样例2中4个数都不满足中间数的定义,因此中间数是-1

\begin{lstlisting}
#include <stdio.h>
int main()
{
	int n, a[1000];
	int i,j,larger,smaller;	// larger,smaller: 比a[i]大或小的数计数
	scanf("%d",&n);
	for(i = 0; i < n; i++) scanf("%d",&a[i]);
	
	for(i = 0; i < n; i++)
	{
		larger = smaller = 0;
		for(j = 0; j < n; j++)
		{
			if(a[j] >= a[i]) larger++; 
			if(a[j] <= a[i]) smaller++;
		}
		if(larger != 0 && larger == smaller) 
		{
			printf("%d ",a[i]); 
			break;  // 最多有一个中间数,其它将与此同值。 
		}
	}
	if(i == n) printf("-1\n"); // 没有中间数 
	return 0;
}
\end{lstlisting}

\begin{note}[要点]
	训练自己的逻辑思维能力。
\end{note}

\section{字符统计}
给出一个字符C和一行文字S,统计在这行文字S中字符C出现的次数。

程序还需要支持大小写敏感选项:

当选项打开时,表示同一个字母的大写和小写看作不同的字符;

当选项关闭时,表示同一个字母的大写和小写看作相同的字符。

输入说明	

输入数据由两行构成。

第一行包含一个字符C和一个数字n。字符C为大小或小写英文字母。数字n表示大小写敏感选项,当数字n为0时表示大小写不敏感,当数字n为1时表示大小写敏感。字符C和数字n之间用空格分隔。

第二行为一个字符串S,字符串由大小写英文字母组成,不含空格和其他字符。字符串S长度不超过100。

输出说明
	
输出字符C在字符串S中出现的次数。

输入样例

样例1输入

L 1

HELLOWorld

样例2输入

L 0

HELLOWorld

输出样例

样例1输出	

2

样例2输出

3

\begin{lstlisting}
#include <stdio.h>
// 估计字符串长度,实际容纳(N-1)个有效字符,预留最后一个字符'\0'
#define N 101 

int main()
{
	char S[N],C,n, i, count=0;
	
	// printf("%d,%d\n",'a','A'); // 97,65 测试大小写字母转换关系 
	
	/************ 测试, 这样的输入得不到正确的值 
	scanf("%c%d",&C,&n);
	gets(S);
	printf("===%c,%d,%s\n",C,n,S); // 测试输入, 错误 
	*************/ 
	
	/************ 测试, 这样的输入得不到正确的值
	scanf("%c%d%s",&C,&n,S);
	printf("===%c,%d,%s\n",C,n,S); // 测试输入, 错误 
	*************/ 
	
	// 测试以下输入正确 
	C=getchar();    // 读取C 
	scanf("%d",&n); // 读取n 
	getchar();      // 消费回车键 
	gets(S);        // 读取S 
	//printf("===%c,%d,%s\n",C,n,S); // 测试输入, 正确 
	
	
	if(n == 0 && C >= 'a' && C <= 'z') // 大小写不敏感, 统一转为大写比较 
	{
		C = C-32; // 小写转大写 
	} 
	
	for(i=0; S[i] != '\0'; i++) 
	{
		if(n==0) // 大小写不敏感, 统一转为大写比较 
		{
			if(S[i]>='a' && S[i]<='z') S[i] = S[i] - 32;  // 小写转大写 
		}
		if(S[i] == C) count++; // 计数 
	}
	printf("%d\n",count);
	return 0;
} 
\end{lstlisting}

\begin{note}[要点]
	\begin{itemize}
		\item 测试输入变量的正确值, 是调试程序的首要点, 尤其是数字, 字符, 字符串混合形式的输入。
		\item 还可以使用 \lstinline|char s1[N], s2[N]; scanf("%s%s",s1,s2)|遇空结束特点, 获取第一行的两部分字符串。
	\end{itemize}
	
\end{note}

\section{复试筛选}	
考研初试成绩公布后需要对m个学生的成绩进行排序,筛选出可以进入复试的前n名学生。
排序规则为首先按照总分排序,总分相同则按英语单科成绩排序,总分和英语成绩也相同时考号小者排在前面。

现给出这m个学生的考研初试成绩,请筛选出可以进入复试的n名学生并按照排名从高到低的顺序依次输出。

输入说明	

输入为m+1行,第一行为两个整数m和n,分别表示总人数和可以进入复试人数,m和n之间用空格分隔,0<n<m<200。

接下来为m行数据,每行包括三项信息,分别表示一个学生的考号(长度不超过20的字符串)、总成绩(小于500的整数)和英语单科成绩(小于100的整数), 这三项之间用空格分隔。

输出说明

按排名从高到低的顺序输出进入复试的这n名学生的信息。

输入样例	

5 3

XD20160001 330 65

XD20160002 330 70

XD20160003 340 60

XD20160004 310 80

XD20160005 360 75

输出样例	

XD20160005 360 75

XD20160003 340 60

XD20160002 330 70

\begin{lstlisting}
// 思路:定义结构体类型和结构体数组,对结构体数组进行排序。 
#include <stdio.h>
#include <string.h>

// 估计结构体数组最大长度 
#define N 200

struct Student 
{
	char no[20];   // 考号 
	int  total;    // 总成绩 
	int  english;  // 英语成绩  
};

// 输入m个考生信息
void input(struct Student *stus, int m)
{
	int i;
	for(i=0;i < m;i++) 
		scanf("%s%d%d",	stus[i].no,&stus[i].total,&stus[i].english); 
}

// 输入n个考生信息
void print(struct Student *stus, int n)
{
	int i;
	for(i=0;i<n;i++) 
		printf("%s %d %d\n",stus[i].no,stus[i].total,stus[i].english); 
}

// 交换两个结构体对象 
void swap(struct Student *p1, struct Student *p2)
{
	struct Student temp;
	temp = *p1; *p1 = *p2; *p2 = temp;
}

/************************************************* 
 选择法排序(降序)
 排序规则为首先按照总分排序,总分相同则按英语单科成绩排序
**************************************************/ 
void sorts(struct Student a[], int n)
{
	int i,j,k;
	for(i = 0; i < n - 1; i++)
	{
		k = i;
		for (j = i+1; j < n; j++)
		{
			// 条件判断语句:善用&&、||运算,简化if else结构 
			// 注意(条件1||条件2||$\cdots$)的截断语义
			if (a[j].total>a[k].total || (a[j].total==a[k].total &&  a[j].english>a[k].english)
			|| (a[j].total==a[k].total &&  a[j].english==a[k].english && strcmp(a[j].no,a[k].no)<0))  
			k = j;
		}
		if (k != i) swap(&a[i],&a[k]);
	} 
}

int main()
{
	struct Student stus[N]; 
	int m,n,i;
	scanf("%d%d",&m,&n);
	input(stus,m); 
	sorts(stus,m);
	print(stus,n);
	return 0;
}
\end{lstlisting}

\begin{note}[要点]
	\begin{itemize}
		\item 条件判断语句:善用 \lstinline$&&, ||$ 运算,简化\lstinline|if else|结构
		\item \lstinline!if (条件1||条件2||条件3||$\cdots$)!, 任一条件为\lstinline|true|, 整个条件表达式为\lstinline|true|,不必进行后续条件的判断。称为``或''条件的截断语义。
		\item 本例是结构体数组排序, 模块化程序设计的典型案例,  详见课件。 
	\end{itemize}
\end{note}

\section{画图}
在一个定义了直角坐标系的纸上,画一个(x1,y1)到(x2,y2)的矩形,指将横坐标范围从x1到x2,纵坐标范围从y1到y2之间的区域涂上颜色。    
下图给出了一个画了两个矩形的例子。第一个矩形是(1,1) 到(4, 4),用绿色和紫色表示。第二个矩形是(2, 3)到(6, 5),用蓝色和紫色表示。

\begin{center}
	\includegraphics*[scale=0.4]{paint}
\end{center}

图中,一共有15个单位的面积被涂上颜色,其中紫色部分被涂了两次,但在计算面积时只计算一次。在实际的涂色过程中,所有的矩形 都涂成统一的颜色,图中显示不同颜色仅为说明方便。给出所有要画的矩形,请问总共有多少个单位的面积被涂上颜色。

输入说明 

输入的第一行包含一个整数n,表示要画的矩形的个数,1<=n<=100   

接下来n行,每行4个非负整数,分别表示要画的矩形的左下角的横坐标与纵坐标,以及右上角的横坐标与纵坐标。0<=横坐标、纵坐标<=100。

输出说明	

输出一个整数,表示有多少个单位的面积被涂上颜色。

输入样例

2 

1 1 4 4 

2 3 6 5 

输出样例

15

\begin{lstlisting}
#define N 100
//定义矩形结构体 
struct Rec
{
	int leftBottomX;
	int leftBottomY;
	int rightTopX;
	int rightTopY;
};

// 返回最大矩形
struct Rec Largest(struct Rec res[], int n)
{
	struct Rec largest=res[0];
	int i;
	for(i=1;i<n;i++)
	{
		if(res[i].leftBottomX < largest.leftBottomX) largest.leftBottomX=res[i].leftBottomX;
		if(res[i].leftBottomY < largest.leftBottomY) largest.leftBottomY=res[i].leftBottomY;
		if(res[i].rightTopX > largest.rightTopX) largest.rightTopX=res[i].rightTopX;
		if(res[i].rightTopY > largest.rightTopY) largest.rightTopY=res[i].rightTopY;
	}
	return largest; 
} 

// rec区域, grid数组元素置1 
void grid1(struct Rec rec, int grid[][N])
{
	int i,j;
	for(i=rec.leftBottomY;i<rec.rightTopY;i++)
	{
		for(j=rec.leftBottomX;j<rec.rightTopX;j++)
		grid[i][j]=1;
	} 
}

// largest区域, grid数组元素初始化为0 
void grid0(struct Rec largest, int grid[][N])
{
	int i,j;
	for(i=largest.leftBottomY;i<largest.rightTopY;i++)
	{
		for(j=largest.leftBottomX;j<largest.rightTopX;j++)
		grid[i][j]=0;
	} 
}

// largest区域, 返回grid数组元素为1的单元数量,即被覆盖的单位面积数 
int gridNum(struct Rec largest, int grid[][N])
{
	int i,j,num=0;
	for(i=largest.leftBottomY;i<largest.rightTopY;i++)
	{
		for(j=largest.leftBottomX;j<largest.rightTopX;j++)
		if(grid[i][j]) num++;
	} 
	return num;
}

void input(struct Rec *recs, int n)
{
	int i;
	for(i=0;i<n;i++) 
	scanf("%d%d%d%d",&recs[i].leftBottomX,&recs[i].leftBottomY,
	&recs[i].rightTopX,&recs[i].rightTopY);
} 

void print(struct Rec *recs, int n)
{
	int i;
	for(i=0;i<n;i++) 
	printf("%d,%d,%d,%d\n",recs[i].leftBottomX,recs[i].leftBottomY,
	recs[i].rightTopX,recs[i].rightTopY);
} 

int main()
{
	int grid[N][N],n,i,num;
	struct Rec recs[N],largest;
	scanf("%d",&n);
	input(recs,n);
	// print(recs,n); // 检查输入 
	largest=Largest(recs,n); // 计算最大矩形 
	// print(&largest,1); // 查看最大矩形是否正确 
	grid0(largest,grid); // 初始化 
	for(i=0;i<n;i++)
	{
		grid1(recs[i],grid); // 置1 
	}
	printf("%d\n",gridNum(largest,grid)); // 统计输出 
	return 0;
}
\end{lstlisting}

\begin{note}[要点]
详见课件。注意使用标志矩阵表示最大矩阵的使用技巧。 进一步练习模块化程序设计方法。
\end{note}

