%%%%%%%%%%%%%%%%%%%%%%%%%% lecture-12
%\begin{frame}[shrink]
%  \frametitle{lecture-12 主要内容}
%  \framesubtitle{用函数实现模块化程序设计}
%  %\tableofcontents[hideallsubsections]
%  \tableofcontents
%\end{frame}

\section{使用函数进行模块化程序设计}

\begin{frame}[shrink,fragile]{为什么使用函数(1)}
\tiny
\vspace{-0.2cm}
\begin{columns}[T]
\column{0.4\textwidth}
\begin{beamerboxesrounded}{函数调用}
\begin{lstlisting}
#include<stdio.h>
#include<math.h>
int main() // 主函数
{
   int a;
   // 库函数调用
   scanf("%d",&a); 
   a=(int)fabs(a); 
   printf("%d\n",a);
   return 0; 
}
\end{lstlisting}
\end{beamerboxesrounded}
\column{0.6\textwidth}
\begin{beamerboxesrounded}{函数定义}
\begin{lstlisting}
int scanf(char format[], args, ...)
{ ...;
  return 整型值;
}
int printf(char format[], args, ...)
{ ...;
  return 整型值;
}
double fabs(double a)
{ // 模拟代码,没有考虑精度
  if(a<0) return -a;
  else return a;
}
\end{lstlisting}
\end{beamerboxesrounded}
\end{columns}
\end{frame}

\begin{frame}[shrink,fragile]{为什么使用函数(2)}
\tiny
\vspace{-0.2cm}
\begin{columns}[T]
\column{0.5\textwidth}
\begin{beamerboxesrounded}{函数调用}
\begin{lstlisting}
#include<stdio.h>
#include<string.h>
int main() // 主函数
{
   int a;
   char s1[81],char s2[81]="1234";
   // 库函数调用 
   strcpy(s1,s2); 
   a=strcmp(s1,s2);
   printf("%d ",a);
   printf("%d\n",strlen(s1));
   return 0; 
}
\end{lstlisting}
\end{beamerboxesrounded}
\column{0.5\textwidth}
\begin{beamerboxesrounded}{函数定义}
\begin{lstlisting}
char[] strcpy(char s1[],char s2[])
{ ...; return s1; 
}
int strcmp(char s1[],char s2[])
{  ...;  return 整型值; // 1,-1,0
}
int strlen(char s[])
{
  int len=0;
  while(s[len]!='\0') len++;
  return len;
}
\end{lstlisting}
\end{beamerboxesrounded}
\end{columns}
\end{frame}

\begin{frame}[shrink,fragile]{利用函数进行功能分解 --- 简化程序设计的有效手段}
\vspace{-0.3cm}
\begin{columns}[T]
\column{0.5\textwidth}
\begin{beamerboxesrounded}{功能分解, 函数调用}
\begin{lstlisting}
#include<stdio.h>
#include<math.h>
int add(int a,int b); // 函数原型声明
void output(double a);
int main() // 主函数
{
  int a=10,b=20,c;
  //传递参数a,b的值, 计算结果赋值给变量c 
  c=add(a,b);  // 函数调用
  output(10.5*c); // 函数调用
  return 0; 
}
\end{lstlisting}
\end{beamerboxesrounded}
\column{0.5\textwidth}
\begin{beamerboxesrounded}{分解功能实现, 函数定义}
\begin{lstlisting}
// 通过参数a,b的值, 进行相关计算, 返回整型数据给调用者
int add(int a,int b)
{  return a+b; // 返回整型值 }
// 通过参数a的值, 进行相关计算, 不需要返回数据
void output(double a)
{  
   printf("%lf\n", sqrt(a)); // 函数中可调用别的函数
}
\end{lstlisting}
\end{beamerboxesrounded}
\end{columns}
%~\\
\end{frame}

\begin{frame}{使用函数进行程序设计的优点}
\begin{enumerate}
	\setlength{\itemsep}{.5cm}
	\item 使用函数可使程序清晰、精炼、简单、灵活。
	\item 函数就是功能。每一个函数用来实现一个特定的功能。函数名应反映其代表的功能。
	\item 在设计较大程序时,往往把它分为若干个程序模块,每一个模块包括一个或多个函数,每个函数实现一个特定的功能。
	\item 一个C程序可由一个主函数和若干个其他函数构成。由主函数调用其他函数,其他函数也可以互相调用。	
\end{enumerate}
\end{frame}

%\subsection{函数三要素: 函数原型声明, 函数定义, 函数调用}

\begin{frame}[shrink,fragile]{函数三要素: 函数原型声明, 函数定义, 函数调用}
\vspace{-0.2cm}
\begin{columns}[T]
\column{0.5\textwidth}
\begin{beamerboxesrounded}{函数原型声明, 函数调用}
\begin{lstlisting}
#include<stdio.h> // 库函数原型声明
// 函数原型声明, 使编译器认识这个函数,如果函数定义在调用之前,可省略声明
int add(int a,int b); 
void output(double a);
int main() // 主函数
{
   int a=10,b=20,c;
   // 函数调用, 执行函数功能
   c=add(a,b); 
   output(10.5*c);
   return 0; 
}
\end{lstlisting}
\end{beamerboxesrounded}
\column{0.5\textwidth}
\begin{beamerboxesrounded}{函数定义, 定义函数功能}
\begin{lstlisting}
// 通过参数a,b的值, 进行相关计算, 返回整型数据给调用者
int add(int a,int b)
{  return a+b; // 返回整型值 }

// 通过参数a的值, 进行相关计算, 不需要返回数据
void output(double a)
{  
   printf("%lf\n", sqrt(a)); // 函数中可调用别的函数
}
\end{lstlisting}
\end{beamerboxesrounded}
\end{columns}
~\\
\end{frame}

%\subsection{函数定义: int\, add(int a,int b)\{\quad\}}

\begin{frame}[shrink,fragile]{函数定义: int\, add(int a,int b)\{\quad\}}
\textbf{函数定义: } 返回类型\, 函数名(参数类型\, 参数, $\cdots$) $\{\cdots ; \}$
\begin{lstlisting}
int add(int a, int b)
{  ...;
   return 整型值; // 必须含return语句, 函数执行结束, 并将返回值返回给调用者
}
double fun(void) // 无参函数, 等效 double fun() 
{  ...;
   return 双精度值; // 必须含return语句, 函数执行结束, 并将返回值返回给调用者
}
void output(double a) // 无返回值函数
{  ...;
   return; // 可选return语句(注意没有表达式), 仅表示函数执行结束
}
\end{lstlisting}
\end{frame}

\begin{frame}[shrink,fragile]{函数调用时数据类型的隐式转换}
\begin{lstlisting}
int a=2,b=3,c;
// pow函数原型: double pow(double x,double y);
// 编译器自动把a,b"隐式"转换为double
// 计算$a^b$的结果是double类型, 赋值语句隐式转换为int, 但是会引起警告信息
c=pow(a,b);
c=(int)pow(a,b); // 将函数的返回值强制转换为int, 不会有警告信息
c=(int)pow((double)a,(double)b); // 例如vs2019等编译器要求显示强制转换
\end{lstlisting}
\end{frame}

\begin{frame}[shrink,fragile]{整数pow函数问题}
\vspace{-0.3cm}
\lstinline|double pow(double x,double y);| 数学库函数真对双精度浮点数设计。用它计算整数$x^y$会有复杂的精度问题, 因此整数运算尽量不要用此函数。 
\begin{lstlisting}
#include<stdio.h>
#include<math.h>   // 数学库函数
int main()
{
	int x=2,y=3;
	// 有些编译器会输出99,124, 因为转换前是99.999, 124.999 
	printf("%d,%d\n",(int)pow(10,x),(int)pow(5,y)); 	
	// 奇怪的是下列调用输出结果与上面有可能会输出不一致的结果 
	printf("%d,%d\n",(int)pow(10,2),(int)pow(5,3)); 
	return 0;
}
\end{lstlisting}
\end{frame}

\begin{frame}[shrink,fragile]{自定义整数pow函数}
\vspace{-0.5cm}
\begin{columns}[T]
\column{0.35\linewidth}
\begin{lstlisting}
#include<stdio.h>
// 自定义整数pow函数, 返回$x^y, x,y\ge 0$
int intpow(int x, int y)
{
	int i,p=1;
	// 这里规定: $0^0=1$
	if(x==0 && y==0) 
		return 1;
	// 包含了$x^0=1,x>0$ 
	// $y^0=0,y>0$和$0^0=1$
	for(i=0;i<y;i++)p=p*x;
	return p;
} 
\end{lstlisting}
\column{0.65\linewidth}
\begin{lstlisting}[frame=leftline]
int main()
{
	int x=2,y=3;
	
	printf("%d,%d\n",intpow(10,x),intpow(5,y)); 
	// 100,125
	printf("%d,%d\n",intpow(10,2),intpow(5,3)); 
	// 100,125
	printf("%d,%d\n",intpow(0,2),intpow(0,0)); 
	// 0,1 
	
	return 0;
}
\end{lstlisting}
\end{columns}
\end{frame}

\note{递归版
	\begin{lstlisting}
	int intpow(int x,int y)
	{
		if(x==0 && y==0) return 1; // 这里规定: 0^0=1 
		// 递归 
		if(y==0) return 1; // x^0=1, 结束递归
		else if(y%2) return x*intpow(x,y/2)*intpow(x,y/2);
		return intpow(x,y/2)*intpow(x,y/2);
	}
    \end{lstlisting}
}

\section{实参和形参间的数据传递(值传递和地址传递)}

\begin{frame}[shrink,fragile]{实参和形参间的数据传递(值传递)}
\vspace{-0.3cm}
\begin{columns}[T]
\column{0.55\textwidth}
\begin{lstlisting}
#include<stdio.h> // 库函数原型声明
// 函数原型声明, 使编译器认识这个函数
int max(int x,int y); 
int main() // 主函数
{
   int a=10,b=20,c;
   scanf("%d%d",&a,&b);
   // 把此处的a,b值拷贝给函数形式参数x,y
   c=max(a,b); // 实际参数 
   printf("较大者=%d\n",c);//与下一句等效
   printf("较大者=%d\n",max(a,b));
   return 0; 
}
\end{lstlisting}
\column{0.45\textwidth}
\begin{lstlisting}[frame=leftline]
// 定义函数求形参x,y中的较大者并返回给调用者
int max(int x,int y) // 形式参数
{  
   int z;
   z=x>y ? x : y;
   return z; 
}
\end{lstlisting}
\end{columns}
\textbf{\textcolor{blue}{在调用函数过程中发生的实参与形参间的数据传递称为``虚实结合"。}}
\end{frame}

\begin{frame}[shrink,fragile]{实参和形参间的数据传递(地址传递)}
例: 将数组a中n个整数按相反顺序存放。
\begin{columns}[T]
\column{0.6\textwidth}
\begin{lstlisting}
#include<stdio.h> 
void inv(int x[],int n); // 要求x是地址传递
#define N 100 
int main() 
{
   int i, n, a[N]={1,2,3,4,5,6,7,8,9,10};
   scanf("%d",&n);
   for(i=0; i<n; i++) scanf("%d",&a[i]);
   // 把实参a数组的地址拷贝给形参x, n的值拷贝给形式参数n
   inv(a,n); // 实参a数组的内容被改变
   for(i=0; i<n; i++) printf("%d ",a[i]);
   return 0; 
}
\end{lstlisting}
\column{0.4\textwidth}
\begin{lstlisting}[frame=leftline]
// 倒置数组x的内容, n是x的长度
// 要求x是地址传递
void inv(int x[],int n) 
{  
   int i,temp,m=(n-1)/2;
   //以中间元素为界, 前后元素交换
   for(i=0; i<=m; i++) 
   {
      temp=x[i]; 
      x[i]=x[n-1-i]; 
      x[n-1-i]=temp;
   }
   return; //可选, 函数执行完毕
}
\end{lstlisting}
\end{columns}
~\\
\end{frame}

\begin{frame}[shrink,fragile]{值传递与地址传递的不同点}
\vspace{-0.2cm}
\begin{itemize}
	\item 值传递, 对形参值的改变不会引起实参值的改变。
	\item 地址传递虽然不能改变实参的地址,但是对地址指向内容的改变会引起实参指向内容的改变。
	\item 数组名表示数组元素在内存中的首地址, 数组元素是连续存放的。
	\item 用数组名作为参数传递,就是地址传递。在函数内部对数组元素的改变,就是改变实参数组元素的值。
\end{itemize}
\begin{tikzpicture}
\node[text width=.5\textwidth,draw] (a) {
	\vspace{-0.2cm}
	\begin{lstlisting}
	// a是地址传递, n是值传递
	void fun(double a[],int n) 
	{
		// 改变地址a指向的内容,就是改变实参数组的元素值
		a[1]=20.5; 
		// n的改变不会影响实参的值
		n=30; 
	}
	\end{lstlisting}
};
\node[anchor=west,text width=.7\textwidth,draw] at(a.east)
{
	\vspace{-0.2cm}
	\begin{lstlisting}
	int main()
	{
		double a[2]={0.8,0.3};
		int n=2;
		// 实参a的地址拷贝给形参a,实参n的值拷贝给形参n
		fun(a,n); 
		printf("%d,%lf,%lf\n",n,a[0],a[1]); // 2,0.8,20.5
	}
	\end{lstlisting}
};
\end{tikzpicture}

\end{frame}

\begin{frame}[shrink,fragile]{数组名作为函数参数的注意事项}
\vspace{-0.3cm}
\begin{columns}[T]
\column{0.5\textwidth}
%\begin{beamerboxesrounded}{一维数组}
\begin{lstlisting}
#define M 100 // 估计的数组的第一维长度
#define N 100 // 估计的数组的第二维长度
// 函数定义时省略第一维大小
// a[]表示a接受地址传递,以区别于值传递
void fun(int a[],int n)
{
   int i;
   for(i=0;i<n;i++) 
     printf("%d "a[i]);
   printf("\n");
}
// 函数调用
int x[M]={1,2,3,4,5};
fun(x,5);//调用时仅用数组名传递x的地址
\end{lstlisting}
%\end{beamerboxesrounded}
\column{0.5\textwidth}
%\begin{beamerboxesrounded}{二维数组}
\begin{lstlisting}[frame=leftline]
// 函数定义时省略第一维大小,第二维不能省略
void fun(int a[][N],int m,int n)
{
  int i;
  for(i=0;i<m;i++)
  {
    for(j=0;j<n;j++) 
      printf("%d "a[i][j]);
    printf("\n");
  }
}
// 函数调用
int x[M][N]={{1,2},{3,4},{5,6},
             {7,8},{9,10}};
fun(x,5,2);//调用时仅用数组名传递x的地址
\end{lstlisting}
%\end{beamerboxesrounded}
\end{columns}
~\\
\end{frame}

\section{程序举例}

\begin{frame}[shrink,fragile]{例: PM2.5}
给出一组PM2.5数据,按以下分级标准统计各级天气的天数,并计算出PM2.5平均值。

PM2.5分级标准为: 
一级优$(0\le PM2.5\le 50)$; 二级良$(51\le PM2.5\le 100)$;  三级轻度污染$(101\le PM2.5\le 150)$; 四级中度污染$(151\le PM2.5\le 200)$; 五级重度污染$(201\le PM2.5\le 300)$;六级严重污染$(PM2.5>300)$.

输入说明, 输入分为两行,

第一行是一个整数$n$表示天数$(1<n\le 100)$ 

第二行为$n$个非负整数$Pi(0\le Pi\le 1000)$表示每天的PM2.5值,整数之间用空格分隔。

输出说明, 输出两行数据,

第一行为PM2.5平均值, 结果保留2位小数; 

第二行依次输出一级优, 二级良, 三级轻度污染, 四级中度污染, 五级重度污染, 六级严重污染的天数。
\end{frame}

\begin{frame}[shrink,fragile]{例: PM2.5(非模块化设计)}
\vspace{-0.2cm}
\begin{tikzpicture}
\node[text width=0.65\textwidth,draw] (a) {
\vspace{-0.3cm}
\begin{lstlisting}
#include <stdio.h>
int main()
{
  int i =0,n,pm25,day[6] = {0,0,0,0,0,0},
  	  sum = 0;
  scanf("%d",&n);
  while(i < n) {
    scanf("%d",&pm25);
    sum += pm25;
    if(pm25 >= 0 && pm25 <= 50 ) day[0]++;
    else if(pm25 >= 51 && pm25 <= 100 ) day[1]++;
    else if(pm25 >= 101 && pm25 <= 150 ) day[2]++;
    else if(pm25 >= 151 && pm25 <= 200 ) day[3]++;
    else if(pm25 >= 201 && pm25 <= 300 ) day[4]++;
    else day[5]++;
    i++;
  } 
\end{lstlisting}
};
\node[anchor=north west,text width=0.55\textwidth,draw] at(a.north east) {
\vspace{-0.3cm}
\begin{lstlisting}
	printf("%.2f\n",(float)sum/n);
	for(i = 0; i < 6; i++)
		if(i == 5) printf("%d\n",day[i]);
		else  printf("%d ",day[i]);
	return 0;
}
\end{lstlisting}
};
\end{tikzpicture}
\end{frame}

\begin{frame}[shrink,fragile]{例: PM2.5(模块化设计, 主程序)}
\vspace{-0.2cm}
\begin{tikzpicture}
\node[text width=0.5\textwidth,draw] (a) {
\vspace{-0.3cm}
\begin{lstlisting}
#include <stdio.h>
// 函数声明
float haze(int pm25[], int day[], int n, int m);
#define N 1000 // 估计的最大值
int main()
{
   int i =0,n,pm25[N],day[6];
   scanf("%d",&n);
   while(i < n) 
   {
     scanf("%d",&pm25[i]);
     i++;
   } 
} 
\end{lstlisting}
};
\node[anchor=north west,text width=0.55\textwidth,draw] at(a.north east) {
\vspace{-0.3cm}
\begin{lstlisting}
	 // 函数调用,并输出函数中计算的平均值
	printf("%.2f\n",haze(pm25,day,n,6));
	// 输出函数中计算的数组元素值
	for(i = 0; i < 6; i++)
		if(i == 5) printf("%d\n",day[i]);
		else  printf("%d ",day[i]);
	return 0;
}
\end{lstlisting}
};
\end{tikzpicture}
\end{frame}

\begin{frame}[shrink,fragile]{例: PM2.5(模块化设计, 雾霾统计信息)}
\vspace{-0.4cm}
\begin{tikzpicture}
\node[text width=.8\textwidth] (a) {
\begin{lstlisting}
float haze(int pm25[], int day[], int n, int m)
{
  int i=0, sum=0;
  // 初始化day数组
  for(i=0; i<m; i++) day[i]=0;
  while(i < n) 
  {
    sum += pm25[i];
    if(pm25[i] >= 0 && pm25[i] <= 50 ) day[0]++;
    else if(pm25[i] >= 51 && pm25[i] <= 100 ) day[1]++;
    else if(pm25[i] >= 101 && pm25[i] <= 150 ) day[2]++;
    else if(pm25[i] >= 151 && pm25[i] <= 200 ) day[3]++;
    else if(pm25[i] >= 201 && pm25[i] <= 300 ) day[4]++;
    else day[5]++;
    i++;
  } 
  return (float)sum/n;
} 
\end{lstlisting}
};

\node[anchor=north west,text width=.5\textwidth] (b) at($(a.north east)+(-0.5,-1)$) {
	函数参数说明: pm25[]: PM2.5值, day[]: 不同标准对应的统计天数; n是pm25数组的长度, m是数组day的长度, 返回PM2.5平均值.
};

\node[anchor=north,text width=.5\textwidth,fill=green] at(b.south) {
	\textcolor{blue}{要点:}
	
	用数组做参数, 可获得函数计算结果的多值传递;
	
	而return仅返回一个值。
};
\end{tikzpicture}
\end{frame}

\begin{frame}[shrink,fragile]{例: 数字排序(主程序)}
给定n个整数,请计算每个整数各位数字和,按各位数字和从大到小的顺序输出。
\begin{lstlisting}
#include <stdio.h>
#define N 1000 // 估计的数组大小
int bitsSum(int a); // 计算整数a的各位之和
void sort(int a[], int n); // 从大到小排序
void output(int a[], int n); // 输出
int main()
{
   int i,n; // n是实际数组长度 
   int num[N],sum[N]; // num表示N个整数, sum存放对应整数的各位数字的和  
   scanf("%d",&n);
   for(i=0;i<n;i++) { scanf("%d",&num[i]);  sum[i]=bitsSum(num[i]); }
   sort(sum,n);  // 从大到小排序
   output(sum,n); // 输出
   return 0;
}
\end{lstlisting}
\end{frame}

\begin{frame}[shrink,fragile]{例: 数字排序(子函数: 计算整数的各位之和, 输出)}
\begin{columns}[T]
	\column{0.4\textwidth}
\begin{lstlisting}
// 计算整数a的各位之和
int bitsSum(int a)
{
	int sum=0;
	while(a)
	{
		sum += a%10;
		a /= 10;
	}
	return sum;
}
\end{lstlisting}
\column{0.4\textwidth}
\begin{lstlisting}[frame=leftline]
// 输出, n是数组长度
void output(int a[], int n)
{
	int i;
	for(i=0;i<n;i++) 
		printf("%d ",a[i]);
	printf("\n");
}
\end{lstlisting}
\end{columns}
\end{frame}

\begin{frame}[shrink,fragile]{例: 数字排序(排序子函数---方法1: 冒泡排序)}
\begin{lstlisting}
// 从大到小排序, 冒泡, n是数组长度
 void sort(int a[], int n)
{
   int i,j,flag,temp;
   for(j = 1; j <= n-1; j++) // 第j趟比较
   {
      flag=0;
      for(i = 0; i < n - j; i++) // 相邻两数比较
      {
        if (a[i] < a[i+1]) // 交换
        { 
           temp = a[i]; a[i] = a[i+1]; a[i+1] = temp; 
           flag=1;
        }
      }
      if(!flag) break;
   }
}
\end{lstlisting}
\end{frame}

\begin{frame}[shrink,fragile]{选择法排序}
\begin{tikzpicture}[node distance=0.5cm, on grid,every label/.style={blue,outer sep=0pt,inner sep=2pt,align=center,font=\scriptsize}]
\node[text width=\textwidth] (a) {
	从大到小\textcolor{blue}{选择法排序}: 先将$n$个数中最大的数与$a[0]$对换, $a[0]$就是最大的数;
	
	再将$a[1]\sim a[n-1]$中最大的数与$a[1]$对换,$\cdots$, 每比较一轮,找出一个未经排序的数中最大的一个。共比较$n-1$轮。
	
	\begin{lstlisting}
	int a[]={5,8,2,9,3};  // 待排序的数
	int j,k; // j: 外层循环次数, 共$n-1$次, k: 每次循环选择的最大数下标
	\end{lstlisting}
};

\node[] at ($(a.south west)+(2,0)$) (aa) {$a[0]$};
\node[] (a1) [below= of aa] {$a[1]$};
\node[] (a2) [below= of a1] {$a[2]$};
\node[] (a3) [below= of a2] {$a[3]$};
\node[] (a4) [below= of a3] {$a[4]$};

%% j=0
\node[fill=green] (a0) [right=1cm of aa] {5};
\node[] (a1) [below= of a0] {8};
\node[] (a2) [below= of a1] {2};
\node[fill=red] (a3) [below= of a2] {9};
\node[] (a4) [below= of a3] {3};

\node[] (b0) [right=0.5cm of a0] {9};
\node[] (b1) [below= of b0] {8};
\node[] (b2) [below= of b1] {2};
\node[] (b3) [below= of b2] {5};
\node[] (b4) [below= of b3] {3};
\node[fit=(a0) (b4),label={south:$j=0,k=j$\\找出$k=3$\\交换$a[j],a[k]$},draw,dashed] {};

%% j=1
\node[] (a0) [right=1.5cm of b0] {9};
\node[fill=green] (a1) [below= of a0] {8};
\node[] (a2) [below= of a1] {2};
\node[] (a3) [below= of a2] {5};
\node[] (a4) [below= of a3] {3};

\node[] (b0) [right=0.5cm of a0] {9};
\node[] (b1) [below= of b0] {8};
\node[] (b2) [below= of b1] {2};
\node[] (b3) [below= of b2] {5};
\node[] (b4) [below= of b3] {3};
\node[fit=(a0) (b4),label={south:$j=1,k=j$\\找出$k=1$\\$k=j$无需交换},draw,dashed] {}; 

%% j=2
\node[] (a0) [right=1.5cm of b0] {9};
\node[] (a1) [below= of a0] {8};
\node[fill=green] (a2) [below= of a1] {2};
\node[fill=red] (a3) [below= of a2] {5};
\node[] (a4) [below= of a3] {3};

\node[] (b0) [right=0.5cm of a0] {9};
\node[] (b1) [below= of b0] {8};
\node[] (b2) [below= of b1] {5};
\node[] (b3) [below= of b2] {2};
\node[] (b4) [below= of b3] {3};
\node[fit=(a0) (b4),label={south:$j=2,k=j$\\找出$k=3$\\交换$a[j],a[k]$},draw,dashed] {};

%% j=3
\node[] (a0) [right=1.5cm of b0] {9};
\node[] (a1) [below= of a0] {8};
\node[] (a2) [below= of a1] {5};
\node[fill=green] (a3) [below= of a2] {2};
\node[fill=red] (a4) [below= of a3] {3};

\node[] (b0) [right=0.5cm of a0] {4};
\node[] (b1) [below= of b0] {3};
\node[] (b2) [below= of b1] {2};
\node[] (b3) [below= of b2] {1};
\node[] (b4) [below= of b3] {0};
\node[fit=(a0) (b4),label={south:$j=3,k=j$\\找出$k=4$\\交换$a[j],a[k]$},draw,dashed] {};
\end{tikzpicture}
\end{frame}

\note{
	\begin{lstlisting}
	下标     0 1 2 3 4
	j=0,k=j, 1 2 3 4 5==>max=5,k=4
	j=1,k=j, 5 2 3 4 1==>max=4,k=3
	j=2,k=j, 5 4 3 2 1==>max=3,k=2
	j=3,k=j, 5 4 3 2 1 break
	\end{lstlisting}
	
}

\begin{frame}[shrink,fragile]{例: 数字排序(排序子函数---方法2: 选择法排序)}
\vspace{-0.3cm}
\begin{lstlisting}
// 从大到小排序, 选择
void sort(int a[], int n)
{
   int i,j,k,temp;
   for(j = 0; j <= n-2; j++) // 第j趟比较, 共n-1次循环
   {
     k=j; // 首先假设j就是未经排序的数中, 最大元素的下标
     for(i = j+1; i < n; i++) // 未经排序的数从j+1开始
     {
         if (a[i] > a[k]) k=i; // 如果第i个元素比第k个元素大, 置换k为i
     }
     if(k!=j)//未经排序数中的最大元素与a[j]交换, a[j]及其以前元素是已经排好的数据
     { 
        temp = a[j]; a[j] = a[k]; a[k] = temp; 
     }
   }
}
\end{lstlisting}
\end{frame}

\begin{frame}[shrink,fragile]{例: 数字排序(联动输出: 修改主程序)}
给定n个整数,请计算每个整数各位数字和,按各位数字和从大到小的顺序输出。\textbf{要求联动输出: 整数\quad 该整数的各位之和}
\begin{lstlisting}
#include <stdio.h>
#define N 1000 // 估计的数组大小
int bitsSum(int a); // 计算整数a的各位之和
void sort(int a[], int b[], int n);  // a从大到小排序, b联动改变
void output(int a[], int b[], int n); // 输出a,b
int main()
{
   int i,n; // n是实际数组长度 
   int num[N],sum[N]; // num表示N个整数, sum存放对应整数的各位数字的和  
   scanf("%d",&n);
   for(i=0;i<n;i++) { scanf("%d",&num[i]); sum[i]=bitsSum(num[i]); }
   sort(sum,num,n);  // sum从大到小排序,num联动改变
   output(num,sum,n); // 输出num,sum
   return 0;
}
\end{lstlisting}
\end{frame}

\begin{frame}[shrink,fragile]{例: 数字排序(联动输出: 修改排序子函数)}
\begin{lstlisting}
// a从大到小排序, b联动改变
void sort(int a[], int b[],int n)
{
   int i,j,flag,temp;
   for(j = 1; j <= n-1; j++) // 第j趟比较
   {
       flag=0;
       for(i = 0; i < n - j; i++) // 相邻两数比较
       {
          if (a[i] < a[i+1]) // 同时交换a和b
          { 
             temp = a[i]; a[i] = a[i+1]; a[i+1] = temp; 
             temp = b[i]; b[i] = b[i+1]; b[i+1] = temp; 
             flag=1;
          }
       }
       if(!flag) break;
    }
}
\end{lstlisting}
\end{frame}

\begin{frame}[shrink,fragile]{例: 数字排序(联动输出: 修改输出子函数)}
\begin{lstlisting}
// 输出a,b
void output(int a[], int b[], int n)
{
   int i;
   for(i=0;i<n;i++) printf("%d %d",a[i],b[i]);
   printf("\n");
}
\end{lstlisting}
\end{frame}

\begin{frame}[shrink,fragile]{例: 数字排序(联动输出: 修改主程序,二维数组)}
给定n个整数,请计算每个整数各位数字和,按各位数字和从大到小的顺序输出。\textbf{要求联动输出: 整数\quad 该整数的各位之和}
\begin{lstlisting}
#include <stdio.h>
#define N 1000 // 估计的数组大小
int bitsSum(int a); // 计算整数a的各位之和
void sort(int a[][2], int n);  // 按a的第0列从大到小排序
void output(int a[][2], int n); // 输出a
int main()
{
   int i,n; // n是实际数组长度 
   int num[N][2]; // 第0列表示整数, 第1列是该整数的各位数字的和  
   scanf("%d",&n);
   for(i=0;i<n;i++) 
   { scanf("%d",&num[i][0]); num[i][1]=bitsSum(num[i][0]); }
   sort(num,n);  // sum从大到小排序,num联动改变
   output(num,n); // 输出num
   return 0;
}
\end{lstlisting}
\end{frame}

\begin{frame}[shrink,fragile]{例: 数字排序(联动输出: 修改排序子函数,二维数组)}
\begin{lstlisting}
// 按a的第0列从大到小排序
void sort(int a[][2],int n)
{
   int i,j,flag,temp;
   for(j = 1; j <= n-1; j++) // 第j趟比较
   {
      flag=0;
      for(i = 0; i < n - j; i++) // 相邻两数比较
      {
          if (a[i][0] < a[i+1][0]) // 同时交换a的第0列和第1列
          { 
            temp = a[i][0]; a[i][0] = a[i+1]; a[i+1][0] = temp; 
            temp = a[i][1]; a[i][1] = a[i+1][1]; b[i+1][1] = temp; 
            flag=1;
          }
      }
      if(!flag) break;
   }
}
\end{lstlisting}
\end{frame}

\begin{frame}[shrink,fragile]{例: 数字排序(联动输出: 修改输出子函数, 二维数组)}
\begin{lstlisting}
// 输出a
void output(int a[][2], int n)
{
   int i;
   for(i=0;i<n;i++) printf("%d %d",a[i][0],a[i][1]);
   printf("\n");
}
\end{lstlisting}

\begin{block}{模块化程序设计的优点}
	综上,程序功能的变化,仅修改相应子函数即可, 逻辑清晰。\\
	因此,利用函数进行功能分解是进行复杂程序设计的有效手段。
\end{block}
\end{frame}






