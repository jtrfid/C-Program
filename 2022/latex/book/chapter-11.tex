%%%%%%%%%%%%%%%%%%%%% chapter.tex %%%%%%%%%%%%%%%%%%%%%%%%%%%%%%%%%
%
% sample chapter
%
% Use this file as a template for your own input.
%
%%%%%%%%%%%%%%%%%%%%%%%% Springer-Verlag %%%%%%%%%%%%%%%%%%%%%%%%%%
%\motto{Use the template \emph{chapter.tex} to style the various elements of your chapter content.}
\chapter{提交代码片段题型练习}  

提交代码片段即是根据题目要求,仅复制相关代码片段,然后粘贴至提交页面指定区域即可。

\section{字符串压缩}

\section*{问题描述:}

有一种简单的字符串压缩算法,对于字符串中连续出现的同一个英文字符,用该字符加上连续出现的次数来表示(连续出现次数小于3时不压缩)。

例如,字符串aaaaabbbabaaaaaaaaaaaaabbbb可压缩为a5b3aba13b4。

请设计一个程序,采用该压缩方法对字符串压缩并输出。请编写一个函数compress,采用该压缩方法对字符串src进行压缩。函数定义如下:
char *compress(char *src);
返回值:
指向压缩后的字符串数据
参数:
src:输入/输出参数,输入表示待压缩字符串,输出表示压缩后的字符串

\textcolor{blue}{注意:函数声明已包含在主程序中,不需要自己定义。只需要提交自定义的函数代码。}

\begin{lstlisting}[frame=none]
主程序如下:
#include<stdio.h>
#include<string.h>
#include<stdlib.h>
char *compress(char *src);
int main()
{
	char src[100];
	scanf("%s",src);
	
	char *ps = compress(src);
	
	puts(ps);
	return 0;
}
\end{lstlisting}

\textbf{主函数输入说明:} 输入第一行为源字符串src(长度小于100),该字符串只包含大小写字母。

\textbf{主函数输出说明:} 输出压缩后的字符串。

\textbf{输入样例: } aaaaabbbabaaaaaaaaaaaaabbbb

\textbf{输出样例: } a5b3aba13b4

\section{函数\lstinline|char *compress(char *src)|设计分析}

\begin{enumerate}[(1)]
	\item 关键局部变量:
	
	 \lstinline|char *p1,*p2; // p1,p2指向每次读取src的区间| 
	
	 \lstinline|char dest[100]; // 存储压缩字符串| 
	 
	 \lstinline|int count; // 连续相同字符计数| 
	
	\item 首先指针$p_1,p_2$指向字符串第一个字符, 
	
	$\underset{p_1,p_2}{\underset{\uparrow}{a}}aaaabcccccd$
	
	\item \lstinline|count=0; while(*p1==*p2) {p2++; count++;}|
	
	$\underset{p_1}{\underset{\uparrow}{a}}aaaa\underset{p_2}{\underset{\uparrow}{b}}cccccd$
	
	\item $*p_1$添加至dest末尾.
	
	\lstinline|char tmp[2]; tmp[0]=*p1; tmp[1]='\0'; strcat(dest,tmp);|
	
	\item \lstinline|if (count>=3)|, 将其转换为字符串strCount, 添加至dest末尾. 对于\lstinline|count=1,2|的情况处理简单(略).
	
	\lstinline|strcat(dest,strCount);|
	
	\item \lstinline|p1=p2;| 重复step (2)至(5), 直至src结束.
	
	$aaaaa\underset{p_1,p2}{\underset{\uparrow}{b}}cccccd$ 
	
	\item \lstinline|strcpy(src,dest); return src;| 
	
	 \lstinline|return dest; //直接返回dest是错误的.| 
	 
	 因为指针变量dest是局部变量,dest中存储的地址, 离开函数后会失效。 
	  
\end{enumerate}



\section{解法一:自定义函数将count转为字符串}

\begin{lstlisting}
// 整数转字符串,要求p以'\0'结尾. 
// 参考课件递归函数部分【例: 生成正整数对应的二进制字符串】 
// 完成同样功能stdlib.h中库函数char *itoa(int i,char *s,int radix);  
void IntStr(int count,char *p)
{
	char r[2];
	// 如果多于两位数,递归,未计算的count压入栈中 
	if (count>=10) IntStr(count/10,p); 
	// pop栈中的count 
	r[0]=count%10+'0'; 
	r[1]='\0'; // 确保字符串以'\0'结尾  
	strcat(p,r); 
}

char *compress(char *src)
{
	// p1,p2指向每次读取src的区间,dest存储压缩字符串。 
	char *p1=src,*p2=src,dest[100]={'\0'};
	int count; // 连续相同字符计数 
	char tmp[100];
	
	// 读取p1,p2指向的src中的区间,如果相同字符, p2++; 
	while(*p2!='\0')
	{
		count=0;
		while(*p1==*p2)	{ p2++; count++; }
		if(count==1) { tmp[0]=*p1; tmp[1]='\0'; strcat(dest,tmp); }  // 追加字符本身 
		else if(count==2) // 追加2个字符本身 
		{ tmp[0]=*p1; tmp[1]=*p1; tmp[2]='\0'; strcat(dest,tmp); }
		else // count>=3 
		{
			tmp[0]=*p1; tmp[1]='\0'; strcat(dest,tmp); // 追加字符本身
			tmp[0]='\0'; // 确保字符串以'\0'结尾  
			IntStr(count,tmp); // // 转换count为数字字符串
			strcat(dest,tmp); // 追加count字符串 
		}
		p1=p2; // 准备下轮读取 
	}
	strcpy(src,dest); // src存储的地址与主函数共享 
	return src;  // 返回来自参数的指针变量传输的地址.
	// 不能返回dest. 因为指针变量dest是局部变量,dest中存储的地址, 离开函数后会失效。 
	// return dest;  
}
\end{lstlisting}

\section{解法二:使用stdlib.h中的库函数char *itoa(int i,char *s,int radix), 将count转为字符串}

\begin{lstlisting}
// 使用库函数count转为字符串 
char *compress(char *src)
{
	// p1,p2指向每次读取src的区间,dest存储压缩字符串。 
	char *p1=src,*p2=src,dest[100]={'\0'};
	int count; // 连续相同字符计数 
	char tmp[100];
	
	// 读取p1,p2指向的src中的区间,如果相同字符, p2++; 
	while(*p2!='\0')
	{
		count=0;
		while(*p1==*p2) { p2++; count++; }
		if(count==1) { tmp[0]=*p1; tmp[1]='\0'; strcat(dest,tmp); } // 追加字符本身 
		else if(count==2) // 追加2个字符本身 
		{ tmp[0]=*p1; tmp[1]=*p1; tmp[2]='\0'; strcat(dest,tmp); }
		else // count>=3 
		{
			tmp[0]=*p1; tmp[1]='\0'; strcat(dest,tmp); // 追加字符本身 
			tmp[0]='\0'; // 确保字符串以'\0'结尾  
			itoa(count,tmp,10);//转count为数字字符串, 10表示转为十进制数字 
			strcat(dest,tmp); // 追加count字符串 
		}
		p1=p2; // 准备下轮读取 
	}
	strcpy(src,dest); // src存储的地址与主函数共享 
	return src;  // 返回来自参数的指针变量传输的地址.
	// 不能返回dest. 因为指针变量dest是局部变量,dest中存储的地址, 离开函数后会失效。 
	// return dest;  
}
\end{lstlisting}


