%%%%%%%%%%%%%%%%%%%%% chapter.tex %%%%%%%%%%%%%%%%%%%%%%%%%%%%%%%%%
%
% sample chapter
%
% Use this file as a template for your own input.
%
%%%%%%%%%%%%%%%%%%%%%%%% Springer-Verlag %%%%%%%%%%%%%%%%%%%%%%%%%%
%\motto{Use the template \emph{chapter.tex} to style the various elements of your chapter content.}
\chapter{第7次机试练习: 函数,数组,排序,字符串练习}

\section{消除类游戏}
消除类游戏是深受大众欢迎的一种游戏,游戏在一个包含有n行m列的游戏棋盘上进行,棋盘的每一行每一列的方格上放着一个有颜色的棋子,当一行或一列上有连续三个或更多的相同颜色的棋子时,这些棋子都被消除。当有多处可以被消除时,这些地方的棋子将同时被消除。

现在给你一个n行m列的棋盘,棋盘中的每一个方格上有一个棋子,请给出经过一次消除后的棋盘。

请注意:一个棋子可能在某一行和某一列同时被消除。

输入格式

输入的第一行包含两个整数n, m,用空格分隔,分别表示棋盘的行数和列数。

接下来n行,每行m个整数,用空格分隔,分别表示每一个方格中的棋子的颜色。颜色使用1至9编号。

输出格式

输出n行,每行m个整数,相邻的整数之间使用一个空格分隔,表示经过一次消除后的棋盘。如果一个方格中的棋子被消除,则对应的方格输出0,否则输出棋子的颜色编号。

样例输入1

4 5

2 2 3 1 2

3 4 5 1 4

2 3 2 1 3

2 2 2 4 4

样例输出1

2 2 3 0 2

3 4 5 0 4

2 3 2 0 3

0 0 0 4 4

样例说明

棋盘中第4列的1和第4行的2可以被消除,其他的方格中的棋子均保留。

样例输入2

4 5

2 2 3 1 2

3 1 1 1 1

2 3 2 1 3

2 2 3 3 3

样例输出2

2 2 3 0 2

3 0 0 0 0

2 3 2 0 3

2 2 0 0 0

样例说明

棋盘中所有的1以及最后一行的3可以被同时消除,其他的方格中的棋子均保留。
评测用例规模与约定

所有的评测用例满足:$1 \le n, m \le 30$。

\begin{lstlisting}
#include <stdio.h>
#include <math.h>
#define M 30 // 估计数组最大长度

// 扫描curRow行中可删除的元素, F: 删除标志矩阵, m: 列数 
void delRow(int curRow, int checker[][M], int F[][M],int m)
{
	int i,j, current,count;
	for(i=0;i<m;i++) 
	{
		// if(F[curRow][i]==1) continue; // 如果先扫描行,这就不合适了 
		count=1; 
		current=checker[curRow][i];
		for(j=i+1;j<m;j++)
		{
			if(current==checker[curRow][j]) count++;
			else break;
		}
		if(count>=3)
		{
			for(j=i;j<m;j++)
			{
				if(checker[curRow][j]==current) F[curRow][j]=1;//置删除标志
				else break; 
			}
		}
	} 
}

// 扫描curCol列中可删除的元素, F: 删除标志矩阵, n: 行数 
void delCol(int curCol, int checker[][M], int F[][M],int n)
{
	int i,j, current,count;
	for(i=0;i<n;i++) 
	{
		count=1; 
		current=checker[i][curCol];
		for(j=i+1;j<n;j++)
		{
			if(current==checker[j][curCol]) count++;
			else break;
		}
		if(count>=3)
		{
			for(j=i;j<n;j++)
			{
				if(checker[j][curCol]==current) F[j][curCol]=1;//置删除标志
				else break; 
			}
		}
	} 
}

// 清除所有可删除的元素, F: 删除标志矩阵, n: 行数, m: 列数 
void del(int checker[][M], int F[][M],int n, int m)
{
	int i,j;
	for(i=0;i<n;i++)
	{
		for(j=0;j<m;j++)
		{
			if(F[i][j]==1) checker[i][j]=0;
		}
	}
}

// 读棋盘, n: 行数, m: 列数
void Read(int checker[][M],int n, int m)
{
	int i,j;
	for(i=0;i<n;i++)
	{
		for(j=0;j<m;j++)
		{
			scanf("%d",&checker[i][j]);
		}
	}
}

// 输出棋盘, n: 行数, m: 列数
void output(int checker[][M],int n, int m)
{
	int i,j;
	for(i=0;i<n;i++) // 行 
	{
		for(j=0;j<m;j++) // 列 
		{
			printf("%d ",checker[i][j]); 
		}
		printf("\n");
	} 
}

int main()
{
	// 棋盘,初始化是为了便于测试 
	int checker[M][M]={ {2,2,3,1,2},
						{3,4,5,1,4},
						{2,3,2,1,3},
						{2,2,2,4,4} };
	int n=4,m=5,i,j; // n行,m列 
	/************
	int checker[M][M]={ {2,2,3,1,2},
						{3,1,1,1,1},
						{2,3,2,1,3},
						{2,2,3,3,3} };
	int n=4,m=5,i,j; // n行,m列 
	*************/
	// 标志矩阵 
	int F[M][M]; // 对应元素为1,则表示删除 
	
	scanf("%d%d",&n,&m);
	// 读棋盘 
	Read(checker,n,m);
	
	// 初始化F 
	for(i=0;i<n;i++)
	for(j=0;j<m;j++)
	F[i][j]=0;

	// 扫描行 
	for(i=0;i<n;i++) // 行 
	delRow(i, checker, F, m);
	
	// 扫描列 
	for(j=0;j<m;j++) // 列 
	delCol(j, checker, F, n);
	
	// 清除 
	del(checker, F, n, m);
	
	// 输出棋盘 
	output(checker,n,m); 
	
	return 0;																	
} 
\end{lstlisting}

\begin{note}[要点]
	标志矩阵,体会模块化编程思想,初始化变量的程序调试技巧。
\end{note}

\section{表达式求值}
表达式由两个非负整数x,y和一个运算符op构成,求表达式的值。
这两个整数和运算符的顺序是随机的,可能是``x op y", ``op x y"或者``x y op",例如, ``25 + 3"表示25加3, ``5 30 *"表示5乘以30, "/ 600 15"表示600除以15。

输入说明

输入为一个表达式,表达式由两个非负整数x,y和一个运算符op构成,x,y和op之间以空格分隔,但顺序不确定。

x和y均不大于10000000,op可以是+, -,*, /, \%中的任意一种, 分表表示加法, 减法, 乘法, 除法和求余。

除法按整数除法求值, 输入数据保证除法和求余运算的y值不为0。

输出说明	

输出表达式的值。

输入样例

样例1输入

5 20 *

样例2输入

4 + 8

样例3输入

/ 8 4

输出样例

样例1输出

100

样例2输出

12

样例3输出

2

\begin{lstlisting}
#include <stdio.h>
// 估计字符串最大长度,存储有效字符(N-1)个,预留最后一个字符'\0' 
#define N 20 

// 根据参数,计算表达式的值 
int compute(char op,int x,int y)
{
	int result = -1;
	switch(op)
	{
		case '+': result = x+y; break;
		case '-': result = x-y; break;
		case '*': result = x*y; break;
		case '/': if(y != 0) result = x/y; break;
		case '%': if(y != 0) result = x%y; break;
	}
	return result;
}

// 数字字符串s转为int, 要求s以'\0'结尾 
int strToInt(char *s)// int toInt(char s[]) 
{
	int result=0;
	while(*s) // 等效while(*s != '\0')或while(*s!=0)
	{
		result=result*10+ (*s-'0');
		s++; //移至下一字符 
	}
	return result;
} 

/******************************************************
提取子串函数 
忽略s中空格前缀,复制s中的字符串到subs中,遇空格或'\0'结束
返回subs不含空格。 返回复制后s指针指向(地址) 
要求s和subs以'\0'结尾。
*******************************************************/ 
char* getSubs(char *s, char *subs) 
{
	int start=0; 
	while(*s)
	{
		if(*s==' ') 
		{
			if(start==0) s++; // 忽略s的前缀空格 
			else break; // 是有效字符串后的一个空格 
		}
		else
		{
			start=1; // 开始复制 
			*subs=*s;
			s++;
			subs++;
		}
	}
	*subs='\0'; // 不要忘记结尾符 
	return s;
}

// 解析s, 以空格为分隔符, 分解s为3个字符串 
void parse(char *s,char result[][N])
{
	char *p;
	p=getSubs(s,result[0]);
	p=getSubs(p,result[1]);
	p=getSubs(p,result[2]);
}	

// 如果s是操作符,返回1, 参数op返回该操作符
// 否则, 返回0 
int isOp(char *s, char *op)
{
	if(*s >= '0' && *s <= '9') // 数字
		return 0;
	else // 操作符
	{
		*op=*s;
		return 1;
	}
} 

int main()
{
	char s[N],op;
	char s3[3][N];   
	int x,y;
	gets(s); 
	
	parse(s,s3); // s被分解为3个字符串 
	if(isOp(s3[0],&op)) // op x y
	{
		x=strToInt(s3[1]);
		y=strToInt(s3[2]);
	}
	else if(isOp(s3[1],&op)) // x op y
	{
		x=strToInt(s3[0]);
		y=strToInt(s3[2]);
	}
	else if(isOp(s3[2],&op)) // x y op
	{
		x=strToInt(s3[0]);
		y=strToInt(s3[1]);
	}

	printf("%d\n",compute(op,x,y)); 
	return 0;
}

// 简单实现, 直接读取三个子串, 就不用分解了
int main1() 
{
	char s[N],op;
	char s3[3][N];   
	int x,y;
	scanf("%s%s%s",s3[0],s3[1],s3[2]); // 利用"%s"读字符串遇空格结束特点,直接读取3个字符串。 
	
	// parse(s,s3); // s被分解为3个字符串 
	if(isOp(s3[0],&op)) // op x y
	{
		x=strToInt(s3[1]);
		y=strToInt(s3[2]);
	}
	else if(isOp(s3[1],&op)) // x op y
	{
		x=strToInt(s3[0]);
		y=strToInt(s3[2]);
	}
	else if(isOp(s3[2],&op)) // x y op
	{
		x=strToInt(s3[0]);
		y=strToInt(s3[1]);
	}
	
	printf("%d\n",compute(op,x,y)); 
	return 0;
}
\end{lstlisting}

\begin{note}[要点]
	体会字符串处理, 指针应用, 模块化程序设计思想。
\end{note}

\section{排序}	
给定N个不同的整数,要求对这N个整数按如下规则排序并输出。

规则一:所有的偶数排在奇数前面。

规则二:在规则一的前提下按照从大到小的顺序排序。

输入说明	

数据由两行构成,第一行为整数n (n<=100), 表示待排序整数的数量。第二行是n个整数,每个整数的取值区间都为$[-32768\sim32767]$,整数之间以空格间隔。

输出说明
	
在一行输出排好序的整数,整数之间以空格间隔。

输入样例	

5 

1 2 3 4 5

输出样例
	
4 2 5 3 1

\begin{lstlisting}
# include <stdio.h>
#define N 100

// 选择法从大到小排序 
int sorts(int a[],int n)
{
	int i,j,k,tmp;
	for(i=0;i<n-1;i++)
	{
		k=i; // 未排序中较大者 
		for(j=i+1;j<n;j++)
		if(a[j]>a[k]) k=j;
		if(k!=i)
		{
			tmp=a[i];
			a[i]=a[k];
			a[k]=tmp;
		} 
	}
}

int main()
{
	int even[N],odd[N],n,i,num;
	int evenN=0, oddN=0;
	scanf("%d",&n);
	for(i=0;i<n;i++)
	{
		scanf("%d",&num);
		if(num%2==0) // even
		{
			even[evenN]=num;
			evenN++;
		}
		else
		{
			odd[oddN]=num;
			oddN++;
		}
	} 
	// 排序
	sorts(even,evenN);
	sorts(odd,oddN);
	// 输出 
	for(i=0;i<evenN;i++) printf("%d ",even[i]); 
	for(i=0;i<oddN;i++) printf("%d ",odd[i]); 
	printf("\n"); 
	return 0;
}
\end{lstlisting}

\begin{note}[要点]
	各种形式的排序问题(整数, 字符, 字符串, 结构体)是必须掌握的重点之一。选择法排序或冒泡法排序必须信手拈来。
\end{note}

\section{等差数列}	
请写一个程序,判断给定整数序列能否构成一个等差数列。

输入说明
	
输入数据由两行构成,第一行只有一个整数n(n<100), 表示序列长度(该序列中整数的个数);

第二行为n个整数,每个整数的取值区间都为$[-32768\sim 32767]$, 整数之间以空格间隔。

输出说明	

对输入数据进行判断,不能构成等差数列输出``no", 能构成等差数列输出表示数列公差(相邻两项的差)的绝对值的一个整数。

输入样例
	
样例1输入

6

23 15 4 18 35 11

样例2输入

5

2 6 8 4 10

输出样例
	
样例1输出

no

样例2输出

2

\begin{lstlisting}
# include <stdio.h>
#define N 100

// 选择法从大到小排序 
int sorts(int a[],int n)
{
	int i,j,k,tmp;
	for(i=0;i<n-1;i++)
	{
		k=i; // 未排序中较大者 
		for(j=i+1;j<n;j++)
		if(a[j]>a[k]) k=j;
		if(k!=i)
		{
			tmp=a[i];
			a[i]=a[k];
			a[k]=tmp;
		} 
	}
}

// 排序后的a是否构成等差数列,是返回1,否则返回0
// 若是等差数列, 指针参数tolerance返回公差的绝对值。 
int check(int a[],int n, int *tolerance)
{
	int i;
	*tolerance = a[0]-a[1]; // 大-小,不用求绝对值 
	for(i=1;i<n-1;i++) // 注意检查数组越界问题
	{
		if(a[i]-a[i+1] != *tolerance) return 0;
	} 
	return 1;
}

int main()
{
	int a[N],n,i,tolerance;
	scanf("%d",&n);
	for(i=0;i<n;i++)
	{
		scanf("%d",&a[i]);
	} 
	// 排序
	sorts(a,n);
	// check
	if(check(a,n, &tolerance)) printf("%d\n",tolerance);
	else printf("no\n");
	return 0;
}
\end{lstlisting}

\begin{note}[要点]
	数组越界问题, 排序, 地址传递, 模块化程序设计。
\end{note}

\section{0-1矩阵}	
查找一个只包含0和1的矩阵中每行最长的连续1序列。

输入说明
	
输入第一行为两个整数m和n (0<=m,n<=100)表示二维数组行数和列数,其后为m行数据,每行n个整数(0或1), 输入数据中不会出现同一行有两个最长1序列的情况。

输出说明
	
找出每一行最长的连续1序列,输出其起始位置(从0开始计算)和结束位置(从0开始计算),如果这一行没有1则输出两个-1,然后换行。

输入样例 
	
5 6

1 0 0 1 1 0

0 0 0 0 0 0

1 1 1 1 1 1

1 1 1 0 1 1

0 0 1 1 0 0

输出样例
	
3 4

-1 -1

0 5

0 2

2 3

\begin{lstlisting}
#include <stdio.h>
#define N 100

int main()
{
	// 标志变量start,end, maxLen: 记录本行最长连续1总体情况
	// 标志变量tmpStart, tmpLen: 标志是否开始连续1及连续长度 
	int a[N][N],m,n,i,j,start,end,tmpStart,maxLen,tmpLen;
	scanf("%d%d",&m,&n);
	for(i=0;i<m;i++)
		for(j=0;j<n;j++)
			scanf("%d",&a[i][j]);
	
	// 遍历各行		
	for(i=0;i<m;i++) // 行
	{
		start=-1; end=-1; maxLen=0; // 本行连续1总体情况
		tmpStart=-1; tmpLen=0; // 标志是否开始连续1及连续长度
		for(j=0;j<n;j++) // 列
		{
			if(a[i][j]==1) // 1开始 
			{
				if (tmpStart==-1) tmpStart=j; // 开始记录 
				tmpLen++;
			}
			else // 遇0结算 
			{
				if(tmpStart!=-1 && tmpLen>maxLen) 
				{
					start=tmpStart;
					end=j-1;
					maxLen=tmpLen;
					tmpStart=-1;
					tmpLen=0;
				}
			}
		}
		if(tmpStart!=-1 && tmpLen>maxLen) // 末尾是1的情况 
		{
			start=tmpStart;
			end=n-1;
		}
		printf("%d %d\n",start,end);
	}
	return 0;
}
\end{lstlisting}

\begin{note}[要点]
	合理使用标志变量, 训练自己的逻辑思维能力。
\end{note}

\section{寻找最长的行}	
寻找若干行文本中最长的一行

输入说明	

输入为多个字符串(每个字符串长度不超过100个字符),每个字符串占一行,输入的行为``***end***"时表示输入结束

输出说明
	
输出其中最长的一行长度后换行再输出最长行的内容,如果最长行不止一个,则输出其中的第一行。

输入样例
	
abce

abdf dlfd

***end***

输出样例
	
9

abdf dlfd 

\begin{lstlisting}
#include <stdio.h>
#include <string.h> // str前缀的字符串处理函数需要此头文件 

// 估计字符串长度,实际容纳(N-1)个有效字符,预留最后一个字符'\0' 
#define N 100

int main()
{
	char lines[N],maxLine[N];
	
	// 在输入的同时,即可处理,不必设置一个一维数组存储这些数。
	while(1)
	{
		gets(lines);  // 不能使用scanf("%s",lines),因其遇空格结束 
		if(strcmp(lines,"***end***") == 0) break;
		if(strlen(lines) > strlen(maxLine)) strcpy(maxLine,lines);
	}
	
	printf("%d\n%s\n",strlen(maxLine),maxLine); 
	
	return 0;
} 
\end{lstlisting}

\begin{note}[要点]
	简单处理即可, 不必用二维字符数组存储各个字符串,否则会导致超时。
\end{note}

\section{统计正整数的个数}
统计n个正整数中每个数出现的次数。

输入说明	

第一行是一个整数n(5<n<30), 表示要待统计整数的个数;

第二行是n个整数,每个整数均小于100000

输出说明
	
按照整数从小到大的顺序依次输出不同的整数及其出现次数, 整数和出现次数之间用冒号$(:)$分隔。

输入样例	

12
 
19 223 35 321 2 33 44 223 2 19 2 19

输出样例

2:3

19:3

33:1

35:1

44:1

223:2

321:1

\begin{lstlisting}
// 思路: 首先排序, 再统计每个数的个数.
#include <stdio.h>
int main()
{
	int n,num[30],i=0;
	scanf("%d",&n);
	while(i < n) 
	{ 
		scanf("%d",&num[i]);
		i++; 
	} 
	
	// 冒泡排序
	int p,q,t;
	for(p = 1; p <= n-1; p++)
	{
		for(q = 0; q < n-p; q++)
		{
			if(num[q] > num[q+1])
			{
				t = num[q];
				num[q] = num[q+1];
				num[q+1] = t;
			}
		}
	} 
	
	// 判断重复数
	int x,num_i = 0;
	x = num[0]; 
	for(i = 0; i < n; i++)
	{
		if(num[i] == x) num_i++;
		else 
		{
			printf("%d:%d\n",x,num_i);
			x = num[i]; // 新数 
			num_i = 1;  // 至少有一个数  
		}	
		
		if (i == n - 1)  // 最后一个数 
		printf("%d:%d\n",x,num_i);
	} 
	
	return 0;
} 
\end{lstlisting}

\begin{note}[要点]
	基本思路: 首先排序, 再统计每个数的个数.
\end{note}

