%%%%%%%%%%%%%%%%%%%%% chapter.tex %%%%%%%%%%%%%%%%%%%%%%%%%%%%%%%%%
%
% sample chapter
%
% Use this file as a template for your own input.
%
%%%%%%%%%%%%%%%%%%%%%%%% Springer-Verlag %%%%%%%%%%%%%%%%%%%%%%%%%%
%\motto{Use the template \emph{chapter.tex} to style the various elements of your chapter content.}
\chapter{第5次机试练习: 流程控制, 字符串, 数组}

\section{歌德巴赫猜想}
德巴赫猜想:任意一个大偶数都能分解为两个素数的和,对于输入的一个正偶数,写一个程序来验证歌德巴赫猜想。

由于每个正偶数可能分解成多组素数和,仅输出分解值分别是最小和最大素数的一组,按从小到大顺序输出。

输入说明	

输入一个正偶数n,1<n<1000。

输出说明
	
输出分解出的两个最小和最大素数。

输入样例
	
10

输出样例
	
3 7

\begin{lstlisting}
#include <stdio.h>
#include <math.h>

// 参数n是素数, 返回1, 否则返回0
int isPrime(int n)
{
	int i;
	if(n==2) return 0; // 最小素数是2 
	for(i = 2; i <= sqrt(n); i++) 
	{
		if (n%i == 0) return 0; // n不是素数 
	}
	return 1; // n是素数 
}

int main()
{
	int j,k,num; // num大偶数 
	int flag; // 标志变量:用于标识是否找到符合要求的素数对
	scanf("%d",&num);
	// 大偶数num,分解为两个素数 
	for(j = 3; j < num; j += 2)  //  找出第一个素数(最小的), for #1 
	{
		if(!isPrime(j)) continue; // 如果j不是素数, 继续下一轮迭代 
		
		flag = 0; // 初始化,未找出素数对 
		for(k = j + 2; k < num; k += 2) // 找出第二个素数, for #2
		{
			// j是最小素数, k必然是最大素数
			if(isPrime(k) && j+k == num) 
			{
				printf("%d %d\n",j,k);
				flag = 1; // 找出素数对, 如果没有此设置, 将会输出多组 
				break; // break for #2
			}
		}
		if(flag) break; // break for #1 
	}
	return 0;
}

// 不用标志变量版本 
int main2()
{
	int j,k,num; // num大偶数 
	scanf("%d",&num);
	// 大偶数num,分解为两个素数 
	for(j = 3; j < num; j += 2)  //  找出第一个素数(最小的)  for #1 
	{
		if(!isPrime(j)) continue; // 如果j不是素数, 继续下一轮迭代 
		
		for(k = j + 2; k < num; k += 2) //  找出第二个素数 for #2
		{
			// j是最小素数, k必然是最大素数
			if(isPrime(k) && j+k == num) 
			{
				printf("%d %d\n",j,k);
				return 0; // 找出素数对,结束主函数. 不结束, 将会输出多组  
			}
		}
	}
	return 0;
}

// 优化,根据题意找出一组:最小素数+最大素数=偶数 
int main3()
{
	int j,k,num; // num大偶数 
	scanf("%d",&num);
	// 对于大偶数num,分解为两个素数 
	for(j = 3; j < num; j += 2)  //  找出第一个素数(最小的)  for #1 
	{
		if(!isPrime(j)) continue; // 如果j不是素数, 继续下一轮迭代 
		
		for(k = num-1; k>=2; k -= 2) //  找出第二个素数(最大的) for #2
		{
			// j是最小素数, 判断k是否是最大素数并且二者之和=num 
			if(isPrime(k) && j+k == num) 
			{
				printf("%d %d\n",j,k);
				return 0; // 找出素数对,结束主函数. 不结束, 将会输出多组 
			}
		}
	}
	return 0;
}
\end{lstlisting}

\begin{note}[要点]
	再次体会标志变量的用法及内层循环前的初始化。仔细审题,本题要求一组输出: 最小素数+最大素数=偶数
\end{note}

\section{回文数}
若一个非负整数其各位数字按照正反顺序读完全相同,则称之为回文数,例如12321。
判断输入的整数是否是回文数。若是,则输出该整数各位数字之和,否则输出no。

输入说明	

输入为一个整数n,0<=n<100000000。

输出说明	

若该整数为回文数,则输出整数各位数字之和,否则输出no。

输入样例
	
样例1输入

131

样例2输入

24

输出样例

样例1输出
	
5

样例2输出

no

\begin{lstlisting}
#include <stdio.h>
#include <string.h>

// 思路1:  求该整数的反序组成的整数,如整数1234, 其反序整数即为4321, 如果二者相等即为回文数 
// 判断num是否是回文数,是:返回1;不是,返回0
// 指针参数*sum,返回这个数的各位之和  
int isPalindromic1(int num,int *sum)
{
	int reverse = 0, tmp = num;
	*sum = 0; // 初始换指针内容 
	while(tmp)
	{
		reverse = reverse*10 + tmp%10;
		*sum += tmp%10;
		tmp /= 10;
	}
	if(reverse == num ) return 1;
	else return 0;	
}

// 思路2:  构造数组reverse, 反序存储该整数各位数,按照数组下标,前后数组元素相等则为回文数 
// 判断num是否是回文数,是:返回1;不是,返回0
// 指针参数*sum,返回这个数的各位之和  
int isPalindromic2(int num,int *sum)
{
	int reverse[9], len=0, i=0; // 依题意数组最大长度为9,最多存储9位数。实际长度用len变量表示 
	
	*sum = 0; // 初始换指针内容 
	//构造数组reverse, 反序存储该整数各位数
	while(num)
	{
		reverse[i]=num%10;  
		*sum += num%10;   // 累加各位数字 
		num /= 10;
		len++; // 计算数组实际长度 
		i++; 
	}
	//按照数组下标,前后数组元素相等则为回文数
	for(i=0; i<len/2;i++)
	{
		if(reverse[i] != reverse[len-i-1]) return 0;  // 不是回文数 
	}
	return 1;	// 至此,必然是回文数 
}

// 测试方法1和2的主程序
int main12()
{
	int i,num,sum;
	
	scanf("%d",&num); 
	//if(isPalindromic1(num,&sum)) printf("%d\n",sum);
	if(isPalindromic2(num,&sum)) printf("%d\n",sum);
	else printf("no\n");
	return 0;
} 

// 思路3:  按照字符串处理输入的整数,前后数组元素相等则为回文数 
int main3()
{
	char s[10]; // 留出'\0', 最多存储9位数。实际长度用len变量表示
	int sum=0, len=0,i; 
	
	// 以字符串形式接收输入的整数, 末尾自动追加'\0'  
	gets(s);
	
	// 计算len, 或者len=strlen(s), 同时计算各位数字之和 
	for(len=0;s[i]!='\0';) 
	{
		len++;
		sum=sum+s[i]-'0'; // 计算各位数字之和
		i++;  
	}
	
	//按照数组下标,前后数组元素相等则为回文数
	for(i=0; i<len/2;i++)
	{
		if(s[i] != s[len-i-1]) // 不是回文数 
		{
			printf("no\n");
			return 0;  // 主函数结束 
		} 
	}
	// 至此,必然是回文数 
	printf("%d\n",sum);
	return 0;
}

// 思路4:  按照字符串处理输入的整数,前后数组元素相等则为回文数。
// 使用指针操作 
int main()
{
	char s[10]; // 留出'\0', 最多存储9位数。实际长度用len变量表示
	int sum=0; 
	char *p1=s,*p2=s; // 用于正序和反序遍历s数组,初始指向第一个元素 
	
	// 以字符串形式接收输入的整数,末尾自动追加'\0' 
	gets(s);
	
	// 用p2遍历字符串,同时计算各位数字之和 
	for(;*p2!='\0';p2++) 
	{
		sum=sum+(*p2)-'0'; // 计算各位数字之和
	}
	// 至此,p2指向最后一个元素'\0', 我们使它指向最后一个有效元素: 
	p2--; 
	
	//按照数组下标,前后数组元素相等则为回文数
	for(;p1<p2;p1++,p2--)
	{
		if(*p1 != *p2) // 不是回文数 
		{
			printf("no\n");
			return 0;  // 主函数结束 
		} 
	}
	// 至此,必然是回文数 
	printf("%d\n",sum);
	return 0;
}
\end{lstlisting}

\begin{note}[要点]
	\begin{enumerate}
		\item 掌握函数的地址传递方法。
		\item 使用两个指针变量p1, p2, 其中p1指向待查找子串的首字母, 另一个指向末尾, \lstinline|p1++, p2--|; 是判断字符串是否是回文的有效技巧。
	\end{enumerate}
\end{note}

\section{寻找最大整数}
从键盘输入四个整数,找出其中的最大值并将其输出。

输入说明
	
输入4个整数,用空格分隔

输出说明
	
输出值最大的一个整数

输入样例
	
25 99 -46 0

输出样例
	
99

\begin{lstlisting}
#include <stdio.h>
// 不用存储整数序列, 采用一条循环语句,合并输入和计算,减少出错概率。 
int main() 
{
	int i, num, max;
	// 输入, 并计算 
	for(i = 0; i < 4; i++) 
	{ 
		if (i==0) scanf("%d",&max); // 假定第一个数就是最大的数 
		else
		{
			scanf("%d",&num);
			if(num > max) max=num; 
		} 
	} 
	printf("%d\n",max);
	return 0;
} 

int main1() // 另解, 存储整数序列
{
	int i, num[4], max;
	// 输入 
	for(i = 0; i < 4; i++) 
		scanf("%d",&num[i]);
	// 假定的最大值必须是实际存在的,不要想当然是0,9999,等等。
	max = num[0]; 
	for(i = 0; i < 4; i++)  
		if(max < num[i]) max = num[i];
	
	printf("%d\n",max);
	return 0;
} 
\end{lstlisting}

\begin{note}[要点]
	题目虽然简单,你能体会哪种解法更好? 特别注意假定变量的值必须是实际存在的数。
\end{note}

\section{ISBN号码}
每一本正式出版的图书都有一个ISBN号码与之对应,ISBN码包括9位数字、1位识别码和3位分隔符,其规定格式如"x-xxx-xxxxx-x",
其中符号“-”是分隔符(键盘上的减号),最后一位是识别码,例如0-670-82162-4就是一个标准的ISBN码。

ISBN码的首位数字表示书籍的出版语言,例如0代表英语;

第一个分隔符“-”之后的三位数字代表出版社,例如670代表维京出版社;

第二个分隔之后的五位数字代表该书在出版社的编号;

最后一位为识别码。识别码的计算方法如下:

首位数字乘以1加上次位数字乘以2……以此类推,用所得的结果mod 11,所得的余数即为识别码,如果余数为10,则识别码为大写字母X。
例如ISBN号码0-670-82162-4中的识别码4是这样得到的:
对067082162这9个数字,从左至右,分别乘以1,2,\dots,9,再求和,即$0\times 1+6\times 2+\dots +2\times 9=158$,然后取158 mod 11的结果4作为识别码。

编写程序判断输入的ISBN号码中识别码是否正确,如果正确,则仅输出``Right"; 如果错误,则输出正确的ISBN号码。

输入说明	

输入只有一行,是一个字符序列,表示一本书的ISBN号码(保证输入符合ISBN号码的格式要求)。

输出说明	

输出一行,假如输入的ISBN号码的识别码正确,那么输出``Right",否则,按照规定的格式,输出正确的ISBN号码(包括分隔符``-")。

输入样例	

样例输入1

0-670-82162-4

样例输入2

0-670-82162-0

输出样例	

样例输出1

Right

样例输出2

0-670-82162-4

\begin{lstlisting}
#include <stdio.h>
int main()
{
	char ISBN[14] = "0-670-82162-4";  // ISBN[13]='\0'
	int i,j, code1 = 0, code2;
	
	// 末尾自动添加'\0'.
	scanf("%s",ISBN);  // 或 gets(ISBN); 
	code2 = ISBN[12] == 'X' ? 'X' : ISBN[12]-'0';
	for(i = 0,j = 1; i < 11; i++)
	{
		if (ISBN[i]=='-') continue;
		// 整数与单个数字字符的关系:9 = '9' -'0' 
		code1 += (ISBN[i]-'0')*j;  
		j++;
	}   
	code1 %= 11;
	if (code1 == 10) code1 = 'X';  
	if(code1 == code2) printf("Right\n");
	else 
	{
		if(code1 == 'X') ISBN[12] = 'X';
		else ISBN[12] = code1 + '0';
		printf("%s\n",ISBN); // 或 puts(ISBN);
	}
	return 0;
} 
\end{lstlisting}

\begin{note}[要点]
	\begin{enumerate}
		\item 定义字符数组表示字符串时,且记给\lstinline|'\0'|留一个字符的位置, 表示字符串的结尾。
		\item \lstinline|ASCII编码(整数)=字符-'0';|
		\item \lstinline|字符=ASCII编码(整数)+'0';|
		\item 整数可以表示字符的ASCII编码(整数), 整数和字符类型可以``混用", 详见课件。
		\begin{lstlisting}
		int a; char c='A';
		a = c+1; // c当作整数运算
		printf("%d %c %d %c\n",a,a,c,c); //66 B 65 A 
		\end{lstlisting}
	\end{enumerate}
\end{note}

\section{密码强度}
每个人都有很多密码,你知道你的密码强度吗?假定密码由大写字母、小写字母、数字和非字母数字的符号这四类字符构成,密码强度计算规则如下:

1.	基础分:空密码(密码长度为零)0分, 非空密码1分

2.	加分项1:密码长度超过8位, +1分

3.	加分项2:密码包含两类不同字符+1分, 包含三类不同字符+2分, 包含四类不同字符+3分

按照此规则计算的密码强度为$0\sim 5$。请你设计一个程序计算给出的密码的强度。

输入说明
	
输入为一个密码字符串,字符串长度不超过50个字符。

输出说明
	
输出一个整数表示该密码的强度。

输入样例
	
输入样例1

abcd

输入样例2

ab123

输出样例
	
样例1输出:

1

样例2输出

2

\begin{lstlisting}	
#include <stdio.h>
#include <string.h>
int main()
{
	char p[51]; // 记得给'\0'留位置
	int i,strength = 0;
	// class4[0]=1大写字母, class4[1]=1小写字母, class4[2]=1数字, class4[3]=1非字母数字 
	int class4[4] = {0,0,0,0};  
	
	//scanf("%s",p);  // 不能完整接收含空格的字符串和空密码 
	gets(p); // last char: '\0',直接回车,就是空密码 
	
	// 	1.	基础分:空密码(密码长度为零)0分,非空密码1分 
	if(strlen(p) == 0) strength += 0;
	else  strength += 1;
	
	// 2.	加分项1:密码长度超过8位,+1分 
	if(strlen(p) > 8) strength += 1;
	
	// 3.	加分项2:密码包含两类不同字符+1分,包含三类不同字符+2分,包含四类不同字符+3分 
	for(i = 0; p[i] != '\0'; i++)
	{
		if(p[i] >= 'A' && p[i] <= 'Z') class4[0] = 1; 
		else if(p[i] >= 'a' && p[i] <= 'z') class4[1] = 1; 
		else if(p[i] >= '0' && p[i] <= '9') class4[2] = 1;
		else class4[3] = 1; 
	}
	int c = 0;
	for(i = 0; i < 4; i++) c += class4[i];
	
	if(c >= 4) strength += 3;
	else if(c >= 3) strength += 2;
	else if(c >= 2) strength += 1;
	
	printf("%d\n",strength);
	
	return 0;
}
\end{lstlisting}

\begin{note}[要点]
	字符串处理的典型问题: \lstinline|'\0'|, 字符串相关函数\lstinline|char s1[81],s2[81]; strlen(s1), strcmp(s1,s2), strcpy(s1,s2); scanf("%s",s1), gets(s1)|的区别等, 应该充分掌握。
\end{note}

%%%%%%%%%%%%%%%% 留作下次练习
\begin{comment}

\section{矩阵}
请写一个程序,对于一个m行m列$(2<m<20)$的方阵,求其每一行、每一列及主、辅对角线元素之和,然后按照从大到小的顺序依次输出这些值。

注:主对角线是方阵从左上角到右下角的一条斜线,辅对角线是方阵从右上角到左下角的一条斜线。

输入说明

输入数据的第一行为一个正整数m;

接下来为m行、每行m个整数表示方阵的元素。

输出说明	

从大到小排列的一行整数,每个整数后跟一个空格,最后换行。

输入样例

4

15  8   -2   6

31  24  18  71

-3  -9   27  13

17  21  38  69

输出样例

159 145 144 135 81 60 44 32 28 27

\begin{lstlisting}
#include <stdio.h>
// 估计方阵行列数 
#define M 20

// input, m是实际方阵行列数 
void input(int matrix[][M], int m)
{
int i,j; 
for(i = 0; i < m; i++)
for(j = 0; j < m; j++) 
scanf("%d",&matrix[i][j]);
}

// 计算主对角线之和, m是实际方阵行列数 
int main_diagonal(int matrix[][M], int m)
{	
int i,j,sum = 0; 
for(i = 0; i < m; i++)
{
for(j = 0; j < m; j++) 
{
if(i == j) sum += matrix[i][j];   // 主对角线 
}
}
return sum;
}

// 计算副对角线之和, m是实际方阵行列数 
int counter_diagonal(int matrix[][M], int m)
{	
int i,j,sum = 0;
for(i = 0; i < m; i++)
{
for(j = 0; j < m; j++) 
{ 
if(j == m-i-1) sum += matrix[i][j]; // 副对角线之和 
}
}
return sum;
}

// 计算第i行之和, m是实际方阵行列数 
int sumI(int matrix[][M], int m, int i)
{ 
int j,sum = 0; 
for(j = 0; j < m; j++) // 遍历列 
{
sum += matrix[i][j];  // 第i行之和
}
return sum; 
} 

// 计算第j行之和, m是实际方阵行列数 
int sumJ(int matrix[][M], int m, int j)
{ 
int i,sum = 0; 
for(i = 0; i < m; i++) // 遍历行 
{
sum += matrix[i][j];  // 第j行之和
}
return sum;
} 

// 交换两个元素值 
void swap(int *p1, int *p2)
{
int temp;
temp = *p1; *p1 = *p2; *p2 = temp;
}

// 选择法排序(降序)
void sorts(int a[], int n)
{
int i,j,k;
for(i = 0; i < n-1; i++)
{
k = i;
for (j = i+1; j < n; j++)
if(a[j] > a[k])	k = j;
if (k != i) swap(&a[i],&a[k]);
} 
}

int main()
{
int matrix[M][M],a[2*M+2]; // 以估计行列数,定义数组 
int i,m; 
scanf("%d",&m); // 实际方阵行列数 

input(matrix,m); // input 

// 调用各函数,装配数组a
int n = 0; // 记录数组a的实际长度		
for(i = 0; i < m; i++)
{
a[n++] = sumI(matrix,m,i); // 第i行之和
a[n++] = sumJ(matrix,m,i); // 第i列之和  
}

a[n++] = main_diagonal(matrix,m);    // 主对角线之和 
a[n++] = counter_diagonal(matrix,m); // 副对角线之和  

// 排序数组a 
sorts(a,n);

// 输出 
for(i = 0; i < n; i++)
printf("%d ",a[i]);

printf("\n");

return 0;
}
\end{lstlisting}

\begin{note}[要点]
\begin{enumerate}
\item 思路:定义功能单一的函数,实现简单功能,主程序调用各个函数。
\item 一维数组a[2*M+2]存储相关函数计算结果,排序数组a即是所求。 
\item 避免过多循环嵌套,不易出错,简化程序设计。 
\item 但是缺点是在各函数中分别循环遍历方阵,效率低。
\item 优化方案是不采用独立函数计算,在主函数中一次遍历方阵,计算各值。 
\item 二维数组表示矩阵, 是常见题型, 必须熟练掌握元素的下标规律及其遍历技巧。
\end{enumerate}
\end{note}

\section{消除类游戏}
消除类游戏是深受大众欢迎的一种游戏,游戏在一个包含有n行m列的游戏棋盘上进行,棋盘的每一行每一列的方格上放着一个有颜色的棋子,当一行或一列上有连续三个或更多的相同颜色的棋子时,这些棋子都被消除。当有多处可以被消除时,这些地方的棋子将同时被消除。

现在给你一个n行m列的棋盘,棋盘中的每一个方格上有一个棋子,请给出经过一次消除后的棋盘。

请注意:一个棋子可能在某一行和某一列同时被消除。

输入格式

输入的第一行包含两个整数n, m,用空格分隔,分别表示棋盘的行数和列数。

接下来n行,每行m个整数,用空格分隔,分别表示每一个方格中的棋子的颜色。颜色使用1至9编号。

输出格式

输出n行,每行m个整数,相邻的整数之间使用一个空格分隔,表示经过一次消除后的棋盘。如果一个方格中的棋子被消除,则对应的方格输出0,否则输出棋子的颜色编号。

样例输入1

4 5

2 2 3 1 2

3 4 5 1 4

2 3 2 1 3

2 2 2 4 4

样例输出1

2 2 3 0 2

3 4 5 0 4

2 3 2 0 3

0 0 0 4 4

样例说明

棋盘中第4列的1和第4行的2可以被消除,其他的方格中的棋子均保留。

样例输入2

4 5

2 2 3 1 2

3 1 1 1 1

2 3 2 1 3

2 2 3 3 3

样例输出2

2 2 3 0 2

3 0 0 0 0

2 3 2 0 3

2 2 0 0 0

样例说明

棋盘中所有的1以及最后一行的3可以被同时消除,其他的方格中的棋子均保留。
评测用例规模与约定

所有的评测用例满足:$1 \le n, m \le 30$。

\begin{lstlisting}
#include <stdio.h>
#include <math.h>
#define M 30 // 估计数组最大长度

// 扫描curRow行中可删除的元素, F: 删除标志矩阵, m: 列数 
void delRow(int curRow, int checker[][M], int F[][M],int m)
{
int i,j, current,count;
for(i=0;i<m;i++) 
{
// if(F[curRow][i]==1) continue; // 如果先扫描行,这就不合适了 
count=1; 
current=checker[curRow][i];
for(j=i+1;j<m;j++)
{
if(current==checker[curRow][j]) count++;
else break;
}
if(count>=3)
{
for(j=i;j<m;j++)
{
if(checker[curRow][j]==current) F[curRow][j]=1;//置删除标志
else break; 
}
}
} 
}

// 扫描curCol列中可删除的元素, F: 删除标志矩阵, n: 行数 
void delCol(int curCol, int checker[][M], int F[][M],int n)
{
int i,j, current,count;
for(i=0;i<n;i++) 
{
count=1; 
current=checker[i][curCol];
for(j=i+1;j<n;j++)
{
if(current==checker[j][curCol]) count++;
else break;
}
if(count>=3)
{
for(j=i;j<n;j++)
{
if(checker[j][curCol]==current) F[j][curCol]=1;//置删除标志
else break; 
}
}
} 
}

// 清除所有可删除的元素, F: 删除标志矩阵, n: 行数, m: 列数 
void del(int checker[][M], int F[][M],int n, int m)
{
int i,j;
for(i=0;i<n;i++)
{
for(j=0;j<m;j++)
{
if(F[i][j]==1) checker[i][j]=0;
}
}
}

// 读棋盘, n: 行数, m: 列数
void Read(int checker[][M],int n, int m)
{
int i,j;
for(i=0;i<n;i++)
{
for(j=0;j<m;j++)
{
scanf("%d",&checker[i][j]);
}
}
}

// 输出棋盘, n: 行数, m: 列数
void output(int checker[][M],int n, int m)
{
int i,j;
for(i=0;i<n;i++) // 行 
{
for(j=0;j<m;j++) // 列 
{
printf("%d ",checker[i][j]); 
}
printf("\n");
} 
}

int main()
{
// 棋盘,初始化是为了便于测试 
int checker[M][M]={{2,2,3,1,2},
{3,4,5,1,4},
{2,3,2,1,3},
{2,2,2,4,4}};
int n=4,m=5,i,j; // n行,m列 
/************
int checker[M][M]={{2,2,3,1,2},
{3,1,1,1,1},
{2,3,2,1,3},
{2,2,3,3,3}};
int n=4,m=5,i,j; // n行,m列 
*************/
// 标志矩阵 
int F[M][M]; // 对应元素为1,则表示删除 

scanf("%d%d",&n,&m);
// 读棋盘 
Read(checker,n,m);

// 初始化F 
for(i=0;i<n;i++)
for(j=0;j<m;j++)
F[i][j]=0;

// 扫描行 
for(i=0;i<n;i++) // 行 
delRow(i, checker, F, m);

// 扫描列 
for(j=0;j<m;j++) // 列 
delCol(j, checker, F, n);

// 清除 
del(checker, F, n, m);

// 输出棋盘 
output(checker,n,m); 
return 0;
} 
\end{lstlisting}

\begin{note}[要点]
标志矩阵,体会模块化编程思想,初始化变量的程序调试技巧。
\end{note}

\section{表达式求值}
表达式由两个非负整数x,y和一个运算符op构成,求表达式的值。
这两个整数和运算符的顺序是随机的,可能是``x op y", ``op x y"或者``x y op",例如, ``25 + 3"表示25加3, ``5 30 *"表示5乘以30, "/ 600 15"表示600除以15。

输入说明

输入为一个表达式,表达式由两个非负整数x,y和一个运算符op构成,x,y和op之间以空格分隔,但顺序不确定。

x和y均不大于10000000,op可以是+, -,*, /, \%中的任意一种, 分表表示加法, 减法, 乘法, 除法和求余。

除法按整数除法求值, 输入数据保证除法和求余运算的y值不为0。

输出说明	

输出表达式的值。

输入样例

样例1输入

5 20 *

样例2输入

4 + 8

样例3输入

/ 8 4

输出样例

样例1输出

100

样例2输出

12

样例3输出

2

\begin{lstlisting}
#include <stdio.h>
// 估计字符串最大长度,存储有效字符(N-1)个,预留最后一个字符'\0' 
#define N 20 

// 根据参数,计算表达式的值 
int compute(char op,int x,int y)
{
int result = -1;
switch(op)
{
case '+': result = x+y; break;
case '-': result = x-y; break;
case '*': result = x*y; break;
case '/': if(y != 0) result = x/y; break;
case '%': if(y != 0) result = x%y; break;
}
return result;
}

// 数字字符串s转为int, 要求s以'\0'结尾 
int strToInt(char *s)// int toInt(char s[]) 
{
int result=0;
while(*s) // *s != '\0'
{
result=result*10+ (*s-'0');
s++; //移至下一字符 
}
return result;
} 

/******************************************************
提取子串函数 
忽略s中空格前缀,复制s中的字符串到subs中,遇空格或'\0'结束
返回subs不含空格。 返回复制后s指针指向(地址) 
要求s和subs以'\0'结尾。
*******************************************************/ 
char* getSubs(char *s, char *subs) 
{
int start=0; 
while(*s)
{
if(*s==' ') 
{
if(start==0) s++; // 忽略s的前缀空格 
else break; // 是有效字符串后的一个空格 
}
else
{
start=1; // 开始复制 
*subs=*s;
s++;
subs++;
}
}
*subs='\0'; // 不要忘记结尾符 
return s;
}

// 解析s, 以空格为分隔符, 分解s为3个字符串 
void parse(char *s,char result[][N])
{
char *p;
p=getSubs(s,result[0]);
p=getSubs(p,result[1]);
p=getSubs(p,result[2]);
} 

// 如果s是操作符,返回1, 参数op返回该操作符
// 否则, 返回0 
int isOp(char *s, char *op)
{
if(*s >= '0' && *s <= '9') return 0;
else
{
*op=*s;
return 1;
}
} 

int main()
{
char s[N],op;
char s3[3][N];   
int x,y;
gets(s); 

parse(s,s3); // s被分解为3个字符串 
if(isOp(s3[0],&op)) // op x y
{
x=strToInt(s3[1]);
y=strToInt(s3[2]);
}
else if(isOp(s3[1],&op)) // x op y
{
x=strToInt(s3[0]);
y=strToInt(s3[2]);
}
else if(isOp(s3[2],&op)) // x y op
{
x=strToInt(s3[0]);
y=strToInt(s3[1]);
}

printf("%d\n",compute(op,x,y)); 
return 0;
}

int main1() // 另解,直接读取三个子串, 就不用分解了
{
char s[N],op;
char s3[3][N];   
int x,y;
scanf("%s%s%s",s3[0],s3[1],s3[2]); // 利用"%s"读字符串遇空格结束特点,直接读取3个字符串。 

// parse(s,s3); // s被分解为3个字符串 
if(isOp(s3[0],&op)) // op x y
{
x=strToInt(s3[1]);
y=strToInt(s3[2]);
}
else if(isOp(s3[1],&op)) // x op y
{
x=strToInt(s3[0]);
y=strToInt(s3[2]);
}
else if(isOp(s3[2],&op)) // x y op
{
x=strToInt(s3[0]);
y=strToInt(s3[1]);
}

printf("%d\n",compute(op,x,y)); 
return 0;
}
\end{lstlisting}

\begin{note}[要点]
本例是字符串处理, 指针应用, 模块化程序设计的范例, 上课时做了详细的讲解。详见课件。
\end{note}

\section{马鞍点}	
若一个矩阵中的某元素在其所在行最小而在其所在列最大,则该元素为矩阵的一个马鞍点。请写一个程序,找出给定矩阵的马鞍点。

输入说明

输入数据第一行只有两个整数m和n(0<m<100,0<n<100),分别表示矩阵的行数和列数;

接下来的m行、每行n个整数表示矩阵元素(矩阵中的元素互不相同),整数之间以空格间隔。

输出说明

在一行上输出马鞍点的行号、列号(行号和列号从0开始计数)及元素的值(用一个空格分隔), 之后换行;

若不存在马鞍点,则输出一个字符串``no"后换行。

输入样例	

4  3

11    13    121

407   72    88

23    58    1

134   30    62

输出样例

1 1 72

\begin{lstlisting}
#include <stdio.h>
// 估计的二维数组最大行列数 
#define M 100
#define N 100

// 判断a[row,col]是否是马鞍点, 是: 返回1; 否则返回0
// m,n是二维数组实际行列数 
int compute(int a[][N], int m, int n, int row, int col)
{
int i,element = a[row][col]; 
// element在其所在行最小而在其所在列最大 
for(i = 0; i < n; i++)
if(a[row][i] < element) return 0; // 不是马鞍点,直接返回0 
for(i = 0; i < m; i++)
if(a[i][col] > element) return 0; // 不是马鞍点,直接返回0
return 1;	// 如果执行至此,肯定是马鞍点,直接返回0 
}

int main()
{
int matrix[M][N]; // 按照估计的最大行列数定义二维数组 
int i,j,m,n,flag = 0; 

scanf("%d%d",&m,&n); // 实际行列数 

// input
for(i = 0; i < m; i++)
for(j = 0; j < n; j++) 
scanf("%d",&matrix[i][j]);

// 遍历二维数组, 判断马鞍点		
for(i = 0; i < m; i++)
{
for(j = 0; j < n; j++) 
{
if(compute(matrix,m,n,i,j))
{
printf("%d %d %d\n",i,j,matrix[i][j]);
flag = 1;
}

}
}	
if (!flag) printf("no\n");

return 0;
}
\end{lstlisting}

\begin{note}[要点]
\begin{enumerate}
\item 思路:定义函数计算单个元素a[i,j]是否是马鞍点,主程序遍历二维数组,调用此函数。
\item 避免过多循环嵌套,不易出错,简化程序设计。
\item 掌握二维数组作为函数参数的定义, 调用方式。
\end{enumerate}
\end{note}

\end{comment}
