%%%%%%%%%%%%%%%%%%%%% chapter.tex %%%%%%%%%%%%%%%%%%%%%%%%%%%%%%%%%
%
% sample chapter
%
% Use this file as a template for your own input.
%
%%%%%%%%%%%%%%%%%%%%%%%% Springer-Verlag %%%%%%%%%%%%%%%%%%%%%%%%%%
%\motto{Use the template \emph{chapter.tex} to style the various elements of your chapter content.}
\chapter{第3次机试练习}

\section{整数分析}
给出一个整数n(0<=n<=100000000)。求出该整数的位数,以及组成该整数的所有数字中的最大数字和最小数字。

输入说明
	
输入一个整数n(0<=n<=100000000)

输出说明
	
在一行上依次输出整数n的位数,以及组成该整数的所有数字中的最大数字和最小数字,各个数字之间用空格分隔。

输入样例
	
217

输出样例
	
3 7 1

\begin{lstlisting}
#include <stdio.h>
// 循环除10取余是整数分解的基本技巧
int main()
{
	int i = 0, n, bit, max, min;
	scanf("%d",&n);
	while(n) // 等效于 while(n!=0)
	{
		bit = n%10; // 获取n的最低为
		// 切记: 初始化时,假设的max和min必须是实际存在的数。
		if(i == 0) // 初始化: 原始n的最低位设为最大和最小数字
		{
			max = min = bit;
		}  
		else
		{
			if(bit > max) max = bit;
			if(bit < min) min = bit;
		}
		n /= 10; // 去除最低位
		i++;
	} 
	// (i == 0 ? 1 : i)是条件表达式, 表达式的值是:
	// 如果i==0,则表达式的值为1否则表达式的值是i
	printf("%d %d %d\n",(i == 0 ? 1 : i),max,min); //考虑原始n==0的情况 
	return 0;
} 
\end{lstlisting}

\begin{note}[知识点]
  \begin{enumerate}
  	\item 整数数位分解是基本编程练习之一。
  	\item 切记: 初始化时,假设的max和min必须是实际存在的数。比如不能想当然假设max=1000, min=0.
  	\item 注意审题: ``输入一个整数$n,(0<=n<=100000000)$'', 因此, 0也是一个合法输入。
  \end{enumerate}
  
\end{note}

\section{冰箱温度预测}
编写一个程序,用于预测冰箱断电后经过时间t(以小时为单位)后的温度T。已知计算公式如下所示
\[ T=\frac{4t^2}{t+2}-20 \]

输入说明	

输入两个整数h和m表示冰箱断电后经过的时间,h表示小时,m表示分钟

输出说明
	
输出冰箱断电后经过时间t(以小时为单位)后的温度T,保留两位小数

输入样例
	
2 0

输出样例
	
-16.00

\begin{lstlisting}
#include <stdio.h>
int main()
{
	int h,m;
	float t,T;
	scanf("%d%d",&h,&m);
	t = h + m/60.0;      // 必须是60.0, why?
	T = 4*t*t/(t+2)-20;  // 优先级保证了计算的正确性, why?
	printf("%.2f\n",T);
	
	return 0;
}
\end{lstlisting}

\begin{note}[知识点]
	整数/整数, 表达式的值是整数部分, 自动舍去小数部分。
\end{note}

\section{除法计算器}	
小明的弟弟刚开始学习除法,为了检查弟弟的计算结果是否正确,小明决定设计一个简单计算器程序来验算。

输入说明
	
输入数据由四个整数m,n,q,r构成,m为被除数,n为除数,q和r为小明的弟弟计算出的商和余数。整数之间用空格分隔,所有整数取值范围在$(-100000\sim 100000)$, n不为0。

输出说明
	
如果验算结果正确,输出yes,否则输出正确的商和余数

输入样例:

样例1:

10 3 3 1

样例2:

10 3 3 2

输出样例	

样例1输出:

yes

样例2输出:

3 1

\begin{lstlisting}
#include <stdio.h>
int main()
{
	int m,n,q,r;
	scanf("%d%d%d%d",&m,&n,&q,&r);
	if(m==q*n+r && q==m/n && r==m%n) printf("yes\n");
	else printf("%d %d\n",m/n,m%n); 
	return 0;
} 
\end{lstlisting}

\begin{note}
	改变题设条件,修改此程序,进行各种表达式计算练习, 分析优先级。 如果n=0时, 如何处理。
\end{note}

\section{完全平方数}
若一个整数n能表示成某个整数m的平方的形式,则称这个数为完全平方数。写一个程序判断输入的整数是不是完全平方数。

输入说明
	
输入数据为一个整数n,0<=n<1000000。

输出说明
	
如果n是完全平方数,则输出构成这个完全平方数的整数m,否则输出no。

输入样例
	
样例1:

144

样例2:

15

输出样例
	
样例1输出:

12

样例2输出:

no

\begin{lstlisting}
#include <stdio.h>
#include <math.h>  // 数学函数头文件
int main()
{
	int n,m;
	scanf("%d",&n);
	m=(int)sqrt(n); // sqrt(n)计算的结果为doublel类型, 此语句表示把它转化为int类型, 自动舍去小数部分(不会四舍五入), 并赋值给m。
	if(n==m*m) printf("%d\n",m);
	else printf("no");
	return 0;
} 
\end{lstlisting}

\begin{note}[要点]
	\begin{enumerate}
	\item 数据类型的强制转换, sqrt函数原型: \lstinline|double sqrt(double x);|
	
	\lstinline|m=(int)sqrt(n)|是函数调用语句, 等效于:
	\begin{lstlisting}[frame=none]
	double y=sqrt(n); // n自动由int转化为double类型的数据, 不会损失精度
	m=(int)y; 
	\end{lstlisting}
	\item 注意审题: ``输入一个整数$n, (0<=n<=1000000)$'', 因此, 0也是一个合法输入。
\end{enumerate}
\end{note}

\section{选号程序}
小明决定申请一个新的QQ号码,系统随机生成了若干个号码供他选择。小明的选号原则是:
\begin{enumerate}
	\item 选择所有号码中各位数字之和最大的号码。
	\item 如果有多个号码各位数字之和相同则选择数值最大的号码。
\end{enumerate}

请你写一个程序帮助小明选择一个QQ号码。

输入说明
	
输入数据由两行构成,第一行为一个整数n表示有n个待选号码(0<n<100),第二行有n个正整数,表示各个待选的号码,每个号码长度不超过9位数。每个号码之间用空格分隔,且每个号码都不相同。

输出说明
	
输出根据小明的选号原则选出的号码。

输入样例
	
5

10000 11111 22222 333 1234

输出样例
	
22222

\begin{lstlisting}
#include <stdio.h>
// 在循环语句中, 读取备选qq号, 计算各位之和, 依据筛选条件选取qq号
int main() 
{ 
    // 关键变量含义说明: 
	// select_qq,select_sum表示备选qq及其各位之和
	// qq,sum表示当前读取的qq及其各位和 
	int i,n,select_qq,select_sum,qq,sum,tmp;
	scanf("%d",&n);
	for(i=0;i<n;i++) // 注意条件表达式, 表明i的最大值是n-1, 因为i是0开始的, 因此共执行n次循环
	{
		scanf("%d",&qq); // 读取当前备选qq号
		tmp=qq; // 保存到临时变量中,因为下面的循环语句要更改。 
		sum=0;  // 当前读取qq号的各位之和。 注意: 一定要初始化,否则上一个备选号的sum值会带入本轮循环中。 
		while(tmp) // 计算各位之和 
		{
			sum+=tmp%10;
			tmp/=10;
		}
		// 第1轮迭代(i==0), 当前读取的qq就是所选, 其它根据题设条件选号
		// 因为三个表达式为||运算, 从左到右依次计算各表达式的值, 如果为真,则不会计算后边表达式。
		// 因此, 当i==0时不会其它两个表达式的值, if条件为真。 
		if(i==0 || sum>select_sum || (sum==select_sum && qq>select_qq))
		{ 
			select_qq=qq;
			select_sum=sum;
		} 
	}
	printf("%d",select_qq);
	return 0;
} 

// 解法2: 用二维数组存储所有qq号及其各位和 
#define N 100 // 估计最大数组长度 
int main1() 
{
	// 二维数组No, 第一列表示qq号, 第二列表示该qq号的各位数字之和。 
	int i,n,No[N][2],tmp,sum,max=0,largest=0,select;
	scanf("%d",&n);
	// 筛选条件2
	for(i=0;i<n;i++)
	{
		scanf("%d",&No[i][0]);
		tmp=No[i][0];
		sum=0;  // 一定初始化 
		while(tmp)
		{
			sum+=tmp%10;
			tmp/=10;
		}
		No[i][1]=sum;
		if(sum>=max) max=sum; 
	}
	// 筛选条件1
	for(i=0;i<n;i++)
	{
		if(No[i][1]==max) // 备选号码
		{
			if(No[i][0]>=largest)
			{
				select=No[i][0];
				largest=No[i][0];
			}
		} 
	}
	printf("%d",select);
	return 0;
} 
\end{lstlisting}

\begin{note}[要点]
	\begin{enumerate}
		\item ||和 \&\&运算从左到右执行,取得结果,则不执行后面的表达式。\\
		取得结果的含义是: \\
		if (条件1||条件2||条件3)运算中, 只要有一个条件表达式为真(非0),即整个条件()结果即为真。 \\
		if (条件1 \&\&条件2 \&\& 条件3)运算中, 只要有一个条件表达式为假(0),即整个条件()结果即为假。
		\item 比较两种解法的优缺点。
		\item 本例是循环迭代的范例, 应反复演练, 领会迭代程序的编程技巧。
		\item 试着定义函数, 改写此程序。
		\item 本题不必使用排序算法,使程序复杂化。
	\end{enumerate}	
\end{note}

\section{成绩分级}	
给出一个百分制的成绩,要求输出成绩等级\lstinline|'A','B','C','D','E'|。90分以上为\lstinline|'A',80~89分为'B',70~79分为'C',60~69分为'D',60分以下为'E'|。

输入说明	

输入一个正整数m(0<=m<=100)

输出说明
	
输出一个字符

输入样例
	
59

输出样例
	
E

\begin{lstlisting}
#include <stdio.h>
int main()
{
	int grade;
	scanf("%d",&grade);
	grade /= 10;
	switch(grade)
	{
		case 0: case 1: case 2: case 3: case 4: 
		case 5:  printf("E"); break;
		case 6:  printf("D"); break;
		case 7:  printf("C"); break;
		case 8:  printf("B"); break;
		case 9:
		case 10:  printf("A"); break;
		
	}
	return 0;
} 
\end{lstlisting}

\begin{note}[要点]
	熟练掌握switch语句。
\end{note}

\section{abc组合}	
已知abc+cba=n,其中a,b,c均为一位数,1000<n<2000,编程求出满足条件的a,b,c所有组合。

输入说明
	
一个整数n

输出说明
	
按照整数abc从小到大的顺序,输出a, b, c, 用空格分隔,每输出一组a,b,c后换行.

输入样例
	
1352

输出样例
	
3 7 9

4 7 8

5 7 7

6 7 6

7 7 5

8 7 4

9 7 3

\begin{lstlisting}
#include <stdio.h>

int main()
{
	int n,a,b,c;
	scanf("%d",&n);
	for(a = 0; a <= 9; a++)
	 for(b = 0; b <= 9; b++)
	  for(c = 0; c <= 9; c++)
	    if(a*100+b*10+c + c*100+b*10+a == n)
	printf("%d %d %d\n",a,b,c);
	return 0;
} 
\end{lstlisting}

\begin{note}[思考]
	如何用数字分解技巧改写此题。
\end{note}

\section{工资计算}
小明的公司每个月给小明发工资,而小明拿到的工资为交完个人所得税之后的工资。假设他一个月的税前工资为S元,则他应交的个人所得税按如下公式计算:
\begin{enumerate}
	\item 个人所得税起征点为3500元,若S不超过3500,则不交税,3500元以上的部分才计算个人所得税,令A=S-3500元;
	\item A中不超过1500元的部分,税率3%;
	\item A中超过1500元未超过4500元的部分,税率10%;
	\item A中超过4500元未超过9000元的部分,税率20%;
	\item A中超过9000元未超过35000元的部分,税率25%;
	\item A中超过35000元的部分,税率30%;
\end{enumerate}
例如,如果小明的税前工资为10000元,则A=10000-3500=6500元,其中不超过1500元部分应缴税\lstinline|1500×3%=45|元,超过1500元不超过4500元部分应缴税\lstinline|code(4500-1500)×10%=300|元,超过4500元部分应缴税\lstinline|(6500-4500)×20%=400|元。总共缴税745元,税后所得为9255元。

已知小明这个月税前所得为S元,请问他的税后工资T是多少元。

输入格式

输入为一个整数S,表示小明的税前工资。所有评测数据保证小明的税前工资为一个整百的数。

输出格式

输出一个整数T,表示小明的税后工资。

样例输入

10000

样例输出

9255

评测用例规模与约定
对于所有评测用例,$1\le T\le 100000$。

\begin{lstlisting}
#include <stdio.h>

int main()
{
	int S,T,A;
	float tax=0.0;
	scanf("%d",&S);
	A=S-3500;
	if(A<=0) tax=0;
	else
	{
		if(A<=1500)  tax=A*0.03;	
		else if(A>1500 && A<=4500) 
			tax=1500*0.03+(A-1500)*0.1;
		else if(A>4500 && A<=9000) 
			tax=1500*0.03+(4500-1500)*0.1+(A-4500)*0.2;
		else if(A>9000 && A<=35000) 
			tax=1500*0.03+(4500-1500)*0.1+(9000-4500)*0.2+(A-9000)*0.25;
		else 
			tax=1500*0.03+(4500-1500)*0.1+(9000-4500)*0.2+(35000-9000)*0.25+(A-35000)*0.3;
	} 
	T=S-tax;
	printf("%d\n",T);
	return 0;
} 
\end{lstlisting}

\begin{note}[要点]
	掌握基本条件语句。练习if else语句的各种组合形式。
\end{note}


\section{自然数分解}
任何一个自然数m的立方均可写成m个连续奇数之和。例如:
\begin{align*}
1^3 &=1\\
2^3 &=3+5\\
3^3 &=7+9+11\\
4^3 &=13+15+17+19
\end{align*}

编程实现:输入一自然数n, 求组成$n^3$的n个连续奇数。

输入说明

一个正整数n, 0<n<30 。

输出说明

输出n个连续奇数, 数据之间用空格隔开,并换行

输入样例

4

输出样例

13 15 17 19

\begin{lstlisting}
#include <stdio.h>
// 从估计的第一个奇数开始,循环迭代求解。
int main()
{
	int n,i,j,sum,first;
	scanf("%d",&n);
	
	// 第一个可能的奇数:
	if(n%2) first = n;    // n是奇数 
	else first = n + 1;   // n是偶数 
	
	while(1)
	{
		sum = 0; // 每趟内层循环前,必须置0 
		// 从first开始,n个连续奇数, i:表示连续奇数,j:计数。 
		for(i = first,j = 1; j <= n; i += 2,j++ ) 
		{
			sum += i; // 连续奇数累加 
			if(sum == n*n*n) 
			{
				// 输出 
				for(i = first,j = 1; j <= n; i += 2,j++) 
				{
					if (j == n)printf("%d\n",i);
					else printf("%d ",i);
				}
				return 0; // 函数结束 
			}
		}
		first += 2; 
	}
	return 0;
} 
\end{lstlisting}

\begin{note}[要点]
	再次强调进入内层循环前, 相关变量的初始化; 以及标志变量(如本例first)的使用技巧。
\end{note}


\section{跳一跳}
跳一跳是一款微信小游戏,游戏规则非常简单,只需玩家要从一个方块跳到下一个方块,如果未能成功跳到下一个方块则游戏结束。

计分规则如下:

1. 如果成功跳到下一个方块上,但未跳到方块中心,加1分

2. 如果成功跳到下一个方块上,且刚好跳到方块中心,则第一次加2分,此后连续跳到中心时每次递增2分。也就是说,第一次跳到方块中心加2分,连续第二次跳到方块中心加4分,连续第三次跳到方块中心加6分,$\dots$, 以此类推。

3. 如果未能成功跳到方块上,加0分,且游戏结束

现在给出玩家一局游戏的每次跳跃情况,请计算玩家最终得分。

输入说明
	
输入为若干个非零整数(整数个数小于1000),表示玩家每次的跳跃情况。整数之间用空格分隔,整数取值为0, 1, 2。

0 表示未能成功跳到下一个方块上,

1 表示成功跳到下一个方块上但未跳到方块中心,

2 表示成功跳到下一个方块上,且刚好跳到方块中心。

输入的数据只有最后一个整数是0,其余均非零。

输出说明	

输出一个整数表示该玩家的最终得分。

输入样例	

1 1 2 1 2 2 2 0

输出样例
	
17

\begin{lstlisting}
#include <stdio.h>
// 无限循环, 符合结束条件,break
int main()
{
	// last记录上一次的跳跃情况, num表示连续跳至方框中心次数。
	int score=0, a, i, last=0, num=0; 
	while(1) // 无限循环, a==0时, break;
	{
		scanf("%d",&a);
		if(a==1) score++; 
		if((last==1 || last==0) && a==2) // 第一次跳至中心 
		{
			score=score+2; 
		    num=0; // 连续跳至中心清0
		} 
		if(last==2 && a==2) // 连续跳至中心
		{ 
			score=score+2; 
			num++;
		}  
		if(last==2 && (a==1 || a==0)) // 连续跳至中心结束, 开始清算
		{ 
			for(i=1;i<=num;i++) // 结算递增情况 
			{
				score=score+i*2;
			}
			num=0;  // 已经结算, 清0连续跳至中心次数
		} 
		if(a==0) break; // 结束
		last=a; // 记录上一次的跳跃情况
	}
	printf("%d\n",score);	
	return 0;
} 
\end{lstlisting}

\begin{note}[要点]
	通过本题编程, 有助于训练自己的逻辑思维能力。
\end{note}

