%%%%%%%%%%%%%%%%%%%%%%%%%% lecture-4
\begin{frame}
  \frametitle{lecture-4 主要内容}
  \framesubtitle{最简单的C语言程序设计---顺序程序设计}
  \tableofcontents[hideallsubsections]
\end{frame}

\section{数据输入再讨论}

\begin{frame}[fragile]{数据输入再讨论}
在开发平台,以具体的示例,详细讲解以下内容:
\begin{itemize}
	\item \%d, \%f, \%c, \%lf格式符的使用
	\item char c; scanf(``\%c", \&c); 接收输入的字符
	\item char c; c=getchar()接收输入的字符
	\item 避免数字,字符在一条语句中输入的情况,如:\\ scanf(``\%d\%c\%d",...);
	\item 重点理解字符缓冲区的概念,以及消费无用字符的技巧。
\end{itemize}
\end{frame}




