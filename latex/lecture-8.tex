%%%%%%%%%%%%%%%%%%%%%%%%%% lecture-8
\begin{frame}[shrink]
  \frametitle{lecture-8 主要内容}
  \framesubtitle{循环结构程序设计举例}
  \tableofcontents[hideallsubsections]
\end{frame}

\section{循环结构程序设计举例}

\begin{frame}[shrink,fragile]
$[$例5.7$]$用公式$\frac{\pi}{4}\approx 1-\frac{1}{3}+\frac{1}{5}-\frac{1}{7}+\cdots$求$\pi$的近似值, 直到发现某一项的绝对值小于$10^{-6}$为止(该项不累加)。\\
~\\
\begin{columns}
\column{0.7\textwidth}
解题思路:  找规律
\begin{enumerate}
	\item 每项的分子都是1。
	\item 后一项的分母是前一项的分母加2。
	\item 第1项的符号为正,从第2项起,每一项的符号与前一项的符号相反。
	在每求出一项后,检查它的绝对值是否大于或等于10-6。
\end{enumerate}
\column{0.3\textwidth}
\includegraphics[scale=0.3]{pi}
\end{columns}
\end{frame}

\begin{frame}[shrink,fragile]
\begin{lstlisting}
#include <stdio.h>
#include <math.h> //程序中用到数学函数fabs,应包含头文件math.h
int main()
{
  int sign=1; //sign用来表示数值的符号
  double pi=0.0,n=1.0,term=1.0; //pi开始代表多项式的值,最后代表π的值, n代表分母,term代表当前项的值
  while(fabs(term)>=1e-6) //检查当前项term的绝对值是否大于或等于10-6
  {
    pi=pi+term; //把当前项term累加到pi中
    n=n+2;      //n+2是下一项的分母 
    sign=-sign; //sign代表符号,下一项的符号与上一项符号相反
    term=sign/n; //求出下一项的值term
  }
  pi=pi*4; //多项式的和pi乘以4,才是π的近似值
  printf("pi=%10.8f\n",pi);  //输出π的近似值  
  return 0;
}
\end{lstlisting}
\end{frame}

\begin{frame}[shrink,fragile]
\small
$[$例5.8$]$求Fibonacci(斐波那契)数列的前40个数。这个数列有如下特点: 第1, 2两个数为1, 1。从第3个数开始,该数是其前面两个数之和。即该数列为1,1,2,3,5,8,13,\dots,用数学方式表示为:
\[
\begin{cases}
F_1=1 & (n=1)\\
F_2=1 & (n=2)\\
F_n=F_{n-1}+F_{n-2} & (n\ge 3)\\
\end{cases}
\]
\begin{columns}
	\column{0.4\textwidth}
	这是一个有趣的古典数学问题: 有一对兔子,从出生后第3个月起每个月都生一对兔子。小兔子长到第3个月后每个月又生一对兔子。假设所有兔子都不死,问每个月的兔子总数为多少?
	\column{0.6\textwidth}
	\includegraphics[scale=0.45]{Fn}
	\scriptsize
	不满1个月的为小兔子,满1个月不满2个月的为中兔子,满2个月以上的为老兔子。
\end{columns}
\end{frame}

\begin{frame}[shrink,fragile]
解法一: 利用递推(迭代)公式: $F_1=F_2=1; F_3=F_1+F_2; F_1=F_2; F_2=F_3; $
\begin{columns}
	\column{0.6\textwidth}
	\begin{lstlisting}
    #include <stdio.h>
    int main()
    {
        int f1=1,f2=1,f3;
        int i;
        printf("%12d\n%12d\n",f1,f2);
        for(i=1; i<=38; i++)
        {
            f3=f1+f2;
            printf("%12d\n",f3);
            f1=f2;
            f2=f3;
        }
        return 0;
    }
    \end{lstlisting}
	\column{0.3\textwidth}
	\includegraphics[scale=0.6]{Fn1}\\
	\includegraphics[scale=0.6]{Fno1}
\end{columns}
\end{frame}

\begin{frame}[shrink,fragile]
解法二: 利用递推(迭代)公式: $F_1=F_2=1; F_1=F_1+F_2; F_2=F_1+F_2; $
\begin{columns}
	\column{0.5\textwidth}
\begin{lstlisting}
#include <stdio.h>
int main()
{
  int f1=1,f2=1;
  int i;
  for(i=1; i<=20; i++)
  {
    printf("%12d%12d",f1,f2);
    if(i%2==0) // 等效 if(!(i%2)) 
       printf("\n");
    f1=f1+f2;
    f2=f2+f1;
  }
  return 0;
}
\end{lstlisting}
	\column{0.4\textwidth}
	\includegraphics[scale=0.6]{Fn2}\\
	\includegraphics[scale=0.6]{Fno2}
\end{columns}
\end{frame}

\begin{frame}[shrink,fragile]
\small
$[$例5.9$]$输入一个大于3的整数n,判定它是否为素数(prime,又称质数)。
\centering
\begin{columns}
	\column{0.65\textwidth}
\begin{lstlisting}
#include <stdio.h>
int main()
{
  int n,i;
  printf("please enter a integer number,n=?");
  scanf("%d",&n);
  for (i=2;i<n;i++)
    if(n%i==0) break;
  if(i<n) // for提前结束
    printf("%d is not a prime number.\n",n);
  else // for正常结束
    printf("%d is a prime number.\n",n);
  return 0;
}
\end{lstlisting}
	\column{0.3\textwidth}
	\includegraphics[scale=0.5]{prime}
\end{columns}
\textbf{\textcolor{blue}{只要在循环结束后检查循环变量i的值,就能判定循环是提前结束还是正常结束的。从而判定n是否为素数。这种判断循环结束的方法以后会常用到。}}
\end{frame}

\begin{frame}[shrink,fragile]
\small
\textbf{\textcolor{blue}{优化: }} $n$不必被$2\sim (n-1)$内的各整数去除,只须将$n$被$2\sim\sqrt{n}$之间的整数除即可。因为n的每一对因子, 必然有一个小于n, 另一个大于n。
%\centering
\begin{lstlisting}
#include <stdio.h>
#include <math.h>
int main()
{
   int n,i,k;
   printf("please enter a integer number,n=?");
   scanf("%d",&n);
   k=sqrt(n); // 自动转换为整数(不会四舍五入), 相当于k=(int)sqrt(n);
   for (i=2;i<=k;i++)
     if(n%i==0) break;
   if(i<=k) 
      printf("%d is not a prime number.\n",n);
   else 
      printf("%d is a prime number.\n",n);
   return 0;
}
\end{lstlisting}
\end{frame}

\begin{frame}[shrink,fragile]
\small
\textbf{\textcolor{blue}{使用标志变量, 判断循环结束条件。}}
%\centering
\begin{lstlisting}
int n,i,k,flag=1; // flag: 标志变量
k=sqrt(n); // 自动转换为整数(不会四舍五入), 相当于k=(int)sqrt(n);
for (i=2;i<=k;i++)
   if(n%i==0) { flag=0; break; }
if(!flag) // for提前结束
   printf("%d is not a prime number.\n",n);
else // for正常结束
   printf("%d is a prime number.\n",n);
\end{lstlisting}
\pause
\rule{\textwidth}{1pt} %水平线
\begin{lstlisting}
int n,i,k,flag=1; // flag: 标志变量
k=sqrt(n); // 自动转换为整数(不会四舍五入), 相当于k=(int)sqrt(n);
for (i=2;i<=k && flag;i++)
   if(n%i==0) { flag=0; }
if(!flag) 
   printf("%d is not a prime number.\n",n);
else 
   printf("%d is a prime number.\n",n);
\end{lstlisting}
\end{frame}

\begin{frame}[shrink,fragile]
\small
$[$例5.10$]$求$100\sim 200$间的全部素数。
\pause
\begin{lstlisting}
#include <stdio.h>
#include <math.h>
int main()
{
  int n,i,k;
  for (n=101;i<=200;n+=2) //n从101变化到200, 对每个奇数n进行判定
  {
     k=sqrt(n); // 自动转换为整数(不会四舍五入), 相当于k=(int)sqrt(n);
     for(i=2;i<=k;i++)
        if(n%i==0) break;
     if(i>k) 
        printf("%d ",n);
  }
  printf("\n");
  return 0;
}
\end{lstlisting}
\end{frame}

\begin{frame}[shrink,fragile]
\small
$[$例5.11$]$译密码。为使电文保密,往往按一定规律将其转换成密码,收报人再按约定的规律将其译回原文。例如,可以按以下规律将电文变成密码:将字母A变成字母E,a变成e,即变成其后的第4个字母,W变成A,X变成B,Y变成C,Z变成D。
\pause
\begin{lstlisting}
char c;
c=getchar(); //输入一个字符给字符变量c
while(c!='\n') //检查c的值是否为换行符'\n'  
{
  if((c>='a' && c<='z') || (c>='A' && c<='Z')) //c如果是字母
  {
     if((c>='W' && c<='Z') || (c>='w' && c<='z')) c = c-22; //如果是26个字母中最后4个字母之一就使c-22
     else  c =c + 4;   //如果是前面22个字母之一,就使c + 4
}
printf("%c",c); //输出已改变的字符
c=getchar();    //再输入下一个字符给字符变量c
}
printf("\n");
\end{lstlisting}
\end{frame}

\begin{frame}[shrink,fragile]
\textbf{\textcolor{blue}{在循环条件中接收输入的字符是一种常见技巧。}}
\begin{lstlisting}
char c;
while((c=getchar())!='\n') //检查c的值是否为换行符'\n'  
{
   if((c>='a' && c<='z') || (c>='A' && c<='Z')) //c如果是字母
   {
      if((c>='W' && c<='Z') || (c>='w' && c<='z')) c = c-22; //如果是26个字母中最后4个字母之一就使c-22
      else  c =c + 4;   //如果是前面22个字母之一,就使c + 4
   }
   printf("%c",c); //输出已改变的字符
}
printf("\n");
\end{lstlisting}
\end{frame}

\begin{frame}[fragile]
附加题1: 求$s=a+aa+aaa+\cdots+a\cdots a$, 其中$a$是一个$1\sim 9$的数字。例如$a=2, n=4$时, $s=2+22+222+2222$, $a$和$n$由键盘输入。
\pause
\begin{columns}
\column{0.5\textwidth}
\begin{lstlisting}
int i,s,n,term = 0;
for(i=1,s=0; i<=n; i++) // 初始化循环变量用逗号隔开
{
   term = term*10 + a;
   s += term; 
}
\end{lstlisting}
\end{columns}
\end{frame}

\begin{frame}[fragile]
\small
附加题2: 韩信点兵。韩信有一队兵, 他想知道有多少人, 便让士兵排队报数:\\
按从1至5报数, 最末一个士兵报的数为1;\\
按从1至6报数, 最末一个士兵报的数为5;\\
按从1至7报数, 最末一个士兵报的数为4;\\
按从1至11报数,最末一个士兵报的数为10;\\ 
计算韩信至少有多少兵。 
\pause
\begin{columns}
\column{0.8\textwidth}
\begin{lstlisting}
int x=1;
for(;;x++)  // 循环体仅含if()结构,看作一条语句,'{}'可省略
    if(x%5==1 && x%6==5 && x%7==4 && x%11==10)
    { 
       printf("%d\n",x);  
       break;
    }
\end{lstlisting}
\end{columns}
\end{frame}

\begin{frame}[fragile]
附加题3: 求水仙花数。如果一个三位数的个位数、十位数和百位数的立方和等于该数自身,则称该数为水仙花数。 \\
编程求出所有的水仙花数。\\
~\\  
\pause
\begin{columns}
\column{0.6\textwidth}
解法一: 采用三重循环
\begin{lstlisting}
int i,j,k; // 百、十、个位
for(i=1;i<=9;i++)    // 百位
  for(j=0;j<=9;j++)  // 十位
    for(k=0;k<=9;k++)  // 个位
      if(i*100+j*10+k == i*i*i+j*j*j+k*k*k)  
        printf("%d\n",i*100+j*10+k);
\end{lstlisting}
\end{columns}
\end{frame}

\begin{frame}[fragile]
附加题3: 求水仙花数。如果一个三位数的个位数、十位数和百位数的立方和等于该数自身,则称该数为水仙花数。 \\
编程求出所有的水仙花数。\\
~\\ 
\pause
\begin{columns}
\column{0.6\textwidth}
解法二: 采用一重循环
\begin{lstlisting}
int m,i,j,k; 
for(m=100;m<=999;m++)
{
   i=m/100; j=m/10%10; k=m%10;  
   if(i*100+j*10+k == i*i*i+j*j*j+k*k*k)  
      printf("%d\n",i*100+j*10+k);
}
\end{lstlisting}
\end{columns}
~\\
\pause
\textcolor{blue}{思考: 输出共有多少个水仙数?}
\end{frame}

\begin{frame}[fragile]
附加题3: 求整数区间$[a,b]$中水仙花数的个数。
\pause
\begin{columns}
\column{0.8\textwidth}
\begin{lstlisting}
int n=0; //计数 
int a,b; // a,b 区间
int i,t;   // 循环变量,代表a,b区间的每个数
int sum; // i的各位立方和 
scanf("%d%d",&a,&b);
for(i=a;i<=b;i++) // 考察i是否水仙数
{  
  sum = 0; t=i; // 临时变量记住i; 易遗漏每次内层循环前sum要归0
  while(t!=0) // 累加各位立方 
  { sum+=pow(t%10,3); t=t/10; }
  if(sum==i) n++; // i是水仙数 
}
printf("%d\n",n);
\end{lstlisting}
\end{columns}
\end{frame}

\begin{frame}[fragile]
附加题4: 百钱百鸡, 已知公鸡5个钱1只, 母鸡3个钱1只, 小鸡1个钱3只, 用100个钱买了100只鸡。问公鸡、母鸡、小鸡各几只? 
\vspace{0.5cm}
\pause
\begin{columns}
%\column{0.9\textwidth}
\begin{lstlisting}
int x,y,z; // 公鸡、母鸡、小鸡个数
for(x=0;x<=100;x++) 
   for(y=0;y<=100;y++) 
     for(z=0;z<=100;z++) 
       if(5*x+3*y+z/3 == 100 && x+y+z == 100 && z%3 == 0) // 全部条件! 
          printf("%d,%d,%d\n",x,y,z);
\end{lstlisting}
\end{columns}
\vspace{0.5cm}
\pause
\textcolor{blue}{如何考虑无解的情况?}
\end{frame}

\begin{frame}{注意事项小结}
\begin{enumerate}
\setlength{\itemsep}{.5cm}
\item while( )\{ \}; do \{ \} while( ); for(;;)\{ \}执行顺序;
\item 循环变量的开始和结束条件;
\item 循环体是复合语句时,必须用\{ \}扩起来;
\item 必要时,用break结束整个循环,用continue结束本次循环;
\item 关键是找出循环规律,必要时设计流程图,指导代码实现。	
\end{enumerate}
\end{frame}