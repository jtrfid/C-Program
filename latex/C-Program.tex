% !Mode::"TeX:UTF-8"
% !Mode:: "TeX:UTF-8"
\documentclass[xcolor=svgnames,serif,table,10pt]{beamer}
\mode<presentation>{
% Setup appearance:
\useoutertheme{infolines}
\usetheme{Darmstadt}
\setbeamercovered{transparent}
\setbeamertemplate{caption}[numbered]
\setbeamertemplate{navigation symbols}{}
\setbeamertemplate{blocks}[rounded][shadow=true]
\setbeamertemplate{enumerate items}[circle]

% 修改样式
\setbeamercolor{box}{bg=black!20!orange,fg=white}
\setbeamercolor{block title}{use=sidebar,fg=sidebar.fg!10!white,bg=orange!70!black}
\setbeamercolor{block title example}{use=sidebar,fg=sidebar.fg!10!white,bg=black!60!green}
\setbeamercolor{block title alerted}{use=sidebar,fg=sidebar.fg!10!white,bg=black!50!red}

\setbeamertemplate{headline}
{%
  \begin{beamercolorbox}[shadow=true]{section in head/foot}
  \vskip2pt\insertnavigation{\paperwidth}\vskip2pt
  \end{beamercolorbox}%
}
}
\usepackage{url}
\usepackage{animate}
\usepackage[english]{babel}
\usepackage{times}
\usepackage[T1]{fontenc}
\usepackage{multirow,multicol,longtable}
\usepackage{graphics}
\usepackage{xcolor}
\usepackage[no-math]{fontspec}%-------------------------------------------------- 提供字体选择命令
\usepackage{xunicode}%----------------------------------------------------------- 提供Unicode字符宏
\usepackage{xltxtra}%------------------------------------------------------------ 提供了针对XeTeX的改进并且加入了XeTeX的LOGO
\usepackage[BoldFont,SlantFont,CJKchecksingle]{xeCJK}%--------------------------- 使用xeCJK宏包


%================================== 设置中文字体 ================================%
%\setCJKmainfont{Adobe Heiti Std}%------------------------------------------------设置正文为黑体
%\setCJKmonofont{Adobe Song Std}%-------------------------------------------------设置等距字体
%\setCJKsansfont{Adobe Kaiti Std}%------------------------------------------------设置无衬线字体
% \setCJKfamilyfont{zxzt}{FZShouJinShu-S10S}
% \setCJKfamilyfont{FZDH}{FZDaHei-B02S}
%================================== 设置中文字体 ================================%

%================================== 设置英文字体 ================================%
\setmainfont[Mapping=tex-text]{Times New Roman}%--------------------------------英文衬线字体
\setsansfont[Mapping=tex-text]{Arial}%------------------------------------英文无衬线字体
\setmonofont[Mapping=tex-text]{Courier New}%-------------------------------------英文等宽字体
\newfontfamily\Arial{Arial}
%================================== 设置英文字体 ================================%

%================================== 设置数学字体 ================================%
%\setmathsfont(Digits,Latin,Greek)[Numbers={Lining,Proportional}]{Minion Pro}
%================================== 设置数学字体 ================================%
\punctstyle{kaiming}%------------------------------------------------------------ 开明式标点格式
\usepackage{graphicx}
\usepackage{tikz}
\usetikzlibrary{positioning,backgrounds}
\usetikzlibrary{fadings}
\usetikzlibrary{patterns}
\usetikzlibrary{calc}
\usetikzlibrary{shadings}
\pgfdeclarelayer{background}
\pgfdeclarelayer{foreground}
\pgfsetlayers{background,main,foreground}
\usepackage{xifthen}
\usepackage{colortbl,dcolumn}
\usepackage{enumerate}
\usepackage{pifont}
\usepackage{tabularx}
\usepackage{booktabs}
\usepackage{hyperref}

\usepackage{bm} % 字体加粗的方法
\usetikzlibrary{arrows,automata,matrix,arrows}

%=================================== 数学符号 =================================%
\newcommand{\rtn}{\mathrm{\mathbf{R}}}
\newcommand{\N}{\mathrm{\mathbf{N}}}
\newcommand{\As}{\mathrm{a.s.}}
\newcommand{\Ae}{\mathrm{a.e.}}
\newcommand*{\PR}{\mathrm{\mathbf{P}}}
\newcommand*{\EX}{\mathrm{\mathbf{E}}}
\newcommand{\EXlr}[1]{\mathrm{\mathbf{E}}\left[#1\right]}
\newcommand*{\dif}{\,\mathrm{d}}
\newcommand*{\F}{\mathcal{F}}
\newcommand*{\h}{\mathcal{H}}
\newcommand*{\vp}{\varepsilon}
\newcommand*{\prs}{\dif\PR-\As}
\newcommand*{\dte}{\dif t-\Ae}
\newcommand*{\pts}{\dif\PR\times\dif t-\Ae}
\newcommand{\Ito}{It\^{o}}
\newcommand{\tT}[1][0]{[#1,T]}
\newcommand{\intT}[2][T]{\int^{#1}_{#2}}
\newcommand{\intTe}[1][t]{\intT[t+\varepsilon]{#1}}
\newcommand{\s}{\mathcal{S}}
\newcommand{\me}{\mathrm{e}}
\newcommand{\one}[1]{{\bf 1}_{#1}}
\renewcommand{\M}{{\rm M}}
\newcommand{\Me}[1][t]{M^{\varepsilon}_{#1}}
\newcommand{\Ne}[1][t]{N^{\varepsilon}_{#1}}
\newcommand{\Pe}[1][t]{P^{\varepsilon}_{#1}}
\DeclareMathOperator*{\sgn}{sgn}
% =================================== 数学符号 =================================%

% 定义罗马数字
\makeatletter
\newcommand{\rmnum}[1]{\romannumeral #1}
\newcommand{\Rmnum}[1]{\expandafter\@slowromancap\romannumeral #1@}
\makeatother

% 定义破折号
\newcommand{\pozhehao}{\kern0.3ex\rule[0.8ex]{2em}{0.1ex}\kern0.3ex}
% 中文日期
\def\CJK@today{\the\year 年 \the\month 月}
\newcommand\zhtoday{\CJK@today}

% 中文图表
\renewcommand\figurename{图}
\renewcommand\tablename{表}

\usepackage{verbatim} % \begin{comment} xxxx \end{comment}

\usetheme{Berlin} %主题
%\usecolortheme{sustech} %主题颜色

\usepackage{algorithm}  
\usepackage{algorithmicx}  
\usepackage{algpseudocode}
\floatname{algorithm}{算法}
\renewcommand{\algorithmicrequire}{\textbf{输入:}} 
\renewcommand{\algorithmicensure}{\textbf{输出:}}  

\algrenewcommand{\algorithmiccomment}[1]{ $//$ #1}

\definecolor{mygreen}{rgb}{0,0.6,0}
\definecolor{mygray}{rgb}{0.5,0.5,0.5}
\definecolor{mymauve}{rgb}{0.58,0,0.82}
\usepackage{listings}
\lstset{ %
	backgroundcolor=\color{white},   % choose the background color
	basicstyle=\footnotesize\ttfamily,     % size of fonts used for the code
	columns=fullflexible,
	breaklines=true,                 % automatic line breaking only at whitespace
	captionpos=b,                    % sets the caption-position to bottom
	tabsize=4,
	commentstyle=\color{mygreen},    % comment style
	escapeinside={\%*}{*)},          % if you want to add LaTeX within your code
	keywordstyle=\color{blue},       % keyword style
	stringstyle=\color{mymauve}\ttfamily,     % string literal style
	numbers=left, 
	%	frame=single,
	rulesepcolor=\color{red!20!green!20!blue!20},
	% identifierstyle=\color{red},
	language=c
}

\graphicspath{{figures/},} 

% Author, Title, etc.

\title{计算机导论与程序设计[CS006001-60]}

%% \subtitle{Foreground-constrained Eulerian Video Motion Magnification}

\author[段江涛]{段江涛
\\机电工程学院}

\institute[]{\includegraphics[height=1cm]{xd.jpg}}

\date{\zhtoday}

\setlength{\baselineskip}{22pt}
\renewcommand{\baselinestretch}{1.4}

% The main document

\begin{document}
\setlength{\abovedisplayskip}{1ex}%------------------------------------------ 公式前的距离
\setlength{\belowdisplayskip}{1ex}%------------------------------------------ 公式后的距离

\begin{frame}
  \titlepage
\end{frame}

%%%%%%%%%%%%%%%%%%%%%%%%%%%%%%%%
%%%%%%%%%%%%%%%%%%%%%%%%%%%%%%%%
%%%%%%%%%%%%%%%%%%%%%%%%%%% lecture-1
%\begin{frame}
%  \frametitle{lecture-1 主要内容}
%  \tableofcontents[hideallsubsections]
%\end{frame}

\section{课程介绍}

\begin{frame}[shrink]{课程内容}
\begin{itemize}
	%\setlength{\itemsep}{.5cm}
	\item 计算机导论:了解计算机的基本知识;掌握计算机操作基本技能。
	\item 程序设计:掌握结构化程序设计方法, 训练程序逻辑思维能力。会读、会编、会调试C语言程序。
	\item 学习方法:线上、线下相结合。认真书写课堂笔记, 按时完成上机练习作业, 鼓励大量编程练习。
	\item 教材
	\begin{itemize}
		\item 大学计算机,龚尚福, 贾澎涛,西安电子科技大学出版社
		\item C程序设计第五版, 谭浩强,清华大学出版社
	\end{itemize}
	%\item 智慧教育平台(使用Chrome浏览器): 
	%\href{https://cvnis.xidian.edu.cn/}{ https://cvnis.xidian.edu.cn/}
	\item 线上参考课程资源链接:
	%\href{run:./online resource.pdf}{online resource.pdf}
	\href{./online resource.pdf}{online resource.pdf} % 可以不使用run
\end{itemize}
\end{frame}

\begin{frame}{线上导论部分学习内容}
\begin{enumerate}
	\setlength{\itemsep}{.8cm}
	\item 计算机历史、现状、发展趋势与前沿技术概述
	\item 计算机体系结构及其编码方式
	\item 计算机组成与软件系统
	\item 计算机应用实践
\end{enumerate}
\end{frame}

\begin{frame}{考核}
\begin{enumerate}
	\setlength{\itemsep}{.5cm}
	\item 平时成绩: 10\%~由上机练习,课堂讨论等部分组成。
	\item 导论部分: 20\%~结合线上资源, 自学字处理软件。总结知识点、课堂笔记,撰写课程学习报告。
	\item 期中考试: 30\%~根据机试系统给出的题目编写程序,通过调试得到正确结果并通过\textbf{机试系统提交}。
	\item 期末考试: 40\%~根据机试系统给出的题目编写程序,通过调试得到正确结果并通过\textbf{机试系统提交}。
\end{enumerate}
\end{frame}

\section{导论简介}

\begin{frame}{计算机导论主要内容}
\textbf{总体要求: 了解计算机的基本知识; 掌握计算机操作基本技能。}\\
\begin{itemize}
	\item 计算机系统组成
	\item 计算机工作原理
	\item 操作系统
	\item 字处理: Microsoft Word
	\item 电子表格: Microsoft Excel
	\item 演示文稿: Microsoft PowerPoint
\end{itemize}
\end{frame}

\begin{frame}{计算机工作原理}
\textbf{工作原理: ``存储程序'' + ``程序控制''}
\begin{enumerate}
	\item 以\textbf{二进制}形式表示数据和指令
	\item 将程序存入存储器中,由控制器自动读取并执行
	\item 外部存储器存储的程序和所需数据$\implies$计算机内存$\implies$在程序控制下由CPU周而复始地取出指令、分析指令、执行指令$\implies$操作完成。	
\end{enumerate}
\centering\includegraphics[scale=0.25]{hframe}
\end{frame}

\section{C语言程序设计简介}

\begin{frame}{计算机程序}
\centering
\includegraphics[scale=0.4]{program1}
\end{frame}

\begin{frame}{计算机语言}
\centering
\includegraphics[scale=0.4]{program2}
\end{frame}

\begin{frame}{高级语言的发展}
\centering
\includegraphics[scale=0.4]{program3}
\end{frame}

\begin{frame}{C语言的特点}
\vspace{-0.5cm}
\begin{enumerate}
	\item 语言简洁、紧凑,使用方便、灵活
    \item 运算符丰富
    \item 数据类型丰富
    \item \textcolor{blue}{C语言是完全模块化和结构化的语言}\\
          具有结构化的控制语句(顺序、选择、循环结构)\\
          用函数作为程序的模块单位,便于实现程序的模块化
    \item \textcolor{blue}{兼具高级语言和低级语言的功能}\\
          允许直接访问物理地址\\
          能进行位(bit)操作\\  
          能实现汇编语言的大部分功能\\
          可以直接对硬件进行操作        
\end{enumerate}
\end{frame}

\begin{frame}{程序设计的任务}
\centering
\includegraphics[scale=0.4]{program}
\end{frame}

\begin{frame}[fragile]{第一个C语言程序}
    \begin{lstlisting}
    #include<stdio.h>            // standard input/output编译预处理指令
    int main()                   // 主函数
    {                            // 函数开始标志
       printf("Hello World!");   // printf函数,输出一行信息
       return 0;                 // 函数执行完毕返回函数值0
    }                            // 函数结束标志
    \end{lstlisting}
\end{frame}

\begin{frame}[fragile]{求两个整数之和}
\begin{lstlisting}
#include<stdio.h>         // standard input/output编译预处理指令
int main()                // 主函数
{                         // 函数开始标志
   int a,b,sum;           // 定义a,b,sum为整型变量
   a=123;                 // 对变量a赋值
   b=456;				  // 对变量b赋值
   sum=a+b;               // 计算a+b, 并把结果存放在变量sum中
   printf("sum is %d\n",sum);// printf函数,输出结果
   return 0;                 // 函数执行完毕返回函数值0
}                            // 函数结束标志
\end{lstlisting}
\end{frame}

\begin{frame}[fragile]{求5!的C语言程序}
\begin{lstlisting}
#include<stdio.h>            // standard input/output编译预处理指令
int main()                   // 主函数
{                            // 函数开始标志
	int i,p;  // p表示被乘数, i表示乘数
	p=1;	  // 对变量p赋值
	i=2;      // 对循环变量i赋值
	while(i<=5) // 循环结构
	{  
		p=p*i;  // =右端p记录本语句执行前的值,左端p记录执行后p的值。
		i++;    // i = i + 1
	}
	printf("%d\n",p);        // printf函数, 输出p的计算结果
	return 0;                // 函数执行完毕返回函数值0
}                            // 函数结束标志
\end{lstlisting}
\end{frame}

\section{开发工具}

\begin{frame}{运行C程序的步骤与方法}
\centering
\includegraphics[scale=0.45]{program4}
\end{frame}

\begin{frame}{集成开发环境---编译系统}
\begin{itemize}
\setlength{\itemsep}{.5cm}
\item Bloodshed Dev-C++ 
\item Turbo C
\item Visual C++6.0 
\item Visual Studio(VS2015,VS Community 2019等)
\end{itemize}
\end{frame}

\begin{frame}{Bloodshed Dev-C++集成开发环境}
\includegraphics[scale=0.24]{DevCpp}   
\end{frame}

\begin{frame}{Bloodshed Dev-C++集成开发环境}
\vspace{-0.3cm}
\begin{itemize}
\item 选择“文件”菜单,选择“源文件”, 编辑程序。
\item 保存时,保存为 .cpp或 .c文件。
\item 选择“编译和运行”菜单,生成.exe文件,运行程序。  
\end{itemize}
\centering 
\includegraphics[scale=0.2]{DevCpp}   
\end{frame}

%%%%%%%%%%%%%%%%%%%%%%%%%% lecture-2
\begin{frame}
  \frametitle{lecture-2 算法---程序的灵魂}
  \tableofcontents[hideallsubsections]
\end{frame}

\section{开发工具}

\begin{frame}{运行C程序的步骤与方法}
\centering
\includegraphics[scale=0.45]{program4}
\end{frame}

\begin{frame}{集成开发环境---编译系统}
\begin{itemize}
\setlength{\itemsep}{.5cm}
\item Bloodshed Dev-C++ 
\item Turbo C
\item Visual C++6.0 
\item Visual Studio(VS2015,VS Community 2019等)
\end{itemize}
\end{frame}

\begin{frame}{Bloodshed Dev-C++集成开发环境}
\includegraphics[scale=0.24]{DevCpp}   
\end{frame}

\begin{frame}{Bloodshed Dev-C++集成开发环境}
\begin{itemize}
\item 选择“文件”菜单,选择“源文件”, 编辑程序。
\item 保存时,保存为 .cpp或 .c文件。
\item 选择“编译和运行”菜单,生成.exe文件,运行程序。  
\end{itemize}
\centering 
\includegraphics[scale=0.2]{DevCpp}   
\end{frame}

\section{Algorithm + Data Structures = Programs}

\begin{frame}{数据结构与算法}
\begin{columns}
	\column{0.4\textwidth}
	\begin{block}{}
		算法 + 数据结构 = 程序\\
		Algorithm + Data Structures = Programs
	\end{block}
    %\hspace{20pt}
	\column{0.4\textwidth}
	\includegraphics[scale=0.4]{Wirth}\\
	沃思(Niklaus Wirth)
\end{columns}
\begin{itemize}
	\item 数据结构\\
	对数据的描述。在程序中要指定用到哪些数据,以及这些数据的类型和数据的组织形式。
	\item 	算法\\
	对操作的描述。即要求计算机进行操作的步骤	
\end{itemize}
\end{frame}

\begin{frame}{程序员的工作}
\centering
\includegraphics[scale=0.5]{programer}
\end{frame}

\begin{frame}{算法}
\begin{columns}
	\column{0.4\textwidth}
	\includegraphics[scale=0.25]{programer2}
	\column{0.5\textwidth}
	\begin{block}{算法}
		\begin{itemize}
			\item 广义地说,为解决一个问题而采取的方法和步骤,就称为“算法”。
		    \item 对同一个问题,可以有不同的解题方法和步骤。
		    \item 为了有效地进行解题,不仅需要保证算法正确,还要考虑算法的质量,选择合适的算法。
		\end{itemize} 
	\end{block}
\end{columns}
\end{frame}

\begin{frame}{算法}
\begin{columns}[t]
	\column{0.45\textwidth}
	\begin{block}{数值运算算法}
	   如求一个方程的根, 计算一个函数的定积分等。\\
	   数值运算的目的是求数值解。\\
	   由于数值运算往往有现成的模型,可以运用数值分析方法,因此对数值运算的算法的研究比较深入,算法比较成熟。	
	\end{block}
	\column{0.45\textwidth}
	\begin{block}{非数值运算算法}
		如图书检索, 人事管理等。\\
		计算机在非数值运算方面的应用远超在数值运算方面的应用。
		非数值运算的种类繁多,要求各异,需要使用者参考已有的类似算法,重新设计解决特定问题的专门算法。
	\end{block}
\end{columns}
\end{frame}

\begin{frame}[shrink]{简单的算法举例[例2.1(p17)]}
\begin{example}
	[例2.1(p17)] 求$1\times 2\times 3\times 4\times 5$
\end{example}
\vspace{-0.7cm}
\small
\begin{columns}[t]
	\column{0.45\textwidth}
	\begin{block}{算法(一)步骤}<1->
		\begin{enumerate}
			\item[S1] 先求1乘以2,得到结果2
			\item[S2] 将步骤1得到的乘积2再乘以3,得到结果6
			\item[S3] 将6再乘以4,得24
			\item[S4] 将24再乘以5,得120
		\end{enumerate}	
	\end{block}
    \begin{block}{思考}<3->
    	求$1\times 3\times 5\times 7\times 9$
    \end{block}
	\column{0.45\textwidth}<2->
	\begin{block}{算法(二)步骤}
		\begin{enumerate}
			\item[S1] $p=1$, 表示将1存放在变量p中
			\item[S2] $i=2$, 表示将2存放在变量i中
			\item[S3] $p=p*i$, 使p与i相乘,乘积仍放在变量p中
			\item[S4] $i=i+1$, 使变量i的值加1
			\item[S5] if (i<=5) goto S3\\
			          else 算法结束, 最后得到p的值就是5!的值。
		\end{enumerate}	
	\end{block}
\end{columns}
\end{frame}

\begin{frame}[shrink]{简单的算法举例[例2.2(p18)]}
\begin{example}
	[例2.2(p18)] 有50个学生,要求输出成绩在80分以上的学生的学号和成绩.
\end{example}

\begin{block}{算法步骤}<2->
	float g[50]=\{100,90.5,30.8,$\cdots$\}; \textcolor{red}{// 表示50名学生成绩}\\
	int i = 0;  \textcolor{red}{//表示第i个学生学号}\\
	while(i<50) \\
	\{ \\
	   \qquad if (g[i]>=80) printf("第\%d个学生成绩\%f, " , i+1, g[i]); \\
	   \qquad i = i + 1;\\
	\} 
\end{block}
\end{frame}

\begin{frame}{例2.3(p18): 闰年判定条件.}
\centering
\includegraphics[scale=0.4]{leap}
\end{frame}

\note{
	1、普通闰年bai:公历年份是4的倍数的,一般是闰年。du(zhi如2004年就是闰年);
	
	2、世纪dao闰年:公历年份是整百数的,必须是400的倍数才是闰年(如1900年不是世纪闰年,2000年是世纪闰年)。
}

\begin{frame}[shrink]
\begin{columns}%[T] % align columns
	\begin{column}{1.0\textwidth}
		\begin{algorithm}[H]  
			\caption{例2.3(p18): 判定2000—2500年中的每一年是否为闰年.} %算法的名字
			\begin{algorithmic}[1] %每行显示行号
				\State int year=2000, char R; \textcolor{red}{// R是标志变量, 'Y'或'N'}
				\While{(year<=2500)} % While语句,需要和EndWhile对应
				\State R=`N';  
				\If{(year能被4整除,但是不能被100整除)} R=`Y'; % If 语句,需要和EndIf对应
				\ElsIf {(year能被100整除, 并且能被400整除)} R=`Y';
				\EndIf
				\If{(R=='Y')} printf(``\%d是闰年", year); 
				\Else \quad printf(``\%d不是闰年", year);
				\EndIf	
				\State year = year + 1;
				\EndWhile
			\end{algorithmic}  
		\end{algorithm}
	\end{column}%
\end{columns}
\begin{textblock*}{15cm}(8.5cm,5cm)
	\includegraphics[scale=0.25]{leap}
\end{textblock*}
\end{frame}

\begin{frame}%[shrink]
\begin{columns}%[T] % align columns
	\begin{column}<0->{.8\textwidth}
		%\scriptsize
		\begin{algorithm}[H]  
			\caption{例2.4(p19): 求$1-\frac{1}{2}+\frac{1}{3}-\frac{1}{4}+\cdots+\frac{1}{99}-\frac{1}{100}$.} %算法的名字
			\begin{algorithmic}[1] %每行显示行号
				\State int sign=1, deno=2;
				\State float sum = 1.0; 
				\While{(deno<=100)} % While语句,需要和EndWhile对应
				\State sign = -1 * sign;
				\State sum = sum + sign*1.0/deno; 
				\State deno = deno + 1; 
				\EndWhile
				\State printf(``sum=\%f$\backslash$n", sum);	
			\end{algorithmic}  
		\end{algorithm}
	\end{column}%
	%\hfill%	
	\begin{column}<0->{.20\textwidth}
		\newline
		\newline
		sign: 表示当前项的数值符号\\
		deno: 表示当前项的分母\\
		sum:  表示当前项的累加和
	\end{column}%
\end{columns}
\medskip
\colorbox{green}{问题: 为何使用sign*1.0 ?}
\end{frame}

\begin{frame}%[shrink]
\begin{columns}%[T] % align columns
	\begin{column}<0->{1.0\textwidth}
		%\scriptsize
		\begin{algorithm}[H]  
			\caption{例2.5(p20): 给出一个大于或等于3的正整数,判断它是不是一个素数.} %算法的名字
			\begin{algorithmic}[1] %每行显示行号
				\State int n, i=2;
				\State scanf("\%d",\&n); \textcolor{red}{// 输入n的值.} 
				\While{(i < n)} % While语句,需要和EndWhile对应
				\If{(n能被i整除)} \{ printf(``\%d不是素数", n); return; \}
				\EndIf
				\State i = i + 1;
				\EndWhile
				\State printf(``\%d是素数", n);	
			\end{algorithmic}  
		\end{algorithm}
	\end{column}%
	%\hfill%	
\end{columns}
\begin{block}{Notes}
	实际上, $n$不必检查被$2\sim(n-1)$之间的整数整除, 只须检查能否被$2\sim\sqrt{2}$间的整数整除即可。
\end{block}
\end{frame}

\begin{frame}{算法的特性}
\includegraphics[scale=0.35]{algo}
\end{frame}

\begin{frame}{结构化程序设计方法}
\includegraphics[scale=0.5]{top-down}
\end{frame}

\section{初识C语言程序(作业)}

\begin{frame}[fragile]{求5!的C语言程序。\small{[作业: 请抄写以下各页,并试着分析理解。]}}
\begin{lstlisting}
#include<stdio.h>            // standard input/output编译预处理指令
int main()                   // 主函数
{                            // 函数开始标志
   int i,p;  // p表示被乘数, i表示乘数
   p=1;
   i=2;
   while(i<=5)
   {  
      p=p*i;
      i++; // i = i + 1
   }
   printf("%d\n",p);
   return 0;                 // 函数执行完毕返回函数值0
}                            // 函数结束标志
\end{lstlisting}
\end{frame}

\begin{frame}[fragile]{变量在使用之前首先要定义它的数据类型}
\begin{lstlisting}
#include<stdio.h>            // standard input/output编译预处理指令
int main()                   // 主函数
{                            // 函数开始标志
   int a,b;  // 定义变量a, b为整型数值,同类型变量可以在一条语句中定义。
   float f;  // 定义变量f为单精度浮点数
   double d; // 定义变量d为双精度浮点数
   char c;   // 定义变量c为单个英文字母
   a=10;
   b=20;
   f=10.2;
   d=20.3;
   c='A';
   return 0;                 // 函数执行完毕返回函数值0
}                            // 函数结束标志
\end{lstlisting}
\end{frame}

\begin{frame}{常用格式描述符与数据类型的对应关系}
\begin{tabular}{|c|c|c|}
	\hline 
	\textbf{格式符} & \textbf{对应的数据类型} &  \textbf{备注}\\ 
	\hline 
	\%d & int &  \\ 
	\hline  
	\%f & float &  \\
	\hline
	\%c & char & \\ 
	\hline   
	\%lf & double & \\ 
	\hline 
	\%.2f & float & 保留两位小数, 四舍五入。不适用于scanf()。 \\ 
	\hline 
	\%.2lf & double & 保留两位小数, 四舍五入。不适用于scanf()。 \\ 
	\hline
	\hline   
	\%x & int & 十六进制显示 \\ 
	\hline 
	\%ld & long int &  \\ 
	\hline 
\end{tabular}
\newline
\newline
\textcolor{blue}{详见p73, 表3.6}
\end{frame}

\begin{frame}[fragile]{输出语句printf(``原样输出, \%格式符'', 对应变量值);}
\begin{lstlisting}
#include<stdio.h>            // standard input/output编译预处理指令
int main()                   // 主函数
{                            // 函数开始标志
   int a=10,b;    // 定义变量a, b为整型数值, 定义变量时,可以指定变量的初值
   float f=10.2;  // 定义变量f为单精度浮点数
   double d; // 定义变量d为双精度浮点数
   char c;   // 定义变量c为单个英文字母
   f=10.2;
   d=20.3;
   c='A';
   printf("a=%d,b=%d,c=%c,f=%f,d=%lf\n",a,b,c,f,d); // \n为换行符
   return 0;                 // 函数执行完毕返回函数值0
}                            // 函数结束标志
\end{lstlisting}
\end{frame}

\begin{frame}[fragile]{输入语句scanf(``\%变量格式符'', \&变量名);}
\begin{lstlisting}
#include<stdio.h>            // standard input/output编译预处理指令
int main()                   // 主函数
{                            // 函数开始标志
    int a=10,b;    // 定义变量a, b为整型数值, 定义变量时,可以指定变量的初值
    float f=10.2;  // 定义变量f为单精度浮点数
    double d; // 定义变量d为双精度浮点数
    char c='A';   // 定义变量c为单个英文字母, 字符输入以后讲
    printf("请输入整数a,b, 空格隔开:\n"); // 提示语句[可选]
    scanf("%d%d",&a,&b);
    printf("请输入浮点数f,d, 空格隔开:\n"); // 提示语句[可选]
    scanf("%f%lf",&f,&d);
    printf("a=%d,b=%d,c=%c,f=%f,d=%lf\n",a,b,c,f,d); // \n为换行符
    return 0;                 // 函数执行完毕返回函数值0
}                            // 函数结束标志
\end{lstlisting}
\end{frame}

\begin{frame}[fragile]{if(条件表达式)\{ 表达式为真(非0)时执行语句; \}}
\begin{lstlisting}
#include<stdio.h>            // standard input/output编译预处理指令
int main()                   // 主函数
{                            // 函数开始标志
   int a=10;    // 定义变量a为整型数值, 定义变量时,可以指定变量的初值
   if(a>=10)
   {
      printf("a>=10\n"); // \n为换行符
   }
   else
   {
      printf("a<10\n"); // \n为换行符
   }
   return 0;                 // 函数执行完毕返回函数值0
}                            // 函数结束标志
\end{lstlisting}
\end{frame}

\begin{frame}[fragile]{while(条件表达式)\{ 表达式为真(非0)时执行的语句;\}}
\begin{lstlisting}
#include<stdio.h>            // standard input/output编译预处理指令
int main()                   // 主函数
{                            // 函数开始标志
   int a=10;    // 定义变量a为整型数值, 定义变量时,可以指定变量的初值
   while(a>=0)
   {
     printf("a=%d\n",a); // \n为换行符
     a--; // a= a - 1
   }
   return 0;                 // 函数执行完毕返回函数值0
}                            // 函数结束标志
\end{lstlisting}
\end{frame}




%%%%%%%%%%%%%%%%%%%%%%%%%%% lecture-3
\begin{frame}
  \frametitle{lecture-3 主要内容}
  \framesubtitle{最简单的C语言程序设计---顺序程序设计}
  \tableofcontents[hideallsubsections]
\end{frame}

\section{初识C语言程序(讲解作业)}

\begin{frame}[fragile]{DevC++ 5.0以前的版本}
\begin{lstlisting}
#include<stdio.h>            // standard input/output编译预处理指令
#include <stdlib.h>          // standard library function, eg. for system()函数 
int main()                   // 主函数
{                            // 函数开始标志
   printf("%d\n",p);
   system(“pause”);          // 窗口暂停,DevC++ 5.0以前的版本。[机试系统提交时,一定注释或删除该语句] 
   return 0;                 // 函数执行完毕返回函数值0
}                            // 函数结束标志
\end{lstlisting}
\end{frame}

\begin{frame}[fragile]{求5!的C语言程序}
\begin{lstlisting}
#include<stdio.h>            // standard input/output编译预处理指令
int main()                   // 主函数
{                            // 函数开始标志
   int i,p;  // p表示被乘数, i表示乘数
   p=1;
   i=2;
   while(i<=5)
   {  
      p=p*i;
      i++; // i = i + 1
   }
   printf("%d\n",p);
   return 0;                 // 函数执行完毕返回函数值0
}                            // 函数结束标志
\end{lstlisting}
\end{frame}

\begin{frame}[fragile]{变量在使用之前首先要定义它的数据类型}
\begin{lstlisting}
#include<stdio.h>            // standard input/output编译预处理指令
int main()                   // 主函数
{                            // 函数开始标志
   int a,b;  // 定义变量a, b为整型数值,同类型变量可以在一条语句中定义。
   float f;  // 定义变量f为单精度浮点数
   double d; // 定义变量d为双精度浮点数
   char c;   // 定义变量c为单个英文字母
   a=10;
   b=20;
   f=10.2;
   d=20.3;
   c='A';
   return 0;                 // 函数执行完毕返回函数值0
}                            // 函数结束标志
\end{lstlisting}
\end{frame}

\begin{frame}[fragile]{if(条件表达式)\{ 表达式为真(非0)时执行语句; \}}
\begin{lstlisting}
#include<stdio.h>            // standard input/output编译预处理指令
int main()                   // 主函数
{                            // 函数开始标志
   int a=10;    // 定义变量a为整型数值, 定义变量时,可以指定变量的初值
   if(a>=10)
   {
      printf("a>=10\n"); // \n为换行符
   }
   else
   {
      printf("a<10\n"); // \n为换行符
   }
   return 0;                 // 函数执行完毕返回函数值0
}                            // 函数结束标志
\end{lstlisting}
\end{frame}

\begin{frame}[fragile]{while(条件表达式)\{ 表达式为真(非0)时执行的语句;\}}
\begin{lstlisting}
#include<stdio.h>            // standard input/output编译预处理指令
int main()                   // 主函数
{                            // 函数开始标志
   int a=10;    // 定义变量a为整型数值, 定义变量时,可以指定变量的初值
   while(a>=0)
   {
     printf("a=%d\n",a); // \n为换行符
     a--; // a= a - 1
   }
   return 0;                 // 函数执行完毕返回函数值0
}                            // 函数结束标志
\end{lstlisting}
\end{frame}

\section{数据的输入输出}

\begin{frame}{常用格式描述符与数据类型的对应关系}
\begin{tabular}{|c|c|c|}
	\hline 
	\textbf{格式符} & \textbf{对应的数据类型} &  \textbf{备注}\\ 
	\hline 
	\%d & int &  \\ 
	\hline  
	\%f & float &  \\
	\hline
	\%c & char & \\ 
	\hline   
	\%lf & double & \\ 
	\hline 
	\%.2f & float & 保留两位小数, 四舍五入。不适用于scanf()。 \\ 
	\hline 
	\%.2lf & double & 保留两位小数, 四舍五入。不适用于scanf()。 \\ 
	\hline
	\hline   
	\%x & int & 十六进制显示 \\ 
	\hline 
	\%ld & long int &  \\ 
	\hline 
\end{tabular}
\newline
\newline
\textcolor{blue}{详见p73, 表3.6}
\end{frame}

\begin{frame}[shrink,fragile]{输出语句printf(``原样输出, \%格式符'', 对应变量值);}
\begin{lstlisting}
#include<stdio.h>            // standard input/output编译预处理指令
int main()                   // 主函数
{                            // 函数开始标志
   int a=10,b;    // 定义变量a, b为整型数值, 定义变量时,可以指定变量的初值
   float f=10.2;  // 定义变量f为单精度浮点数
   double d; // 定义变量d为双精度浮点数
   char c;   // 定义变量c为单个英文字母
   f=10.2;
   d=20.356;
   c='A';
   printf("a=%d,b=%d,c=%c,f=%f,d=%.2lf\n",a,b,c,f,d); // %.2f, %.2lf保留两位小数
   return 0;                 // 函数执行完毕返回函数值0
}                            // 函数结束标志
\end{lstlisting}
\textcolor{blue}{变量b没有被赋值, 将是一个随机值。}
\end{frame}

\begin{frame}[fragile]{输入语句scanf(``\%变量格式符'', \&变量名);}
\begin{lstlisting}
#include<stdio.h>            // standard input/output编译预处理指令
int main()                   // 主函数
{                            // 函数开始标志
   int a=10,b;    // 定义变量a, b为整型数值, 定义变量时,可以指定变量的初值
   float f=10.2;  // 定义变量f为单精度浮点数
   double d; // 定义变量d为双精度浮点数
   char c='A';   // 定义变量c为单个英文字母, 字符输入以后讲
   printf("请输入整数和浮点数, 空格隔开:\n"); // 提示语句[可选]
   scanf("%d%f",&a,&f);  // 尽量简单, 不要有其它字符和'\n'
   printf("请输入两个浮点数, 空格隔开:\n"); // 提示语句[可选]
   scanf("%f%lf",&f,&d);
   printf("a=%d,b=%d,c=%c,f=%f,d=%lf\n",a,b,c,f,d); // \n为换行符
   return 0;                 // 函数执行完毕返回函数值0
}                            // 函数结束标志
\end{lstlisting}
\end{frame}

\begin{frame}[fragile]{字符输出函数putchar}
\begin{lstlisting}
#include<stdio.h>
int main()
{
   char a = 'B',b = 'O',c = 'Y'; //定义3个字符变量并初始化
   putchar(a); //向显示器输出字符B
   putchar(b); //向显示器输出字符O
   putchar(c); //向显示器输出字符Y
   putchar ('\n'); //向显示器输出一个换行符
   return 0;
}
\end{lstlisting}
\end{frame}

\begin{frame}[fragile]{字符输入函数getchar, 遇到回车, 开始从缓冲区中接收字符。}
\begin{lstlisting}
#include<stdio.h>
int main()
{
   char a,b,c;  //定义字符变量a,b,c
   a = getchar();  //从键盘输入一个字符,送给字符变量a
   b = getchar();  //从键盘输入一个字符,送给字符变量b
   c = getchar();  //从键盘输入一个字符,送给字符变量c
   putchar(a);  //将变量a的值输出
   putchar(b);  //将变量b的值输出 
   putchar(c);  //将变量c的值输出
   printf("\na=%d,b=%d,c=%d,a=%c,b=%c,c=%c\n",a,b,c,a,b,c);
   return 0;
}
\end{lstlisting}
\end{frame}

\begin{frame}[fragile]{字符输入函数getchar, 遇到回车, 开始从缓冲区中接收字符。}
\begin{lstlisting}
char a,b,c;  //定义字符变量a,b,c
a = getchar();  //从键盘输入一个字符,送给字符变量a
b = getchar();  //从键盘输入一个字符,送给字符变量b
c = getchar();  //从键盘输入一个字符,送给字符变量c
putchar(a);  //将变量a的值输出
putchar(b);  //将变量b的值输出 
putchar(c);  //将变量c的值输出
printf("\na=%d,b=%d,c=%d,a=%c,b=%c,c=%c\n",a,b,c,a,b,c);
\end{lstlisting}
从键盘输入abc回车, 观察结果, 应该是正确的结果。
遇到回车, 开始从缓冲区中接收字符。
\end{frame}

\begin{frame}[fragile]{字符输入函数getchar, 遇到回车, 开始从缓冲区中接收字符。}
\begin{lstlisting}
a = getchar();  //从键盘输入一个字符,送给字符变量a='a'
b = getchar();  //从键盘输入一个字符,送给字符变量b='\n'
c = getchar();  //从键盘输入一个字符,送给字符变量c='b'
putchar(a);  //将变量a的值输出a
putchar(b);  //将变量b的值输出\n 
putchar(c);  //将变量c的值输出b
printf("\na=%d,b=%d,c=%d,a=%c,b=%c,c=%c\n",a,b,c,a,b,c);
\end{lstlisting}
再运行一次程序, 输入a回车, 输入b回车, 输入c回车, 观察结果。\\
\includegraphics[scale=0.5]{abc}
\end{frame}

\begin{frame}{开发平台上演示讲解}
在开发平台,以具体的示例,详细讲解以下内容:
\begin{itemize}
	\item int, float, double, char 数据类型, sizeof()函数
	\item \%d, \%f, \%c, \%lf格式符的使用(见ppt中的表格)
	\item if()\{\quad\}, while()\{\quad\}简单语句
	\item char c; scanf(``\%c", \&c); 接收输入的字符
	\item char c; c=getchar()接收输入的字符, putchar()输出一个字符
	\item 避免数字,字符在一条语句中输入的情况,如:\\ scanf(``\%d\%c\%d",...);
	\item 重点理解字符缓冲区的概念,以及消费无用字符的技巧。
\end{itemize}
\end{frame}

\note
{
根据学生的反馈,数据的输入语句没有完全听明白。 \\
下一讲, 进一步详细讲解。\\
欢迎同学们在群里踊跃发言,任何不理解的知识点请指出来,我将在以后的讲课中,有针对性的讲清楚大家有疑惑的问题。一些与编程无关的俏皮话之类的东西就不要发了,争取把我们这个群建立成纯净的,对大家学习课程有帮助的程序设计讨论群。力争100名学生一个都不掉队,考个好成绩。加油!!
}





%%%%%%%%%%%%%%%%%%% last frame
\begin{frame}[plain]{}
  \begin{center}
    \begin{tikzpicture}
      \node[above,xscale=1.2,yscale=1.2]{\Huge 欢迎批评指正!};
    \end{tikzpicture}
  \end{center}
\end{frame}

\end{document}


%%%%下面的内容不参与文档的编译。使用者在想用某个东西时直接可通过查阅,并复制黏贴和修改使用。

\iffalse  %注释开始

\section{第一部分}

%垂直居中
\begin{frame}
  \begin{center}
  需要居中的内容!
  \end{center}
\end{frame}
或者
\begin{frame}
  \centering
  一些内容
\end{frame}

%幻灯片标题的使用
\begin{frame}
\frametitle{第一部分第一张幻灯}
  一些内容
\end{frame}

%项目编号的使用
\begin{frame}
  \frametitle{条目}
  \begin{itemize}
  \item 项目1
  \item 项目2
  \item 项目3
  \item 项目4
    \begin{itemize}
    \item 二级项目1
    \item 二级项目2
    \end{itemize}
  \end{itemize}
\end{frame}

%表格的使用
\begin{frame}
  \frametitle{表格}
  \begin{table}[htbp!]
  	\small %\tiny,\scriptsize,\footnotesize,\small,\normalsize,\large,\LARGE,\huge,\Huge
    \centering
    \caption{主流机器学习框架}
    \begin{tabular}{c|c|c|c|c}
      \toprule[1pt]
      机器学习库	& 机构 & 支持语言  & 平台 & Tensor \\
      \toprule[1pt]
      TensorFlow	& Google & C++,Python &跨平台 & Good \\
 	  \hline
      Pytorch	&  Facebook& Python & 跨平台 & Good \\
 	  \bottomrule[1pt]
    \end{tabular}
  \end{table}
\end{frame}

%区块的使用
\begin{frame}
  \frametitle{分析}
  \begin{block}{XXX 算法}
	\begin{itemize}
		\item 步骤1
	 	\item 步骤2
	 	\item 步骤3
	 \end{itemize}
  \end{block}
\end{frame}

%使用区块来强调内容
\begin{frame}
  \frametitle{强调}
  \begin{itemize}
  \item 这是内容
  \end{itemize}
  \only<1>\begin{block}{}
    这里蹦出来一个强调!
  \end{block}
\end{frame}

%section中目录的使用
\begin{frame}
  \frametitle{技术影响力}
    \tableofcontents[currentsection,hideallsubsections]
\end{frame}

%插入图片
\begin{frame}
\begin{figure}[!h]
  \centering
  % Requires \usepackage{graphicx}
  \includegraphics[width=2cm]{pics/logo.jpg}\\
  \caption{logo图片样例}\label{pic6}
\end{figure}
\end{frame}

%分栏实现图文混排
\begin{frame}
分栏前面的一些内容!!
\begin{columns}%0.6 0.4表示相对比例
\column{0.6\textwidth}%<1->
分栏的左侧,文字叙述。
\column{0.4\textwidth}%<1->
分栏的右侧插入了图片。
 \begin{figure}[!h]
  \centering
  % Requires \usepackage{graphicx}
  \includegraphics[width=4cm]{pics/logo.jpg}\\
  \caption{logo图片样例}\label{pic6}
\end{figure}
\end{columns}
分栏后面的一些内容!!
\end{frame}

%% last frame
\begin{frame}[plain]{}
\begin{center}
	\begin{tikzpicture}
	\node[above,xscale=1.2,yscale=1.2]{\Huge 欢迎批评指正!};
	\end{tikzpicture}
\end{center}
\end{frame}

% 大括号
$\left\{ .....  \right\}$
$\left[ .....  \right]$
$\left( ..... \right)$
$\left\{ .....  \right.$ %有左无右

$\delta_{jk}=
\begin{cases}
1, & (i=k)\\
0, & (i\ne k)
\end{cases}
$

\tikzstyle{int}=[draw, fill=blue!20, minimum size=2em]
\tikzstyle{init} = [pin edge={to-,thin,black}]
\begin{tikzpicture}[node distance=2.5cm,auto,>=latex']
\node [int, pin={[init]above:$v_0$}] (a) {$\frac{1}{s}$};
\node (b) [left of=a,node distance=2cm, coordinate] {a};
\node [int, pin={[init]above:$p_0$}] (c) [right of=a] {$\frac{1}{s}$};
\node [coordinate] (end) [right of=c, node distance=2cm]{};
\path[->] (b) edge node {$a$} (a);
\path[->] (a) edge node {$v$} (c);
\draw[->] (c) edge node {$p$} (end) ;
\end{tikzpicture}

%%%%%%%%%% 上下文字
$P(H_1|H_0)\mathop{=}\limits^{def}P_F$\\    %上下文字在\[... \]和$...$表现不一致
\[P(H_1|H_0)\mathop{=}\limits^{def}P_F \]
\[P(H_1|H_0)\mathop{=}^{def}P_F \]
\begin{align*}
P(H_1|H_0)\mathop{=}^{def}P_F
\end{align*}

\begin{enumerate}
	\setcounter{enumi}{3} %设定起始编号 
	\item aaa
	\item bbb
\end{enumerate}

%%%%%%%%%%%
\\
~\\ %一行空白

\newline
\newline
%%%%%%%%%%%%

\bigskip  或\medskip

%%%%%%%%%%%%%%%% 自动压缩,以显示全部内容
\begin{frame}[shrink]
\frametitle{标题} %需要页面标题时,这样设置,而不能\begin{frame}{标题}[shrink]
\end{frame}

%%%%%%%%%%%% 最后一列左对齐---&&, 单个&分割,则是右对齐
\begin{align*}
E[a(t)]&=\int_{-\infty}^{\infty}a(t)p(x)dx\\
&=a(t)\int_{-\infty}^{\infty}p(x)dx &&\text{by $a(t)$是常数}\\
&=a(t)\cdot 1 &&by \int_{-\infty}^{\infty}p(x)dx=1\\
&=a(t)
\end{align*}

%%%%%%%%%%%%%%%%有色文本
\textcolor{blue}{This text is in blue} 
\colorbox{yellow}{This text is highlighted in yellow} 
\colorbox{yellow}{ 
	\textcolor{red}{ 
		\textbf{ 
			Bold text in red, highlighted in yellow 
		} 
	} 
} 

block 普通环境
theorem 定理环境
lemma 引理环境
proof 证明环境
corollary 推论环境
example 示例环境
alertblock 警示环境

\begin{frame}{算法}
\begin{columns}[T] % align columns
	\begin{column}<0->{.5\textwidth}
		\begin{algorithm}[H]  
			\caption{algorithm caption} %算法的名字
			\hspace*{0.02in} {\bf Input:} %算法的输入, \hspace*{0.02in}用来控制位置,同时利用 \\ 进行换行
			input parameters A, B, C\\
			\hspace*{0.02in} {\bf Output:} %算法的结果输出
			output result 
			\begin{algorithmic}[1] %每行显示行号  
				\State some description % \State 后写一般语句
				\For{condition} % For 语句,需要和EndFor对应
				\State ...
				\If{condition} % If 语句,需要和EndIf对应
				\Else
				\EndIf
				\EndFor
				\While{condition} % While语句,需要和EndWhile对应
				\EndWhile
				\State \Return result	
			\end{algorithmic}  
		\end{algorithm}
	\end{column}%
	\hfill%	
	\begin{column}<0->{.40\textwidth}
		
	\end{column}%
\end{columns}
\end{frame}

\begin{frame}[fragile]{代码}
\fontspec{Consolas}
%不能用Tab键
\begin{lstlisting}
inline int gcd(int a, int b) { 
return b==0?a:gcd(b,a%b)
}
inline int lcm(int a, int b) {
return a/gcd(a,b)*b;
}
\end{lstlisting}
\end{frame}

\fi   %注释结束
