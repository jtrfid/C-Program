%%%%%%%%%%%%%%%%%%%%%%%%%% lecture-7
\begin{frame}[shrink]
  \frametitle{lecture-10 主要内容}
  \framesubtitle{利用数组处理批量数据}
  \tableofcontents[hideallsubsections]
\end{frame}

\begin{frame}[shrink,fragile]{为什么需要数组}
\begin{itemize}
	\item 要向计算机输入全班50个学生的成绩一门课的成绩; 
	\item 用50个float型简单变量表示学生的成绩
	\begin{itemize}
		\item[-] \textbf{烦琐},如果有1000名学生怎么办呢?
		\item[-] 没有反映出这些\textbf{数据间的内在联系},实际上这些数据是同一个班级、同一门课程的成绩,它们具有相同的属性。
	\end{itemize}
\end{itemize}
\begin{lstlisting}
float s0,s1,s2,...,s49; // 50名学生一门课的成绩
\end{lstlisting}
\pause
\begin{lstlisting}
float s[50]; // 50名学生一门课的成绩
int i; // 表示数组下标
for(i=0;i<50;i++) scanf("%f",&s[i]);
\end{lstlisting}
\begin{block}{数组}
	\begin{enumerate}
		\item 数组是一组有序数据的集合。数组中各数据的排列是有一定规律的,下标代表数据在数组中的序号。
		\item 用数组名和下标即可唯一地确定数组中的元素。
		\item 数组中的每一个元素都属于同一个数据类型。
	\end{enumerate}
\end{block}
\end{frame}

\section{定义数组: int a[10];}

\begin{frame}[shrink,fragile]{定义数组: int a[10];}
定义一维数组: \\
\textcolor{blue}{元素类型\quad 数组名[常量表达式(表示元素个数---数组的长度)]}
\begin{lstlisting}
#define NUM 100
float s[50]; // 50名学生一门课的成绩
int a[10]; // 10个元素的整型数组
double b[NUM]; // 常量NUM个元素的double数组
char c[50]; // 50个元素的char型数组
\end{lstlisting}
\begin{block}{Notes}
	数组元素的下标从0开始,int a[10]; 10个整型元素,则最大下标值为9,不存在数组元素a[10]
\end{block}
\begin{tabular}{|c|c|c|c|c|c|c|c|c|c|}
	\hline 
	a[0] & a[1] & a[2] & a[3] & a[4] & a[5] & a[6] & a[7] & a[8] & a[9] \\ 
	\hline 
\end{tabular} 
\end{frame}

\section{引用数组: int i=0; a[i]}

\begin{frame}[shrink,fragile]{引用数组: int i=0; a[i]}
\begin{lstlisting}
int a[10]; // 10个元素的整型数组
int i;
a[0]=10; // 给a数组的第一个元素赋值
a[9]=10; // 给a数组的最后一个元素赋值
printf("%d,%d",a[0],a[9]);
for(i=0;i<10;i++) a[i] = i+1; // 给数组的第i个元素赋值
for(i=0;i<10;i++) printf("%d\t",a[i]); // 输出数组a的10个元素
for(i=0;i<10;i++) scanf("%d",&a[i]); // 输入10个整数, 存入数组a中。注意'&'
\end{lstlisting}
\begin{block}{Notes}
	数组元素的下标从0开始,int a[10]; 10个整型元素,则最大下标值为9,不存在数组元素a[10]
\end{block}
\end{frame}

\begin{frame}[shrink,fragile]
$[$例6.1$]$对10个字符型数组元素依次赋值为'a','b',$\cdots$。要求按逆序输出。
\pause
\begin{lstlisting}
char c[10];
int i;
for(i=0;i<9;i++) c[i]='a'+i;
for(i=9;i>0;i--) printf("%c ", c[i]);
printf("\n");
\end{lstlisting}
\end{frame}

\section{初始化数组: int a[5]=\{1,2,3,4,5\};}

\begin{frame}[shrink,fragile]{初始化数组: int a[5]=\{1,2,3,4,5\};}
为了使程序简洁,常在定义数组的同时给各数组元素赋值,这称为数组的初始化。
\begin{lstlisting}
int a[10]={1,2,3,4,5,6,7,8,9,10}; // 在定义数组时对全部数组元素赋予初值。
char c[10]={'a','b'}; // 可以只给数组中的一部分元素赋值。其他元素的值不确定
double d[]={10.0,10.2,10.3}; // 等效于 double d[3]={10.0,10.2,10.3}
d[2] = 20.2; // 修改第3个元素
\end{lstlisting}
\end{frame}

\begin{frame}[shrink,fragile]
$[$例6.2$]$用数组来处理求Fibonacci数列问题。
\begin{lstlisting}
#include <stdio.h>
int main()
{
  int i;
  int f[20]={1,1};  //对最前面两个元素f[0]和f[1]赋初值1
  for(i=2;i<20;i++)
     f[i]=f[i-2]+f[i-1]; //先后求出f[2]~f[19]的值
  for(i=0;i<20;i++)
  {
     if(i%5==0) printf("\n"); //控制每输出5个数后换行
     printf("%12d",f[i]); //输出一个数
  }
  printf("\n");
  return 0;
}
\end{lstlisting}
\end{frame}

\section{冒泡排序}

\begin{frame}[shrink,fragile]{冒泡排序(第1趟)}
int a[4]=\{9,8,5,0\}; // n个元素, 要求从小到大顺序排列
\begin{columns}[T]
	\column{0.3\textwidth}
	\begin{tabular}{|c|}
		\hline 
		9 \\ 
		\hline 
		8 \\ 
		\hline 
		5 \\ 
		\hline
		0 \\
	    \hline 
	\end{tabular}\\ 
    第1趟原始数据
	\column{0.2\textwidth}
	\pause
	\begin{tabular}{|c|}
		\hline 
		8 \\ 
		\hline 
		9 \\ 
		\hline 
		5 \\ 
		\hline
		0 \\
		\hline  
	\end{tabular}\\ 
    第1趟第1次相邻两数比较
    \column{0.2\textwidth}
    \pause
    \begin{tabular}{|c|}
    	\hline 
    	8 \\ 
    	\hline 
    	5 \\ 
    	\hline 
    	9 \\ 
    	\hline 
    	0 \\
    	\hline 
    \end{tabular}\\ 
    第1趟第2次相邻两数比较
    \column{0.2\textwidth}
    \pause
    \begin{tabular}{|c|}
    	\hline 
    	8 \\ 
    	\hline 
    	5 \\ 
    	\hline 
    	0 \\ 
    	\hline 
    	\colorbox{yellow}{9} \\
    	\hline 
    \end{tabular}\\ 
    第1趟第3次相邻两数比较(第1个大数沉底)
\end{columns}
~\\
\textcolor{blue}{第1趟,相邻两数比较3-0次}
\end{frame}

\begin{frame}[shrink,fragile]{冒泡排序(第2趟)}
int a[4]=\{9,8,5,0\}; // 要求从小到大顺序排列
\begin{columns}[T]
	\column{0.3\textwidth}
	\begin{tabular}{|c|}
		\hline 
		8 \\ 
		\hline 
		5 \\ 
		\hline 
		0 \\ 
		\hline
		9 \\
		\hline 
	\end{tabular}\\ 
	第2趟原始数据
	\column{0.2\textwidth}
	\pause
	\begin{tabular}{|c|}
		\hline 
		5 \\ 
		\hline 
		8 \\ 
		\hline 
		0 \\ 
		\hline
		9 \\
		\hline  
	\end{tabular}\\ 
	第2趟第1次相邻两数比较
	\column{0.2\textwidth}
	\pause
	\begin{tabular}{|c|}
		\hline 
		5 \\ 
		\hline 
		0 \\ 
		\hline 
		\colorbox{yellow}{8} \\ 
		\hline 
		\colorbox{yellow}{9} \\
		\hline 
	\end{tabular}\\ 
	第2趟第2次相邻两数比较(第2大沉底)
\end{columns}
~\\
\textcolor{blue}{第2趟,相邻两数比较3-1次}
\end{frame}

\begin{frame}[shrink,fragile]{冒泡排序(第3趟)}
int a[4]=\{9,8,5,0\}; // 要求从小到大顺序排列
\begin{columns}[T]
	\column{0.3\textwidth}
	\begin{tabular}{|c|}
		\hline 
		5 \\ 
		\hline 
		0 \\ 
		\hline 
		8 \\ 
		\hline
		9 \\
		\hline 
	\end{tabular}\\ 
	第3趟原始数据
	\column{0.2\textwidth}
	\pause
	\begin{tabular}{|c|}
		\hline 
		0 \\ 
		\hline 
		\colorbox{yellow}{5} \\ 
		\hline 
		\colorbox{yellow}{8} \\ 
		\hline
		\colorbox{yellow}{9} \\
		\hline  
	\end{tabular}\\ 
	第3趟第1次相邻两数比较(第3大沉底)
\end{columns}
~\\
\textcolor{blue}{第3趟,相邻两数比较3-2次}
\end{frame}

\begin{frame}[shrink,fragile]{冒泡排序(总结)}
\begin{itemize}
	\setlength{\itemsep}{.5cm}
	\item n: 表示n个元素
	\item j=1,2,..n-1
	\item 表示第j趟排序,相邻元素两两比较,必要时交换。
	\item 第j趟排序进行n-j次相邻元素两两比较;
	\item 最多进行n-1趟排序。
	\item 注意排序趟数,两两比较的边界条件。用从2个数,3个数,4个数排序演练
\end{itemize}
\end{frame}

\begin{frame}[shrink,fragile]{冒泡排序(核心程序)}
\begin{lstlisting}
int a[10]={9,8,7,6,5,4,3,2,1,0},i,j,t;
for(j=0;j<9;j++) //进行9次循环,实现9趟比较
{
  for(i=0;i<9-j;i++) //在每一趟中进行9-j次比较
  {
    if(a[i]>a[i+1]) //相邻两个数比较 
       { t=a[i]; a[i]=a[i+1]; a[i+1]=t; } // 交换
  }
}
printf("the sorted numbers :\n");
for(i=0;i<10;i++)
   printf("%d ",a[i]);
\end{lstlisting}
\end{frame}

\begin{frame}[shrink,fragile]{冒泡排序(优化)}
\textbf{\textcolor{blue}{优化:第$j$趟排序中, 没有进行相邻元素的交换,表示数据已经排序好, 没有必要进行此后的$(n-1-j)$趟排序。}}
\begin{lstlisting}
int a[10]={7,8,7,6,5,6,7,8,9,10},i,j,t,flag;
for(j=0;j<9;j++) //进行9次循环,实现9趟比较
{
   flag = 0; // 每趟排序,初始化flag,表示未进行交换
   for(i=0;i<9-j;i++) //在每一趟中进行9-j次比较
   {
     if(a[i]>a[i+1]) //相邻两个数比较 
       { t=a[i]; a[i]=a[i+1]; a[i+1]=t; flag=1; } // 交换, 设置标志变量
   }
   if(!flag) break; // 表示第j趟未交换,排序好了!
}
printf("the sorted numbers :\n");
for(i=0;i<10;i++) printf("%d ",a[i]);
\end{lstlisting}
\end{frame}

\begin{frame}[shrink,fragile]{冒泡排序(输出每趟排序的结果)}
\textbf{\textcolor{blue}{优化:第$j$趟排序中, 没有进行相邻元素的交换,表示数据已经排序好, 没有必要进行此后的$(n-1-j)$趟排序。}}
\begin{lstlisting}
int a[10]={7,8,7,6,5,6,7,8,9,10},i,j,t,flag;
for(j=0;j<9;j++) //进行9次循环,实现9趟比较
{
   flag = 0; // 每趟排序,初始化flag,表示未进行交换
   for(i=0;i<9-j;i++) //在每一趟中进行9-j次比较
   {
      if(a[i]>a[i+1]) //相邻两个数比较 
       { t=a[i]; a[i]=a[i+1]; a[i+1]=t; flag=1; } // 交换, 设置标志变量
   }
   printf("第%d趟排序: \n", j)
   for(t=0;t<10;t++) printf("%d ",a[t]); // 临时变量t的复用
   if(!flag) break; // 表示第j趟未交换,排序好了!
}
printf("the sorted numbers :\n");
for(i=0;i<10;i++) printf("%d ",a[i]);
\end{lstlisting}
\end{frame}