%%%%%%%%%%%%%%%%%%%%%%%%%% lecture-9
\begin{frame}[shrink]
  \frametitle{lecture-9 主要内容}
  \framesubtitle{循环结构程序设计举例(续)}
  \tableofcontents[hideallsubsections]
\end{frame}

\section{循环结构程序设计举例(续)}

\begin{frame}[fragile]
附加题1: 求$s=a+aa+aaa+\cdots+a\cdots a$, 其中$a$是一个$1\sim 9$的数字。例如$a=2, n=4$时, $s=2+22+222+2222$, $a$和$n$由键盘输入。
\pause
\begin{columns}
\column{0.5\textwidth}
\begin{lstlisting}
int i,s,n,term = 0;
for(i=1,s=0; i<=n; i++) // 初始化循环变量用逗号隔开
{
   term = term*10 + a;
   s += term; 
}
\end{lstlisting}
\end{columns}
\end{frame}

\begin{frame}[fragile]
\small
附加题2: 韩信点兵。韩信有一队兵, 他想知道有多少人, 便让士兵排队报数:\\
按从1至5报数, 最末一个士兵报的数为1;\\
按从1至6报数, 最末一个士兵报的数为5;\\
按从1至7报数, 最末一个士兵报的数为4;\\
按从1至11报数,最末一个士兵报的数为10;\\ 
计算韩信至少有多少兵。 
\pause
\begin{columns}
\column{0.8\textwidth}
\begin{lstlisting}
int x=1;
for(;;x++)  // 循环体仅含if()结构,看作一条语句,'{}'可省略
    if(x%5==1 && x%6==5 && x%7==4 && x%11==10)
    { 
       printf("%d\n",x);  
       break;
    }
\end{lstlisting}
\end{columns}
\end{frame}

\begin{frame}[fragile]
附加题3-1: 求水仙花数。如果一个三位数的个位数、十位数和百位数的立方和等于该数自身,则称该数为水仙花数。 \\
编程求出所有的水仙花数。\\
~\\  
\pause
\begin{columns}
\column{0.6\textwidth}
解法一: 采用三重循环
\begin{lstlisting}
int i,j,k; // 百、十、个位
for(i=1;i<=9;i++)    // 百位
  for(j=0;j<=9;j++)  // 十位
    for(k=0;k<=9;k++)  // 个位
      if(i*100+j*10+k == i*i*i+j*j*j+k*k*k)  
        printf("%d\n",i*100+j*10+k);
\end{lstlisting}
\end{columns}
\end{frame}

\begin{frame}[fragile]
附加题3-2: 求水仙花数。如果一个三位数的个位数、十位数和百位数的立方和等于该数自身,则称该数为水仙花数。 \\
编程求出所有的水仙花数。\\
~\\ 
\pause
\begin{columns}
\column{0.6\textwidth}
解法二: 采用一重循环
\begin{lstlisting}
int m,i,j,k; 
for(m=100;m<=999;m++)
{
   i=m/100; j=m/10%10; k=m%10;  
   if(i*100+j*10+k == i*i*i+j*j*j+k*k*k)  
      printf("%d\n",i*100+j*10+k);
}
\end{lstlisting}
\end{columns}
~\\
\pause
\textcolor{blue}{思考: 输出共有多少个水仙数?}
\end{frame}

\begin{frame}[fragile]
附加题3-3: 求整数区间$[a,b]$中水仙花数的个数。
\pause
\begin{columns}
\column{0.8\textwidth}
\begin{lstlisting}
int n=0; //计数 
int a,b; // a,b 区间
int i,t;   // 循环变量,代表a,b区间的每个数
int sum; // i的各位立方和 
scanf("%d%d",&a,&b);
for(i=a;i<=b;i++) // 考察i是否水仙数
{  
  sum = 0; t=i; // 临时变量记住i; 易遗漏每次内层循环前sum要归0
  while(t!=0) // 累加各位立方 
  { sum+=pow(t%10,3); t=t/10; }
  if(sum==i) n++; // i是水仙数 
}
printf("%d\n",n);
\end{lstlisting}
\end{columns}
\end{frame}

\begin{frame}[fragile]
附加题4: 百钱百鸡, 已知公鸡5个钱1只, 母鸡3个钱1只, 小鸡1个钱3只, 用100个钱买了100只鸡。问公鸡、母鸡、小鸡各几只? 
\vspace{0.5cm}
\pause
\begin{columns}
%\column{0.9\textwidth}
\begin{lstlisting}
int x,y,z; // 公鸡、母鸡、小鸡个数
for(x=0;x<=100;x++) 
   for(y=0;y<=100;y++) 
     for(z=0;z<=100;z++) 
       if(5*x+3*y+z/3 == 100 && x+y+z == 100 && z%3 == 0) // 全部条件! 
          printf("%d,%d,%d\n",x,y,z);
\end{lstlisting}
\end{columns}
\vspace{0.5cm}
\pause
\textcolor{blue}{如何考虑无解的情况?}
\end{frame}

\begin{frame}[shrink,fragile]
附加题5-1: 求整数$a,b$的最大公约数,当两个数中有一个为0时,公约数是不为0的那个整数; 当两个整数互质时最大公约数为1。
输入两个整数a和b,求最大公约数。 
\pause
\begin{lstlisting}
#include <stdio.h>
int main()
{
  int a,b,t,i;
  scanf("%d%d",&a,&b); // 机试系统不要想当然给提示语句, 除非题目要求 
  if(a<b) { t=a; a=b; b=t; } // 交换a,b,使a是较大者 
  for(i=b;i>0;i--)
    if(a%i==0 && b%i==0){ t=i; break; } // 求得最大公约数 
  if(i==0)  // 如果循环结束,还未求得公约数,
  {
    if(b==0) t = a;
    else t=1; // a,b互质 
  } 
  printf("%d\n",t);
  return 0;
}
\end{lstlisting}
\end{frame}

\begin{frame}[shrink,fragile]
附加题5-2: 求整数$a,b$的最大公约数, 伪代码分析
\begin{lstlisting}
a,b的最大公约数,a=mb+r, m=a/b; r=a%b
while(1)
{
  r = a%b; 
  if(r==0) break; // b就是最大公约数
  else if(b%r==0) break; // r就是最大公约数,因为: n=b/r; r=nb; a=mb+nb=(m+n)b;
  a=b; b=r; // 准备下一轮迭代   
}
\end{lstlisting}
\rule{\textwidth}{1pt} %水平线
\pause
\begin{lstlisting}
int a,b,r,t;
scanf("%d%d",&a,&b); // 机试系统不要想当然给提示语句, 除非题目要求
if(a<b) { t=a; a=b; b=t; } // 交换a,b,使a是较大者
while(1)
{
   if(b==0) { t=a; break; }
   r = a%b; 
   if(r==0) {t=b; break;} // b就是最大公约数
   else if(r==1) {t=1; break;} // a,b互质 
   else if(b%r==0) {t=r; break;} // r就是最大公约数,因为: n=b/r; r=nb; a=mb+nb=(m+n)b;
   a=b; b=r; // 准备下一轮迭代   
}
\end{lstlisting}
\end{frame}

\begin{frame}[shrink,fragile]
附加题6: 给出一个百分制的成绩,要求输出成绩等级'A','B','C','D','E'。90分以上为'A',$80\sim 89$分为'B',$70\sim 79$分为'C',$60\sim 69$分为'D',60分以下为'E'。
\begin{lstlisting}
int grade;
scanf("%d",&grade);
grade /= 10; // 等效于 grade=grade/10;
switch(grade)
{
  case 0: case 1: case 2: case 3: case 4: 
  case 5:  printf("E"); break;
  case 6:  printf("D"); break;
  case 7:  printf("C"); break;
  case 8:  printf("B"); break;
  case 9:
  case 10:  printf("A"); break;
}
\end{lstlisting}
\pause
\textcolor{blue}{思考: 如果输入成绩等级, 输出分数段, 如何修改程序? }
\end{frame}

\begin{frame}{注意事项小结}
\begin{enumerate}
\setlength{\itemsep}{.5cm}
\item while( )\{ \}; do \{ \} while( ); for(;;)\{ \}执行顺序;
\item 循环变量的开始和结束条件;
\item 循环体是复合语句时,必须用\{ \}扩起来;
\item 必要时,用break结束整个循环,用continue结束本次循环;
\item 关键是找出循环规律,必要时设计流程图,指导代码实现。	
\end{enumerate}
\end{frame}