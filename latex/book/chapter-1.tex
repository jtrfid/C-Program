%%%%%%%%%%%%%%%%%%%%% chapter.tex %%%%%%%%%%%%%%%%%%%%%%%%%%%%%%%%%
%
% sample chapter
%
% Use this file as a template for your own input.
%
%%%%%%%%%%%%%%%%%%%%%%%% Springer-Verlag %%%%%%%%%%%%%%%%%%%%%%%%%%
%\motto{Use the template \emph{chapter.tex} to style the various elements of your chapter content.}
\chapter{第1次机试(课外)练习}

\section{计算球体重量}
已知铁的比重是7.86(克/立方厘米),金的比重是19.3(克/立方厘米)。写一个程序,分别计算出给定直径的铁球与金球的质量,假定PI=3.1415926

输入说明:\\
输入两个整数,分别表示铁球与金球的直径(单位为毫米)

输出说明:\\
输出两个浮点数,分别表示铁球与金球的质量(单位为克),小数点后保留3位小数,两个浮点数之间用空格分隔

输入样例:\\
100 100

输出样例:\\
4115.486  10105.456

提示: \\
用\lstinline|scanf|输入,用\lstinline|printf|输出,保留3位小数的格式控制字符为\lstinline|%.3f|

\begin{lstlisting}
#include<stdio.h>
#include<math.h>   // 数学库函数 
#define PI 3.1415926       
int main()                   
{  
	int a,b;
	scanf("%d%d",&a,&b);
	float v1= 4.0/3.0*pow(a/2.0/10,3)*PI;
	float v2= 4.0/3.0*pow(b/2.0/10,3)*PI;
	printf("%.3f %.3f\n",7.86*v1,19.3*v2); 
	return 0;           
}                   
\end{lstlisting}

\begin{note}[要点]
	\begin{enumerate}
		\item 整数除以整数, 结果为整数。\\
		4.0/3.0结果是浮点数,4/3结果是整数
		\item pow函数原型: \lstinline|double pow(double x,double y)|\\
		当形参数整数整数时, 由于精度问题,推荐不使用此函数计算$x^y$. 使用循环语句, 易计算$x^y$。 如果必要, 可自定义函数: \lstinline|int mypow(int x,int y)|。 见课件。
	\end{enumerate}
\end{note}

\section{温度转化}
已知华氏温度到摄氏温度的转换公式为:摄氏温度= (华氏温度- 32)×5/9,写程序将给定的华氏温度转换为摄氏温度输出。

输入说明:\\
只有一个整数,表示输入的华氏温度

输出说明:\\
输出一个表示摄氏温度的实数,小数点后保留2位有效数字,多余部分四舍五入

输入样例:\\
50

输出样例:\\
10.00

提示:\\
用scanf输入,用printf输出,保留2位小数的格式控制字符为%.2f

\begin{lstlisting}
#include <stdio.h>

int main()
{
	int f;
	float c;
	scanf("%d",&f);
	c = (f-32)*5.0/9;     // (1)
	//c = (f-32)*5/9;     // (2)
	printf("%.2f\n",c);
	return 0;
} 
\end{lstlisting}

\begin{note}[思考]
	为何语句(1),(2)计算结果不一致, 哪一条语句正确?
\end{note}


\section{整数简单运算}
编写程序,计算用户输入的两个整数的和、差、乘积(*)和商(/)。

输入格式:输入两个整数,整数之间用空格分隔。

输出格式:输出四个整数结果,分别表示和、差、积和商,每输出一个结果换行。

输入样例:\\
3 4

输出样例:\\
7  \\
-1 \\
12 \\
0

\begin{lstlisting}
#include<stdio.h>
int main()                   
{  
	int a,b;
	scanf("%d%d",&a,&b);
	printf("%d\n%d\n%d\n%d\n",a+b,a-b,a*b,a/b); 
	return 0;           
}             
\end{lstlisting}

\begin{note}[思考]
	b=0时如何处理?
\end{note}

\section{A+B+C}
通过键盘输入三个整数a,b,c,求3个整数之和。

输入说明:

三整形数据通过键盘输入,输入的数据介于-100000和100000之间,整数之间以空格、跳格或换行分隔。 

输出说明:

输出3个数的和。

输入样例:

-6  0  39

输出样例:

33

\begin{lstlisting}
#include<stdio.h>  
int main()                   
{  
	int a,b,c;
	scanf("%d%d%d",&a,&b,&c);
	printf("%d\n",a+b+c); 
	return 0;           
}           
\end{lstlisting}

\section{字符输入输出}
通过键盘输入5个大写字母,输出其对应的小写字母,并在末尾加上“!”。

输入说明:

5个大写字母通过键盘输入,字母之间以竖线“|”分隔。

输出说明:

输出5个大写字母对应的小写字母,之间无分隔,并在末尾加上‘!’。

输入样例:

H|E|L|L|O

输出样例:

hello!

\begin{lstlisting}
#include<stdio.h>     
int main()                   
{  
	char c1,c2,c3,c4,c5;
	scanf("%c|%c|%c|%c|%c",&c1,&c2,&c3,&c4,&c5);
	c1+=32; c2+=32; c3+=32; c4+=32; c5+=32;
	printf("%c%c%c%c%c!",c1,c2,c3,c4,c5); 
	return 0;           
}           
\end{lstlisting}

\begin{note}[要点]
	\lstinline|scanf("原样输入",...);|
\end{note}

\begin{note}(大小写字符转化关系)
	小写字符ASCII码=大写字符ASCII码+32
\end{note}

\section{数字字符}
通过键盘输入1个整数\lstinline|a(0<=a<=4)|,1个数字字符\lstinline|b('0'<=b<='5')|求a+b。

输入说明:

整形数据、数字字符通过键盘输入,输入的整形数据介于0和4之间,输入的数字字符介于`0'和`5'之间,二个输入数之间用``,"分隔。

输出说明:

分别以整数形式及字符形式输出a+b,输出的二个数之间用``,"分隔。

输入样例:

3,5

输出样例:

56,8

\begin{lstlisting}
#include<stdio.h>    
int main()                   
{  
	int a;
	char b;
	scanf("%d,%c",&a,&b);
	printf("%d,%c",a+b,a+b); 
	return 0;           
}             
\end{lstlisting}

\begin{note}(scanf函数)
	\lstinline|scanf("原样输入",...);|
\end{note}

\section{实数运算}
通过键盘输入长方体的长、宽、高,求长方体的体积V(单精度)。

输入说明:

十进制形式输入长、宽、高,输入数据间用空格分隔。

输出说明:

单精度形式输出长方体体积V,保留小数点后3位,左对齐。

输入样例:

15  8.12  6.66

输出样例:

811.188


\begin{lstlisting}
#include<stdio.h>   
int main()                   
{  
	float a,b,c;
	scanf("%f%f%f",&a,&b,&c);
	printf("%.3f",a*b*c); 
	return 0;           
}                  
\end{lstlisting}

\section{四则运算}
输入两个整数和一个四则运算符,根据运算符计算并输出其运算结果(和、差、积、商、余之一)。注意做整除及求余运算时,除数不能为零。

输入说明:

使用scanf()函数输入两个整数和一个运算符,格式见输入样例。

输出说明:

输出使用printf()函数,格式见输出样例。

输入样例:

\lstinline|5%2|

输出样例:

\lstinline|5%2=1|

\begin{lstlisting}
#include<stdio.h>    
int main()                   
{  
	int a,b;
	char op;
	scanf("%d%c%d",&a,&op,&b);
	switch(op)
	{
		case '+': printf("%d%c%d=%d\n",a,op,b,a+b); break;
		case '-': printf("%d%c%d=%d\n",a,op,b,a-b); break;
		case '*': printf("%d%c%d=%d\n",a,op,b,a*b); break;
		case '/': printf("%d%c%d=%d\n",a,op,b,a/b); break;
		case '%': printf("%d%c%d=%d\n",a,op,b,a%b); break;
	}
	return 0;           
}                 
\end{lstlisting}

\section{数位输出}
输入一个5位整数,求出其各数位数值,并按照从高位到低位的顺序输出,如:输入12345,输出为1 2 3 4 5。

输入说明:

输入一个五位正整数。

输出说明:

按数位从高到低依次输出,各数位之间以一个空格相分隔。

输入样例:

96237

输出样例:

9 6 2 3 7

\begin{lstlisting}
#include<stdio.h>      
int main()                   
{  
	int a,b=10000,i=5;
	scanf("%d",&a);
	while(i>=1)
	{
		if (i==1)printf("%d\n",a/b);
		else printf("%d ",a/b);
		a = a-a/b*b;
		b/=10;
		i--;
	} 
	return 0;           
}               
\end{lstlisting}

\section{阶梯电价计费}
电价分三个档次,[0,110]度电,每度电0.5元;(110,210]度电,超出110部分每度电0.55元,超过210度电,超出210部分每度电0.70元,给出一个家庭一月用电量,请计算出应缴的电费(四舍五入,保留小数点后两位小数)。

输入说明:

输入数据为一个正实数,表示一月用电量

输出说明:

输出应缴电费,四舍五入保留2位小数。

输入样例:

输入样例1

100

输入样例2

200

输入样例3

329.75

输出样例:

输出样例1

50.00

输出样例2

104.50

输出样例3

193.83

\begin{lstlisting}
#include <stdio.h>
int main()
{
	float sum,u1=0.5,u2=0.55,u3=0.70; // 用电量,每度电单价
	float fee = 0; // 应缴电费
	
	scanf("%f",&sum);
	
	if (sum > 210) 
	{
		fee = (sum-210)*u3;
		sum = 210;
	}
	if (sum > 110)
	{
		fee += (sum-110)*u2;
		sum = 110;
	}
	fee += sum*u1;
	
	printf("%.2f\n",fee);
	return 0;
} 

int main2()  // 另解
{
	float sum,u1=0.5,u2=0.55,u3=0.70; // 用电量,每度电单价
	float fee = 0; // 应缴电费 
	scanf("%f",&sum);
	
	if (sum >= 210)
		fee = 110*u1 + (210-110)*u2 + (sum-210)*u3;
	else if (sum >= 110)
		fee = 110*u1 + (sum-110)*u2;
	else 
		fee = sum*u1; 

	printf("%.2f\n",fee);
	return 0;
} 

int main3() // 另解
{
	float sum,u1=0.5,u2=0.55,u3=0.70; // 用电量,每度电单价
	float fee = 0; // 应缴电费 
	
	scanf("%f",&sum);
	
	if (sum <= 110) fee = sum*u1;
	else if (sum <= 210)
	{
		fee = 110*u1;
		sum -= 110; 
		fee += sum*u2;
	}
	else // sum > 210
	{
		fee = 110*u1; 
		fee += (210-110)*u2;
		sum -= 210;
		fee += sum*u3;
	}
	
	printf("%.2f\n",fee);
	return 0;
} 
\end{lstlisting}

\begin{note}[四舍五入问题]
	不同的编译系统,处理结果可能不一致,\lstinline|printf("%.2f\n",fee);|默认输出即可。
\end{note}

\section{计算某月天数}
每年的1,3,5,7,8,10,12月有31天,4,6,9,11月有30天,闰年2月29天,其他年份2月28天,给定年份和月份求该月的天数

输入说明:

输入由两个正整数a和b构成,a表示年份,b表示月份,a和b之间用空格分隔

输出说明:

根据年份和月份计算该月天数并输出

输入样例	

输入样例1

2000 3

输入样例2

2001 2

输出样例
	
输出样例1

31

输出样例2

28

\begin{lstlisting}
#include <stdio.h>
int main()
{
	int a,b,t = 0;
	scanf("%d%d",&a,&b);
	if((a%4 == 0 && a%100 !=0) || (a%100 == 0 && a%400 == 0))
	{
		if(b == 2) t = 29;
	}
	else if (b == 2) t = 28;
	
	if(b == 1 || b == 3 || b == 5 || b == 7 || b == 8 || b == 10 || b == 12) t = 31;
	else if(b == 4 || b == 6 || b == 9 || b == 11) t = 30; 
	
	printf("%d\n",t);
	return 0;	
}
\end{lstlisting}

\begin{note}(逻辑运算符)
	\&\&, ||, !
\end{note}

\section{计算整数各位数字之和}
假设n是一个由最多9位数字(d9,\dots, d1)组成的正整数。编写一个程序计算n的每一位数字之和。

输入说明:
	
输入数据为一个正整数n

输出说明:
	
对整数n输出它的各位数字之和后换行

输入样例:	

3704

输出样例:	

14

\begin{lstlisting}
#include <stdio.h>
int main()
{
	int n,sum = 0; // 注意初始化sum
	scanf("%d",&n);
	while(n)
	{
		sum += n%10;
		n /= 10;
	}
	printf("%d",sum);
	return 0;	
}
\end{lstlisting}

\begin{note}(计算整数各位数字)
	循环语句: 除10, 取余, 是基本技巧。
\end{note}

\section{完数}
请写一个程序,给出指定整数范围[a,b]内的所有完数,0 < a < b < 10000。
一个数如果恰好等于除它本身外的所有因子之和,这个数就称为``完数"。
例如6是完数,因为6=1+2+3

输入说明	

输入为两个整数a和b,a和b之间用空格分隔

输出说明	

输出[a,b]内的所有完数,每个数字占一行

输入样例	

1 10

输出样例	

6

\begin{lstlisting}
#include <stdio.h>
int main()
{
	int i,j,n1,n2,sum = 0;
	scanf("%d%d",&n1,&n2);
	for(i = n1; i <= n2; i++)
	{
		if(i == 1) continue; // 避免输出1,1不是完数 
		sum = 1; // 注意必须在此处初始化sum
		for(j = 2; j < i; j++)
		{
			if(i%j == 0) sum += j; 
		}
		if(sum == i) printf("%d\n",i);
	} 

	return 0;	
}
\end{lstlisting}

\begin{note}[特别注意]
	且记:进入内层循环前, 相关变量的初始化问题。	
\end{note}

\section{最大公约数}	
最大公约数(GCD)指某几个整数共有因子中最大的一个,最大公约数具有如下性质,

gcd(a,0)=a

gcd(a,1)=1

因此当两个数中有一个为0时,gcd是不为0的那个整数,当两个整数互质时最大公约数为1。

输入两个整数a和b,求最大公约数

输入说明:

输入为两个正整数a和b(0<=a,b<10000),a和b之间用空格分隔,

输出说明:

输出其最大公约数

输入样例:

样例1输入
	
2 4

样例2输入:

12 6

样例3输入:

3 5

输出样例:

样例1输出
	
2

样例2输出

6

样例3输出

1

\begin{lstlisting}
#include <stdio.h>
// 递归函数
int gcd(int a,int b)
{
	if(b==0) return a;
	if(a%b==0) return b;
	return gcd(b,a%b);
}

int main() // 调用递归函数
{
	int a,b,t;
	scanf("%d%d",&a,&b);
	if(a<b) { t=a; a=b; b=t; } // 交换a,b 
	printf("%d\n",gcd(a,b));
	return 0;	
}

int main1() // 另解(循环语句)
{
	int a,b,r,t;
	scanf("%d%d",&a,&b); // 机试系统不要想当然给提示语句, 除非题目要求
	if(a<b) { t=a; a=b; b=t; } // 交换a,b,使a是较大者
	while(1)
	{
		if(b==0) { t=a; break; }  // 分母为0时, a就是最大公约数 
		r = a%b; 
		if(r==0) {t=b; break;}    // 本轮循环的a(上轮循环的b)就是最大公约数
		
		a=b; b=r; // 准备下一轮迭代   
	}
	printf("%d\n",t);
	return 0;	
}
\end{lstlisting}

\begin{note}(欧几里得定理)
	见课件。
\end{note}

\section{角谷定理}
角谷定理定义如下:
对于一个大于1的整数n,如果n是偶数,则n = n / 2。如果n是奇数,则n = 3 * n +1,反复操作后,n一定为1。

例如输入22的变化过程: 22 ->11 -> 34 -> 17 -> 52 -> 26 -> 13 -> 40 -> 20 -> 10 -> 5 -> 16 -> 8 -> 4 -> 2 -> 1,数据变化次数为15。

输入一个大于1的整数,求经过多少次变化可得到自然数1。

输入说明	

输入为一个整数n,1<n<100000。

输出说明	

输出变为1需要的次数

输入样例
	
样例1输入

22

样例2输入

33

输出样例

样例1输出
	
15

样例2输出

26

\begin{lstlisting}
#include <stdio.h>
int main()
{
	int n,i=0;
	scanf("%d",&n);
	while(n!=1)
	{
		if(n%2==0) n=n/2;
		else n=3*n+1;
		i++;
	} 
	printf("%d\n",i);
	return 0;	
}
\end{lstlisting}
