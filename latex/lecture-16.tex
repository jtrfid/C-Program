%%%%%%%%%%%%%%%%%%%%%%%%%% lecture-14
\begin{frame}[shrink]
  \frametitle{lecture-15 主要内容}
  \framesubtitle{指针应用题, 期中考试总结}
  %\tableofcontents[hideallsubsections]
  \tableofcontents
\end{frame}

\section{指针应用题}

\begin{frame}[shrink,fragile]{例: 表达式求值}
表达式由两个非负整数x,y和一个运算符op构成,求表达式的值。\\
这两个整数和运算符的顺序是随机的,可能是``x op y", ``op x y"或者``x y op",例如,``25 + 3"表示25加3,``5 30 *" 表示5乘以30, ``/ 600 15"表示600除以15。\\
输入说明\\	
输入为一个表达式, 表达式由两个非负整数x, y和一个运算符op构成, x,y和op之间以空格分隔, 但顺序不确定。
x和y均不大于10000000, op可以是+, -, *, /, \%中的任意一种,分表表示加法,减法,乘法,除法和求余。
除法按整数除法求值,输入数据保证除法和求余运算的y值不为0。\\
输出说明\\	
输出表达式的值。\\
\medskip
\begin{columns}
	\column{0.3\textwidth}
	输入样例\\	
	样例1输入\\
	5 20 *\\
	样例2输入\\
	4 + 8\\
	样例3输入\\
	/ 8 4
	\column{0.3\textwidth}
	输出样例\\	
	样例1输出\\
	100\\
	样例2输出\\
	12\\
	样例3输出\\
	2
\end{columns}
\medskip
\end{frame}

\begin{frame}[shrink,fragile]{例: 表达式求值---解题思路}
\begin{itemize}
	\item \lstinline|gets|函数读取字符串,遍历字符串,根据op字符是"非数字字符"的特点,判断表达式的三种形式。\\
	注意: 因为字符串中含有空格, 不能使用\lstinline|scanf("%s",-);|读取字符串。
	\item 编写计算函数, 根据op,x,y计算表达式的值。
	\item 编写独立函数, 提取[*p1,*p2]之间的数字字符串,并返回对应的整数值(x或y)。
	\item 主程序,解析字符串, 调用上述函数。
\end{itemize}
\end{frame}

\begin{frame}[shrink,fragile]{例: 表达式求值---计算函数}
\begin{lstlisting}
// 根据参数,计算表达式的值 
int compute(char op,int x,int y)
{
   int result = -1;
   switch(op)
   {
     case '+': result = x+y; break;
     case '-': result = x-y; break;
     case '*': result = x*y; break;
     case '/': if(y != 0) result = x/y; break;
     case '%': if(y != 0) result = x%y; break;
   }
   return result;
}
\end{lstlisting}
\end{frame}

\begin{frame}[shrink,fragile]{例: 表达式求值---提取字符串中的数字}
\begin{lstlisting}
// 提取[*p1,*p2]之间的数字字符串,并返回对应的整数值(xy)。  
int xy(char *p1,char *p2)
{
  int result = 0;
  // 删除前后缀空格 
  while(*p1 == ' ') p1++; // 使p1,p2指向非空格
  while(*p2 == ' ') p2--; 
  for(; p1 <= p2; p1++)
  {
     result += *p1 -'0'; // 1 = '1' - '0'
     if(p1 != p2) result *= 10; // 非最后一个字符,左移一位十进制数 
   }
   return result;
}
\end{lstlisting}
\end{frame}

\begin{frame}[shrink,fragile]{例: 表达式求值---主程序}
\begin{lstlisting}
#include <stdio.h>
// 估计字符串最大长度,存储有效字符(N-1)个,预留最后一个字符'\0' 
#define N 20 
// 根据参数,计算表达式的值 
int compute(char op,int x,int y); // 函数声明
// 提取[*p1,*p2]之间的数字字符串,并返回对应的整数值(xy)。  
int xy(char *p1,char *p2);
int main()
{
   char s[N],*p1,*p2,op; // 借助p1,p2指针变量,扫描字符串 
   int x,y;
   gets(s); // 不能使用scanf("%s",s); 空格将会终止
   p1 = s; // p1指向字符串首地址 
   while(*p1 == ' ') p1++; // 使p1指向非空格
   if (*p1 < '0' || *p1 > '9') // 首先判断第一个字符是运算符, op x y
   { 续1 }
   else // x y op 或 x op y 
   { 续2,续3}
   printf("%d\n",compute(op,x,y)); 
   return 0;
}
\end{lstlisting}
\end{frame}

\begin{frame}[shrink,fragile]{例: 表达式求值---主程序(续1)}
\begin{lstlisting}
// 首先判断第一个字符是运算符, op x y
if (*p1 < '0' || *p1 > '9')
{
  op = *p1;
  p1++; // 掠过op 
  while(*p1 == ' ') p1++; // 使p1指向非空格
  p2 = p1; 
  while(*p2 != ' ') p2++; // x空格
  x = xy(p1,p2);
  p1 = p2;
  while(*p2 != '\0') p2++; // p2指向'\0' 
  y = xy(p1,p2 - 1); // p2 - 1是最后一个有效字符 
} 
\end{lstlisting}
\end{frame}

\begin{frame}[shrink,fragile]{例: 表达式求值---主程序(续2)}
\begin{lstlisting}
else // x y op 或 x op y 
{
   p2 = s; // p2指向字符串首地址 
   while(*p2 != '\0') p2++; // p2指向'\0'
   p2--;
   while(*p2 == ' ') p2--; // p2指向非空格
   if (*p2 < '0' || *p2 > '9') // x y op 
   {
     op = *p2;
     p2--; // 掠过op
     while(*p2 == ' ') p2--; // p2指向非空格
     p1 = p2;
     while(*p1 != ' ') p1--; // 空格y
     y = xy(p1,p2);
     x = xy(s,p1);
   }
   else // x op y 
   { 续3 }
}
\end{lstlisting}
\end{frame}

\begin{frame}[shrink,fragile]{例: 表达式求值---主程序(续3)}
\begin{lstlisting}
else // x op y 
{
  p1 = s;
  while(*p1 == ' ') p1++; // 使p1指向非空格
  p2 = p1;
  while(*p2 != ' ') p2++; // x空格
  x = xy(p1,p2);
  while(*p2 == ' ') p2++; // op
  op = *p2;
  p1 = p2 + 1; // 掠过op
  p2 = p1;
  while(*p2 != '\0') p2++; // p2指向'\0'
  y = xy(p1,p2 - 1); // p2 - 1是最后一个有效字符 	
} 
\end{lstlisting}
\end{frame}

\section{期中考试总结}

\begin{frame}[shrink,fragile]{期中考试总结}
\begin{enumerate}
	\item 评分规则
	\begin{itemize}
		\item 将5个题的得分按从高到低排序, 记为S1, S2, S3, S4, S5
		\item 总分=S1*0.30+S2*0.25+S3*0.20+S4*0.15+S5*0.1
	\end{itemize}
	\item 第1题: 自由落体, 基本计算, 仅有1人没得满分。
	\item 第2题: 运费: 5人未得满分, 3人0分\\ 
	与lecture-6.pdf中的例题基本一致。考查switch语句或if else选择语句。
	\item 第3题: 二进制字符转换, 考查一重循环, 本题平均成绩最低44。\\
	lecture-8.pdf p15-16, 循环语句中接收字符的常见技巧。\\
	lecture-5.pdf 数位=ASCII编码值-'0'\\
	lecture-14.pdf 以递归形式输出一个整数的二进制位。\\
	课件及上机练习中涉及各种数字分解程序设计技巧。
	\item 第4题, 考查一重循环, 平均成绩86。\\
	lecture-9.pdf p24 原题, 迭代计算是基本编程要领, 必须掌握。
	\item 质数求和, 考查二重循环, 平均成绩51。\\
	lecture-8.pdf p8,p17 100$\sim$200间的素数。
	\item 本班题目简单, 其他班有类似机试练习中选QQ号题。期末考试会复杂些。
\end{enumerate}
\end{frame}

\begin{frame}[fragile]{第3题---二进制字符转换, 简单实现}
10101d $\implies$ 21\\
101,1  $\implies$ 5
\begin{lstlisting}
char ch;
int sum=0;
while(1)
{
    ch=getchar(); //或 scanf("%c",&ch);
    if(ch!='0' && ch !='1') break;
    sum=sum*2+ch-'0';
} 
printf("%d\n",sum);
\end{lstlisting}
\end{frame}

\begin{frame}[fragile]{第3题---二进制字符转换, 字符串数组实现}
\begin{lstlisting}
#define N 31 // 估计最大数组大小, 预留'\0'
char ch[N];
int sum=0,i;
gets(ch);
for(i=0;;i++)
{
   if(ch[i]!='0' && ch[i] !='1') break;
   sum=sum*2+ch[i]-'0';
} 
printf("%d\n",sum);
\end{lstlisting}
\end{frame}

\begin{frame}[fragile]{第3题---二进制字符转换, 指针实现}
\begin{lstlisting}
#define N 31 // 估计最大数组大小, 预留'\0'
char ch[N], *p=ch;
int sum=0;
gets(ch);
for(;;)
{
  if(*p != '0' && *p !='1') break;
  sum=sum*2+ (*p - '0');
  p++; // 指向下一个字符
} 
printf("%d\n",sum);
\end{lstlisting}
\end{frame}




